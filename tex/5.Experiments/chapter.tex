\expandafter\ifx\csname MasterFile\endcsname\relax
\def\SubFile{hoge}
\documentclass[a4j,12pt,twoside,openany]{jreport}
%\nofiles %tocファイルを更新させない
%\documentclass[12pt,a4j,twoside,openany]{jsbook}
\usepackage[dvipdfmx]{graphicx}
\usepackage{../dspc} % ベースラインスキップの指定
\usepackage{../slashbox} % 表に斜線を入れる
%\usepackage{../mediabb}
\usepackage{fancyvrb} % Verbatim環境
\usepackage{fancyhdr} % Headerの下線付き章見出し
\usepackage{here} % float[H]
\usepackage{multirow}
\usepackage{hhline} % 表の罫線の角を美しくする
\usepackage{amsmath} %コレがないとcasesが動かない
\usepackage{amsfonts} % 数学用フォント
\usepackage{bm} % 数式環境での bold
\usepackage{algorithm}
\usepackage{algorithmicx}
\usepackage[noend]{algpseudocode}
\usepackage[flushleft]{threeparttable} % 脚注付きテーブル
\usepackage{enumitem}
\usepackage{comment}
\usepackage{fancybox}
%\usepackage{csvsimple,booktabs,siunitx}
%\usepackage{filecontents}


\setlength{\evensidemargin}{5pt}
\setlength{\oddsidemargin}{40pt}
%\setlength{\headheight}{16.5pt}
%%\setlength{\headheight}{30pt}
\setcounter{secnumdepth}{3}
\setlist[description]{leftmargin=2\parindent,labelindent=\parindent}

\makeatletter
\def\@makechapterhead#1{%
	\vspace*{50\p@}%
	{
		\parindent \z@ \raggedright \normalfont
		\ifnum \c@secnumdepth >\m@ne
		% \if@mainmatter
			\huge\bfseries\@chapapp\thechapter\@chappos
			\par\nobreak
			\vskip 20\p@
		% \fi
		\fi
		\interlinepenalty\@M
		\Huge\bfseries #1\par\nobreak
		\vskip 40\p@
	}
}

%新しいコマンド定義
\newcounter{linenumber}
\newenvironment{listing}{%
  \begin{list}{%
    \small\arabic{linenumber}:}{%
      \usecounter{linenumber}%
      \setlength{\baselineskip}{18pt}%
      \setlength{\itemsep}{0pt}%
      \setlength{\parsep}{0pt}}}%
 {\end{list}}
\newcommand{\figcaption}[1]{\def\@captype{figure}\caption{#1}}
\newcommand{\tblcaption}[1]{\def\@captype{table}\caption{#1}}
\newcommand{\norm}[1]{\left\| #1 \right\|}
\newcommand{\cc}[1]{\multicolumn{1}{|c|}{#1}}
\newcommand{\circled}[1]{\raisebox{.5pt}{\textcircled{\raisebox{-.9pt} {#1}}}}
\newcommand{\specialcell}[2][c]{%
  \begin{tabular}[#1]{@{}c@{}}#2\end{tabular}}
\makeatother
%===============================================================================
\expandafter\ifx\csname SubFile\endcsname\relax
\begin{document}
\def\MasterFile{hoge}
%-------------------------------------------------------------------------------
%\maketitle
\thispagestyle{empty}
\begin{titlepage}

% 題名
\def\title{分散表現を用いた\\話題変化判定}
% 補助題名
\def\subtitle{卒業論文}
% 著者
\def\author{芳野 魁}
% 入学年度(平成)
\def\year{29}
% 学籍番号
\def\number{26115162}
% 指導教官
\def\kyoukan{伊藤 孝行}
% 指導教官役職
\def\kyoukanrank{教授}
% 提出日
\def\teisyutubi{平成29年12月28日}

\pagestyle{empty}

\begin{center}

\vspace*{20mm}
{\Large\mc 平成29年度 \hspace{7mm} 卒 業 論 文}
\vspace{15mm}

%\setlength{\unitlength}{1mm}
\begin{picture}(100,60)
  \put(0,0){\makebox(100,60){\huge\bf\shortstack{\title}}}
\end{picture}
\\
%\begin{picture}(100,5)
%  \put(0,0){\makebox(100,5){\Large\bf\shortstack{\subtitle}}}
%\end{picture}
\end{center}
\vspace{10mm}
\begin{flushright}
\begin{tabular}{ll}
{\large 提出日} & {\large {\teisyutubi}} \\
{\large 所属}  & {\large 名古屋工業大学 情報工学科} \\
{\large 指導教員} & {\large {\kyoukan} {\kyoukanrank}} \\
 & \\
{\large 入学年度} & {\large 平成26年度入学}\\
{\large 学籍番号} &{\large {\number}} \\
 & \\
%{\large 氏名} & {\huge {\author}}
{\large 氏名} & {\huge\mc {\author}}
\end{tabular}
\end{flushright}

\end{titlepage}

%\addcontentsline{toc}{chapter}{表紙}
\thispagestyle{empty}
\mbox{}\newpage
%===============================================================================
%\frontmatter
%===============================================================================
%\mainmatter
%-------------------------------------------------------------------------------
\pagenumbering{arabic}
\cleardoublepage
\expandafter\ifx\csname MasterFile\endcsname\relax
	\def\SubFile{hoge}
	\documentclass[a4j,12pt,twoside,openany]{jreport}
%\nofiles %tocファイルを更新させない
%\documentclass[12pt,a4j,twoside,openany]{jsbook}
\usepackage[dvipdfmx]{graphicx}
\usepackage{../dspc} % ベースラインスキップの指定
\usepackage{../slashbox} % 表に斜線を入れる
%\usepackage{../mediabb}
\usepackage{fancyvrb} % Verbatim環境
\usepackage{fancyhdr} % Headerの下線付き章見出し
\usepackage{here} % float[H]
\usepackage{multirow}
\usepackage{hhline} % 表の罫線の角を美しくする
\usepackage{amsmath} %コレがないとcasesが動かない
\usepackage{amsfonts} % 数学用フォント
\usepackage{bm} % 数式環境での bold
\usepackage{algorithm}
\usepackage{algorithmicx}
\usepackage[noend]{algpseudocode}
\usepackage[flushleft]{threeparttable} % 脚注付きテーブル
\usepackage{enumitem}
\usepackage{comment}
\usepackage{fancybox}
%\usepackage{csvsimple,booktabs,siunitx}
%\usepackage{filecontents}


\setlength{\evensidemargin}{5pt}
\setlength{\oddsidemargin}{40pt}
%\setlength{\headheight}{16.5pt}
%%\setlength{\headheight}{30pt}
\setcounter{secnumdepth}{3}
\setlist[description]{leftmargin=2\parindent,labelindent=\parindent}

\makeatletter
\def\@makechapterhead#1{%
	\vspace*{50\p@}%
	{
		\parindent \z@ \raggedright \normalfont
		\ifnum \c@secnumdepth >\m@ne
		% \if@mainmatter
			\huge\bfseries\@chapapp\thechapter\@chappos
			\par\nobreak
			\vskip 20\p@
		% \fi
		\fi
		\interlinepenalty\@M
		\Huge\bfseries #1\par\nobreak
		\vskip 40\p@
	}
}

%新しいコマンド定義
\newcounter{linenumber}
\newenvironment{listing}{%
  \begin{list}{%
    \small\arabic{linenumber}:}{%
      \usecounter{linenumber}%
      \setlength{\baselineskip}{18pt}%
      \setlength{\itemsep}{0pt}%
      \setlength{\parsep}{0pt}}}%
 {\end{list}}
\newcommand{\figcaption}[1]{\def\@captype{figure}\caption{#1}}
\newcommand{\tblcaption}[1]{\def\@captype{table}\caption{#1}}
\newcommand{\norm}[1]{\left\| #1 \right\|}
\newcommand{\cc}[1]{\multicolumn{1}{|c|}{#1}}
\newcommand{\circled}[1]{\raisebox{.5pt}{\textcircled{\raisebox{-.9pt} {#1}}}}
\newcommand{\specialcell}[2][c]{%
  \begin{tabular}[#1]{@{}c@{}}#2\end{tabular}}
\makeatother
%===============================================================================
\expandafter\ifx\csname SubFile\endcsname\relax
\begin{document}
\def\MasterFile{hoge}
%-------------------------------------------------------------------------------
%\maketitle
\thispagestyle{empty}
\input{../hyoushi/title}
%\addcontentsline{toc}{chapter}{表紙}
\thispagestyle{empty}
\mbox{}\newpage
%===============================================================================
%\frontmatter
%===============================================================================
%\mainmatter
%-------------------------------------------------------------------------------
\pagenumbering{arabic}
\cleardoublepage
\input{../0.Abstract/chapter}
%-------------------------------------------------------------------------------
\clearpage
\addcontentsline{toc}{chapter}{目次}
\tableofcontents

\clearpage
\addcontentsline{toc}{chapter}{図目次}
\listoffigures

\clearpage
\addcontentsline{toc}{chapter}{表目次}
\listoftables

%-------------------------------------------------------------------------------

%=====================
\pagestyle{fancy} % Headerをつける
\renewcommand{\sectionmark}[1]{\markright{\thesection\ \ \ #1}}
\renewcommand{\chaptermark}[1]{\markboth{#1}{}}
\lhead{}
\chead{}
\lfoot{}
\rfoot{}%-------------------------------------------------------------------------------
\input{../1.Introduction/chapter}
%-------------------------------------------------------------------------------
\input{../2.Related_Work/chapter}
%-------------------------------------------------------------------------------
\input{../3.The_Model/chapter}
%-------------------------------------------------------------------------------
\input{../4.Implementation/chapter}
%-------------------------------------------------------------------------------
\input{../5.Experiments/chapter}
%-------------------------------------------------------------------------------
\input{../6.Conclusion/chapter}

%===============================================================================
\pagestyle{plain}
%-------------------------------------------------------------------------------
\input{../7.Acknowledgement/chapter} %謝辞
%-------------------------------------------------------------------------------
\def\BibFile{../Bibliograhoy/database2}
\input{../Bibliography/chapter} %参考文献
% %===============================================================================
\appendix
\input{../A.Mypaper/chapter} % 投稿論文リスト
\input{../B.SIG-CCI2/chapter} %
\input{../C.IJCAI-16/chapter} %
%===============================================================================
\end{document}
\fi

	\begin{document}
	\setcounter{chapter}{0}
	\fi
  %-------------------------------------------------------------------------------
\cleardoublepage
\chapter{序論}
\label{intro:chapter}
%本章では, 本研究を行なうに至った背景と目的について述べる.その後,本論文の構成について述べる.
\section{研究の背景}
\label{intro:background}
近年,Web上での大規模な議論活動が活発になっているが,現在一般的に使われている "2ちゃんねる" や "Twitter" といったシステムでは整理や収束を行うことが困難である.困難である原因として,議論の管理を行う者がいないことが挙げられる.
つまり,議論を整理・収束させるには議論のマネジメントを行う人物が必要である.
%
大規模意見集約システムCOLLAGREE\cite{collagreeTest}ではファシリテーターと呼ばれる人物が議論のマネジメントを行っている.
しかし,ファシリテーターは人間であり,長時間に渡って大人数での議論の動向をマネジメントし続けるのは困難である.
COLLAGREEで大規模な議論を収束させるためには,ファシリテーターが必要な時には画面を見るようにして,他の時は見なくても済むようにすることで画面に向き合う時間を減らす工夫があることが望ましい.ファシリテーターが画面を見るべきタイミングは議論の話題が変化したときである.以前の議論の内容から外れた発言がされた時,ファシリテーターが適切な発言をすることで,脱線や炎上を避けて議論を収束させることができる.
すなわち,ファシリテーターの代わりに自動的に議論中の話題の変化を観測することが求められている.
%
現在,COLLAGREE上で使用されている議論支援システムは「(1)投稿支援システム」と「(2)議論可視化システム」の2つに大別できる.
投稿支援システムはポイント機能やファシリテーションフレーズ簡易投稿機能のように,ユーザーが投稿をする際に何らかの補助やリアクションを行う.現行の機能では選択肢の提示に留まっており,作業量を減らすことには繋がりにくい.
一方,議論可視化システムは議論ツリーやキーワード抽出のように,ユーザーにスレッドとは異なる議論の見方を提供する.
\ref{Fig:argTree1}に議論ツリーの例を示す.
\begin{figure}[htbp]
 \begin{center}
  \includegraphics[width=\textwidth]{../images/2.Related_Work/argTree1.png}
  \caption{議論ツリー}
  \label{Fig:argTree1}
  \vspace{-10pt}
 \end{center}
\end{figure}
現行の機能では議論を見やすくすることに重点が置かれており,議論の把握の助けにはなるが画面に向き合う時間を減らすことにはなりにくい.むしろ,作業量を増やすことになり得ることもある.
従って,現行の支援機能ではファシリテーターの作業量の減少には繋がりにくい.
%
近年,自然言語処理の分野において分散表現が多くの研究で使われており,機械翻訳を始めとする単語の意味が重要となる分野で精度の向上が確認されている.分散表現を用いることで,人間に近い精度で話題の変化を観測することが可能となる.
%
以上のような背景を踏まえて,分散表現を用いて,話題の変化を観測し,話題の変化が確認された時にファシリテーターに伝えることが望ましい.
話題の変化の観測は,発言中に現れる単語の類似度の計算と見なすことができる.
分散表現を用いることで単語間の類似度を求めることができる,値が大きいほど単語がそれぞれ類似した実数ベクトルであることを表す.単語Aと単語Bの実数ベクトルが類似しているとは,単語Aと共に使われることの多い単語と単語Bと共に使われることの多い単語が多く共通していることを示す.故に,分散表現を使って単語の類似度を計算することができる.
%
発言文から単語を選ぶ際には自動要約を用いる.発言文から重要でない単語を取り除くことで関連度の計算の精度を高めることが可能となる.
要約の手法としてはokapi BM25 \cite{okapiBM25}とLexRankを組み合わせた抽出的要約手法を用いる.
\begin{comment}
%======================================= 社会的背景
2013年頃からWeb上での大規模な議論活動が活発になり,大規模な人数での議論が期待されている.
大規模な議論では意見を共有することは可能であるが,議論を整理させることや収束させることは難しい.以上から大規模意見集約システムCOLLAGREEが開発された.本システムではWeb上で適切に大規模な議論を行うことができるように議論をマネジメントするファシリテーターを導入した\cite{collagreeTest}.
過去の実験ではファシリテーターの存在が議論の集約に大きな役割を果たしていることが認識されており,大規模な議論のためにファシリテータは必要である.しかし,議論の規模に伴って議論時間が長くなる傾向があり,同時にファシリテーターは常に議論の動向を見続ける必要がある.故に,議論の規模が大きくなればなるほどファシリテーターは長時間かつ大規模な議論の動向の監視によって大きな負担がかかる.大規模な議論が増加する傾向を踏まえるとファシリテーターにかかる負担を軽減する支援が必要である.\\
以上の問題を解決するため,話題の変化を追い,重要な話題の転換点をファシリテーターの代わりに検出することが有用であると考える.必要な時にだけファシリテーターが画面を見れば良いようにすることでファシリテーターの負担軽減が期待できる.
%========================================= 現行手法問題点背景
%議論支援に関する先行研究において,既存の手法は全てが文字列を文字列のまま扱う手法である.
%既存手法は殆どがパターンマッチングと重み付けの2つに区分することができる.
%パターンマッチングでは事前に単語を登録して,単語がマッチした場合に処理を行うが,処理それぞれに対して単語を登録しなければならず手間が膨大になってしまう.また,単語の意味が考慮されておらず,手作業で登録を行うので登録漏れがあった場合に単語の意味に関係なく処理を行うことが不可能となってしまう.
%重み付けは単語の出現頻度や文章の長さを使用して単語・文章に順位を付ける手法で必ずしも単語の登録が必要でないため多くの研究で使用されている.
%しかし,重み付けもまた単語を文字列のまま扱っており,意味までは考慮されていない.故に ,人間なら対応できる似た単語でも1文字違うだけで対処が困難となる.
議論支援に関する先行研究においてファシリテーターに対する支援を目的としたものは無く,殆どが議論の活性化や可視化を目的としている.
%=================================新手法
近年,自然言語処理の分野において分散表現が多くの研究で使われており.分散表現は文字列である単語を辞書データを使用して実数ベクトルへと変換する.辞書データにない単語には対応できないが,多様な処理を1つの辞書データで行うことができる.また,実数ベクトルの各数値が単語の意味を表現するものとなっており,数値を使用して処理を行うことができる.
分散表現を用いることで既存手法より人間の感覚に近しい処理を行うことができる.
%=================================
以上のような背景を踏まえて,分散表現を用いてファシリテーターの代わりに話題の変化を判定し,知らせることを目指す.
話題転換の検出は発言同士の近さ,すなわち発言に含まれる単語意味の近さと見ることができる.
分散表現ではベクトル同士の内積計算を行うことで単語同士の意味の近さを計算することができる.
また,分散表現を使用することで機械翻訳を始めとする複数の分野で精度の向上が確認されている.
\end{comment}
\section{研究の目的}
\label{intro:taget}
本論文では,分散表現を用いて議論中での発言に含まれる単語の関連度を計算し,話題の変化を観測する手法を提案する.

\section{本論文の構成}
本論文の構成を以下に示す.
\ref{relwork:chapter} 章では要約手法に関する研究と,分散表現に関する先行研究を紹介する.
次に,\ref{model:chapter}章では発言の要約手法の説明を行い,\ref{impl:chapter}章では分散表現を用いた単語集合間の関連度計算について説明する.
そして,\ref{exp:chapter}章では話題転換点の検出の評価実験について説明する.
最後に\ref{con:chapter}章で本論文のまとめと考察を示す.

 %-------------------------------------------------------------------------------
 \expandafter\ifx\csname MasterFile\endcsname\relax
	\def\BibFile{hoge}
	\expandafter\ifx\csname MasterFile\endcsname\relax
	\def\SubFile{hoge}
	\input{../thesis/thesis}
	\begin{document}
	\setcounter{chapter}{0}
	\fi
  %-------------------------------------------------------------------------------
\cleardoublepage
\chapter{序論}
\label{intro:chapter}
%本章では, 本研究を行なうに至った背景と目的について述べる.その後,本論文の構成について述べる.
\section{研究の背景}
\label{intro:background}
近年,Web上での大規模な議論活動が活発になっているが,現在一般的に使われている "2ちゃんねる" や "Twitter" といったシステムでは整理や収束を行うことが困難である.困難である原因として,議論の管理を行う者がいないことが挙げられる.
つまり,議論を整理・収束させるには議論のマネジメントを行う人物が必要である.
%
大規模意見集約システムCOLLAGREE\cite{collagreeTest}ではファシリテーターと呼ばれる人物が議論のマネジメントを行っている.
しかし,ファシリテーターは人間であり,長時間に渡って大人数での議論の動向をマネジメントし続けるのは困難である.
COLLAGREEで大規模な議論を収束させるためには,ファシリテーターが必要な時には画面を見るようにして,他の時は見なくても済むようにすることで画面に向き合う時間を減らす工夫があることが望ましい.ファシリテーターが画面を見るべきタイミングは議論の話題が変化したときである.以前の議論の内容から外れた発言がされた時,ファシリテーターが適切な発言をすることで,脱線や炎上を避けて議論を収束させることができる.
すなわち,ファシリテーターの代わりに自動的に議論中の話題の変化を観測することが求められている.
%
現在,COLLAGREE上で使用されている議論支援システムは「(1)投稿支援システム」と「(2)議論可視化システム」の2つに大別できる.
投稿支援システムはポイント機能やファシリテーションフレーズ簡易投稿機能のように,ユーザーが投稿をする際に何らかの補助やリアクションを行う.現行の機能では選択肢の提示に留まっており,作業量を減らすことには繋がりにくい.
一方,議論可視化システムは議論ツリーやキーワード抽出のように,ユーザーにスレッドとは異なる議論の見方を提供する.
\ref{Fig:argTree1}に議論ツリーの例を示す.
\begin{figure}[htbp]
 \begin{center}
  \includegraphics[width=\textwidth]{../images/2.Related_Work/argTree1.png}
  \caption{議論ツリー}
  \label{Fig:argTree1}
  \vspace{-10pt}
 \end{center}
\end{figure}
現行の機能では議論を見やすくすることに重点が置かれており,議論の把握の助けにはなるが画面に向き合う時間を減らすことにはなりにくい.むしろ,作業量を増やすことになり得ることもある.
従って,現行の支援機能ではファシリテーターの作業量の減少には繋がりにくい.
%
近年,自然言語処理の分野において分散表現が多くの研究で使われており,機械翻訳を始めとする単語の意味が重要となる分野で精度の向上が確認されている.分散表現を用いることで,人間に近い精度で話題の変化を観測することが可能となる.
%
以上のような背景を踏まえて,分散表現を用いて,話題の変化を観測し,話題の変化が確認された時にファシリテーターに伝えることが望ましい.
話題の変化の観測は,発言中に現れる単語の類似度の計算と見なすことができる.
分散表現を用いることで単語間の類似度を求めることができる,値が大きいほど単語がそれぞれ類似した実数ベクトルであることを表す.単語Aと単語Bの実数ベクトルが類似しているとは,単語Aと共に使われることの多い単語と単語Bと共に使われることの多い単語が多く共通していることを示す.故に,分散表現を使って単語の類似度を計算することができる.
%
発言文から単語を選ぶ際には自動要約を用いる.発言文から重要でない単語を取り除くことで関連度の計算の精度を高めることが可能となる.
要約の手法としてはokapi BM25 \cite{okapiBM25}とLexRankを組み合わせた抽出的要約手法を用いる.
\begin{comment}
%======================================= 社会的背景
2013年頃からWeb上での大規模な議論活動が活発になり,大規模な人数での議論が期待されている.
大規模な議論では意見を共有することは可能であるが,議論を整理させることや収束させることは難しい.以上から大規模意見集約システムCOLLAGREEが開発された.本システムではWeb上で適切に大規模な議論を行うことができるように議論をマネジメントするファシリテーターを導入した\cite{collagreeTest}.
過去の実験ではファシリテーターの存在が議論の集約に大きな役割を果たしていることが認識されており,大規模な議論のためにファシリテータは必要である.しかし,議論の規模に伴って議論時間が長くなる傾向があり,同時にファシリテーターは常に議論の動向を見続ける必要がある.故に,議論の規模が大きくなればなるほどファシリテーターは長時間かつ大規模な議論の動向の監視によって大きな負担がかかる.大規模な議論が増加する傾向を踏まえるとファシリテーターにかかる負担を軽減する支援が必要である.\\
以上の問題を解決するため,話題の変化を追い,重要な話題の転換点をファシリテーターの代わりに検出することが有用であると考える.必要な時にだけファシリテーターが画面を見れば良いようにすることでファシリテーターの負担軽減が期待できる.
%========================================= 現行手法問題点背景
%議論支援に関する先行研究において,既存の手法は全てが文字列を文字列のまま扱う手法である.
%既存手法は殆どがパターンマッチングと重み付けの2つに区分することができる.
%パターンマッチングでは事前に単語を登録して,単語がマッチした場合に処理を行うが,処理それぞれに対して単語を登録しなければならず手間が膨大になってしまう.また,単語の意味が考慮されておらず,手作業で登録を行うので登録漏れがあった場合に単語の意味に関係なく処理を行うことが不可能となってしまう.
%重み付けは単語の出現頻度や文章の長さを使用して単語・文章に順位を付ける手法で必ずしも単語の登録が必要でないため多くの研究で使用されている.
%しかし,重み付けもまた単語を文字列のまま扱っており,意味までは考慮されていない.故に ,人間なら対応できる似た単語でも1文字違うだけで対処が困難となる.
議論支援に関する先行研究においてファシリテーターに対する支援を目的としたものは無く,殆どが議論の活性化や可視化を目的としている.
%=================================新手法
近年,自然言語処理の分野において分散表現が多くの研究で使われており.分散表現は文字列である単語を辞書データを使用して実数ベクトルへと変換する.辞書データにない単語には対応できないが,多様な処理を1つの辞書データで行うことができる.また,実数ベクトルの各数値が単語の意味を表現するものとなっており,数値を使用して処理を行うことができる.
分散表現を用いることで既存手法より人間の感覚に近しい処理を行うことができる.
%=================================
以上のような背景を踏まえて,分散表現を用いてファシリテーターの代わりに話題の変化を判定し,知らせることを目指す.
話題転換の検出は発言同士の近さ,すなわち発言に含まれる単語意味の近さと見ることができる.
分散表現ではベクトル同士の内積計算を行うことで単語同士の意味の近さを計算することができる.
また,分散表現を使用することで機械翻訳を始めとする複数の分野で精度の向上が確認されている.
\end{comment}
\section{研究の目的}
\label{intro:taget}
本論文では,分散表現を用いて議論中での発言に含まれる単語の関連度を計算し,話題の変化を観測する手法を提案する.

\section{本論文の構成}
本論文の構成を以下に示す.
\ref{relwork:chapter} 章では要約手法に関する研究と,分散表現に関する先行研究を紹介する.
次に,\ref{model:chapter}章では発言の要約手法の説明を行い,\ref{impl:chapter}章では分散表現を用いた単語集合間の関連度計算について説明する.
そして,\ref{exp:chapter}章では話題転換点の検出の評価実験について説明する.
最後に\ref{con:chapter}章で本論文のまとめと考察を示す.

 %-------------------------------------------------------------------------------
 \expandafter\ifx\csname MasterFile\endcsname\relax
	\def\BibFile{hoge}
	\input{../Bibliography/chapter}
  \fi
  %-------------------------------------------------------------------------------
  \expandafter\ifx\csname MasterFile\endcsname\relax
  \end{document}
  \fi

  \fi
  %-------------------------------------------------------------------------------
  \expandafter\ifx\csname MasterFile\endcsname\relax
  \end{document}
  \fi

%-------------------------------------------------------------------------------
\clearpage
\addcontentsline{toc}{chapter}{目次}
\tableofcontents

\clearpage
\addcontentsline{toc}{chapter}{図目次}
\listoffigures

\clearpage
\addcontentsline{toc}{chapter}{表目次}
\listoftables

%-------------------------------------------------------------------------------

%=====================
\pagestyle{fancy} % Headerをつける
\renewcommand{\sectionmark}[1]{\markright{\thesection\ \ \ #1}}
\renewcommand{\chaptermark}[1]{\markboth{#1}{}}
\lhead{}
\chead{}
\lfoot{}
\rfoot{}%-------------------------------------------------------------------------------
\expandafter\ifx\csname MasterFile\endcsname\relax
	\def\SubFile{hoge}
	\documentclass[a4j,12pt,twoside,openany]{jreport}
%\nofiles %tocファイルを更新させない
%\documentclass[12pt,a4j,twoside,openany]{jsbook}
\usepackage[dvipdfmx]{graphicx}
\usepackage{../dspc} % ベースラインスキップの指定
\usepackage{../slashbox} % 表に斜線を入れる
%\usepackage{../mediabb}
\usepackage{fancyvrb} % Verbatim環境
\usepackage{fancyhdr} % Headerの下線付き章見出し
\usepackage{here} % float[H]
\usepackage{multirow}
\usepackage{hhline} % 表の罫線の角を美しくする
\usepackage{amsmath} %コレがないとcasesが動かない
\usepackage{amsfonts} % 数学用フォント
\usepackage{bm} % 数式環境での bold
\usepackage{algorithm}
\usepackage{algorithmicx}
\usepackage[noend]{algpseudocode}
\usepackage[flushleft]{threeparttable} % 脚注付きテーブル
\usepackage{enumitem}
\usepackage{comment}
\usepackage{fancybox}
%\usepackage{csvsimple,booktabs,siunitx}
%\usepackage{filecontents}


\setlength{\evensidemargin}{5pt}
\setlength{\oddsidemargin}{40pt}
%\setlength{\headheight}{16.5pt}
%%\setlength{\headheight}{30pt}
\setcounter{secnumdepth}{3}
\setlist[description]{leftmargin=2\parindent,labelindent=\parindent}

\makeatletter
\def\@makechapterhead#1{%
	\vspace*{50\p@}%
	{
		\parindent \z@ \raggedright \normalfont
		\ifnum \c@secnumdepth >\m@ne
		% \if@mainmatter
			\huge\bfseries\@chapapp\thechapter\@chappos
			\par\nobreak
			\vskip 20\p@
		% \fi
		\fi
		\interlinepenalty\@M
		\Huge\bfseries #1\par\nobreak
		\vskip 40\p@
	}
}

%新しいコマンド定義
\newcounter{linenumber}
\newenvironment{listing}{%
  \begin{list}{%
    \small\arabic{linenumber}:}{%
      \usecounter{linenumber}%
      \setlength{\baselineskip}{18pt}%
      \setlength{\itemsep}{0pt}%
      \setlength{\parsep}{0pt}}}%
 {\end{list}}
\newcommand{\figcaption}[1]{\def\@captype{figure}\caption{#1}}
\newcommand{\tblcaption}[1]{\def\@captype{table}\caption{#1}}
\newcommand{\norm}[1]{\left\| #1 \right\|}
\newcommand{\cc}[1]{\multicolumn{1}{|c|}{#1}}
\newcommand{\circled}[1]{\raisebox{.5pt}{\textcircled{\raisebox{-.9pt} {#1}}}}
\newcommand{\specialcell}[2][c]{%
  \begin{tabular}[#1]{@{}c@{}}#2\end{tabular}}
\makeatother
%===============================================================================
\expandafter\ifx\csname SubFile\endcsname\relax
\begin{document}
\def\MasterFile{hoge}
%-------------------------------------------------------------------------------
%\maketitle
\thispagestyle{empty}
\input{../hyoushi/title}
%\addcontentsline{toc}{chapter}{表紙}
\thispagestyle{empty}
\mbox{}\newpage
%===============================================================================
%\frontmatter
%===============================================================================
%\mainmatter
%-------------------------------------------------------------------------------
\pagenumbering{arabic}
\cleardoublepage
\input{../0.Abstract/chapter}
%-------------------------------------------------------------------------------
\clearpage
\addcontentsline{toc}{chapter}{目次}
\tableofcontents

\clearpage
\addcontentsline{toc}{chapter}{図目次}
\listoffigures

\clearpage
\addcontentsline{toc}{chapter}{表目次}
\listoftables

%-------------------------------------------------------------------------------

%=====================
\pagestyle{fancy} % Headerをつける
\renewcommand{\sectionmark}[1]{\markright{\thesection\ \ \ #1}}
\renewcommand{\chaptermark}[1]{\markboth{#1}{}}
\lhead{}
\chead{}
\lfoot{}
\rfoot{}%-------------------------------------------------------------------------------
\input{../1.Introduction/chapter}
%-------------------------------------------------------------------------------
\input{../2.Related_Work/chapter}
%-------------------------------------------------------------------------------
\input{../3.The_Model/chapter}
%-------------------------------------------------------------------------------
\input{../4.Implementation/chapter}
%-------------------------------------------------------------------------------
\input{../5.Experiments/chapter}
%-------------------------------------------------------------------------------
\input{../6.Conclusion/chapter}

%===============================================================================
\pagestyle{plain}
%-------------------------------------------------------------------------------
\input{../7.Acknowledgement/chapter} %謝辞
%-------------------------------------------------------------------------------
\def\BibFile{../Bibliograhoy/database2}
\input{../Bibliography/chapter} %参考文献
% %===============================================================================
\appendix
\input{../A.Mypaper/chapter} % 投稿論文リスト
\input{../B.SIG-CCI2/chapter} %
\input{../C.IJCAI-16/chapter} %
%===============================================================================
\end{document}
\fi

	\begin{document}
	\setcounter{chapter}{0}
	\fi
  %-------------------------------------------------------------------------------
\cleardoublepage
\chapter{序論}
\label{intro:chapter}
%本章では, 本研究を行なうに至った背景と目的について述べる.その後,本論文の構成について述べる.
\section{研究の背景}
\label{intro:background}
近年,Web上での大規模な議論活動が活発になっているが,現在一般的に使われている "2ちゃんねる" や "Twitter" といったシステムでは整理や収束を行うことが困難である.困難である原因として,議論の管理を行う者がいないことが挙げられる.
つまり,議論を整理・収束させるには議論のマネジメントを行う人物が必要である.
%
大規模意見集約システムCOLLAGREE\cite{collagreeTest}ではファシリテーターと呼ばれる人物が議論のマネジメントを行っている.
しかし,ファシリテーターは人間であり,長時間に渡って大人数での議論の動向をマネジメントし続けるのは困難である.
COLLAGREEで大規模な議論を収束させるためには,ファシリテーターが必要な時には画面を見るようにして,他の時は見なくても済むようにすることで画面に向き合う時間を減らす工夫があることが望ましい.ファシリテーターが画面を見るべきタイミングは議論の話題が変化したときである.以前の議論の内容から外れた発言がされた時,ファシリテーターが適切な発言をすることで,脱線や炎上を避けて議論を収束させることができる.
すなわち,ファシリテーターの代わりに自動的に議論中の話題の変化を観測することが求められている.
%
現在,COLLAGREE上で使用されている議論支援システムは「(1)投稿支援システム」と「(2)議論可視化システム」の2つに大別できる.
投稿支援システムはポイント機能やファシリテーションフレーズ簡易投稿機能のように,ユーザーが投稿をする際に何らかの補助やリアクションを行う.現行の機能では選択肢の提示に留まっており,作業量を減らすことには繋がりにくい.
一方,議論可視化システムは議論ツリーやキーワード抽出のように,ユーザーにスレッドとは異なる議論の見方を提供する.
\ref{Fig:argTree1}に議論ツリーの例を示す.
\begin{figure}[htbp]
 \begin{center}
  \includegraphics[width=\textwidth]{../images/2.Related_Work/argTree1.png}
  \caption{議論ツリー}
  \label{Fig:argTree1}
  \vspace{-10pt}
 \end{center}
\end{figure}
現行の機能では議論を見やすくすることに重点が置かれており,議論の把握の助けにはなるが画面に向き合う時間を減らすことにはなりにくい.むしろ,作業量を増やすことになり得ることもある.
従って,現行の支援機能ではファシリテーターの作業量の減少には繋がりにくい.
%
近年,自然言語処理の分野において分散表現が多くの研究で使われており,機械翻訳を始めとする単語の意味が重要となる分野で精度の向上が確認されている.分散表現を用いることで,人間に近い精度で話題の変化を観測することが可能となる.
%
以上のような背景を踏まえて,分散表現を用いて,話題の変化を観測し,話題の変化が確認された時にファシリテーターに伝えることが望ましい.
話題の変化の観測は,発言中に現れる単語の類似度の計算と見なすことができる.
分散表現を用いることで単語間の類似度を求めることができる,値が大きいほど単語がそれぞれ類似した実数ベクトルであることを表す.単語Aと単語Bの実数ベクトルが類似しているとは,単語Aと共に使われることの多い単語と単語Bと共に使われることの多い単語が多く共通していることを示す.故に,分散表現を使って単語の類似度を計算することができる.
%
発言文から単語を選ぶ際には自動要約を用いる.発言文から重要でない単語を取り除くことで関連度の計算の精度を高めることが可能となる.
要約の手法としてはokapi BM25 \cite{okapiBM25}とLexRankを組み合わせた抽出的要約手法を用いる.
\begin{comment}
%======================================= 社会的背景
2013年頃からWeb上での大規模な議論活動が活発になり,大規模な人数での議論が期待されている.
大規模な議論では意見を共有することは可能であるが,議論を整理させることや収束させることは難しい.以上から大規模意見集約システムCOLLAGREEが開発された.本システムではWeb上で適切に大規模な議論を行うことができるように議論をマネジメントするファシリテーターを導入した\cite{collagreeTest}.
過去の実験ではファシリテーターの存在が議論の集約に大きな役割を果たしていることが認識されており,大規模な議論のためにファシリテータは必要である.しかし,議論の規模に伴って議論時間が長くなる傾向があり,同時にファシリテーターは常に議論の動向を見続ける必要がある.故に,議論の規模が大きくなればなるほどファシリテーターは長時間かつ大規模な議論の動向の監視によって大きな負担がかかる.大規模な議論が増加する傾向を踏まえるとファシリテーターにかかる負担を軽減する支援が必要である.\\
以上の問題を解決するため,話題の変化を追い,重要な話題の転換点をファシリテーターの代わりに検出することが有用であると考える.必要な時にだけファシリテーターが画面を見れば良いようにすることでファシリテーターの負担軽減が期待できる.
%========================================= 現行手法問題点背景
%議論支援に関する先行研究において,既存の手法は全てが文字列を文字列のまま扱う手法である.
%既存手法は殆どがパターンマッチングと重み付けの2つに区分することができる.
%パターンマッチングでは事前に単語を登録して,単語がマッチした場合に処理を行うが,処理それぞれに対して単語を登録しなければならず手間が膨大になってしまう.また,単語の意味が考慮されておらず,手作業で登録を行うので登録漏れがあった場合に単語の意味に関係なく処理を行うことが不可能となってしまう.
%重み付けは単語の出現頻度や文章の長さを使用して単語・文章に順位を付ける手法で必ずしも単語の登録が必要でないため多くの研究で使用されている.
%しかし,重み付けもまた単語を文字列のまま扱っており,意味までは考慮されていない.故に ,人間なら対応できる似た単語でも1文字違うだけで対処が困難となる.
議論支援に関する先行研究においてファシリテーターに対する支援を目的としたものは無く,殆どが議論の活性化や可視化を目的としている.
%=================================新手法
近年,自然言語処理の分野において分散表現が多くの研究で使われており.分散表現は文字列である単語を辞書データを使用して実数ベクトルへと変換する.辞書データにない単語には対応できないが,多様な処理を1つの辞書データで行うことができる.また,実数ベクトルの各数値が単語の意味を表現するものとなっており,数値を使用して処理を行うことができる.
分散表現を用いることで既存手法より人間の感覚に近しい処理を行うことができる.
%=================================
以上のような背景を踏まえて,分散表現を用いてファシリテーターの代わりに話題の変化を判定し,知らせることを目指す.
話題転換の検出は発言同士の近さ,すなわち発言に含まれる単語意味の近さと見ることができる.
分散表現ではベクトル同士の内積計算を行うことで単語同士の意味の近さを計算することができる.
また,分散表現を使用することで機械翻訳を始めとする複数の分野で精度の向上が確認されている.
\end{comment}
\section{研究の目的}
\label{intro:taget}
本論文では,分散表現を用いて議論中での発言に含まれる単語の関連度を計算し,話題の変化を観測する手法を提案する.

\section{本論文の構成}
本論文の構成を以下に示す.
\ref{relwork:chapter} 章では要約手法に関する研究と,分散表現に関する先行研究を紹介する.
次に,\ref{model:chapter}章では発言の要約手法の説明を行い,\ref{impl:chapter}章では分散表現を用いた単語集合間の関連度計算について説明する.
そして,\ref{exp:chapter}章では話題転換点の検出の評価実験について説明する.
最後に\ref{con:chapter}章で本論文のまとめと考察を示す.

 %-------------------------------------------------------------------------------
 \expandafter\ifx\csname MasterFile\endcsname\relax
	\def\BibFile{hoge}
	\expandafter\ifx\csname MasterFile\endcsname\relax
	\def\SubFile{hoge}
	\input{../thesis/thesis}
	\begin{document}
	\setcounter{chapter}{0}
	\fi
  %-------------------------------------------------------------------------------
\cleardoublepage
\chapter{序論}
\label{intro:chapter}
%本章では, 本研究を行なうに至った背景と目的について述べる.その後,本論文の構成について述べる.
\section{研究の背景}
\label{intro:background}
近年,Web上での大規模な議論活動が活発になっているが,現在一般的に使われている "2ちゃんねる" や "Twitter" といったシステムでは整理や収束を行うことが困難である.困難である原因として,議論の管理を行う者がいないことが挙げられる.
つまり,議論を整理・収束させるには議論のマネジメントを行う人物が必要である.
%
大規模意見集約システムCOLLAGREE\cite{collagreeTest}ではファシリテーターと呼ばれる人物が議論のマネジメントを行っている.
しかし,ファシリテーターは人間であり,長時間に渡って大人数での議論の動向をマネジメントし続けるのは困難である.
COLLAGREEで大規模な議論を収束させるためには,ファシリテーターが必要な時には画面を見るようにして,他の時は見なくても済むようにすることで画面に向き合う時間を減らす工夫があることが望ましい.ファシリテーターが画面を見るべきタイミングは議論の話題が変化したときである.以前の議論の内容から外れた発言がされた時,ファシリテーターが適切な発言をすることで,脱線や炎上を避けて議論を収束させることができる.
すなわち,ファシリテーターの代わりに自動的に議論中の話題の変化を観測することが求められている.
%
現在,COLLAGREE上で使用されている議論支援システムは「(1)投稿支援システム」と「(2)議論可視化システム」の2つに大別できる.
投稿支援システムはポイント機能やファシリテーションフレーズ簡易投稿機能のように,ユーザーが投稿をする際に何らかの補助やリアクションを行う.現行の機能では選択肢の提示に留まっており,作業量を減らすことには繋がりにくい.
一方,議論可視化システムは議論ツリーやキーワード抽出のように,ユーザーにスレッドとは異なる議論の見方を提供する.
\ref{Fig:argTree1}に議論ツリーの例を示す.
\begin{figure}[htbp]
 \begin{center}
  \includegraphics[width=\textwidth]{../images/2.Related_Work/argTree1.png}
  \caption{議論ツリー}
  \label{Fig:argTree1}
  \vspace{-10pt}
 \end{center}
\end{figure}
現行の機能では議論を見やすくすることに重点が置かれており,議論の把握の助けにはなるが画面に向き合う時間を減らすことにはなりにくい.むしろ,作業量を増やすことになり得ることもある.
従って,現行の支援機能ではファシリテーターの作業量の減少には繋がりにくい.
%
近年,自然言語処理の分野において分散表現が多くの研究で使われており,機械翻訳を始めとする単語の意味が重要となる分野で精度の向上が確認されている.分散表現を用いることで,人間に近い精度で話題の変化を観測することが可能となる.
%
以上のような背景を踏まえて,分散表現を用いて,話題の変化を観測し,話題の変化が確認された時にファシリテーターに伝えることが望ましい.
話題の変化の観測は,発言中に現れる単語の類似度の計算と見なすことができる.
分散表現を用いることで単語間の類似度を求めることができる,値が大きいほど単語がそれぞれ類似した実数ベクトルであることを表す.単語Aと単語Bの実数ベクトルが類似しているとは,単語Aと共に使われることの多い単語と単語Bと共に使われることの多い単語が多く共通していることを示す.故に,分散表現を使って単語の類似度を計算することができる.
%
発言文から単語を選ぶ際には自動要約を用いる.発言文から重要でない単語を取り除くことで関連度の計算の精度を高めることが可能となる.
要約の手法としてはokapi BM25 \cite{okapiBM25}とLexRankを組み合わせた抽出的要約手法を用いる.
\begin{comment}
%======================================= 社会的背景
2013年頃からWeb上での大規模な議論活動が活発になり,大規模な人数での議論が期待されている.
大規模な議論では意見を共有することは可能であるが,議論を整理させることや収束させることは難しい.以上から大規模意見集約システムCOLLAGREEが開発された.本システムではWeb上で適切に大規模な議論を行うことができるように議論をマネジメントするファシリテーターを導入した\cite{collagreeTest}.
過去の実験ではファシリテーターの存在が議論の集約に大きな役割を果たしていることが認識されており,大規模な議論のためにファシリテータは必要である.しかし,議論の規模に伴って議論時間が長くなる傾向があり,同時にファシリテーターは常に議論の動向を見続ける必要がある.故に,議論の規模が大きくなればなるほどファシリテーターは長時間かつ大規模な議論の動向の監視によって大きな負担がかかる.大規模な議論が増加する傾向を踏まえるとファシリテーターにかかる負担を軽減する支援が必要である.\\
以上の問題を解決するため,話題の変化を追い,重要な話題の転換点をファシリテーターの代わりに検出することが有用であると考える.必要な時にだけファシリテーターが画面を見れば良いようにすることでファシリテーターの負担軽減が期待できる.
%========================================= 現行手法問題点背景
%議論支援に関する先行研究において,既存の手法は全てが文字列を文字列のまま扱う手法である.
%既存手法は殆どがパターンマッチングと重み付けの2つに区分することができる.
%パターンマッチングでは事前に単語を登録して,単語がマッチした場合に処理を行うが,処理それぞれに対して単語を登録しなければならず手間が膨大になってしまう.また,単語の意味が考慮されておらず,手作業で登録を行うので登録漏れがあった場合に単語の意味に関係なく処理を行うことが不可能となってしまう.
%重み付けは単語の出現頻度や文章の長さを使用して単語・文章に順位を付ける手法で必ずしも単語の登録が必要でないため多くの研究で使用されている.
%しかし,重み付けもまた単語を文字列のまま扱っており,意味までは考慮されていない.故に ,人間なら対応できる似た単語でも1文字違うだけで対処が困難となる.
議論支援に関する先行研究においてファシリテーターに対する支援を目的としたものは無く,殆どが議論の活性化や可視化を目的としている.
%=================================新手法
近年,自然言語処理の分野において分散表現が多くの研究で使われており.分散表現は文字列である単語を辞書データを使用して実数ベクトルへと変換する.辞書データにない単語には対応できないが,多様な処理を1つの辞書データで行うことができる.また,実数ベクトルの各数値が単語の意味を表現するものとなっており,数値を使用して処理を行うことができる.
分散表現を用いることで既存手法より人間の感覚に近しい処理を行うことができる.
%=================================
以上のような背景を踏まえて,分散表現を用いてファシリテーターの代わりに話題の変化を判定し,知らせることを目指す.
話題転換の検出は発言同士の近さ,すなわち発言に含まれる単語意味の近さと見ることができる.
分散表現ではベクトル同士の内積計算を行うことで単語同士の意味の近さを計算することができる.
また,分散表現を使用することで機械翻訳を始めとする複数の分野で精度の向上が確認されている.
\end{comment}
\section{研究の目的}
\label{intro:taget}
本論文では,分散表現を用いて議論中での発言に含まれる単語の関連度を計算し,話題の変化を観測する手法を提案する.

\section{本論文の構成}
本論文の構成を以下に示す.
\ref{relwork:chapter} 章では要約手法に関する研究と,分散表現に関する先行研究を紹介する.
次に,\ref{model:chapter}章では発言の要約手法の説明を行い,\ref{impl:chapter}章では分散表現を用いた単語集合間の関連度計算について説明する.
そして,\ref{exp:chapter}章では話題転換点の検出の評価実験について説明する.
最後に\ref{con:chapter}章で本論文のまとめと考察を示す.

 %-------------------------------------------------------------------------------
 \expandafter\ifx\csname MasterFile\endcsname\relax
	\def\BibFile{hoge}
	\input{../Bibliography/chapter}
  \fi
  %-------------------------------------------------------------------------------
  \expandafter\ifx\csname MasterFile\endcsname\relax
  \end{document}
  \fi

  \fi
  %-------------------------------------------------------------------------------
  \expandafter\ifx\csname MasterFile\endcsname\relax
  \end{document}
  \fi

%-------------------------------------------------------------------------------
\expandafter\ifx\csname MasterFile\endcsname\relax
	\def\SubFile{hoge}
	\documentclass[a4j,12pt,twoside,openany]{jreport}
%\nofiles %tocファイルを更新させない
%\documentclass[12pt,a4j,twoside,openany]{jsbook}
\usepackage[dvipdfmx]{graphicx}
\usepackage{../dspc} % ベースラインスキップの指定
\usepackage{../slashbox} % 表に斜線を入れる
%\usepackage{../mediabb}
\usepackage{fancyvrb} % Verbatim環境
\usepackage{fancyhdr} % Headerの下線付き章見出し
\usepackage{here} % float[H]
\usepackage{multirow}
\usepackage{hhline} % 表の罫線の角を美しくする
\usepackage{amsmath} %コレがないとcasesが動かない
\usepackage{amsfonts} % 数学用フォント
\usepackage{bm} % 数式環境での bold
\usepackage{algorithm}
\usepackage{algorithmicx}
\usepackage[noend]{algpseudocode}
\usepackage[flushleft]{threeparttable} % 脚注付きテーブル
\usepackage{enumitem}
\usepackage{comment}
\usepackage{fancybox}
%\usepackage{csvsimple,booktabs,siunitx}
%\usepackage{filecontents}


\setlength{\evensidemargin}{5pt}
\setlength{\oddsidemargin}{40pt}
%\setlength{\headheight}{16.5pt}
%%\setlength{\headheight}{30pt}
\setcounter{secnumdepth}{3}
\setlist[description]{leftmargin=2\parindent,labelindent=\parindent}

\makeatletter
\def\@makechapterhead#1{%
	\vspace*{50\p@}%
	{
		\parindent \z@ \raggedright \normalfont
		\ifnum \c@secnumdepth >\m@ne
		% \if@mainmatter
			\huge\bfseries\@chapapp\thechapter\@chappos
			\par\nobreak
			\vskip 20\p@
		% \fi
		\fi
		\interlinepenalty\@M
		\Huge\bfseries #1\par\nobreak
		\vskip 40\p@
	}
}

%新しいコマンド定義
\newcounter{linenumber}
\newenvironment{listing}{%
  \begin{list}{%
    \small\arabic{linenumber}:}{%
      \usecounter{linenumber}%
      \setlength{\baselineskip}{18pt}%
      \setlength{\itemsep}{0pt}%
      \setlength{\parsep}{0pt}}}%
 {\end{list}}
\newcommand{\figcaption}[1]{\def\@captype{figure}\caption{#1}}
\newcommand{\tblcaption}[1]{\def\@captype{table}\caption{#1}}
\newcommand{\norm}[1]{\left\| #1 \right\|}
\newcommand{\cc}[1]{\multicolumn{1}{|c|}{#1}}
\newcommand{\circled}[1]{\raisebox{.5pt}{\textcircled{\raisebox{-.9pt} {#1}}}}
\newcommand{\specialcell}[2][c]{%
  \begin{tabular}[#1]{@{}c@{}}#2\end{tabular}}
\makeatother
%===============================================================================
\expandafter\ifx\csname SubFile\endcsname\relax
\begin{document}
\def\MasterFile{hoge}
%-------------------------------------------------------------------------------
%\maketitle
\thispagestyle{empty}
\input{../hyoushi/title}
%\addcontentsline{toc}{chapter}{表紙}
\thispagestyle{empty}
\mbox{}\newpage
%===============================================================================
%\frontmatter
%===============================================================================
%\mainmatter
%-------------------------------------------------------------------------------
\pagenumbering{arabic}
\cleardoublepage
\input{../0.Abstract/chapter}
%-------------------------------------------------------------------------------
\clearpage
\addcontentsline{toc}{chapter}{目次}
\tableofcontents

\clearpage
\addcontentsline{toc}{chapter}{図目次}
\listoffigures

\clearpage
\addcontentsline{toc}{chapter}{表目次}
\listoftables

%-------------------------------------------------------------------------------

%=====================
\pagestyle{fancy} % Headerをつける
\renewcommand{\sectionmark}[1]{\markright{\thesection\ \ \ #1}}
\renewcommand{\chaptermark}[1]{\markboth{#1}{}}
\lhead{}
\chead{}
\lfoot{}
\rfoot{}%-------------------------------------------------------------------------------
\input{../1.Introduction/chapter}
%-------------------------------------------------------------------------------
\input{../2.Related_Work/chapter}
%-------------------------------------------------------------------------------
\input{../3.The_Model/chapter}
%-------------------------------------------------------------------------------
\input{../4.Implementation/chapter}
%-------------------------------------------------------------------------------
\input{../5.Experiments/chapter}
%-------------------------------------------------------------------------------
\input{../6.Conclusion/chapter}

%===============================================================================
\pagestyle{plain}
%-------------------------------------------------------------------------------
\input{../7.Acknowledgement/chapter} %謝辞
%-------------------------------------------------------------------------------
\def\BibFile{../Bibliograhoy/database2}
\input{../Bibliography/chapter} %参考文献
% %===============================================================================
\appendix
\input{../A.Mypaper/chapter} % 投稿論文リスト
\input{../B.SIG-CCI2/chapter} %
\input{../C.IJCAI-16/chapter} %
%===============================================================================
\end{document}
\fi

	\begin{document}
	\setcounter{chapter}{0}
	\fi
  %-------------------------------------------------------------------------------
\cleardoublepage
\chapter{序論}
\label{intro:chapter}
%本章では, 本研究を行なうに至った背景と目的について述べる.その後,本論文の構成について述べる.
\section{研究の背景}
\label{intro:background}
近年,Web上での大規模な議論活動が活発になっているが,現在一般的に使われている "2ちゃんねる" や "Twitter" といったシステムでは整理や収束を行うことが困難である.困難である原因として,議論の管理を行う者がいないことが挙げられる.
つまり,議論を整理・収束させるには議論のマネジメントを行う人物が必要である.
%
大規模意見集約システムCOLLAGREE\cite{collagreeTest}ではファシリテーターと呼ばれる人物が議論のマネジメントを行っている.
しかし,ファシリテーターは人間であり,長時間に渡って大人数での議論の動向をマネジメントし続けるのは困難である.
COLLAGREEで大規模な議論を収束させるためには,ファシリテーターが必要な時には画面を見るようにして,他の時は見なくても済むようにすることで画面に向き合う時間を減らす工夫があることが望ましい.ファシリテーターが画面を見るべきタイミングは議論の話題が変化したときである.以前の議論の内容から外れた発言がされた時,ファシリテーターが適切な発言をすることで,脱線や炎上を避けて議論を収束させることができる.
すなわち,ファシリテーターの代わりに自動的に議論中の話題の変化を観測することが求められている.
%
現在,COLLAGREE上で使用されている議論支援システムは「(1)投稿支援システム」と「(2)議論可視化システム」の2つに大別できる.
投稿支援システムはポイント機能やファシリテーションフレーズ簡易投稿機能のように,ユーザーが投稿をする際に何らかの補助やリアクションを行う.現行の機能では選択肢の提示に留まっており,作業量を減らすことには繋がりにくい.
一方,議論可視化システムは議論ツリーやキーワード抽出のように,ユーザーにスレッドとは異なる議論の見方を提供する.
\ref{Fig:argTree1}に議論ツリーの例を示す.
\begin{figure}[htbp]
 \begin{center}
  \includegraphics[width=\textwidth]{../images/2.Related_Work/argTree1.png}
  \caption{議論ツリー}
  \label{Fig:argTree1}
  \vspace{-10pt}
 \end{center}
\end{figure}
現行の機能では議論を見やすくすることに重点が置かれており,議論の把握の助けにはなるが画面に向き合う時間を減らすことにはなりにくい.むしろ,作業量を増やすことになり得ることもある.
従って,現行の支援機能ではファシリテーターの作業量の減少には繋がりにくい.
%
近年,自然言語処理の分野において分散表現が多くの研究で使われており,機械翻訳を始めとする単語の意味が重要となる分野で精度の向上が確認されている.分散表現を用いることで,人間に近い精度で話題の変化を観測することが可能となる.
%
以上のような背景を踏まえて,分散表現を用いて,話題の変化を観測し,話題の変化が確認された時にファシリテーターに伝えることが望ましい.
話題の変化の観測は,発言中に現れる単語の類似度の計算と見なすことができる.
分散表現を用いることで単語間の類似度を求めることができる,値が大きいほど単語がそれぞれ類似した実数ベクトルであることを表す.単語Aと単語Bの実数ベクトルが類似しているとは,単語Aと共に使われることの多い単語と単語Bと共に使われることの多い単語が多く共通していることを示す.故に,分散表現を使って単語の類似度を計算することができる.
%
発言文から単語を選ぶ際には自動要約を用いる.発言文から重要でない単語を取り除くことで関連度の計算の精度を高めることが可能となる.
要約の手法としてはokapi BM25 \cite{okapiBM25}とLexRankを組み合わせた抽出的要約手法を用いる.
\begin{comment}
%======================================= 社会的背景
2013年頃からWeb上での大規模な議論活動が活発になり,大規模な人数での議論が期待されている.
大規模な議論では意見を共有することは可能であるが,議論を整理させることや収束させることは難しい.以上から大規模意見集約システムCOLLAGREEが開発された.本システムではWeb上で適切に大規模な議論を行うことができるように議論をマネジメントするファシリテーターを導入した\cite{collagreeTest}.
過去の実験ではファシリテーターの存在が議論の集約に大きな役割を果たしていることが認識されており,大規模な議論のためにファシリテータは必要である.しかし,議論の規模に伴って議論時間が長くなる傾向があり,同時にファシリテーターは常に議論の動向を見続ける必要がある.故に,議論の規模が大きくなればなるほどファシリテーターは長時間かつ大規模な議論の動向の監視によって大きな負担がかかる.大規模な議論が増加する傾向を踏まえるとファシリテーターにかかる負担を軽減する支援が必要である.\\
以上の問題を解決するため,話題の変化を追い,重要な話題の転換点をファシリテーターの代わりに検出することが有用であると考える.必要な時にだけファシリテーターが画面を見れば良いようにすることでファシリテーターの負担軽減が期待できる.
%========================================= 現行手法問題点背景
%議論支援に関する先行研究において,既存の手法は全てが文字列を文字列のまま扱う手法である.
%既存手法は殆どがパターンマッチングと重み付けの2つに区分することができる.
%パターンマッチングでは事前に単語を登録して,単語がマッチした場合に処理を行うが,処理それぞれに対して単語を登録しなければならず手間が膨大になってしまう.また,単語の意味が考慮されておらず,手作業で登録を行うので登録漏れがあった場合に単語の意味に関係なく処理を行うことが不可能となってしまう.
%重み付けは単語の出現頻度や文章の長さを使用して単語・文章に順位を付ける手法で必ずしも単語の登録が必要でないため多くの研究で使用されている.
%しかし,重み付けもまた単語を文字列のまま扱っており,意味までは考慮されていない.故に ,人間なら対応できる似た単語でも1文字違うだけで対処が困難となる.
議論支援に関する先行研究においてファシリテーターに対する支援を目的としたものは無く,殆どが議論の活性化や可視化を目的としている.
%=================================新手法
近年,自然言語処理の分野において分散表現が多くの研究で使われており.分散表現は文字列である単語を辞書データを使用して実数ベクトルへと変換する.辞書データにない単語には対応できないが,多様な処理を1つの辞書データで行うことができる.また,実数ベクトルの各数値が単語の意味を表現するものとなっており,数値を使用して処理を行うことができる.
分散表現を用いることで既存手法より人間の感覚に近しい処理を行うことができる.
%=================================
以上のような背景を踏まえて,分散表現を用いてファシリテーターの代わりに話題の変化を判定し,知らせることを目指す.
話題転換の検出は発言同士の近さ,すなわち発言に含まれる単語意味の近さと見ることができる.
分散表現ではベクトル同士の内積計算を行うことで単語同士の意味の近さを計算することができる.
また,分散表現を使用することで機械翻訳を始めとする複数の分野で精度の向上が確認されている.
\end{comment}
\section{研究の目的}
\label{intro:taget}
本論文では,分散表現を用いて議論中での発言に含まれる単語の関連度を計算し,話題の変化を観測する手法を提案する.

\section{本論文の構成}
本論文の構成を以下に示す.
\ref{relwork:chapter} 章では要約手法に関する研究と,分散表現に関する先行研究を紹介する.
次に,\ref{model:chapter}章では発言の要約手法の説明を行い,\ref{impl:chapter}章では分散表現を用いた単語集合間の関連度計算について説明する.
そして,\ref{exp:chapter}章では話題転換点の検出の評価実験について説明する.
最後に\ref{con:chapter}章で本論文のまとめと考察を示す.

 %-------------------------------------------------------------------------------
 \expandafter\ifx\csname MasterFile\endcsname\relax
	\def\BibFile{hoge}
	\expandafter\ifx\csname MasterFile\endcsname\relax
	\def\SubFile{hoge}
	\input{../thesis/thesis}
	\begin{document}
	\setcounter{chapter}{0}
	\fi
  %-------------------------------------------------------------------------------
\cleardoublepage
\chapter{序論}
\label{intro:chapter}
%本章では, 本研究を行なうに至った背景と目的について述べる.その後,本論文の構成について述べる.
\section{研究の背景}
\label{intro:background}
近年,Web上での大規模な議論活動が活発になっているが,現在一般的に使われている "2ちゃんねる" や "Twitter" といったシステムでは整理や収束を行うことが困難である.困難である原因として,議論の管理を行う者がいないことが挙げられる.
つまり,議論を整理・収束させるには議論のマネジメントを行う人物が必要である.
%
大規模意見集約システムCOLLAGREE\cite{collagreeTest}ではファシリテーターと呼ばれる人物が議論のマネジメントを行っている.
しかし,ファシリテーターは人間であり,長時間に渡って大人数での議論の動向をマネジメントし続けるのは困難である.
COLLAGREEで大規模な議論を収束させるためには,ファシリテーターが必要な時には画面を見るようにして,他の時は見なくても済むようにすることで画面に向き合う時間を減らす工夫があることが望ましい.ファシリテーターが画面を見るべきタイミングは議論の話題が変化したときである.以前の議論の内容から外れた発言がされた時,ファシリテーターが適切な発言をすることで,脱線や炎上を避けて議論を収束させることができる.
すなわち,ファシリテーターの代わりに自動的に議論中の話題の変化を観測することが求められている.
%
現在,COLLAGREE上で使用されている議論支援システムは「(1)投稿支援システム」と「(2)議論可視化システム」の2つに大別できる.
投稿支援システムはポイント機能やファシリテーションフレーズ簡易投稿機能のように,ユーザーが投稿をする際に何らかの補助やリアクションを行う.現行の機能では選択肢の提示に留まっており,作業量を減らすことには繋がりにくい.
一方,議論可視化システムは議論ツリーやキーワード抽出のように,ユーザーにスレッドとは異なる議論の見方を提供する.
\ref{Fig:argTree1}に議論ツリーの例を示す.
\begin{figure}[htbp]
 \begin{center}
  \includegraphics[width=\textwidth]{../images/2.Related_Work/argTree1.png}
  \caption{議論ツリー}
  \label{Fig:argTree1}
  \vspace{-10pt}
 \end{center}
\end{figure}
現行の機能では議論を見やすくすることに重点が置かれており,議論の把握の助けにはなるが画面に向き合う時間を減らすことにはなりにくい.むしろ,作業量を増やすことになり得ることもある.
従って,現行の支援機能ではファシリテーターの作業量の減少には繋がりにくい.
%
近年,自然言語処理の分野において分散表現が多くの研究で使われており,機械翻訳を始めとする単語の意味が重要となる分野で精度の向上が確認されている.分散表現を用いることで,人間に近い精度で話題の変化を観測することが可能となる.
%
以上のような背景を踏まえて,分散表現を用いて,話題の変化を観測し,話題の変化が確認された時にファシリテーターに伝えることが望ましい.
話題の変化の観測は,発言中に現れる単語の類似度の計算と見なすことができる.
分散表現を用いることで単語間の類似度を求めることができる,値が大きいほど単語がそれぞれ類似した実数ベクトルであることを表す.単語Aと単語Bの実数ベクトルが類似しているとは,単語Aと共に使われることの多い単語と単語Bと共に使われることの多い単語が多く共通していることを示す.故に,分散表現を使って単語の類似度を計算することができる.
%
発言文から単語を選ぶ際には自動要約を用いる.発言文から重要でない単語を取り除くことで関連度の計算の精度を高めることが可能となる.
要約の手法としてはokapi BM25 \cite{okapiBM25}とLexRankを組み合わせた抽出的要約手法を用いる.
\begin{comment}
%======================================= 社会的背景
2013年頃からWeb上での大規模な議論活動が活発になり,大規模な人数での議論が期待されている.
大規模な議論では意見を共有することは可能であるが,議論を整理させることや収束させることは難しい.以上から大規模意見集約システムCOLLAGREEが開発された.本システムではWeb上で適切に大規模な議論を行うことができるように議論をマネジメントするファシリテーターを導入した\cite{collagreeTest}.
過去の実験ではファシリテーターの存在が議論の集約に大きな役割を果たしていることが認識されており,大規模な議論のためにファシリテータは必要である.しかし,議論の規模に伴って議論時間が長くなる傾向があり,同時にファシリテーターは常に議論の動向を見続ける必要がある.故に,議論の規模が大きくなればなるほどファシリテーターは長時間かつ大規模な議論の動向の監視によって大きな負担がかかる.大規模な議論が増加する傾向を踏まえるとファシリテーターにかかる負担を軽減する支援が必要である.\\
以上の問題を解決するため,話題の変化を追い,重要な話題の転換点をファシリテーターの代わりに検出することが有用であると考える.必要な時にだけファシリテーターが画面を見れば良いようにすることでファシリテーターの負担軽減が期待できる.
%========================================= 現行手法問題点背景
%議論支援に関する先行研究において,既存の手法は全てが文字列を文字列のまま扱う手法である.
%既存手法は殆どがパターンマッチングと重み付けの2つに区分することができる.
%パターンマッチングでは事前に単語を登録して,単語がマッチした場合に処理を行うが,処理それぞれに対して単語を登録しなければならず手間が膨大になってしまう.また,単語の意味が考慮されておらず,手作業で登録を行うので登録漏れがあった場合に単語の意味に関係なく処理を行うことが不可能となってしまう.
%重み付けは単語の出現頻度や文章の長さを使用して単語・文章に順位を付ける手法で必ずしも単語の登録が必要でないため多くの研究で使用されている.
%しかし,重み付けもまた単語を文字列のまま扱っており,意味までは考慮されていない.故に ,人間なら対応できる似た単語でも1文字違うだけで対処が困難となる.
議論支援に関する先行研究においてファシリテーターに対する支援を目的としたものは無く,殆どが議論の活性化や可視化を目的としている.
%=================================新手法
近年,自然言語処理の分野において分散表現が多くの研究で使われており.分散表現は文字列である単語を辞書データを使用して実数ベクトルへと変換する.辞書データにない単語には対応できないが,多様な処理を1つの辞書データで行うことができる.また,実数ベクトルの各数値が単語の意味を表現するものとなっており,数値を使用して処理を行うことができる.
分散表現を用いることで既存手法より人間の感覚に近しい処理を行うことができる.
%=================================
以上のような背景を踏まえて,分散表現を用いてファシリテーターの代わりに話題の変化を判定し,知らせることを目指す.
話題転換の検出は発言同士の近さ,すなわち発言に含まれる単語意味の近さと見ることができる.
分散表現ではベクトル同士の内積計算を行うことで単語同士の意味の近さを計算することができる.
また,分散表現を使用することで機械翻訳を始めとする複数の分野で精度の向上が確認されている.
\end{comment}
\section{研究の目的}
\label{intro:taget}
本論文では,分散表現を用いて議論中での発言に含まれる単語の関連度を計算し,話題の変化を観測する手法を提案する.

\section{本論文の構成}
本論文の構成を以下に示す.
\ref{relwork:chapter} 章では要約手法に関する研究と,分散表現に関する先行研究を紹介する.
次に,\ref{model:chapter}章では発言の要約手法の説明を行い,\ref{impl:chapter}章では分散表現を用いた単語集合間の関連度計算について説明する.
そして,\ref{exp:chapter}章では話題転換点の検出の評価実験について説明する.
最後に\ref{con:chapter}章で本論文のまとめと考察を示す.

 %-------------------------------------------------------------------------------
 \expandafter\ifx\csname MasterFile\endcsname\relax
	\def\BibFile{hoge}
	\input{../Bibliography/chapter}
  \fi
  %-------------------------------------------------------------------------------
  \expandafter\ifx\csname MasterFile\endcsname\relax
  \end{document}
  \fi

  \fi
  %-------------------------------------------------------------------------------
  \expandafter\ifx\csname MasterFile\endcsname\relax
  \end{document}
  \fi

%-------------------------------------------------------------------------------
\expandafter\ifx\csname MasterFile\endcsname\relax
	\def\SubFile{hoge}
	\documentclass[a4j,12pt,twoside,openany]{jreport}
%\nofiles %tocファイルを更新させない
%\documentclass[12pt,a4j,twoside,openany]{jsbook}
\usepackage[dvipdfmx]{graphicx}
\usepackage{../dspc} % ベースラインスキップの指定
\usepackage{../slashbox} % 表に斜線を入れる
%\usepackage{../mediabb}
\usepackage{fancyvrb} % Verbatim環境
\usepackage{fancyhdr} % Headerの下線付き章見出し
\usepackage{here} % float[H]
\usepackage{multirow}
\usepackage{hhline} % 表の罫線の角を美しくする
\usepackage{amsmath} %コレがないとcasesが動かない
\usepackage{amsfonts} % 数学用フォント
\usepackage{bm} % 数式環境での bold
\usepackage{algorithm}
\usepackage{algorithmicx}
\usepackage[noend]{algpseudocode}
\usepackage[flushleft]{threeparttable} % 脚注付きテーブル
\usepackage{enumitem}
\usepackage{comment}
\usepackage{fancybox}
%\usepackage{csvsimple,booktabs,siunitx}
%\usepackage{filecontents}


\setlength{\evensidemargin}{5pt}
\setlength{\oddsidemargin}{40pt}
%\setlength{\headheight}{16.5pt}
%%\setlength{\headheight}{30pt}
\setcounter{secnumdepth}{3}
\setlist[description]{leftmargin=2\parindent,labelindent=\parindent}

\makeatletter
\def\@makechapterhead#1{%
	\vspace*{50\p@}%
	{
		\parindent \z@ \raggedright \normalfont
		\ifnum \c@secnumdepth >\m@ne
		% \if@mainmatter
			\huge\bfseries\@chapapp\thechapter\@chappos
			\par\nobreak
			\vskip 20\p@
		% \fi
		\fi
		\interlinepenalty\@M
		\Huge\bfseries #1\par\nobreak
		\vskip 40\p@
	}
}

%新しいコマンド定義
\newcounter{linenumber}
\newenvironment{listing}{%
  \begin{list}{%
    \small\arabic{linenumber}:}{%
      \usecounter{linenumber}%
      \setlength{\baselineskip}{18pt}%
      \setlength{\itemsep}{0pt}%
      \setlength{\parsep}{0pt}}}%
 {\end{list}}
\newcommand{\figcaption}[1]{\def\@captype{figure}\caption{#1}}
\newcommand{\tblcaption}[1]{\def\@captype{table}\caption{#1}}
\newcommand{\norm}[1]{\left\| #1 \right\|}
\newcommand{\cc}[1]{\multicolumn{1}{|c|}{#1}}
\newcommand{\circled}[1]{\raisebox{.5pt}{\textcircled{\raisebox{-.9pt} {#1}}}}
\newcommand{\specialcell}[2][c]{%
  \begin{tabular}[#1]{@{}c@{}}#2\end{tabular}}
\makeatother
%===============================================================================
\expandafter\ifx\csname SubFile\endcsname\relax
\begin{document}
\def\MasterFile{hoge}
%-------------------------------------------------------------------------------
%\maketitle
\thispagestyle{empty}
\input{../hyoushi/title}
%\addcontentsline{toc}{chapter}{表紙}
\thispagestyle{empty}
\mbox{}\newpage
%===============================================================================
%\frontmatter
%===============================================================================
%\mainmatter
%-------------------------------------------------------------------------------
\pagenumbering{arabic}
\cleardoublepage
\input{../0.Abstract/chapter}
%-------------------------------------------------------------------------------
\clearpage
\addcontentsline{toc}{chapter}{目次}
\tableofcontents

\clearpage
\addcontentsline{toc}{chapter}{図目次}
\listoffigures

\clearpage
\addcontentsline{toc}{chapter}{表目次}
\listoftables

%-------------------------------------------------------------------------------

%=====================
\pagestyle{fancy} % Headerをつける
\renewcommand{\sectionmark}[1]{\markright{\thesection\ \ \ #1}}
\renewcommand{\chaptermark}[1]{\markboth{#1}{}}
\lhead{}
\chead{}
\lfoot{}
\rfoot{}%-------------------------------------------------------------------------------
\input{../1.Introduction/chapter}
%-------------------------------------------------------------------------------
\input{../2.Related_Work/chapter}
%-------------------------------------------------------------------------------
\input{../3.The_Model/chapter}
%-------------------------------------------------------------------------------
\input{../4.Implementation/chapter}
%-------------------------------------------------------------------------------
\input{../5.Experiments/chapter}
%-------------------------------------------------------------------------------
\input{../6.Conclusion/chapter}

%===============================================================================
\pagestyle{plain}
%-------------------------------------------------------------------------------
\input{../7.Acknowledgement/chapter} %謝辞
%-------------------------------------------------------------------------------
\def\BibFile{../Bibliograhoy/database2}
\input{../Bibliography/chapter} %参考文献
% %===============================================================================
\appendix
\input{../A.Mypaper/chapter} % 投稿論文リスト
\input{../B.SIG-CCI2/chapter} %
\input{../C.IJCAI-16/chapter} %
%===============================================================================
\end{document}
\fi

	\begin{document}
	\setcounter{chapter}{0}
	\fi
  %-------------------------------------------------------------------------------
\cleardoublepage
\chapter{序論}
\label{intro:chapter}
%本章では, 本研究を行なうに至った背景と目的について述べる.その後,本論文の構成について述べる.
\section{研究の背景}
\label{intro:background}
近年,Web上での大規模な議論活動が活発になっているが,現在一般的に使われている "2ちゃんねる" や "Twitter" といったシステムでは整理や収束を行うことが困難である.困難である原因として,議論の管理を行う者がいないことが挙げられる.
つまり,議論を整理・収束させるには議論のマネジメントを行う人物が必要である.
%
大規模意見集約システムCOLLAGREE\cite{collagreeTest}ではファシリテーターと呼ばれる人物が議論のマネジメントを行っている.
しかし,ファシリテーターは人間であり,長時間に渡って大人数での議論の動向をマネジメントし続けるのは困難である.
COLLAGREEで大規模な議論を収束させるためには,ファシリテーターが必要な時には画面を見るようにして,他の時は見なくても済むようにすることで画面に向き合う時間を減らす工夫があることが望ましい.ファシリテーターが画面を見るべきタイミングは議論の話題が変化したときである.以前の議論の内容から外れた発言がされた時,ファシリテーターが適切な発言をすることで,脱線や炎上を避けて議論を収束させることができる.
すなわち,ファシリテーターの代わりに自動的に議論中の話題の変化を観測することが求められている.
%
現在,COLLAGREE上で使用されている議論支援システムは「(1)投稿支援システム」と「(2)議論可視化システム」の2つに大別できる.
投稿支援システムはポイント機能やファシリテーションフレーズ簡易投稿機能のように,ユーザーが投稿をする際に何らかの補助やリアクションを行う.現行の機能では選択肢の提示に留まっており,作業量を減らすことには繋がりにくい.
一方,議論可視化システムは議論ツリーやキーワード抽出のように,ユーザーにスレッドとは異なる議論の見方を提供する.
\ref{Fig:argTree1}に議論ツリーの例を示す.
\begin{figure}[htbp]
 \begin{center}
  \includegraphics[width=\textwidth]{../images/2.Related_Work/argTree1.png}
  \caption{議論ツリー}
  \label{Fig:argTree1}
  \vspace{-10pt}
 \end{center}
\end{figure}
現行の機能では議論を見やすくすることに重点が置かれており,議論の把握の助けにはなるが画面に向き合う時間を減らすことにはなりにくい.むしろ,作業量を増やすことになり得ることもある.
従って,現行の支援機能ではファシリテーターの作業量の減少には繋がりにくい.
%
近年,自然言語処理の分野において分散表現が多くの研究で使われており,機械翻訳を始めとする単語の意味が重要となる分野で精度の向上が確認されている.分散表現を用いることで,人間に近い精度で話題の変化を観測することが可能となる.
%
以上のような背景を踏まえて,分散表現を用いて,話題の変化を観測し,話題の変化が確認された時にファシリテーターに伝えることが望ましい.
話題の変化の観測は,発言中に現れる単語の類似度の計算と見なすことができる.
分散表現を用いることで単語間の類似度を求めることができる,値が大きいほど単語がそれぞれ類似した実数ベクトルであることを表す.単語Aと単語Bの実数ベクトルが類似しているとは,単語Aと共に使われることの多い単語と単語Bと共に使われることの多い単語が多く共通していることを示す.故に,分散表現を使って単語の類似度を計算することができる.
%
発言文から単語を選ぶ際には自動要約を用いる.発言文から重要でない単語を取り除くことで関連度の計算の精度を高めることが可能となる.
要約の手法としてはokapi BM25 \cite{okapiBM25}とLexRankを組み合わせた抽出的要約手法を用いる.
\begin{comment}
%======================================= 社会的背景
2013年頃からWeb上での大規模な議論活動が活発になり,大規模な人数での議論が期待されている.
大規模な議論では意見を共有することは可能であるが,議論を整理させることや収束させることは難しい.以上から大規模意見集約システムCOLLAGREEが開発された.本システムではWeb上で適切に大規模な議論を行うことができるように議論をマネジメントするファシリテーターを導入した\cite{collagreeTest}.
過去の実験ではファシリテーターの存在が議論の集約に大きな役割を果たしていることが認識されており,大規模な議論のためにファシリテータは必要である.しかし,議論の規模に伴って議論時間が長くなる傾向があり,同時にファシリテーターは常に議論の動向を見続ける必要がある.故に,議論の規模が大きくなればなるほどファシリテーターは長時間かつ大規模な議論の動向の監視によって大きな負担がかかる.大規模な議論が増加する傾向を踏まえるとファシリテーターにかかる負担を軽減する支援が必要である.\\
以上の問題を解決するため,話題の変化を追い,重要な話題の転換点をファシリテーターの代わりに検出することが有用であると考える.必要な時にだけファシリテーターが画面を見れば良いようにすることでファシリテーターの負担軽減が期待できる.
%========================================= 現行手法問題点背景
%議論支援に関する先行研究において,既存の手法は全てが文字列を文字列のまま扱う手法である.
%既存手法は殆どがパターンマッチングと重み付けの2つに区分することができる.
%パターンマッチングでは事前に単語を登録して,単語がマッチした場合に処理を行うが,処理それぞれに対して単語を登録しなければならず手間が膨大になってしまう.また,単語の意味が考慮されておらず,手作業で登録を行うので登録漏れがあった場合に単語の意味に関係なく処理を行うことが不可能となってしまう.
%重み付けは単語の出現頻度や文章の長さを使用して単語・文章に順位を付ける手法で必ずしも単語の登録が必要でないため多くの研究で使用されている.
%しかし,重み付けもまた単語を文字列のまま扱っており,意味までは考慮されていない.故に ,人間なら対応できる似た単語でも1文字違うだけで対処が困難となる.
議論支援に関する先行研究においてファシリテーターに対する支援を目的としたものは無く,殆どが議論の活性化や可視化を目的としている.
%=================================新手法
近年,自然言語処理の分野において分散表現が多くの研究で使われており.分散表現は文字列である単語を辞書データを使用して実数ベクトルへと変換する.辞書データにない単語には対応できないが,多様な処理を1つの辞書データで行うことができる.また,実数ベクトルの各数値が単語の意味を表現するものとなっており,数値を使用して処理を行うことができる.
分散表現を用いることで既存手法より人間の感覚に近しい処理を行うことができる.
%=================================
以上のような背景を踏まえて,分散表現を用いてファシリテーターの代わりに話題の変化を判定し,知らせることを目指す.
話題転換の検出は発言同士の近さ,すなわち発言に含まれる単語意味の近さと見ることができる.
分散表現ではベクトル同士の内積計算を行うことで単語同士の意味の近さを計算することができる.
また,分散表現を使用することで機械翻訳を始めとする複数の分野で精度の向上が確認されている.
\end{comment}
\section{研究の目的}
\label{intro:taget}
本論文では,分散表現を用いて議論中での発言に含まれる単語の関連度を計算し,話題の変化を観測する手法を提案する.

\section{本論文の構成}
本論文の構成を以下に示す.
\ref{relwork:chapter} 章では要約手法に関する研究と,分散表現に関する先行研究を紹介する.
次に,\ref{model:chapter}章では発言の要約手法の説明を行い,\ref{impl:chapter}章では分散表現を用いた単語集合間の関連度計算について説明する.
そして,\ref{exp:chapter}章では話題転換点の検出の評価実験について説明する.
最後に\ref{con:chapter}章で本論文のまとめと考察を示す.

 %-------------------------------------------------------------------------------
 \expandafter\ifx\csname MasterFile\endcsname\relax
	\def\BibFile{hoge}
	\expandafter\ifx\csname MasterFile\endcsname\relax
	\def\SubFile{hoge}
	\input{../thesis/thesis}
	\begin{document}
	\setcounter{chapter}{0}
	\fi
  %-------------------------------------------------------------------------------
\cleardoublepage
\chapter{序論}
\label{intro:chapter}
%本章では, 本研究を行なうに至った背景と目的について述べる.その後,本論文の構成について述べる.
\section{研究の背景}
\label{intro:background}
近年,Web上での大規模な議論活動が活発になっているが,現在一般的に使われている "2ちゃんねる" や "Twitter" といったシステムでは整理や収束を行うことが困難である.困難である原因として,議論の管理を行う者がいないことが挙げられる.
つまり,議論を整理・収束させるには議論のマネジメントを行う人物が必要である.
%
大規模意見集約システムCOLLAGREE\cite{collagreeTest}ではファシリテーターと呼ばれる人物が議論のマネジメントを行っている.
しかし,ファシリテーターは人間であり,長時間に渡って大人数での議論の動向をマネジメントし続けるのは困難である.
COLLAGREEで大規模な議論を収束させるためには,ファシリテーターが必要な時には画面を見るようにして,他の時は見なくても済むようにすることで画面に向き合う時間を減らす工夫があることが望ましい.ファシリテーターが画面を見るべきタイミングは議論の話題が変化したときである.以前の議論の内容から外れた発言がされた時,ファシリテーターが適切な発言をすることで,脱線や炎上を避けて議論を収束させることができる.
すなわち,ファシリテーターの代わりに自動的に議論中の話題の変化を観測することが求められている.
%
現在,COLLAGREE上で使用されている議論支援システムは「(1)投稿支援システム」と「(2)議論可視化システム」の2つに大別できる.
投稿支援システムはポイント機能やファシリテーションフレーズ簡易投稿機能のように,ユーザーが投稿をする際に何らかの補助やリアクションを行う.現行の機能では選択肢の提示に留まっており,作業量を減らすことには繋がりにくい.
一方,議論可視化システムは議論ツリーやキーワード抽出のように,ユーザーにスレッドとは異なる議論の見方を提供する.
\ref{Fig:argTree1}に議論ツリーの例を示す.
\begin{figure}[htbp]
 \begin{center}
  \includegraphics[width=\textwidth]{../images/2.Related_Work/argTree1.png}
  \caption{議論ツリー}
  \label{Fig:argTree1}
  \vspace{-10pt}
 \end{center}
\end{figure}
現行の機能では議論を見やすくすることに重点が置かれており,議論の把握の助けにはなるが画面に向き合う時間を減らすことにはなりにくい.むしろ,作業量を増やすことになり得ることもある.
従って,現行の支援機能ではファシリテーターの作業量の減少には繋がりにくい.
%
近年,自然言語処理の分野において分散表現が多くの研究で使われており,機械翻訳を始めとする単語の意味が重要となる分野で精度の向上が確認されている.分散表現を用いることで,人間に近い精度で話題の変化を観測することが可能となる.
%
以上のような背景を踏まえて,分散表現を用いて,話題の変化を観測し,話題の変化が確認された時にファシリテーターに伝えることが望ましい.
話題の変化の観測は,発言中に現れる単語の類似度の計算と見なすことができる.
分散表現を用いることで単語間の類似度を求めることができる,値が大きいほど単語がそれぞれ類似した実数ベクトルであることを表す.単語Aと単語Bの実数ベクトルが類似しているとは,単語Aと共に使われることの多い単語と単語Bと共に使われることの多い単語が多く共通していることを示す.故に,分散表現を使って単語の類似度を計算することができる.
%
発言文から単語を選ぶ際には自動要約を用いる.発言文から重要でない単語を取り除くことで関連度の計算の精度を高めることが可能となる.
要約の手法としてはokapi BM25 \cite{okapiBM25}とLexRankを組み合わせた抽出的要約手法を用いる.
\begin{comment}
%======================================= 社会的背景
2013年頃からWeb上での大規模な議論活動が活発になり,大規模な人数での議論が期待されている.
大規模な議論では意見を共有することは可能であるが,議論を整理させることや収束させることは難しい.以上から大規模意見集約システムCOLLAGREEが開発された.本システムではWeb上で適切に大規模な議論を行うことができるように議論をマネジメントするファシリテーターを導入した\cite{collagreeTest}.
過去の実験ではファシリテーターの存在が議論の集約に大きな役割を果たしていることが認識されており,大規模な議論のためにファシリテータは必要である.しかし,議論の規模に伴って議論時間が長くなる傾向があり,同時にファシリテーターは常に議論の動向を見続ける必要がある.故に,議論の規模が大きくなればなるほどファシリテーターは長時間かつ大規模な議論の動向の監視によって大きな負担がかかる.大規模な議論が増加する傾向を踏まえるとファシリテーターにかかる負担を軽減する支援が必要である.\\
以上の問題を解決するため,話題の変化を追い,重要な話題の転換点をファシリテーターの代わりに検出することが有用であると考える.必要な時にだけファシリテーターが画面を見れば良いようにすることでファシリテーターの負担軽減が期待できる.
%========================================= 現行手法問題点背景
%議論支援に関する先行研究において,既存の手法は全てが文字列を文字列のまま扱う手法である.
%既存手法は殆どがパターンマッチングと重み付けの2つに区分することができる.
%パターンマッチングでは事前に単語を登録して,単語がマッチした場合に処理を行うが,処理それぞれに対して単語を登録しなければならず手間が膨大になってしまう.また,単語の意味が考慮されておらず,手作業で登録を行うので登録漏れがあった場合に単語の意味に関係なく処理を行うことが不可能となってしまう.
%重み付けは単語の出現頻度や文章の長さを使用して単語・文章に順位を付ける手法で必ずしも単語の登録が必要でないため多くの研究で使用されている.
%しかし,重み付けもまた単語を文字列のまま扱っており,意味までは考慮されていない.故に ,人間なら対応できる似た単語でも1文字違うだけで対処が困難となる.
議論支援に関する先行研究においてファシリテーターに対する支援を目的としたものは無く,殆どが議論の活性化や可視化を目的としている.
%=================================新手法
近年,自然言語処理の分野において分散表現が多くの研究で使われており.分散表現は文字列である単語を辞書データを使用して実数ベクトルへと変換する.辞書データにない単語には対応できないが,多様な処理を1つの辞書データで行うことができる.また,実数ベクトルの各数値が単語の意味を表現するものとなっており,数値を使用して処理を行うことができる.
分散表現を用いることで既存手法より人間の感覚に近しい処理を行うことができる.
%=================================
以上のような背景を踏まえて,分散表現を用いてファシリテーターの代わりに話題の変化を判定し,知らせることを目指す.
話題転換の検出は発言同士の近さ,すなわち発言に含まれる単語意味の近さと見ることができる.
分散表現ではベクトル同士の内積計算を行うことで単語同士の意味の近さを計算することができる.
また,分散表現を使用することで機械翻訳を始めとする複数の分野で精度の向上が確認されている.
\end{comment}
\section{研究の目的}
\label{intro:taget}
本論文では,分散表現を用いて議論中での発言に含まれる単語の関連度を計算し,話題の変化を観測する手法を提案する.

\section{本論文の構成}
本論文の構成を以下に示す.
\ref{relwork:chapter} 章では要約手法に関する研究と,分散表現に関する先行研究を紹介する.
次に,\ref{model:chapter}章では発言の要約手法の説明を行い,\ref{impl:chapter}章では分散表現を用いた単語集合間の関連度計算について説明する.
そして,\ref{exp:chapter}章では話題転換点の検出の評価実験について説明する.
最後に\ref{con:chapter}章で本論文のまとめと考察を示す.

 %-------------------------------------------------------------------------------
 \expandafter\ifx\csname MasterFile\endcsname\relax
	\def\BibFile{hoge}
	\input{../Bibliography/chapter}
  \fi
  %-------------------------------------------------------------------------------
  \expandafter\ifx\csname MasterFile\endcsname\relax
  \end{document}
  \fi

  \fi
  %-------------------------------------------------------------------------------
  \expandafter\ifx\csname MasterFile\endcsname\relax
  \end{document}
  \fi

%-------------------------------------------------------------------------------
\expandafter\ifx\csname MasterFile\endcsname\relax
	\def\SubFile{hoge}
	\documentclass[a4j,12pt,twoside,openany]{jreport}
%\nofiles %tocファイルを更新させない
%\documentclass[12pt,a4j,twoside,openany]{jsbook}
\usepackage[dvipdfmx]{graphicx}
\usepackage{../dspc} % ベースラインスキップの指定
\usepackage{../slashbox} % 表に斜線を入れる
%\usepackage{../mediabb}
\usepackage{fancyvrb} % Verbatim環境
\usepackage{fancyhdr} % Headerの下線付き章見出し
\usepackage{here} % float[H]
\usepackage{multirow}
\usepackage{hhline} % 表の罫線の角を美しくする
\usepackage{amsmath} %コレがないとcasesが動かない
\usepackage{amsfonts} % 数学用フォント
\usepackage{bm} % 数式環境での bold
\usepackage{algorithm}
\usepackage{algorithmicx}
\usepackage[noend]{algpseudocode}
\usepackage[flushleft]{threeparttable} % 脚注付きテーブル
\usepackage{enumitem}
\usepackage{comment}
\usepackage{fancybox}
%\usepackage{csvsimple,booktabs,siunitx}
%\usepackage{filecontents}


\setlength{\evensidemargin}{5pt}
\setlength{\oddsidemargin}{40pt}
%\setlength{\headheight}{16.5pt}
%%\setlength{\headheight}{30pt}
\setcounter{secnumdepth}{3}
\setlist[description]{leftmargin=2\parindent,labelindent=\parindent}

\makeatletter
\def\@makechapterhead#1{%
	\vspace*{50\p@}%
	{
		\parindent \z@ \raggedright \normalfont
		\ifnum \c@secnumdepth >\m@ne
		% \if@mainmatter
			\huge\bfseries\@chapapp\thechapter\@chappos
			\par\nobreak
			\vskip 20\p@
		% \fi
		\fi
		\interlinepenalty\@M
		\Huge\bfseries #1\par\nobreak
		\vskip 40\p@
	}
}

%新しいコマンド定義
\newcounter{linenumber}
\newenvironment{listing}{%
  \begin{list}{%
    \small\arabic{linenumber}:}{%
      \usecounter{linenumber}%
      \setlength{\baselineskip}{18pt}%
      \setlength{\itemsep}{0pt}%
      \setlength{\parsep}{0pt}}}%
 {\end{list}}
\newcommand{\figcaption}[1]{\def\@captype{figure}\caption{#1}}
\newcommand{\tblcaption}[1]{\def\@captype{table}\caption{#1}}
\newcommand{\norm}[1]{\left\| #1 \right\|}
\newcommand{\cc}[1]{\multicolumn{1}{|c|}{#1}}
\newcommand{\circled}[1]{\raisebox{.5pt}{\textcircled{\raisebox{-.9pt} {#1}}}}
\newcommand{\specialcell}[2][c]{%
  \begin{tabular}[#1]{@{}c@{}}#2\end{tabular}}
\makeatother
%===============================================================================
\expandafter\ifx\csname SubFile\endcsname\relax
\begin{document}
\def\MasterFile{hoge}
%-------------------------------------------------------------------------------
%\maketitle
\thispagestyle{empty}
\input{../hyoushi/title}
%\addcontentsline{toc}{chapter}{表紙}
\thispagestyle{empty}
\mbox{}\newpage
%===============================================================================
%\frontmatter
%===============================================================================
%\mainmatter
%-------------------------------------------------------------------------------
\pagenumbering{arabic}
\cleardoublepage
\input{../0.Abstract/chapter}
%-------------------------------------------------------------------------------
\clearpage
\addcontentsline{toc}{chapter}{目次}
\tableofcontents

\clearpage
\addcontentsline{toc}{chapter}{図目次}
\listoffigures

\clearpage
\addcontentsline{toc}{chapter}{表目次}
\listoftables

%-------------------------------------------------------------------------------

%=====================
\pagestyle{fancy} % Headerをつける
\renewcommand{\sectionmark}[1]{\markright{\thesection\ \ \ #1}}
\renewcommand{\chaptermark}[1]{\markboth{#1}{}}
\lhead{}
\chead{}
\lfoot{}
\rfoot{}%-------------------------------------------------------------------------------
\input{../1.Introduction/chapter}
%-------------------------------------------------------------------------------
\input{../2.Related_Work/chapter}
%-------------------------------------------------------------------------------
\input{../3.The_Model/chapter}
%-------------------------------------------------------------------------------
\input{../4.Implementation/chapter}
%-------------------------------------------------------------------------------
\input{../5.Experiments/chapter}
%-------------------------------------------------------------------------------
\input{../6.Conclusion/chapter}

%===============================================================================
\pagestyle{plain}
%-------------------------------------------------------------------------------
\input{../7.Acknowledgement/chapter} %謝辞
%-------------------------------------------------------------------------------
\def\BibFile{../Bibliograhoy/database2}
\input{../Bibliography/chapter} %参考文献
% %===============================================================================
\appendix
\input{../A.Mypaper/chapter} % 投稿論文リスト
\input{../B.SIG-CCI2/chapter} %
\input{../C.IJCAI-16/chapter} %
%===============================================================================
\end{document}
\fi

	\begin{document}
	\setcounter{chapter}{0}
	\fi
  %-------------------------------------------------------------------------------
\cleardoublepage
\chapter{序論}
\label{intro:chapter}
%本章では, 本研究を行なうに至った背景と目的について述べる.その後,本論文の構成について述べる.
\section{研究の背景}
\label{intro:background}
近年,Web上での大規模な議論活動が活発になっているが,現在一般的に使われている "2ちゃんねる" や "Twitter" といったシステムでは整理や収束を行うことが困難である.困難である原因として,議論の管理を行う者がいないことが挙げられる.
つまり,議論を整理・収束させるには議論のマネジメントを行う人物が必要である.
%
大規模意見集約システムCOLLAGREE\cite{collagreeTest}ではファシリテーターと呼ばれる人物が議論のマネジメントを行っている.
しかし,ファシリテーターは人間であり,長時間に渡って大人数での議論の動向をマネジメントし続けるのは困難である.
COLLAGREEで大規模な議論を収束させるためには,ファシリテーターが必要な時には画面を見るようにして,他の時は見なくても済むようにすることで画面に向き合う時間を減らす工夫があることが望ましい.ファシリテーターが画面を見るべきタイミングは議論の話題が変化したときである.以前の議論の内容から外れた発言がされた時,ファシリテーターが適切な発言をすることで,脱線や炎上を避けて議論を収束させることができる.
すなわち,ファシリテーターの代わりに自動的に議論中の話題の変化を観測することが求められている.
%
現在,COLLAGREE上で使用されている議論支援システムは「(1)投稿支援システム」と「(2)議論可視化システム」の2つに大別できる.
投稿支援システムはポイント機能やファシリテーションフレーズ簡易投稿機能のように,ユーザーが投稿をする際に何らかの補助やリアクションを行う.現行の機能では選択肢の提示に留まっており,作業量を減らすことには繋がりにくい.
一方,議論可視化システムは議論ツリーやキーワード抽出のように,ユーザーにスレッドとは異なる議論の見方を提供する.
\ref{Fig:argTree1}に議論ツリーの例を示す.
\begin{figure}[htbp]
 \begin{center}
  \includegraphics[width=\textwidth]{../images/2.Related_Work/argTree1.png}
  \caption{議論ツリー}
  \label{Fig:argTree1}
  \vspace{-10pt}
 \end{center}
\end{figure}
現行の機能では議論を見やすくすることに重点が置かれており,議論の把握の助けにはなるが画面に向き合う時間を減らすことにはなりにくい.むしろ,作業量を増やすことになり得ることもある.
従って,現行の支援機能ではファシリテーターの作業量の減少には繋がりにくい.
%
近年,自然言語処理の分野において分散表現が多くの研究で使われており,機械翻訳を始めとする単語の意味が重要となる分野で精度の向上が確認されている.分散表現を用いることで,人間に近い精度で話題の変化を観測することが可能となる.
%
以上のような背景を踏まえて,分散表現を用いて,話題の変化を観測し,話題の変化が確認された時にファシリテーターに伝えることが望ましい.
話題の変化の観測は,発言中に現れる単語の類似度の計算と見なすことができる.
分散表現を用いることで単語間の類似度を求めることができる,値が大きいほど単語がそれぞれ類似した実数ベクトルであることを表す.単語Aと単語Bの実数ベクトルが類似しているとは,単語Aと共に使われることの多い単語と単語Bと共に使われることの多い単語が多く共通していることを示す.故に,分散表現を使って単語の類似度を計算することができる.
%
発言文から単語を選ぶ際には自動要約を用いる.発言文から重要でない単語を取り除くことで関連度の計算の精度を高めることが可能となる.
要約の手法としてはokapi BM25 \cite{okapiBM25}とLexRankを組み合わせた抽出的要約手法を用いる.
\begin{comment}
%======================================= 社会的背景
2013年頃からWeb上での大規模な議論活動が活発になり,大規模な人数での議論が期待されている.
大規模な議論では意見を共有することは可能であるが,議論を整理させることや収束させることは難しい.以上から大規模意見集約システムCOLLAGREEが開発された.本システムではWeb上で適切に大規模な議論を行うことができるように議論をマネジメントするファシリテーターを導入した\cite{collagreeTest}.
過去の実験ではファシリテーターの存在が議論の集約に大きな役割を果たしていることが認識されており,大規模な議論のためにファシリテータは必要である.しかし,議論の規模に伴って議論時間が長くなる傾向があり,同時にファシリテーターは常に議論の動向を見続ける必要がある.故に,議論の規模が大きくなればなるほどファシリテーターは長時間かつ大規模な議論の動向の監視によって大きな負担がかかる.大規模な議論が増加する傾向を踏まえるとファシリテーターにかかる負担を軽減する支援が必要である.\\
以上の問題を解決するため,話題の変化を追い,重要な話題の転換点をファシリテーターの代わりに検出することが有用であると考える.必要な時にだけファシリテーターが画面を見れば良いようにすることでファシリテーターの負担軽減が期待できる.
%========================================= 現行手法問題点背景
%議論支援に関する先行研究において,既存の手法は全てが文字列を文字列のまま扱う手法である.
%既存手法は殆どがパターンマッチングと重み付けの2つに区分することができる.
%パターンマッチングでは事前に単語を登録して,単語がマッチした場合に処理を行うが,処理それぞれに対して単語を登録しなければならず手間が膨大になってしまう.また,単語の意味が考慮されておらず,手作業で登録を行うので登録漏れがあった場合に単語の意味に関係なく処理を行うことが不可能となってしまう.
%重み付けは単語の出現頻度や文章の長さを使用して単語・文章に順位を付ける手法で必ずしも単語の登録が必要でないため多くの研究で使用されている.
%しかし,重み付けもまた単語を文字列のまま扱っており,意味までは考慮されていない.故に ,人間なら対応できる似た単語でも1文字違うだけで対処が困難となる.
議論支援に関する先行研究においてファシリテーターに対する支援を目的としたものは無く,殆どが議論の活性化や可視化を目的としている.
%=================================新手法
近年,自然言語処理の分野において分散表現が多くの研究で使われており.分散表現は文字列である単語を辞書データを使用して実数ベクトルへと変換する.辞書データにない単語には対応できないが,多様な処理を1つの辞書データで行うことができる.また,実数ベクトルの各数値が単語の意味を表現するものとなっており,数値を使用して処理を行うことができる.
分散表現を用いることで既存手法より人間の感覚に近しい処理を行うことができる.
%=================================
以上のような背景を踏まえて,分散表現を用いてファシリテーターの代わりに話題の変化を判定し,知らせることを目指す.
話題転換の検出は発言同士の近さ,すなわち発言に含まれる単語意味の近さと見ることができる.
分散表現ではベクトル同士の内積計算を行うことで単語同士の意味の近さを計算することができる.
また,分散表現を使用することで機械翻訳を始めとする複数の分野で精度の向上が確認されている.
\end{comment}
\section{研究の目的}
\label{intro:taget}
本論文では,分散表現を用いて議論中での発言に含まれる単語の関連度を計算し,話題の変化を観測する手法を提案する.

\section{本論文の構成}
本論文の構成を以下に示す.
\ref{relwork:chapter} 章では要約手法に関する研究と,分散表現に関する先行研究を紹介する.
次に,\ref{model:chapter}章では発言の要約手法の説明を行い,\ref{impl:chapter}章では分散表現を用いた単語集合間の関連度計算について説明する.
そして,\ref{exp:chapter}章では話題転換点の検出の評価実験について説明する.
最後に\ref{con:chapter}章で本論文のまとめと考察を示す.

 %-------------------------------------------------------------------------------
 \expandafter\ifx\csname MasterFile\endcsname\relax
	\def\BibFile{hoge}
	\expandafter\ifx\csname MasterFile\endcsname\relax
	\def\SubFile{hoge}
	\input{../thesis/thesis}
	\begin{document}
	\setcounter{chapter}{0}
	\fi
  %-------------------------------------------------------------------------------
\cleardoublepage
\chapter{序論}
\label{intro:chapter}
%本章では, 本研究を行なうに至った背景と目的について述べる.その後,本論文の構成について述べる.
\section{研究の背景}
\label{intro:background}
近年,Web上での大規模な議論活動が活発になっているが,現在一般的に使われている "2ちゃんねる" や "Twitter" といったシステムでは整理や収束を行うことが困難である.困難である原因として,議論の管理を行う者がいないことが挙げられる.
つまり,議論を整理・収束させるには議論のマネジメントを行う人物が必要である.
%
大規模意見集約システムCOLLAGREE\cite{collagreeTest}ではファシリテーターと呼ばれる人物が議論のマネジメントを行っている.
しかし,ファシリテーターは人間であり,長時間に渡って大人数での議論の動向をマネジメントし続けるのは困難である.
COLLAGREEで大規模な議論を収束させるためには,ファシリテーターが必要な時には画面を見るようにして,他の時は見なくても済むようにすることで画面に向き合う時間を減らす工夫があることが望ましい.ファシリテーターが画面を見るべきタイミングは議論の話題が変化したときである.以前の議論の内容から外れた発言がされた時,ファシリテーターが適切な発言をすることで,脱線や炎上を避けて議論を収束させることができる.
すなわち,ファシリテーターの代わりに自動的に議論中の話題の変化を観測することが求められている.
%
現在,COLLAGREE上で使用されている議論支援システムは「(1)投稿支援システム」と「(2)議論可視化システム」の2つに大別できる.
投稿支援システムはポイント機能やファシリテーションフレーズ簡易投稿機能のように,ユーザーが投稿をする際に何らかの補助やリアクションを行う.現行の機能では選択肢の提示に留まっており,作業量を減らすことには繋がりにくい.
一方,議論可視化システムは議論ツリーやキーワード抽出のように,ユーザーにスレッドとは異なる議論の見方を提供する.
\ref{Fig:argTree1}に議論ツリーの例を示す.
\begin{figure}[htbp]
 \begin{center}
  \includegraphics[width=\textwidth]{../images/2.Related_Work/argTree1.png}
  \caption{議論ツリー}
  \label{Fig:argTree1}
  \vspace{-10pt}
 \end{center}
\end{figure}
現行の機能では議論を見やすくすることに重点が置かれており,議論の把握の助けにはなるが画面に向き合う時間を減らすことにはなりにくい.むしろ,作業量を増やすことになり得ることもある.
従って,現行の支援機能ではファシリテーターの作業量の減少には繋がりにくい.
%
近年,自然言語処理の分野において分散表現が多くの研究で使われており,機械翻訳を始めとする単語の意味が重要となる分野で精度の向上が確認されている.分散表現を用いることで,人間に近い精度で話題の変化を観測することが可能となる.
%
以上のような背景を踏まえて,分散表現を用いて,話題の変化を観測し,話題の変化が確認された時にファシリテーターに伝えることが望ましい.
話題の変化の観測は,発言中に現れる単語の類似度の計算と見なすことができる.
分散表現を用いることで単語間の類似度を求めることができる,値が大きいほど単語がそれぞれ類似した実数ベクトルであることを表す.単語Aと単語Bの実数ベクトルが類似しているとは,単語Aと共に使われることの多い単語と単語Bと共に使われることの多い単語が多く共通していることを示す.故に,分散表現を使って単語の類似度を計算することができる.
%
発言文から単語を選ぶ際には自動要約を用いる.発言文から重要でない単語を取り除くことで関連度の計算の精度を高めることが可能となる.
要約の手法としてはokapi BM25 \cite{okapiBM25}とLexRankを組み合わせた抽出的要約手法を用いる.
\begin{comment}
%======================================= 社会的背景
2013年頃からWeb上での大規模な議論活動が活発になり,大規模な人数での議論が期待されている.
大規模な議論では意見を共有することは可能であるが,議論を整理させることや収束させることは難しい.以上から大規模意見集約システムCOLLAGREEが開発された.本システムではWeb上で適切に大規模な議論を行うことができるように議論をマネジメントするファシリテーターを導入した\cite{collagreeTest}.
過去の実験ではファシリテーターの存在が議論の集約に大きな役割を果たしていることが認識されており,大規模な議論のためにファシリテータは必要である.しかし,議論の規模に伴って議論時間が長くなる傾向があり,同時にファシリテーターは常に議論の動向を見続ける必要がある.故に,議論の規模が大きくなればなるほどファシリテーターは長時間かつ大規模な議論の動向の監視によって大きな負担がかかる.大規模な議論が増加する傾向を踏まえるとファシリテーターにかかる負担を軽減する支援が必要である.\\
以上の問題を解決するため,話題の変化を追い,重要な話題の転換点をファシリテーターの代わりに検出することが有用であると考える.必要な時にだけファシリテーターが画面を見れば良いようにすることでファシリテーターの負担軽減が期待できる.
%========================================= 現行手法問題点背景
%議論支援に関する先行研究において,既存の手法は全てが文字列を文字列のまま扱う手法である.
%既存手法は殆どがパターンマッチングと重み付けの2つに区分することができる.
%パターンマッチングでは事前に単語を登録して,単語がマッチした場合に処理を行うが,処理それぞれに対して単語を登録しなければならず手間が膨大になってしまう.また,単語の意味が考慮されておらず,手作業で登録を行うので登録漏れがあった場合に単語の意味に関係なく処理を行うことが不可能となってしまう.
%重み付けは単語の出現頻度や文章の長さを使用して単語・文章に順位を付ける手法で必ずしも単語の登録が必要でないため多くの研究で使用されている.
%しかし,重み付けもまた単語を文字列のまま扱っており,意味までは考慮されていない.故に ,人間なら対応できる似た単語でも1文字違うだけで対処が困難となる.
議論支援に関する先行研究においてファシリテーターに対する支援を目的としたものは無く,殆どが議論の活性化や可視化を目的としている.
%=================================新手法
近年,自然言語処理の分野において分散表現が多くの研究で使われており.分散表現は文字列である単語を辞書データを使用して実数ベクトルへと変換する.辞書データにない単語には対応できないが,多様な処理を1つの辞書データで行うことができる.また,実数ベクトルの各数値が単語の意味を表現するものとなっており,数値を使用して処理を行うことができる.
分散表現を用いることで既存手法より人間の感覚に近しい処理を行うことができる.
%=================================
以上のような背景を踏まえて,分散表現を用いてファシリテーターの代わりに話題の変化を判定し,知らせることを目指す.
話題転換の検出は発言同士の近さ,すなわち発言に含まれる単語意味の近さと見ることができる.
分散表現ではベクトル同士の内積計算を行うことで単語同士の意味の近さを計算することができる.
また,分散表現を使用することで機械翻訳を始めとする複数の分野で精度の向上が確認されている.
\end{comment}
\section{研究の目的}
\label{intro:taget}
本論文では,分散表現を用いて議論中での発言に含まれる単語の関連度を計算し,話題の変化を観測する手法を提案する.

\section{本論文の構成}
本論文の構成を以下に示す.
\ref{relwork:chapter} 章では要約手法に関する研究と,分散表現に関する先行研究を紹介する.
次に,\ref{model:chapter}章では発言の要約手法の説明を行い,\ref{impl:chapter}章では分散表現を用いた単語集合間の関連度計算について説明する.
そして,\ref{exp:chapter}章では話題転換点の検出の評価実験について説明する.
最後に\ref{con:chapter}章で本論文のまとめと考察を示す.

 %-------------------------------------------------------------------------------
 \expandafter\ifx\csname MasterFile\endcsname\relax
	\def\BibFile{hoge}
	\input{../Bibliography/chapter}
  \fi
  %-------------------------------------------------------------------------------
  \expandafter\ifx\csname MasterFile\endcsname\relax
  \end{document}
  \fi

  \fi
  %-------------------------------------------------------------------------------
  \expandafter\ifx\csname MasterFile\endcsname\relax
  \end{document}
  \fi

%-------------------------------------------------------------------------------
\expandafter\ifx\csname MasterFile\endcsname\relax
	\def\SubFile{hoge}
	\documentclass[a4j,12pt,twoside,openany]{jreport}
%\nofiles %tocファイルを更新させない
%\documentclass[12pt,a4j,twoside,openany]{jsbook}
\usepackage[dvipdfmx]{graphicx}
\usepackage{../dspc} % ベースラインスキップの指定
\usepackage{../slashbox} % 表に斜線を入れる
%\usepackage{../mediabb}
\usepackage{fancyvrb} % Verbatim環境
\usepackage{fancyhdr} % Headerの下線付き章見出し
\usepackage{here} % float[H]
\usepackage{multirow}
\usepackage{hhline} % 表の罫線の角を美しくする
\usepackage{amsmath} %コレがないとcasesが動かない
\usepackage{amsfonts} % 数学用フォント
\usepackage{bm} % 数式環境での bold
\usepackage{algorithm}
\usepackage{algorithmicx}
\usepackage[noend]{algpseudocode}
\usepackage[flushleft]{threeparttable} % 脚注付きテーブル
\usepackage{enumitem}
\usepackage{comment}
\usepackage{fancybox}
%\usepackage{csvsimple,booktabs,siunitx}
%\usepackage{filecontents}


\setlength{\evensidemargin}{5pt}
\setlength{\oddsidemargin}{40pt}
%\setlength{\headheight}{16.5pt}
%%\setlength{\headheight}{30pt}
\setcounter{secnumdepth}{3}
\setlist[description]{leftmargin=2\parindent,labelindent=\parindent}

\makeatletter
\def\@makechapterhead#1{%
	\vspace*{50\p@}%
	{
		\parindent \z@ \raggedright \normalfont
		\ifnum \c@secnumdepth >\m@ne
		% \if@mainmatter
			\huge\bfseries\@chapapp\thechapter\@chappos
			\par\nobreak
			\vskip 20\p@
		% \fi
		\fi
		\interlinepenalty\@M
		\Huge\bfseries #1\par\nobreak
		\vskip 40\p@
	}
}

%新しいコマンド定義
\newcounter{linenumber}
\newenvironment{listing}{%
  \begin{list}{%
    \small\arabic{linenumber}:}{%
      \usecounter{linenumber}%
      \setlength{\baselineskip}{18pt}%
      \setlength{\itemsep}{0pt}%
      \setlength{\parsep}{0pt}}}%
 {\end{list}}
\newcommand{\figcaption}[1]{\def\@captype{figure}\caption{#1}}
\newcommand{\tblcaption}[1]{\def\@captype{table}\caption{#1}}
\newcommand{\norm}[1]{\left\| #1 \right\|}
\newcommand{\cc}[1]{\multicolumn{1}{|c|}{#1}}
\newcommand{\circled}[1]{\raisebox{.5pt}{\textcircled{\raisebox{-.9pt} {#1}}}}
\newcommand{\specialcell}[2][c]{%
  \begin{tabular}[#1]{@{}c@{}}#2\end{tabular}}
\makeatother
%===============================================================================
\expandafter\ifx\csname SubFile\endcsname\relax
\begin{document}
\def\MasterFile{hoge}
%-------------------------------------------------------------------------------
%\maketitle
\thispagestyle{empty}
\input{../hyoushi/title}
%\addcontentsline{toc}{chapter}{表紙}
\thispagestyle{empty}
\mbox{}\newpage
%===============================================================================
%\frontmatter
%===============================================================================
%\mainmatter
%-------------------------------------------------------------------------------
\pagenumbering{arabic}
\cleardoublepage
\input{../0.Abstract/chapter}
%-------------------------------------------------------------------------------
\clearpage
\addcontentsline{toc}{chapter}{目次}
\tableofcontents

\clearpage
\addcontentsline{toc}{chapter}{図目次}
\listoffigures

\clearpage
\addcontentsline{toc}{chapter}{表目次}
\listoftables

%-------------------------------------------------------------------------------

%=====================
\pagestyle{fancy} % Headerをつける
\renewcommand{\sectionmark}[1]{\markright{\thesection\ \ \ #1}}
\renewcommand{\chaptermark}[1]{\markboth{#1}{}}
\lhead{}
\chead{}
\lfoot{}
\rfoot{}%-------------------------------------------------------------------------------
\input{../1.Introduction/chapter}
%-------------------------------------------------------------------------------
\input{../2.Related_Work/chapter}
%-------------------------------------------------------------------------------
\input{../3.The_Model/chapter}
%-------------------------------------------------------------------------------
\input{../4.Implementation/chapter}
%-------------------------------------------------------------------------------
\input{../5.Experiments/chapter}
%-------------------------------------------------------------------------------
\input{../6.Conclusion/chapter}

%===============================================================================
\pagestyle{plain}
%-------------------------------------------------------------------------------
\input{../7.Acknowledgement/chapter} %謝辞
%-------------------------------------------------------------------------------
\def\BibFile{../Bibliograhoy/database2}
\input{../Bibliography/chapter} %参考文献
% %===============================================================================
\appendix
\input{../A.Mypaper/chapter} % 投稿論文リスト
\input{../B.SIG-CCI2/chapter} %
\input{../C.IJCAI-16/chapter} %
%===============================================================================
\end{document}
\fi

	\begin{document}
	\setcounter{chapter}{0}
	\fi
  %-------------------------------------------------------------------------------
\cleardoublepage
\chapter{序論}
\label{intro:chapter}
%本章では, 本研究を行なうに至った背景と目的について述べる.その後,本論文の構成について述べる.
\section{研究の背景}
\label{intro:background}
近年,Web上での大規模な議論活動が活発になっているが,現在一般的に使われている "2ちゃんねる" や "Twitter" といったシステムでは整理や収束を行うことが困難である.困難である原因として,議論の管理を行う者がいないことが挙げられる.
つまり,議論を整理・収束させるには議論のマネジメントを行う人物が必要である.
%
大規模意見集約システムCOLLAGREE\cite{collagreeTest}ではファシリテーターと呼ばれる人物が議論のマネジメントを行っている.
しかし,ファシリテーターは人間であり,長時間に渡って大人数での議論の動向をマネジメントし続けるのは困難である.
COLLAGREEで大規模な議論を収束させるためには,ファシリテーターが必要な時には画面を見るようにして,他の時は見なくても済むようにすることで画面に向き合う時間を減らす工夫があることが望ましい.ファシリテーターが画面を見るべきタイミングは議論の話題が変化したときである.以前の議論の内容から外れた発言がされた時,ファシリテーターが適切な発言をすることで,脱線や炎上を避けて議論を収束させることができる.
すなわち,ファシリテーターの代わりに自動的に議論中の話題の変化を観測することが求められている.
%
現在,COLLAGREE上で使用されている議論支援システムは「(1)投稿支援システム」と「(2)議論可視化システム」の2つに大別できる.
投稿支援システムはポイント機能やファシリテーションフレーズ簡易投稿機能のように,ユーザーが投稿をする際に何らかの補助やリアクションを行う.現行の機能では選択肢の提示に留まっており,作業量を減らすことには繋がりにくい.
一方,議論可視化システムは議論ツリーやキーワード抽出のように,ユーザーにスレッドとは異なる議論の見方を提供する.
\ref{Fig:argTree1}に議論ツリーの例を示す.
\begin{figure}[htbp]
 \begin{center}
  \includegraphics[width=\textwidth]{../images/2.Related_Work/argTree1.png}
  \caption{議論ツリー}
  \label{Fig:argTree1}
  \vspace{-10pt}
 \end{center}
\end{figure}
現行の機能では議論を見やすくすることに重点が置かれており,議論の把握の助けにはなるが画面に向き合う時間を減らすことにはなりにくい.むしろ,作業量を増やすことになり得ることもある.
従って,現行の支援機能ではファシリテーターの作業量の減少には繋がりにくい.
%
近年,自然言語処理の分野において分散表現が多くの研究で使われており,機械翻訳を始めとする単語の意味が重要となる分野で精度の向上が確認されている.分散表現を用いることで,人間に近い精度で話題の変化を観測することが可能となる.
%
以上のような背景を踏まえて,分散表現を用いて,話題の変化を観測し,話題の変化が確認された時にファシリテーターに伝えることが望ましい.
話題の変化の観測は,発言中に現れる単語の類似度の計算と見なすことができる.
分散表現を用いることで単語間の類似度を求めることができる,値が大きいほど単語がそれぞれ類似した実数ベクトルであることを表す.単語Aと単語Bの実数ベクトルが類似しているとは,単語Aと共に使われることの多い単語と単語Bと共に使われることの多い単語が多く共通していることを示す.故に,分散表現を使って単語の類似度を計算することができる.
%
発言文から単語を選ぶ際には自動要約を用いる.発言文から重要でない単語を取り除くことで関連度の計算の精度を高めることが可能となる.
要約の手法としてはokapi BM25 \cite{okapiBM25}とLexRankを組み合わせた抽出的要約手法を用いる.
\begin{comment}
%======================================= 社会的背景
2013年頃からWeb上での大規模な議論活動が活発になり,大規模な人数での議論が期待されている.
大規模な議論では意見を共有することは可能であるが,議論を整理させることや収束させることは難しい.以上から大規模意見集約システムCOLLAGREEが開発された.本システムではWeb上で適切に大規模な議論を行うことができるように議論をマネジメントするファシリテーターを導入した\cite{collagreeTest}.
過去の実験ではファシリテーターの存在が議論の集約に大きな役割を果たしていることが認識されており,大規模な議論のためにファシリテータは必要である.しかし,議論の規模に伴って議論時間が長くなる傾向があり,同時にファシリテーターは常に議論の動向を見続ける必要がある.故に,議論の規模が大きくなればなるほどファシリテーターは長時間かつ大規模な議論の動向の監視によって大きな負担がかかる.大規模な議論が増加する傾向を踏まえるとファシリテーターにかかる負担を軽減する支援が必要である.\\
以上の問題を解決するため,話題の変化を追い,重要な話題の転換点をファシリテーターの代わりに検出することが有用であると考える.必要な時にだけファシリテーターが画面を見れば良いようにすることでファシリテーターの負担軽減が期待できる.
%========================================= 現行手法問題点背景
%議論支援に関する先行研究において,既存の手法は全てが文字列を文字列のまま扱う手法である.
%既存手法は殆どがパターンマッチングと重み付けの2つに区分することができる.
%パターンマッチングでは事前に単語を登録して,単語がマッチした場合に処理を行うが,処理それぞれに対して単語を登録しなければならず手間が膨大になってしまう.また,単語の意味が考慮されておらず,手作業で登録を行うので登録漏れがあった場合に単語の意味に関係なく処理を行うことが不可能となってしまう.
%重み付けは単語の出現頻度や文章の長さを使用して単語・文章に順位を付ける手法で必ずしも単語の登録が必要でないため多くの研究で使用されている.
%しかし,重み付けもまた単語を文字列のまま扱っており,意味までは考慮されていない.故に ,人間なら対応できる似た単語でも1文字違うだけで対処が困難となる.
議論支援に関する先行研究においてファシリテーターに対する支援を目的としたものは無く,殆どが議論の活性化や可視化を目的としている.
%=================================新手法
近年,自然言語処理の分野において分散表現が多くの研究で使われており.分散表現は文字列である単語を辞書データを使用して実数ベクトルへと変換する.辞書データにない単語には対応できないが,多様な処理を1つの辞書データで行うことができる.また,実数ベクトルの各数値が単語の意味を表現するものとなっており,数値を使用して処理を行うことができる.
分散表現を用いることで既存手法より人間の感覚に近しい処理を行うことができる.
%=================================
以上のような背景を踏まえて,分散表現を用いてファシリテーターの代わりに話題の変化を判定し,知らせることを目指す.
話題転換の検出は発言同士の近さ,すなわち発言に含まれる単語意味の近さと見ることができる.
分散表現ではベクトル同士の内積計算を行うことで単語同士の意味の近さを計算することができる.
また,分散表現を使用することで機械翻訳を始めとする複数の分野で精度の向上が確認されている.
\end{comment}
\section{研究の目的}
\label{intro:taget}
本論文では,分散表現を用いて議論中での発言に含まれる単語の関連度を計算し,話題の変化を観測する手法を提案する.

\section{本論文の構成}
本論文の構成を以下に示す.
\ref{relwork:chapter} 章では要約手法に関する研究と,分散表現に関する先行研究を紹介する.
次に,\ref{model:chapter}章では発言の要約手法の説明を行い,\ref{impl:chapter}章では分散表現を用いた単語集合間の関連度計算について説明する.
そして,\ref{exp:chapter}章では話題転換点の検出の評価実験について説明する.
最後に\ref{con:chapter}章で本論文のまとめと考察を示す.

 %-------------------------------------------------------------------------------
 \expandafter\ifx\csname MasterFile\endcsname\relax
	\def\BibFile{hoge}
	\expandafter\ifx\csname MasterFile\endcsname\relax
	\def\SubFile{hoge}
	\input{../thesis/thesis}
	\begin{document}
	\setcounter{chapter}{0}
	\fi
  %-------------------------------------------------------------------------------
\cleardoublepage
\chapter{序論}
\label{intro:chapter}
%本章では, 本研究を行なうに至った背景と目的について述べる.その後,本論文の構成について述べる.
\section{研究の背景}
\label{intro:background}
近年,Web上での大規模な議論活動が活発になっているが,現在一般的に使われている "2ちゃんねる" や "Twitter" といったシステムでは整理や収束を行うことが困難である.困難である原因として,議論の管理を行う者がいないことが挙げられる.
つまり,議論を整理・収束させるには議論のマネジメントを行う人物が必要である.
%
大規模意見集約システムCOLLAGREE\cite{collagreeTest}ではファシリテーターと呼ばれる人物が議論のマネジメントを行っている.
しかし,ファシリテーターは人間であり,長時間に渡って大人数での議論の動向をマネジメントし続けるのは困難である.
COLLAGREEで大規模な議論を収束させるためには,ファシリテーターが必要な時には画面を見るようにして,他の時は見なくても済むようにすることで画面に向き合う時間を減らす工夫があることが望ましい.ファシリテーターが画面を見るべきタイミングは議論の話題が変化したときである.以前の議論の内容から外れた発言がされた時,ファシリテーターが適切な発言をすることで,脱線や炎上を避けて議論を収束させることができる.
すなわち,ファシリテーターの代わりに自動的に議論中の話題の変化を観測することが求められている.
%
現在,COLLAGREE上で使用されている議論支援システムは「(1)投稿支援システム」と「(2)議論可視化システム」の2つに大別できる.
投稿支援システムはポイント機能やファシリテーションフレーズ簡易投稿機能のように,ユーザーが投稿をする際に何らかの補助やリアクションを行う.現行の機能では選択肢の提示に留まっており,作業量を減らすことには繋がりにくい.
一方,議論可視化システムは議論ツリーやキーワード抽出のように,ユーザーにスレッドとは異なる議論の見方を提供する.
\ref{Fig:argTree1}に議論ツリーの例を示す.
\begin{figure}[htbp]
 \begin{center}
  \includegraphics[width=\textwidth]{../images/2.Related_Work/argTree1.png}
  \caption{議論ツリー}
  \label{Fig:argTree1}
  \vspace{-10pt}
 \end{center}
\end{figure}
現行の機能では議論を見やすくすることに重点が置かれており,議論の把握の助けにはなるが画面に向き合う時間を減らすことにはなりにくい.むしろ,作業量を増やすことになり得ることもある.
従って,現行の支援機能ではファシリテーターの作業量の減少には繋がりにくい.
%
近年,自然言語処理の分野において分散表現が多くの研究で使われており,機械翻訳を始めとする単語の意味が重要となる分野で精度の向上が確認されている.分散表現を用いることで,人間に近い精度で話題の変化を観測することが可能となる.
%
以上のような背景を踏まえて,分散表現を用いて,話題の変化を観測し,話題の変化が確認された時にファシリテーターに伝えることが望ましい.
話題の変化の観測は,発言中に現れる単語の類似度の計算と見なすことができる.
分散表現を用いることで単語間の類似度を求めることができる,値が大きいほど単語がそれぞれ類似した実数ベクトルであることを表す.単語Aと単語Bの実数ベクトルが類似しているとは,単語Aと共に使われることの多い単語と単語Bと共に使われることの多い単語が多く共通していることを示す.故に,分散表現を使って単語の類似度を計算することができる.
%
発言文から単語を選ぶ際には自動要約を用いる.発言文から重要でない単語を取り除くことで関連度の計算の精度を高めることが可能となる.
要約の手法としてはokapi BM25 \cite{okapiBM25}とLexRankを組み合わせた抽出的要約手法を用いる.
\begin{comment}
%======================================= 社会的背景
2013年頃からWeb上での大規模な議論活動が活発になり,大規模な人数での議論が期待されている.
大規模な議論では意見を共有することは可能であるが,議論を整理させることや収束させることは難しい.以上から大規模意見集約システムCOLLAGREEが開発された.本システムではWeb上で適切に大規模な議論を行うことができるように議論をマネジメントするファシリテーターを導入した\cite{collagreeTest}.
過去の実験ではファシリテーターの存在が議論の集約に大きな役割を果たしていることが認識されており,大規模な議論のためにファシリテータは必要である.しかし,議論の規模に伴って議論時間が長くなる傾向があり,同時にファシリテーターは常に議論の動向を見続ける必要がある.故に,議論の規模が大きくなればなるほどファシリテーターは長時間かつ大規模な議論の動向の監視によって大きな負担がかかる.大規模な議論が増加する傾向を踏まえるとファシリテーターにかかる負担を軽減する支援が必要である.\\
以上の問題を解決するため,話題の変化を追い,重要な話題の転換点をファシリテーターの代わりに検出することが有用であると考える.必要な時にだけファシリテーターが画面を見れば良いようにすることでファシリテーターの負担軽減が期待できる.
%========================================= 現行手法問題点背景
%議論支援に関する先行研究において,既存の手法は全てが文字列を文字列のまま扱う手法である.
%既存手法は殆どがパターンマッチングと重み付けの2つに区分することができる.
%パターンマッチングでは事前に単語を登録して,単語がマッチした場合に処理を行うが,処理それぞれに対して単語を登録しなければならず手間が膨大になってしまう.また,単語の意味が考慮されておらず,手作業で登録を行うので登録漏れがあった場合に単語の意味に関係なく処理を行うことが不可能となってしまう.
%重み付けは単語の出現頻度や文章の長さを使用して単語・文章に順位を付ける手法で必ずしも単語の登録が必要でないため多くの研究で使用されている.
%しかし,重み付けもまた単語を文字列のまま扱っており,意味までは考慮されていない.故に ,人間なら対応できる似た単語でも1文字違うだけで対処が困難となる.
議論支援に関する先行研究においてファシリテーターに対する支援を目的としたものは無く,殆どが議論の活性化や可視化を目的としている.
%=================================新手法
近年,自然言語処理の分野において分散表現が多くの研究で使われており.分散表現は文字列である単語を辞書データを使用して実数ベクトルへと変換する.辞書データにない単語には対応できないが,多様な処理を1つの辞書データで行うことができる.また,実数ベクトルの各数値が単語の意味を表現するものとなっており,数値を使用して処理を行うことができる.
分散表現を用いることで既存手法より人間の感覚に近しい処理を行うことができる.
%=================================
以上のような背景を踏まえて,分散表現を用いてファシリテーターの代わりに話題の変化を判定し,知らせることを目指す.
話題転換の検出は発言同士の近さ,すなわち発言に含まれる単語意味の近さと見ることができる.
分散表現ではベクトル同士の内積計算を行うことで単語同士の意味の近さを計算することができる.
また,分散表現を使用することで機械翻訳を始めとする複数の分野で精度の向上が確認されている.
\end{comment}
\section{研究の目的}
\label{intro:taget}
本論文では,分散表現を用いて議論中での発言に含まれる単語の関連度を計算し,話題の変化を観測する手法を提案する.

\section{本論文の構成}
本論文の構成を以下に示す.
\ref{relwork:chapter} 章では要約手法に関する研究と,分散表現に関する先行研究を紹介する.
次に,\ref{model:chapter}章では発言の要約手法の説明を行い,\ref{impl:chapter}章では分散表現を用いた単語集合間の関連度計算について説明する.
そして,\ref{exp:chapter}章では話題転換点の検出の評価実験について説明する.
最後に\ref{con:chapter}章で本論文のまとめと考察を示す.

 %-------------------------------------------------------------------------------
 \expandafter\ifx\csname MasterFile\endcsname\relax
	\def\BibFile{hoge}
	\input{../Bibliography/chapter}
  \fi
  %-------------------------------------------------------------------------------
  \expandafter\ifx\csname MasterFile\endcsname\relax
  \end{document}
  \fi

  \fi
  %-------------------------------------------------------------------------------
  \expandafter\ifx\csname MasterFile\endcsname\relax
  \end{document}
  \fi

%-------------------------------------------------------------------------------
\expandafter\ifx\csname MasterFile\endcsname\relax
	\def\SubFile{hoge}
	\documentclass[a4j,12pt,twoside,openany]{jreport}
%\nofiles %tocファイルを更新させない
%\documentclass[12pt,a4j,twoside,openany]{jsbook}
\usepackage[dvipdfmx]{graphicx}
\usepackage{../dspc} % ベースラインスキップの指定
\usepackage{../slashbox} % 表に斜線を入れる
%\usepackage{../mediabb}
\usepackage{fancyvrb} % Verbatim環境
\usepackage{fancyhdr} % Headerの下線付き章見出し
\usepackage{here} % float[H]
\usepackage{multirow}
\usepackage{hhline} % 表の罫線の角を美しくする
\usepackage{amsmath} %コレがないとcasesが動かない
\usepackage{amsfonts} % 数学用フォント
\usepackage{bm} % 数式環境での bold
\usepackage{algorithm}
\usepackage{algorithmicx}
\usepackage[noend]{algpseudocode}
\usepackage[flushleft]{threeparttable} % 脚注付きテーブル
\usepackage{enumitem}
\usepackage{comment}
\usepackage{fancybox}
%\usepackage{csvsimple,booktabs,siunitx}
%\usepackage{filecontents}


\setlength{\evensidemargin}{5pt}
\setlength{\oddsidemargin}{40pt}
%\setlength{\headheight}{16.5pt}
%%\setlength{\headheight}{30pt}
\setcounter{secnumdepth}{3}
\setlist[description]{leftmargin=2\parindent,labelindent=\parindent}

\makeatletter
\def\@makechapterhead#1{%
	\vspace*{50\p@}%
	{
		\parindent \z@ \raggedright \normalfont
		\ifnum \c@secnumdepth >\m@ne
		% \if@mainmatter
			\huge\bfseries\@chapapp\thechapter\@chappos
			\par\nobreak
			\vskip 20\p@
		% \fi
		\fi
		\interlinepenalty\@M
		\Huge\bfseries #1\par\nobreak
		\vskip 40\p@
	}
}

%新しいコマンド定義
\newcounter{linenumber}
\newenvironment{listing}{%
  \begin{list}{%
    \small\arabic{linenumber}:}{%
      \usecounter{linenumber}%
      \setlength{\baselineskip}{18pt}%
      \setlength{\itemsep}{0pt}%
      \setlength{\parsep}{0pt}}}%
 {\end{list}}
\newcommand{\figcaption}[1]{\def\@captype{figure}\caption{#1}}
\newcommand{\tblcaption}[1]{\def\@captype{table}\caption{#1}}
\newcommand{\norm}[1]{\left\| #1 \right\|}
\newcommand{\cc}[1]{\multicolumn{1}{|c|}{#1}}
\newcommand{\circled}[1]{\raisebox{.5pt}{\textcircled{\raisebox{-.9pt} {#1}}}}
\newcommand{\specialcell}[2][c]{%
  \begin{tabular}[#1]{@{}c@{}}#2\end{tabular}}
\makeatother
%===============================================================================
\expandafter\ifx\csname SubFile\endcsname\relax
\begin{document}
\def\MasterFile{hoge}
%-------------------------------------------------------------------------------
%\maketitle
\thispagestyle{empty}
\input{../hyoushi/title}
%\addcontentsline{toc}{chapter}{表紙}
\thispagestyle{empty}
\mbox{}\newpage
%===============================================================================
%\frontmatter
%===============================================================================
%\mainmatter
%-------------------------------------------------------------------------------
\pagenumbering{arabic}
\cleardoublepage
\input{../0.Abstract/chapter}
%-------------------------------------------------------------------------------
\clearpage
\addcontentsline{toc}{chapter}{目次}
\tableofcontents

\clearpage
\addcontentsline{toc}{chapter}{図目次}
\listoffigures

\clearpage
\addcontentsline{toc}{chapter}{表目次}
\listoftables

%-------------------------------------------------------------------------------

%=====================
\pagestyle{fancy} % Headerをつける
\renewcommand{\sectionmark}[1]{\markright{\thesection\ \ \ #1}}
\renewcommand{\chaptermark}[1]{\markboth{#1}{}}
\lhead{}
\chead{}
\lfoot{}
\rfoot{}%-------------------------------------------------------------------------------
\input{../1.Introduction/chapter}
%-------------------------------------------------------------------------------
\input{../2.Related_Work/chapter}
%-------------------------------------------------------------------------------
\input{../3.The_Model/chapter}
%-------------------------------------------------------------------------------
\input{../4.Implementation/chapter}
%-------------------------------------------------------------------------------
\input{../5.Experiments/chapter}
%-------------------------------------------------------------------------------
\input{../6.Conclusion/chapter}

%===============================================================================
\pagestyle{plain}
%-------------------------------------------------------------------------------
\input{../7.Acknowledgement/chapter} %謝辞
%-------------------------------------------------------------------------------
\def\BibFile{../Bibliograhoy/database2}
\input{../Bibliography/chapter} %参考文献
% %===============================================================================
\appendix
\input{../A.Mypaper/chapter} % 投稿論文リスト
\input{../B.SIG-CCI2/chapter} %
\input{../C.IJCAI-16/chapter} %
%===============================================================================
\end{document}
\fi

	\begin{document}
	\setcounter{chapter}{0}
	\fi
  %-------------------------------------------------------------------------------
\cleardoublepage
\chapter{序論}
\label{intro:chapter}
%本章では, 本研究を行なうに至った背景と目的について述べる.その後,本論文の構成について述べる.
\section{研究の背景}
\label{intro:background}
近年,Web上での大規模な議論活動が活発になっているが,現在一般的に使われている "2ちゃんねる" や "Twitter" といったシステムでは整理や収束を行うことが困難である.困難である原因として,議論の管理を行う者がいないことが挙げられる.
つまり,議論を整理・収束させるには議論のマネジメントを行う人物が必要である.
%
大規模意見集約システムCOLLAGREE\cite{collagreeTest}ではファシリテーターと呼ばれる人物が議論のマネジメントを行っている.
しかし,ファシリテーターは人間であり,長時間に渡って大人数での議論の動向をマネジメントし続けるのは困難である.
COLLAGREEで大規模な議論を収束させるためには,ファシリテーターが必要な時には画面を見るようにして,他の時は見なくても済むようにすることで画面に向き合う時間を減らす工夫があることが望ましい.ファシリテーターが画面を見るべきタイミングは議論の話題が変化したときである.以前の議論の内容から外れた発言がされた時,ファシリテーターが適切な発言をすることで,脱線や炎上を避けて議論を収束させることができる.
すなわち,ファシリテーターの代わりに自動的に議論中の話題の変化を観測することが求められている.
%
現在,COLLAGREE上で使用されている議論支援システムは「(1)投稿支援システム」と「(2)議論可視化システム」の2つに大別できる.
投稿支援システムはポイント機能やファシリテーションフレーズ簡易投稿機能のように,ユーザーが投稿をする際に何らかの補助やリアクションを行う.現行の機能では選択肢の提示に留まっており,作業量を減らすことには繋がりにくい.
一方,議論可視化システムは議論ツリーやキーワード抽出のように,ユーザーにスレッドとは異なる議論の見方を提供する.
\ref{Fig:argTree1}に議論ツリーの例を示す.
\begin{figure}[htbp]
 \begin{center}
  \includegraphics[width=\textwidth]{../images/2.Related_Work/argTree1.png}
  \caption{議論ツリー}
  \label{Fig:argTree1}
  \vspace{-10pt}
 \end{center}
\end{figure}
現行の機能では議論を見やすくすることに重点が置かれており,議論の把握の助けにはなるが画面に向き合う時間を減らすことにはなりにくい.むしろ,作業量を増やすことになり得ることもある.
従って,現行の支援機能ではファシリテーターの作業量の減少には繋がりにくい.
%
近年,自然言語処理の分野において分散表現が多くの研究で使われており,機械翻訳を始めとする単語の意味が重要となる分野で精度の向上が確認されている.分散表現を用いることで,人間に近い精度で話題の変化を観測することが可能となる.
%
以上のような背景を踏まえて,分散表現を用いて,話題の変化を観測し,話題の変化が確認された時にファシリテーターに伝えることが望ましい.
話題の変化の観測は,発言中に現れる単語の類似度の計算と見なすことができる.
分散表現を用いることで単語間の類似度を求めることができる,値が大きいほど単語がそれぞれ類似した実数ベクトルであることを表す.単語Aと単語Bの実数ベクトルが類似しているとは,単語Aと共に使われることの多い単語と単語Bと共に使われることの多い単語が多く共通していることを示す.故に,分散表現を使って単語の類似度を計算することができる.
%
発言文から単語を選ぶ際には自動要約を用いる.発言文から重要でない単語を取り除くことで関連度の計算の精度を高めることが可能となる.
要約の手法としてはokapi BM25 \cite{okapiBM25}とLexRankを組み合わせた抽出的要約手法を用いる.
\begin{comment}
%======================================= 社会的背景
2013年頃からWeb上での大規模な議論活動が活発になり,大規模な人数での議論が期待されている.
大規模な議論では意見を共有することは可能であるが,議論を整理させることや収束させることは難しい.以上から大規模意見集約システムCOLLAGREEが開発された.本システムではWeb上で適切に大規模な議論を行うことができるように議論をマネジメントするファシリテーターを導入した\cite{collagreeTest}.
過去の実験ではファシリテーターの存在が議論の集約に大きな役割を果たしていることが認識されており,大規模な議論のためにファシリテータは必要である.しかし,議論の規模に伴って議論時間が長くなる傾向があり,同時にファシリテーターは常に議論の動向を見続ける必要がある.故に,議論の規模が大きくなればなるほどファシリテーターは長時間かつ大規模な議論の動向の監視によって大きな負担がかかる.大規模な議論が増加する傾向を踏まえるとファシリテーターにかかる負担を軽減する支援が必要である.\\
以上の問題を解決するため,話題の変化を追い,重要な話題の転換点をファシリテーターの代わりに検出することが有用であると考える.必要な時にだけファシリテーターが画面を見れば良いようにすることでファシリテーターの負担軽減が期待できる.
%========================================= 現行手法問題点背景
%議論支援に関する先行研究において,既存の手法は全てが文字列を文字列のまま扱う手法である.
%既存手法は殆どがパターンマッチングと重み付けの2つに区分することができる.
%パターンマッチングでは事前に単語を登録して,単語がマッチした場合に処理を行うが,処理それぞれに対して単語を登録しなければならず手間が膨大になってしまう.また,単語の意味が考慮されておらず,手作業で登録を行うので登録漏れがあった場合に単語の意味に関係なく処理を行うことが不可能となってしまう.
%重み付けは単語の出現頻度や文章の長さを使用して単語・文章に順位を付ける手法で必ずしも単語の登録が必要でないため多くの研究で使用されている.
%しかし,重み付けもまた単語を文字列のまま扱っており,意味までは考慮されていない.故に ,人間なら対応できる似た単語でも1文字違うだけで対処が困難となる.
議論支援に関する先行研究においてファシリテーターに対する支援を目的としたものは無く,殆どが議論の活性化や可視化を目的としている.
%=================================新手法
近年,自然言語処理の分野において分散表現が多くの研究で使われており.分散表現は文字列である単語を辞書データを使用して実数ベクトルへと変換する.辞書データにない単語には対応できないが,多様な処理を1つの辞書データで行うことができる.また,実数ベクトルの各数値が単語の意味を表現するものとなっており,数値を使用して処理を行うことができる.
分散表現を用いることで既存手法より人間の感覚に近しい処理を行うことができる.
%=================================
以上のような背景を踏まえて,分散表現を用いてファシリテーターの代わりに話題の変化を判定し,知らせることを目指す.
話題転換の検出は発言同士の近さ,すなわち発言に含まれる単語意味の近さと見ることができる.
分散表現ではベクトル同士の内積計算を行うことで単語同士の意味の近さを計算することができる.
また,分散表現を使用することで機械翻訳を始めとする複数の分野で精度の向上が確認されている.
\end{comment}
\section{研究の目的}
\label{intro:taget}
本論文では,分散表現を用いて議論中での発言に含まれる単語の関連度を計算し,話題の変化を観測する手法を提案する.

\section{本論文の構成}
本論文の構成を以下に示す.
\ref{relwork:chapter} 章では要約手法に関する研究と,分散表現に関する先行研究を紹介する.
次に,\ref{model:chapter}章では発言の要約手法の説明を行い,\ref{impl:chapter}章では分散表現を用いた単語集合間の関連度計算について説明する.
そして,\ref{exp:chapter}章では話題転換点の検出の評価実験について説明する.
最後に\ref{con:chapter}章で本論文のまとめと考察を示す.

 %-------------------------------------------------------------------------------
 \expandafter\ifx\csname MasterFile\endcsname\relax
	\def\BibFile{hoge}
	\expandafter\ifx\csname MasterFile\endcsname\relax
	\def\SubFile{hoge}
	\input{../thesis/thesis}
	\begin{document}
	\setcounter{chapter}{0}
	\fi
  %-------------------------------------------------------------------------------
\cleardoublepage
\chapter{序論}
\label{intro:chapter}
%本章では, 本研究を行なうに至った背景と目的について述べる.その後,本論文の構成について述べる.
\section{研究の背景}
\label{intro:background}
近年,Web上での大規模な議論活動が活発になっているが,現在一般的に使われている "2ちゃんねる" や "Twitter" といったシステムでは整理や収束を行うことが困難である.困難である原因として,議論の管理を行う者がいないことが挙げられる.
つまり,議論を整理・収束させるには議論のマネジメントを行う人物が必要である.
%
大規模意見集約システムCOLLAGREE\cite{collagreeTest}ではファシリテーターと呼ばれる人物が議論のマネジメントを行っている.
しかし,ファシリテーターは人間であり,長時間に渡って大人数での議論の動向をマネジメントし続けるのは困難である.
COLLAGREEで大規模な議論を収束させるためには,ファシリテーターが必要な時には画面を見るようにして,他の時は見なくても済むようにすることで画面に向き合う時間を減らす工夫があることが望ましい.ファシリテーターが画面を見るべきタイミングは議論の話題が変化したときである.以前の議論の内容から外れた発言がされた時,ファシリテーターが適切な発言をすることで,脱線や炎上を避けて議論を収束させることができる.
すなわち,ファシリテーターの代わりに自動的に議論中の話題の変化を観測することが求められている.
%
現在,COLLAGREE上で使用されている議論支援システムは「(1)投稿支援システム」と「(2)議論可視化システム」の2つに大別できる.
投稿支援システムはポイント機能やファシリテーションフレーズ簡易投稿機能のように,ユーザーが投稿をする際に何らかの補助やリアクションを行う.現行の機能では選択肢の提示に留まっており,作業量を減らすことには繋がりにくい.
一方,議論可視化システムは議論ツリーやキーワード抽出のように,ユーザーにスレッドとは異なる議論の見方を提供する.
\ref{Fig:argTree1}に議論ツリーの例を示す.
\begin{figure}[htbp]
 \begin{center}
  \includegraphics[width=\textwidth]{../images/2.Related_Work/argTree1.png}
  \caption{議論ツリー}
  \label{Fig:argTree1}
  \vspace{-10pt}
 \end{center}
\end{figure}
現行の機能では議論を見やすくすることに重点が置かれており,議論の把握の助けにはなるが画面に向き合う時間を減らすことにはなりにくい.むしろ,作業量を増やすことになり得ることもある.
従って,現行の支援機能ではファシリテーターの作業量の減少には繋がりにくい.
%
近年,自然言語処理の分野において分散表現が多くの研究で使われており,機械翻訳を始めとする単語の意味が重要となる分野で精度の向上が確認されている.分散表現を用いることで,人間に近い精度で話題の変化を観測することが可能となる.
%
以上のような背景を踏まえて,分散表現を用いて,話題の変化を観測し,話題の変化が確認された時にファシリテーターに伝えることが望ましい.
話題の変化の観測は,発言中に現れる単語の類似度の計算と見なすことができる.
分散表現を用いることで単語間の類似度を求めることができる,値が大きいほど単語がそれぞれ類似した実数ベクトルであることを表す.単語Aと単語Bの実数ベクトルが類似しているとは,単語Aと共に使われることの多い単語と単語Bと共に使われることの多い単語が多く共通していることを示す.故に,分散表現を使って単語の類似度を計算することができる.
%
発言文から単語を選ぶ際には自動要約を用いる.発言文から重要でない単語を取り除くことで関連度の計算の精度を高めることが可能となる.
要約の手法としてはokapi BM25 \cite{okapiBM25}とLexRankを組み合わせた抽出的要約手法を用いる.
\begin{comment}
%======================================= 社会的背景
2013年頃からWeb上での大規模な議論活動が活発になり,大規模な人数での議論が期待されている.
大規模な議論では意見を共有することは可能であるが,議論を整理させることや収束させることは難しい.以上から大規模意見集約システムCOLLAGREEが開発された.本システムではWeb上で適切に大規模な議論を行うことができるように議論をマネジメントするファシリテーターを導入した\cite{collagreeTest}.
過去の実験ではファシリテーターの存在が議論の集約に大きな役割を果たしていることが認識されており,大規模な議論のためにファシリテータは必要である.しかし,議論の規模に伴って議論時間が長くなる傾向があり,同時にファシリテーターは常に議論の動向を見続ける必要がある.故に,議論の規模が大きくなればなるほどファシリテーターは長時間かつ大規模な議論の動向の監視によって大きな負担がかかる.大規模な議論が増加する傾向を踏まえるとファシリテーターにかかる負担を軽減する支援が必要である.\\
以上の問題を解決するため,話題の変化を追い,重要な話題の転換点をファシリテーターの代わりに検出することが有用であると考える.必要な時にだけファシリテーターが画面を見れば良いようにすることでファシリテーターの負担軽減が期待できる.
%========================================= 現行手法問題点背景
%議論支援に関する先行研究において,既存の手法は全てが文字列を文字列のまま扱う手法である.
%既存手法は殆どがパターンマッチングと重み付けの2つに区分することができる.
%パターンマッチングでは事前に単語を登録して,単語がマッチした場合に処理を行うが,処理それぞれに対して単語を登録しなければならず手間が膨大になってしまう.また,単語の意味が考慮されておらず,手作業で登録を行うので登録漏れがあった場合に単語の意味に関係なく処理を行うことが不可能となってしまう.
%重み付けは単語の出現頻度や文章の長さを使用して単語・文章に順位を付ける手法で必ずしも単語の登録が必要でないため多くの研究で使用されている.
%しかし,重み付けもまた単語を文字列のまま扱っており,意味までは考慮されていない.故に ,人間なら対応できる似た単語でも1文字違うだけで対処が困難となる.
議論支援に関する先行研究においてファシリテーターに対する支援を目的としたものは無く,殆どが議論の活性化や可視化を目的としている.
%=================================新手法
近年,自然言語処理の分野において分散表現が多くの研究で使われており.分散表現は文字列である単語を辞書データを使用して実数ベクトルへと変換する.辞書データにない単語には対応できないが,多様な処理を1つの辞書データで行うことができる.また,実数ベクトルの各数値が単語の意味を表現するものとなっており,数値を使用して処理を行うことができる.
分散表現を用いることで既存手法より人間の感覚に近しい処理を行うことができる.
%=================================
以上のような背景を踏まえて,分散表現を用いてファシリテーターの代わりに話題の変化を判定し,知らせることを目指す.
話題転換の検出は発言同士の近さ,すなわち発言に含まれる単語意味の近さと見ることができる.
分散表現ではベクトル同士の内積計算を行うことで単語同士の意味の近さを計算することができる.
また,分散表現を使用することで機械翻訳を始めとする複数の分野で精度の向上が確認されている.
\end{comment}
\section{研究の目的}
\label{intro:taget}
本論文では,分散表現を用いて議論中での発言に含まれる単語の関連度を計算し,話題の変化を観測する手法を提案する.

\section{本論文の構成}
本論文の構成を以下に示す.
\ref{relwork:chapter} 章では要約手法に関する研究と,分散表現に関する先行研究を紹介する.
次に,\ref{model:chapter}章では発言の要約手法の説明を行い,\ref{impl:chapter}章では分散表現を用いた単語集合間の関連度計算について説明する.
そして,\ref{exp:chapter}章では話題転換点の検出の評価実験について説明する.
最後に\ref{con:chapter}章で本論文のまとめと考察を示す.

 %-------------------------------------------------------------------------------
 \expandafter\ifx\csname MasterFile\endcsname\relax
	\def\BibFile{hoge}
	\input{../Bibliography/chapter}
  \fi
  %-------------------------------------------------------------------------------
  \expandafter\ifx\csname MasterFile\endcsname\relax
  \end{document}
  \fi

  \fi
  %-------------------------------------------------------------------------------
  \expandafter\ifx\csname MasterFile\endcsname\relax
  \end{document}
  \fi


%===============================================================================
\pagestyle{plain}
%-------------------------------------------------------------------------------
\expandafter\ifx\csname MasterFile\endcsname\relax
	\def\SubFile{hoge}
	\documentclass[a4j,12pt,twoside,openany]{jreport}
%\nofiles %tocファイルを更新させない
%\documentclass[12pt,a4j,twoside,openany]{jsbook}
\usepackage[dvipdfmx]{graphicx}
\usepackage{../dspc} % ベースラインスキップの指定
\usepackage{../slashbox} % 表に斜線を入れる
%\usepackage{../mediabb}
\usepackage{fancyvrb} % Verbatim環境
\usepackage{fancyhdr} % Headerの下線付き章見出し
\usepackage{here} % float[H]
\usepackage{multirow}
\usepackage{hhline} % 表の罫線の角を美しくする
\usepackage{amsmath} %コレがないとcasesが動かない
\usepackage{amsfonts} % 数学用フォント
\usepackage{bm} % 数式環境での bold
\usepackage{algorithm}
\usepackage{algorithmicx}
\usepackage[noend]{algpseudocode}
\usepackage[flushleft]{threeparttable} % 脚注付きテーブル
\usepackage{enumitem}
\usepackage{comment}
\usepackage{fancybox}
%\usepackage{csvsimple,booktabs,siunitx}
%\usepackage{filecontents}


\setlength{\evensidemargin}{5pt}
\setlength{\oddsidemargin}{40pt}
%\setlength{\headheight}{16.5pt}
%%\setlength{\headheight}{30pt}
\setcounter{secnumdepth}{3}
\setlist[description]{leftmargin=2\parindent,labelindent=\parindent}

\makeatletter
\def\@makechapterhead#1{%
	\vspace*{50\p@}%
	{
		\parindent \z@ \raggedright \normalfont
		\ifnum \c@secnumdepth >\m@ne
		% \if@mainmatter
			\huge\bfseries\@chapapp\thechapter\@chappos
			\par\nobreak
			\vskip 20\p@
		% \fi
		\fi
		\interlinepenalty\@M
		\Huge\bfseries #1\par\nobreak
		\vskip 40\p@
	}
}

%新しいコマンド定義
\newcounter{linenumber}
\newenvironment{listing}{%
  \begin{list}{%
    \small\arabic{linenumber}:}{%
      \usecounter{linenumber}%
      \setlength{\baselineskip}{18pt}%
      \setlength{\itemsep}{0pt}%
      \setlength{\parsep}{0pt}}}%
 {\end{list}}
\newcommand{\figcaption}[1]{\def\@captype{figure}\caption{#1}}
\newcommand{\tblcaption}[1]{\def\@captype{table}\caption{#1}}
\newcommand{\norm}[1]{\left\| #1 \right\|}
\newcommand{\cc}[1]{\multicolumn{1}{|c|}{#1}}
\newcommand{\circled}[1]{\raisebox{.5pt}{\textcircled{\raisebox{-.9pt} {#1}}}}
\newcommand{\specialcell}[2][c]{%
  \begin{tabular}[#1]{@{}c@{}}#2\end{tabular}}
\makeatother
%===============================================================================
\expandafter\ifx\csname SubFile\endcsname\relax
\begin{document}
\def\MasterFile{hoge}
%-------------------------------------------------------------------------------
%\maketitle
\thispagestyle{empty}
\input{../hyoushi/title}
%\addcontentsline{toc}{chapter}{表紙}
\thispagestyle{empty}
\mbox{}\newpage
%===============================================================================
%\frontmatter
%===============================================================================
%\mainmatter
%-------------------------------------------------------------------------------
\pagenumbering{arabic}
\cleardoublepage
\input{../0.Abstract/chapter}
%-------------------------------------------------------------------------------
\clearpage
\addcontentsline{toc}{chapter}{目次}
\tableofcontents

\clearpage
\addcontentsline{toc}{chapter}{図目次}
\listoffigures

\clearpage
\addcontentsline{toc}{chapter}{表目次}
\listoftables

%-------------------------------------------------------------------------------

%=====================
\pagestyle{fancy} % Headerをつける
\renewcommand{\sectionmark}[1]{\markright{\thesection\ \ \ #1}}
\renewcommand{\chaptermark}[1]{\markboth{#1}{}}
\lhead{}
\chead{}
\lfoot{}
\rfoot{}%-------------------------------------------------------------------------------
\input{../1.Introduction/chapter}
%-------------------------------------------------------------------------------
\input{../2.Related_Work/chapter}
%-------------------------------------------------------------------------------
\input{../3.The_Model/chapter}
%-------------------------------------------------------------------------------
\input{../4.Implementation/chapter}
%-------------------------------------------------------------------------------
\input{../5.Experiments/chapter}
%-------------------------------------------------------------------------------
\input{../6.Conclusion/chapter}

%===============================================================================
\pagestyle{plain}
%-------------------------------------------------------------------------------
\input{../7.Acknowledgement/chapter} %謝辞
%-------------------------------------------------------------------------------
\def\BibFile{../Bibliograhoy/database2}
\input{../Bibliography/chapter} %参考文献
% %===============================================================================
\appendix
\input{../A.Mypaper/chapter} % 投稿論文リスト
\input{../B.SIG-CCI2/chapter} %
\input{../C.IJCAI-16/chapter} %
%===============================================================================
\end{document}
\fi

	\begin{document}
	\setcounter{chapter}{0}
	\fi
  %-------------------------------------------------------------------------------
\cleardoublepage
\chapter{序論}
\label{intro:chapter}
%本章では, 本研究を行なうに至った背景と目的について述べる.その後,本論文の構成について述べる.
\section{研究の背景}
\label{intro:background}
近年,Web上での大規模な議論活動が活発になっているが,現在一般的に使われている "2ちゃんねる" や "Twitter" といったシステムでは整理や収束を行うことが困難である.困難である原因として,議論の管理を行う者がいないことが挙げられる.
つまり,議論を整理・収束させるには議論のマネジメントを行う人物が必要である.
%
大規模意見集約システムCOLLAGREE\cite{collagreeTest}ではファシリテーターと呼ばれる人物が議論のマネジメントを行っている.
しかし,ファシリテーターは人間であり,長時間に渡って大人数での議論の動向をマネジメントし続けるのは困難である.
COLLAGREEで大規模な議論を収束させるためには,ファシリテーターが必要な時には画面を見るようにして,他の時は見なくても済むようにすることで画面に向き合う時間を減らす工夫があることが望ましい.ファシリテーターが画面を見るべきタイミングは議論の話題が変化したときである.以前の議論の内容から外れた発言がされた時,ファシリテーターが適切な発言をすることで,脱線や炎上を避けて議論を収束させることができる.
すなわち,ファシリテーターの代わりに自動的に議論中の話題の変化を観測することが求められている.
%
現在,COLLAGREE上で使用されている議論支援システムは「(1)投稿支援システム」と「(2)議論可視化システム」の2つに大別できる.
投稿支援システムはポイント機能やファシリテーションフレーズ簡易投稿機能のように,ユーザーが投稿をする際に何らかの補助やリアクションを行う.現行の機能では選択肢の提示に留まっており,作業量を減らすことには繋がりにくい.
一方,議論可視化システムは議論ツリーやキーワード抽出のように,ユーザーにスレッドとは異なる議論の見方を提供する.
\ref{Fig:argTree1}に議論ツリーの例を示す.
\begin{figure}[htbp]
 \begin{center}
  \includegraphics[width=\textwidth]{../images/2.Related_Work/argTree1.png}
  \caption{議論ツリー}
  \label{Fig:argTree1}
  \vspace{-10pt}
 \end{center}
\end{figure}
現行の機能では議論を見やすくすることに重点が置かれており,議論の把握の助けにはなるが画面に向き合う時間を減らすことにはなりにくい.むしろ,作業量を増やすことになり得ることもある.
従って,現行の支援機能ではファシリテーターの作業量の減少には繋がりにくい.
%
近年,自然言語処理の分野において分散表現が多くの研究で使われており,機械翻訳を始めとする単語の意味が重要となる分野で精度の向上が確認されている.分散表現を用いることで,人間に近い精度で話題の変化を観測することが可能となる.
%
以上のような背景を踏まえて,分散表現を用いて,話題の変化を観測し,話題の変化が確認された時にファシリテーターに伝えることが望ましい.
話題の変化の観測は,発言中に現れる単語の類似度の計算と見なすことができる.
分散表現を用いることで単語間の類似度を求めることができる,値が大きいほど単語がそれぞれ類似した実数ベクトルであることを表す.単語Aと単語Bの実数ベクトルが類似しているとは,単語Aと共に使われることの多い単語と単語Bと共に使われることの多い単語が多く共通していることを示す.故に,分散表現を使って単語の類似度を計算することができる.
%
発言文から単語を選ぶ際には自動要約を用いる.発言文から重要でない単語を取り除くことで関連度の計算の精度を高めることが可能となる.
要約の手法としてはokapi BM25 \cite{okapiBM25}とLexRankを組み合わせた抽出的要約手法を用いる.
\begin{comment}
%======================================= 社会的背景
2013年頃からWeb上での大規模な議論活動が活発になり,大規模な人数での議論が期待されている.
大規模な議論では意見を共有することは可能であるが,議論を整理させることや収束させることは難しい.以上から大規模意見集約システムCOLLAGREEが開発された.本システムではWeb上で適切に大規模な議論を行うことができるように議論をマネジメントするファシリテーターを導入した\cite{collagreeTest}.
過去の実験ではファシリテーターの存在が議論の集約に大きな役割を果たしていることが認識されており,大規模な議論のためにファシリテータは必要である.しかし,議論の規模に伴って議論時間が長くなる傾向があり,同時にファシリテーターは常に議論の動向を見続ける必要がある.故に,議論の規模が大きくなればなるほどファシリテーターは長時間かつ大規模な議論の動向の監視によって大きな負担がかかる.大規模な議論が増加する傾向を踏まえるとファシリテーターにかかる負担を軽減する支援が必要である.\\
以上の問題を解決するため,話題の変化を追い,重要な話題の転換点をファシリテーターの代わりに検出することが有用であると考える.必要な時にだけファシリテーターが画面を見れば良いようにすることでファシリテーターの負担軽減が期待できる.
%========================================= 現行手法問題点背景
%議論支援に関する先行研究において,既存の手法は全てが文字列を文字列のまま扱う手法である.
%既存手法は殆どがパターンマッチングと重み付けの2つに区分することができる.
%パターンマッチングでは事前に単語を登録して,単語がマッチした場合に処理を行うが,処理それぞれに対して単語を登録しなければならず手間が膨大になってしまう.また,単語の意味が考慮されておらず,手作業で登録を行うので登録漏れがあった場合に単語の意味に関係なく処理を行うことが不可能となってしまう.
%重み付けは単語の出現頻度や文章の長さを使用して単語・文章に順位を付ける手法で必ずしも単語の登録が必要でないため多くの研究で使用されている.
%しかし,重み付けもまた単語を文字列のまま扱っており,意味までは考慮されていない.故に ,人間なら対応できる似た単語でも1文字違うだけで対処が困難となる.
議論支援に関する先行研究においてファシリテーターに対する支援を目的としたものは無く,殆どが議論の活性化や可視化を目的としている.
%=================================新手法
近年,自然言語処理の分野において分散表現が多くの研究で使われており.分散表現は文字列である単語を辞書データを使用して実数ベクトルへと変換する.辞書データにない単語には対応できないが,多様な処理を1つの辞書データで行うことができる.また,実数ベクトルの各数値が単語の意味を表現するものとなっており,数値を使用して処理を行うことができる.
分散表現を用いることで既存手法より人間の感覚に近しい処理を行うことができる.
%=================================
以上のような背景を踏まえて,分散表現を用いてファシリテーターの代わりに話題の変化を判定し,知らせることを目指す.
話題転換の検出は発言同士の近さ,すなわち発言に含まれる単語意味の近さと見ることができる.
分散表現ではベクトル同士の内積計算を行うことで単語同士の意味の近さを計算することができる.
また,分散表現を使用することで機械翻訳を始めとする複数の分野で精度の向上が確認されている.
\end{comment}
\section{研究の目的}
\label{intro:taget}
本論文では,分散表現を用いて議論中での発言に含まれる単語の関連度を計算し,話題の変化を観測する手法を提案する.

\section{本論文の構成}
本論文の構成を以下に示す.
\ref{relwork:chapter} 章では要約手法に関する研究と,分散表現に関する先行研究を紹介する.
次に,\ref{model:chapter}章では発言の要約手法の説明を行い,\ref{impl:chapter}章では分散表現を用いた単語集合間の関連度計算について説明する.
そして,\ref{exp:chapter}章では話題転換点の検出の評価実験について説明する.
最後に\ref{con:chapter}章で本論文のまとめと考察を示す.

 %-------------------------------------------------------------------------------
 \expandafter\ifx\csname MasterFile\endcsname\relax
	\def\BibFile{hoge}
	\expandafter\ifx\csname MasterFile\endcsname\relax
	\def\SubFile{hoge}
	\input{../thesis/thesis}
	\begin{document}
	\setcounter{chapter}{0}
	\fi
  %-------------------------------------------------------------------------------
\cleardoublepage
\chapter{序論}
\label{intro:chapter}
%本章では, 本研究を行なうに至った背景と目的について述べる.その後,本論文の構成について述べる.
\section{研究の背景}
\label{intro:background}
近年,Web上での大規模な議論活動が活発になっているが,現在一般的に使われている "2ちゃんねる" や "Twitter" といったシステムでは整理や収束を行うことが困難である.困難である原因として,議論の管理を行う者がいないことが挙げられる.
つまり,議論を整理・収束させるには議論のマネジメントを行う人物が必要である.
%
大規模意見集約システムCOLLAGREE\cite{collagreeTest}ではファシリテーターと呼ばれる人物が議論のマネジメントを行っている.
しかし,ファシリテーターは人間であり,長時間に渡って大人数での議論の動向をマネジメントし続けるのは困難である.
COLLAGREEで大規模な議論を収束させるためには,ファシリテーターが必要な時には画面を見るようにして,他の時は見なくても済むようにすることで画面に向き合う時間を減らす工夫があることが望ましい.ファシリテーターが画面を見るべきタイミングは議論の話題が変化したときである.以前の議論の内容から外れた発言がされた時,ファシリテーターが適切な発言をすることで,脱線や炎上を避けて議論を収束させることができる.
すなわち,ファシリテーターの代わりに自動的に議論中の話題の変化を観測することが求められている.
%
現在,COLLAGREE上で使用されている議論支援システムは「(1)投稿支援システム」と「(2)議論可視化システム」の2つに大別できる.
投稿支援システムはポイント機能やファシリテーションフレーズ簡易投稿機能のように,ユーザーが投稿をする際に何らかの補助やリアクションを行う.現行の機能では選択肢の提示に留まっており,作業量を減らすことには繋がりにくい.
一方,議論可視化システムは議論ツリーやキーワード抽出のように,ユーザーにスレッドとは異なる議論の見方を提供する.
\ref{Fig:argTree1}に議論ツリーの例を示す.
\begin{figure}[htbp]
 \begin{center}
  \includegraphics[width=\textwidth]{../images/2.Related_Work/argTree1.png}
  \caption{議論ツリー}
  \label{Fig:argTree1}
  \vspace{-10pt}
 \end{center}
\end{figure}
現行の機能では議論を見やすくすることに重点が置かれており,議論の把握の助けにはなるが画面に向き合う時間を減らすことにはなりにくい.むしろ,作業量を増やすことになり得ることもある.
従って,現行の支援機能ではファシリテーターの作業量の減少には繋がりにくい.
%
近年,自然言語処理の分野において分散表現が多くの研究で使われており,機械翻訳を始めとする単語の意味が重要となる分野で精度の向上が確認されている.分散表現を用いることで,人間に近い精度で話題の変化を観測することが可能となる.
%
以上のような背景を踏まえて,分散表現を用いて,話題の変化を観測し,話題の変化が確認された時にファシリテーターに伝えることが望ましい.
話題の変化の観測は,発言中に現れる単語の類似度の計算と見なすことができる.
分散表現を用いることで単語間の類似度を求めることができる,値が大きいほど単語がそれぞれ類似した実数ベクトルであることを表す.単語Aと単語Bの実数ベクトルが類似しているとは,単語Aと共に使われることの多い単語と単語Bと共に使われることの多い単語が多く共通していることを示す.故に,分散表現を使って単語の類似度を計算することができる.
%
発言文から単語を選ぶ際には自動要約を用いる.発言文から重要でない単語を取り除くことで関連度の計算の精度を高めることが可能となる.
要約の手法としてはokapi BM25 \cite{okapiBM25}とLexRankを組み合わせた抽出的要約手法を用いる.
\begin{comment}
%======================================= 社会的背景
2013年頃からWeb上での大規模な議論活動が活発になり,大規模な人数での議論が期待されている.
大規模な議論では意見を共有することは可能であるが,議論を整理させることや収束させることは難しい.以上から大規模意見集約システムCOLLAGREEが開発された.本システムではWeb上で適切に大規模な議論を行うことができるように議論をマネジメントするファシリテーターを導入した\cite{collagreeTest}.
過去の実験ではファシリテーターの存在が議論の集約に大きな役割を果たしていることが認識されており,大規模な議論のためにファシリテータは必要である.しかし,議論の規模に伴って議論時間が長くなる傾向があり,同時にファシリテーターは常に議論の動向を見続ける必要がある.故に,議論の規模が大きくなればなるほどファシリテーターは長時間かつ大規模な議論の動向の監視によって大きな負担がかかる.大規模な議論が増加する傾向を踏まえるとファシリテーターにかかる負担を軽減する支援が必要である.\\
以上の問題を解決するため,話題の変化を追い,重要な話題の転換点をファシリテーターの代わりに検出することが有用であると考える.必要な時にだけファシリテーターが画面を見れば良いようにすることでファシリテーターの負担軽減が期待できる.
%========================================= 現行手法問題点背景
%議論支援に関する先行研究において,既存の手法は全てが文字列を文字列のまま扱う手法である.
%既存手法は殆どがパターンマッチングと重み付けの2つに区分することができる.
%パターンマッチングでは事前に単語を登録して,単語がマッチした場合に処理を行うが,処理それぞれに対して単語を登録しなければならず手間が膨大になってしまう.また,単語の意味が考慮されておらず,手作業で登録を行うので登録漏れがあった場合に単語の意味に関係なく処理を行うことが不可能となってしまう.
%重み付けは単語の出現頻度や文章の長さを使用して単語・文章に順位を付ける手法で必ずしも単語の登録が必要でないため多くの研究で使用されている.
%しかし,重み付けもまた単語を文字列のまま扱っており,意味までは考慮されていない.故に ,人間なら対応できる似た単語でも1文字違うだけで対処が困難となる.
議論支援に関する先行研究においてファシリテーターに対する支援を目的としたものは無く,殆どが議論の活性化や可視化を目的としている.
%=================================新手法
近年,自然言語処理の分野において分散表現が多くの研究で使われており.分散表現は文字列である単語を辞書データを使用して実数ベクトルへと変換する.辞書データにない単語には対応できないが,多様な処理を1つの辞書データで行うことができる.また,実数ベクトルの各数値が単語の意味を表現するものとなっており,数値を使用して処理を行うことができる.
分散表現を用いることで既存手法より人間の感覚に近しい処理を行うことができる.
%=================================
以上のような背景を踏まえて,分散表現を用いてファシリテーターの代わりに話題の変化を判定し,知らせることを目指す.
話題転換の検出は発言同士の近さ,すなわち発言に含まれる単語意味の近さと見ることができる.
分散表現ではベクトル同士の内積計算を行うことで単語同士の意味の近さを計算することができる.
また,分散表現を使用することで機械翻訳を始めとする複数の分野で精度の向上が確認されている.
\end{comment}
\section{研究の目的}
\label{intro:taget}
本論文では,分散表現を用いて議論中での発言に含まれる単語の関連度を計算し,話題の変化を観測する手法を提案する.

\section{本論文の構成}
本論文の構成を以下に示す.
\ref{relwork:chapter} 章では要約手法に関する研究と,分散表現に関する先行研究を紹介する.
次に,\ref{model:chapter}章では発言の要約手法の説明を行い,\ref{impl:chapter}章では分散表現を用いた単語集合間の関連度計算について説明する.
そして,\ref{exp:chapter}章では話題転換点の検出の評価実験について説明する.
最後に\ref{con:chapter}章で本論文のまとめと考察を示す.

 %-------------------------------------------------------------------------------
 \expandafter\ifx\csname MasterFile\endcsname\relax
	\def\BibFile{hoge}
	\input{../Bibliography/chapter}
  \fi
  %-------------------------------------------------------------------------------
  \expandafter\ifx\csname MasterFile\endcsname\relax
  \end{document}
  \fi

  \fi
  %-------------------------------------------------------------------------------
  \expandafter\ifx\csname MasterFile\endcsname\relax
  \end{document}
  \fi
 %謝辞
%-------------------------------------------------------------------------------
\def\BibFile{../Bibliograhoy/database2}
\expandafter\ifx\csname MasterFile\endcsname\relax
	\def\SubFile{hoge}
	\documentclass[a4j,12pt,twoside,openany]{jreport}
%\nofiles %tocファイルを更新させない
%\documentclass[12pt,a4j,twoside,openany]{jsbook}
\usepackage[dvipdfmx]{graphicx}
\usepackage{../dspc} % ベースラインスキップの指定
\usepackage{../slashbox} % 表に斜線を入れる
%\usepackage{../mediabb}
\usepackage{fancyvrb} % Verbatim環境
\usepackage{fancyhdr} % Headerの下線付き章見出し
\usepackage{here} % float[H]
\usepackage{multirow}
\usepackage{hhline} % 表の罫線の角を美しくする
\usepackage{amsmath} %コレがないとcasesが動かない
\usepackage{amsfonts} % 数学用フォント
\usepackage{bm} % 数式環境での bold
\usepackage{algorithm}
\usepackage{algorithmicx}
\usepackage[noend]{algpseudocode}
\usepackage[flushleft]{threeparttable} % 脚注付きテーブル
\usepackage{enumitem}
\usepackage{comment}
\usepackage{fancybox}
%\usepackage{csvsimple,booktabs,siunitx}
%\usepackage{filecontents}


\setlength{\evensidemargin}{5pt}
\setlength{\oddsidemargin}{40pt}
%\setlength{\headheight}{16.5pt}
%%\setlength{\headheight}{30pt}
\setcounter{secnumdepth}{3}
\setlist[description]{leftmargin=2\parindent,labelindent=\parindent}

\makeatletter
\def\@makechapterhead#1{%
	\vspace*{50\p@}%
	{
		\parindent \z@ \raggedright \normalfont
		\ifnum \c@secnumdepth >\m@ne
		% \if@mainmatter
			\huge\bfseries\@chapapp\thechapter\@chappos
			\par\nobreak
			\vskip 20\p@
		% \fi
		\fi
		\interlinepenalty\@M
		\Huge\bfseries #1\par\nobreak
		\vskip 40\p@
	}
}

%新しいコマンド定義
\newcounter{linenumber}
\newenvironment{listing}{%
  \begin{list}{%
    \small\arabic{linenumber}:}{%
      \usecounter{linenumber}%
      \setlength{\baselineskip}{18pt}%
      \setlength{\itemsep}{0pt}%
      \setlength{\parsep}{0pt}}}%
 {\end{list}}
\newcommand{\figcaption}[1]{\def\@captype{figure}\caption{#1}}
\newcommand{\tblcaption}[1]{\def\@captype{table}\caption{#1}}
\newcommand{\norm}[1]{\left\| #1 \right\|}
\newcommand{\cc}[1]{\multicolumn{1}{|c|}{#1}}
\newcommand{\circled}[1]{\raisebox{.5pt}{\textcircled{\raisebox{-.9pt} {#1}}}}
\newcommand{\specialcell}[2][c]{%
  \begin{tabular}[#1]{@{}c@{}}#2\end{tabular}}
\makeatother
%===============================================================================
\expandafter\ifx\csname SubFile\endcsname\relax
\begin{document}
\def\MasterFile{hoge}
%-------------------------------------------------------------------------------
%\maketitle
\thispagestyle{empty}
\input{../hyoushi/title}
%\addcontentsline{toc}{chapter}{表紙}
\thispagestyle{empty}
\mbox{}\newpage
%===============================================================================
%\frontmatter
%===============================================================================
%\mainmatter
%-------------------------------------------------------------------------------
\pagenumbering{arabic}
\cleardoublepage
\input{../0.Abstract/chapter}
%-------------------------------------------------------------------------------
\clearpage
\addcontentsline{toc}{chapter}{目次}
\tableofcontents

\clearpage
\addcontentsline{toc}{chapter}{図目次}
\listoffigures

\clearpage
\addcontentsline{toc}{chapter}{表目次}
\listoftables

%-------------------------------------------------------------------------------

%=====================
\pagestyle{fancy} % Headerをつける
\renewcommand{\sectionmark}[1]{\markright{\thesection\ \ \ #1}}
\renewcommand{\chaptermark}[1]{\markboth{#1}{}}
\lhead{}
\chead{}
\lfoot{}
\rfoot{}%-------------------------------------------------------------------------------
\input{../1.Introduction/chapter}
%-------------------------------------------------------------------------------
\input{../2.Related_Work/chapter}
%-------------------------------------------------------------------------------
\input{../3.The_Model/chapter}
%-------------------------------------------------------------------------------
\input{../4.Implementation/chapter}
%-------------------------------------------------------------------------------
\input{../5.Experiments/chapter}
%-------------------------------------------------------------------------------
\input{../6.Conclusion/chapter}

%===============================================================================
\pagestyle{plain}
%-------------------------------------------------------------------------------
\input{../7.Acknowledgement/chapter} %謝辞
%-------------------------------------------------------------------------------
\def\BibFile{../Bibliograhoy/database2}
\input{../Bibliography/chapter} %参考文献
% %===============================================================================
\appendix
\input{../A.Mypaper/chapter} % 投稿論文リスト
\input{../B.SIG-CCI2/chapter} %
\input{../C.IJCAI-16/chapter} %
%===============================================================================
\end{document}
\fi

	\begin{document}
	\setcounter{chapter}{0}
	\fi
  %-------------------------------------------------------------------------------
\cleardoublepage
\chapter{序論}
\label{intro:chapter}
%本章では, 本研究を行なうに至った背景と目的について述べる.その後,本論文の構成について述べる.
\section{研究の背景}
\label{intro:background}
近年,Web上での大規模な議論活動が活発になっているが,現在一般的に使われている "2ちゃんねる" や "Twitter" といったシステムでは整理や収束を行うことが困難である.困難である原因として,議論の管理を行う者がいないことが挙げられる.
つまり,議論を整理・収束させるには議論のマネジメントを行う人物が必要である.
%
大規模意見集約システムCOLLAGREE\cite{collagreeTest}ではファシリテーターと呼ばれる人物が議論のマネジメントを行っている.
しかし,ファシリテーターは人間であり,長時間に渡って大人数での議論の動向をマネジメントし続けるのは困難である.
COLLAGREEで大規模な議論を収束させるためには,ファシリテーターが必要な時には画面を見るようにして,他の時は見なくても済むようにすることで画面に向き合う時間を減らす工夫があることが望ましい.ファシリテーターが画面を見るべきタイミングは議論の話題が変化したときである.以前の議論の内容から外れた発言がされた時,ファシリテーターが適切な発言をすることで,脱線や炎上を避けて議論を収束させることができる.
すなわち,ファシリテーターの代わりに自動的に議論中の話題の変化を観測することが求められている.
%
現在,COLLAGREE上で使用されている議論支援システムは「(1)投稿支援システム」と「(2)議論可視化システム」の2つに大別できる.
投稿支援システムはポイント機能やファシリテーションフレーズ簡易投稿機能のように,ユーザーが投稿をする際に何らかの補助やリアクションを行う.現行の機能では選択肢の提示に留まっており,作業量を減らすことには繋がりにくい.
一方,議論可視化システムは議論ツリーやキーワード抽出のように,ユーザーにスレッドとは異なる議論の見方を提供する.
\ref{Fig:argTree1}に議論ツリーの例を示す.
\begin{figure}[htbp]
 \begin{center}
  \includegraphics[width=\textwidth]{../images/2.Related_Work/argTree1.png}
  \caption{議論ツリー}
  \label{Fig:argTree1}
  \vspace{-10pt}
 \end{center}
\end{figure}
現行の機能では議論を見やすくすることに重点が置かれており,議論の把握の助けにはなるが画面に向き合う時間を減らすことにはなりにくい.むしろ,作業量を増やすことになり得ることもある.
従って,現行の支援機能ではファシリテーターの作業量の減少には繋がりにくい.
%
近年,自然言語処理の分野において分散表現が多くの研究で使われており,機械翻訳を始めとする単語の意味が重要となる分野で精度の向上が確認されている.分散表現を用いることで,人間に近い精度で話題の変化を観測することが可能となる.
%
以上のような背景を踏まえて,分散表現を用いて,話題の変化を観測し,話題の変化が確認された時にファシリテーターに伝えることが望ましい.
話題の変化の観測は,発言中に現れる単語の類似度の計算と見なすことができる.
分散表現を用いることで単語間の類似度を求めることができる,値が大きいほど単語がそれぞれ類似した実数ベクトルであることを表す.単語Aと単語Bの実数ベクトルが類似しているとは,単語Aと共に使われることの多い単語と単語Bと共に使われることの多い単語が多く共通していることを示す.故に,分散表現を使って単語の類似度を計算することができる.
%
発言文から単語を選ぶ際には自動要約を用いる.発言文から重要でない単語を取り除くことで関連度の計算の精度を高めることが可能となる.
要約の手法としてはokapi BM25 \cite{okapiBM25}とLexRankを組み合わせた抽出的要約手法を用いる.
\begin{comment}
%======================================= 社会的背景
2013年頃からWeb上での大規模な議論活動が活発になり,大規模な人数での議論が期待されている.
大規模な議論では意見を共有することは可能であるが,議論を整理させることや収束させることは難しい.以上から大規模意見集約システムCOLLAGREEが開発された.本システムではWeb上で適切に大規模な議論を行うことができるように議論をマネジメントするファシリテーターを導入した\cite{collagreeTest}.
過去の実験ではファシリテーターの存在が議論の集約に大きな役割を果たしていることが認識されており,大規模な議論のためにファシリテータは必要である.しかし,議論の規模に伴って議論時間が長くなる傾向があり,同時にファシリテーターは常に議論の動向を見続ける必要がある.故に,議論の規模が大きくなればなるほどファシリテーターは長時間かつ大規模な議論の動向の監視によって大きな負担がかかる.大規模な議論が増加する傾向を踏まえるとファシリテーターにかかる負担を軽減する支援が必要である.\\
以上の問題を解決するため,話題の変化を追い,重要な話題の転換点をファシリテーターの代わりに検出することが有用であると考える.必要な時にだけファシリテーターが画面を見れば良いようにすることでファシリテーターの負担軽減が期待できる.
%========================================= 現行手法問題点背景
%議論支援に関する先行研究において,既存の手法は全てが文字列を文字列のまま扱う手法である.
%既存手法は殆どがパターンマッチングと重み付けの2つに区分することができる.
%パターンマッチングでは事前に単語を登録して,単語がマッチした場合に処理を行うが,処理それぞれに対して単語を登録しなければならず手間が膨大になってしまう.また,単語の意味が考慮されておらず,手作業で登録を行うので登録漏れがあった場合に単語の意味に関係なく処理を行うことが不可能となってしまう.
%重み付けは単語の出現頻度や文章の長さを使用して単語・文章に順位を付ける手法で必ずしも単語の登録が必要でないため多くの研究で使用されている.
%しかし,重み付けもまた単語を文字列のまま扱っており,意味までは考慮されていない.故に ,人間なら対応できる似た単語でも1文字違うだけで対処が困難となる.
議論支援に関する先行研究においてファシリテーターに対する支援を目的としたものは無く,殆どが議論の活性化や可視化を目的としている.
%=================================新手法
近年,自然言語処理の分野において分散表現が多くの研究で使われており.分散表現は文字列である単語を辞書データを使用して実数ベクトルへと変換する.辞書データにない単語には対応できないが,多様な処理を1つの辞書データで行うことができる.また,実数ベクトルの各数値が単語の意味を表現するものとなっており,数値を使用して処理を行うことができる.
分散表現を用いることで既存手法より人間の感覚に近しい処理を行うことができる.
%=================================
以上のような背景を踏まえて,分散表現を用いてファシリテーターの代わりに話題の変化を判定し,知らせることを目指す.
話題転換の検出は発言同士の近さ,すなわち発言に含まれる単語意味の近さと見ることができる.
分散表現ではベクトル同士の内積計算を行うことで単語同士の意味の近さを計算することができる.
また,分散表現を使用することで機械翻訳を始めとする複数の分野で精度の向上が確認されている.
\end{comment}
\section{研究の目的}
\label{intro:taget}
本論文では,分散表現を用いて議論中での発言に含まれる単語の関連度を計算し,話題の変化を観測する手法を提案する.

\section{本論文の構成}
本論文の構成を以下に示す.
\ref{relwork:chapter} 章では要約手法に関する研究と,分散表現に関する先行研究を紹介する.
次に,\ref{model:chapter}章では発言の要約手法の説明を行い,\ref{impl:chapter}章では分散表現を用いた単語集合間の関連度計算について説明する.
そして,\ref{exp:chapter}章では話題転換点の検出の評価実験について説明する.
最後に\ref{con:chapter}章で本論文のまとめと考察を示す.

 %-------------------------------------------------------------------------------
 \expandafter\ifx\csname MasterFile\endcsname\relax
	\def\BibFile{hoge}
	\expandafter\ifx\csname MasterFile\endcsname\relax
	\def\SubFile{hoge}
	\input{../thesis/thesis}
	\begin{document}
	\setcounter{chapter}{0}
	\fi
  %-------------------------------------------------------------------------------
\cleardoublepage
\chapter{序論}
\label{intro:chapter}
%本章では, 本研究を行なうに至った背景と目的について述べる.その後,本論文の構成について述べる.
\section{研究の背景}
\label{intro:background}
近年,Web上での大規模な議論活動が活発になっているが,現在一般的に使われている "2ちゃんねる" や "Twitter" といったシステムでは整理や収束を行うことが困難である.困難である原因として,議論の管理を行う者がいないことが挙げられる.
つまり,議論を整理・収束させるには議論のマネジメントを行う人物が必要である.
%
大規模意見集約システムCOLLAGREE\cite{collagreeTest}ではファシリテーターと呼ばれる人物が議論のマネジメントを行っている.
しかし,ファシリテーターは人間であり,長時間に渡って大人数での議論の動向をマネジメントし続けるのは困難である.
COLLAGREEで大規模な議論を収束させるためには,ファシリテーターが必要な時には画面を見るようにして,他の時は見なくても済むようにすることで画面に向き合う時間を減らす工夫があることが望ましい.ファシリテーターが画面を見るべきタイミングは議論の話題が変化したときである.以前の議論の内容から外れた発言がされた時,ファシリテーターが適切な発言をすることで,脱線や炎上を避けて議論を収束させることができる.
すなわち,ファシリテーターの代わりに自動的に議論中の話題の変化を観測することが求められている.
%
現在,COLLAGREE上で使用されている議論支援システムは「(1)投稿支援システム」と「(2)議論可視化システム」の2つに大別できる.
投稿支援システムはポイント機能やファシリテーションフレーズ簡易投稿機能のように,ユーザーが投稿をする際に何らかの補助やリアクションを行う.現行の機能では選択肢の提示に留まっており,作業量を減らすことには繋がりにくい.
一方,議論可視化システムは議論ツリーやキーワード抽出のように,ユーザーにスレッドとは異なる議論の見方を提供する.
\ref{Fig:argTree1}に議論ツリーの例を示す.
\begin{figure}[htbp]
 \begin{center}
  \includegraphics[width=\textwidth]{../images/2.Related_Work/argTree1.png}
  \caption{議論ツリー}
  \label{Fig:argTree1}
  \vspace{-10pt}
 \end{center}
\end{figure}
現行の機能では議論を見やすくすることに重点が置かれており,議論の把握の助けにはなるが画面に向き合う時間を減らすことにはなりにくい.むしろ,作業量を増やすことになり得ることもある.
従って,現行の支援機能ではファシリテーターの作業量の減少には繋がりにくい.
%
近年,自然言語処理の分野において分散表現が多くの研究で使われており,機械翻訳を始めとする単語の意味が重要となる分野で精度の向上が確認されている.分散表現を用いることで,人間に近い精度で話題の変化を観測することが可能となる.
%
以上のような背景を踏まえて,分散表現を用いて,話題の変化を観測し,話題の変化が確認された時にファシリテーターに伝えることが望ましい.
話題の変化の観測は,発言中に現れる単語の類似度の計算と見なすことができる.
分散表現を用いることで単語間の類似度を求めることができる,値が大きいほど単語がそれぞれ類似した実数ベクトルであることを表す.単語Aと単語Bの実数ベクトルが類似しているとは,単語Aと共に使われることの多い単語と単語Bと共に使われることの多い単語が多く共通していることを示す.故に,分散表現を使って単語の類似度を計算することができる.
%
発言文から単語を選ぶ際には自動要約を用いる.発言文から重要でない単語を取り除くことで関連度の計算の精度を高めることが可能となる.
要約の手法としてはokapi BM25 \cite{okapiBM25}とLexRankを組み合わせた抽出的要約手法を用いる.
\begin{comment}
%======================================= 社会的背景
2013年頃からWeb上での大規模な議論活動が活発になり,大規模な人数での議論が期待されている.
大規模な議論では意見を共有することは可能であるが,議論を整理させることや収束させることは難しい.以上から大規模意見集約システムCOLLAGREEが開発された.本システムではWeb上で適切に大規模な議論を行うことができるように議論をマネジメントするファシリテーターを導入した\cite{collagreeTest}.
過去の実験ではファシリテーターの存在が議論の集約に大きな役割を果たしていることが認識されており,大規模な議論のためにファシリテータは必要である.しかし,議論の規模に伴って議論時間が長くなる傾向があり,同時にファシリテーターは常に議論の動向を見続ける必要がある.故に,議論の規模が大きくなればなるほどファシリテーターは長時間かつ大規模な議論の動向の監視によって大きな負担がかかる.大規模な議論が増加する傾向を踏まえるとファシリテーターにかかる負担を軽減する支援が必要である.\\
以上の問題を解決するため,話題の変化を追い,重要な話題の転換点をファシリテーターの代わりに検出することが有用であると考える.必要な時にだけファシリテーターが画面を見れば良いようにすることでファシリテーターの負担軽減が期待できる.
%========================================= 現行手法問題点背景
%議論支援に関する先行研究において,既存の手法は全てが文字列を文字列のまま扱う手法である.
%既存手法は殆どがパターンマッチングと重み付けの2つに区分することができる.
%パターンマッチングでは事前に単語を登録して,単語がマッチした場合に処理を行うが,処理それぞれに対して単語を登録しなければならず手間が膨大になってしまう.また,単語の意味が考慮されておらず,手作業で登録を行うので登録漏れがあった場合に単語の意味に関係なく処理を行うことが不可能となってしまう.
%重み付けは単語の出現頻度や文章の長さを使用して単語・文章に順位を付ける手法で必ずしも単語の登録が必要でないため多くの研究で使用されている.
%しかし,重み付けもまた単語を文字列のまま扱っており,意味までは考慮されていない.故に ,人間なら対応できる似た単語でも1文字違うだけで対処が困難となる.
議論支援に関する先行研究においてファシリテーターに対する支援を目的としたものは無く,殆どが議論の活性化や可視化を目的としている.
%=================================新手法
近年,自然言語処理の分野において分散表現が多くの研究で使われており.分散表現は文字列である単語を辞書データを使用して実数ベクトルへと変換する.辞書データにない単語には対応できないが,多様な処理を1つの辞書データで行うことができる.また,実数ベクトルの各数値が単語の意味を表現するものとなっており,数値を使用して処理を行うことができる.
分散表現を用いることで既存手法より人間の感覚に近しい処理を行うことができる.
%=================================
以上のような背景を踏まえて,分散表現を用いてファシリテーターの代わりに話題の変化を判定し,知らせることを目指す.
話題転換の検出は発言同士の近さ,すなわち発言に含まれる単語意味の近さと見ることができる.
分散表現ではベクトル同士の内積計算を行うことで単語同士の意味の近さを計算することができる.
また,分散表現を使用することで機械翻訳を始めとする複数の分野で精度の向上が確認されている.
\end{comment}
\section{研究の目的}
\label{intro:taget}
本論文では,分散表現を用いて議論中での発言に含まれる単語の関連度を計算し,話題の変化を観測する手法を提案する.

\section{本論文の構成}
本論文の構成を以下に示す.
\ref{relwork:chapter} 章では要約手法に関する研究と,分散表現に関する先行研究を紹介する.
次に,\ref{model:chapter}章では発言の要約手法の説明を行い,\ref{impl:chapter}章では分散表現を用いた単語集合間の関連度計算について説明する.
そして,\ref{exp:chapter}章では話題転換点の検出の評価実験について説明する.
最後に\ref{con:chapter}章で本論文のまとめと考察を示す.

 %-------------------------------------------------------------------------------
 \expandafter\ifx\csname MasterFile\endcsname\relax
	\def\BibFile{hoge}
	\input{../Bibliography/chapter}
  \fi
  %-------------------------------------------------------------------------------
  \expandafter\ifx\csname MasterFile\endcsname\relax
  \end{document}
  \fi

  \fi
  %-------------------------------------------------------------------------------
  \expandafter\ifx\csname MasterFile\endcsname\relax
  \end{document}
  \fi
 %参考文献
% %===============================================================================
\appendix
\expandafter\ifx\csname MasterFile\endcsname\relax
	\def\SubFile{hoge}
	\documentclass[a4j,12pt,twoside,openany]{jreport}
%\nofiles %tocファイルを更新させない
%\documentclass[12pt,a4j,twoside,openany]{jsbook}
\usepackage[dvipdfmx]{graphicx}
\usepackage{../dspc} % ベースラインスキップの指定
\usepackage{../slashbox} % 表に斜線を入れる
%\usepackage{../mediabb}
\usepackage{fancyvrb} % Verbatim環境
\usepackage{fancyhdr} % Headerの下線付き章見出し
\usepackage{here} % float[H]
\usepackage{multirow}
\usepackage{hhline} % 表の罫線の角を美しくする
\usepackage{amsmath} %コレがないとcasesが動かない
\usepackage{amsfonts} % 数学用フォント
\usepackage{bm} % 数式環境での bold
\usepackage{algorithm}
\usepackage{algorithmicx}
\usepackage[noend]{algpseudocode}
\usepackage[flushleft]{threeparttable} % 脚注付きテーブル
\usepackage{enumitem}
\usepackage{comment}
\usepackage{fancybox}
%\usepackage{csvsimple,booktabs,siunitx}
%\usepackage{filecontents}


\setlength{\evensidemargin}{5pt}
\setlength{\oddsidemargin}{40pt}
%\setlength{\headheight}{16.5pt}
%%\setlength{\headheight}{30pt}
\setcounter{secnumdepth}{3}
\setlist[description]{leftmargin=2\parindent,labelindent=\parindent}

\makeatletter
\def\@makechapterhead#1{%
	\vspace*{50\p@}%
	{
		\parindent \z@ \raggedright \normalfont
		\ifnum \c@secnumdepth >\m@ne
		% \if@mainmatter
			\huge\bfseries\@chapapp\thechapter\@chappos
			\par\nobreak
			\vskip 20\p@
		% \fi
		\fi
		\interlinepenalty\@M
		\Huge\bfseries #1\par\nobreak
		\vskip 40\p@
	}
}

%新しいコマンド定義
\newcounter{linenumber}
\newenvironment{listing}{%
  \begin{list}{%
    \small\arabic{linenumber}:}{%
      \usecounter{linenumber}%
      \setlength{\baselineskip}{18pt}%
      \setlength{\itemsep}{0pt}%
      \setlength{\parsep}{0pt}}}%
 {\end{list}}
\newcommand{\figcaption}[1]{\def\@captype{figure}\caption{#1}}
\newcommand{\tblcaption}[1]{\def\@captype{table}\caption{#1}}
\newcommand{\norm}[1]{\left\| #1 \right\|}
\newcommand{\cc}[1]{\multicolumn{1}{|c|}{#1}}
\newcommand{\circled}[1]{\raisebox{.5pt}{\textcircled{\raisebox{-.9pt} {#1}}}}
\newcommand{\specialcell}[2][c]{%
  \begin{tabular}[#1]{@{}c@{}}#2\end{tabular}}
\makeatother
%===============================================================================
\expandafter\ifx\csname SubFile\endcsname\relax
\begin{document}
\def\MasterFile{hoge}
%-------------------------------------------------------------------------------
%\maketitle
\thispagestyle{empty}
\input{../hyoushi/title}
%\addcontentsline{toc}{chapter}{表紙}
\thispagestyle{empty}
\mbox{}\newpage
%===============================================================================
%\frontmatter
%===============================================================================
%\mainmatter
%-------------------------------------------------------------------------------
\pagenumbering{arabic}
\cleardoublepage
\input{../0.Abstract/chapter}
%-------------------------------------------------------------------------------
\clearpage
\addcontentsline{toc}{chapter}{目次}
\tableofcontents

\clearpage
\addcontentsline{toc}{chapter}{図目次}
\listoffigures

\clearpage
\addcontentsline{toc}{chapter}{表目次}
\listoftables

%-------------------------------------------------------------------------------

%=====================
\pagestyle{fancy} % Headerをつける
\renewcommand{\sectionmark}[1]{\markright{\thesection\ \ \ #1}}
\renewcommand{\chaptermark}[1]{\markboth{#1}{}}
\lhead{}
\chead{}
\lfoot{}
\rfoot{}%-------------------------------------------------------------------------------
\input{../1.Introduction/chapter}
%-------------------------------------------------------------------------------
\input{../2.Related_Work/chapter}
%-------------------------------------------------------------------------------
\input{../3.The_Model/chapter}
%-------------------------------------------------------------------------------
\input{../4.Implementation/chapter}
%-------------------------------------------------------------------------------
\input{../5.Experiments/chapter}
%-------------------------------------------------------------------------------
\input{../6.Conclusion/chapter}

%===============================================================================
\pagestyle{plain}
%-------------------------------------------------------------------------------
\input{../7.Acknowledgement/chapter} %謝辞
%-------------------------------------------------------------------------------
\def\BibFile{../Bibliograhoy/database2}
\input{../Bibliography/chapter} %参考文献
% %===============================================================================
\appendix
\input{../A.Mypaper/chapter} % 投稿論文リスト
\input{../B.SIG-CCI2/chapter} %
\input{../C.IJCAI-16/chapter} %
%===============================================================================
\end{document}
\fi

	\begin{document}
	\setcounter{chapter}{0}
	\fi
  %-------------------------------------------------------------------------------
\cleardoublepage
\chapter{序論}
\label{intro:chapter}
%本章では, 本研究を行なうに至った背景と目的について述べる.その後,本論文の構成について述べる.
\section{研究の背景}
\label{intro:background}
近年,Web上での大規模な議論活動が活発になっているが,現在一般的に使われている "2ちゃんねる" や "Twitter" といったシステムでは整理や収束を行うことが困難である.困難である原因として,議論の管理を行う者がいないことが挙げられる.
つまり,議論を整理・収束させるには議論のマネジメントを行う人物が必要である.
%
大規模意見集約システムCOLLAGREE\cite{collagreeTest}ではファシリテーターと呼ばれる人物が議論のマネジメントを行っている.
しかし,ファシリテーターは人間であり,長時間に渡って大人数での議論の動向をマネジメントし続けるのは困難である.
COLLAGREEで大規模な議論を収束させるためには,ファシリテーターが必要な時には画面を見るようにして,他の時は見なくても済むようにすることで画面に向き合う時間を減らす工夫があることが望ましい.ファシリテーターが画面を見るべきタイミングは議論の話題が変化したときである.以前の議論の内容から外れた発言がされた時,ファシリテーターが適切な発言をすることで,脱線や炎上を避けて議論を収束させることができる.
すなわち,ファシリテーターの代わりに自動的に議論中の話題の変化を観測することが求められている.
%
現在,COLLAGREE上で使用されている議論支援システムは「(1)投稿支援システム」と「(2)議論可視化システム」の2つに大別できる.
投稿支援システムはポイント機能やファシリテーションフレーズ簡易投稿機能のように,ユーザーが投稿をする際に何らかの補助やリアクションを行う.現行の機能では選択肢の提示に留まっており,作業量を減らすことには繋がりにくい.
一方,議論可視化システムは議論ツリーやキーワード抽出のように,ユーザーにスレッドとは異なる議論の見方を提供する.
\ref{Fig:argTree1}に議論ツリーの例を示す.
\begin{figure}[htbp]
 \begin{center}
  \includegraphics[width=\textwidth]{../images/2.Related_Work/argTree1.png}
  \caption{議論ツリー}
  \label{Fig:argTree1}
  \vspace{-10pt}
 \end{center}
\end{figure}
現行の機能では議論を見やすくすることに重点が置かれており,議論の把握の助けにはなるが画面に向き合う時間を減らすことにはなりにくい.むしろ,作業量を増やすことになり得ることもある.
従って,現行の支援機能ではファシリテーターの作業量の減少には繋がりにくい.
%
近年,自然言語処理の分野において分散表現が多くの研究で使われており,機械翻訳を始めとする単語の意味が重要となる分野で精度の向上が確認されている.分散表現を用いることで,人間に近い精度で話題の変化を観測することが可能となる.
%
以上のような背景を踏まえて,分散表現を用いて,話題の変化を観測し,話題の変化が確認された時にファシリテーターに伝えることが望ましい.
話題の変化の観測は,発言中に現れる単語の類似度の計算と見なすことができる.
分散表現を用いることで単語間の類似度を求めることができる,値が大きいほど単語がそれぞれ類似した実数ベクトルであることを表す.単語Aと単語Bの実数ベクトルが類似しているとは,単語Aと共に使われることの多い単語と単語Bと共に使われることの多い単語が多く共通していることを示す.故に,分散表現を使って単語の類似度を計算することができる.
%
発言文から単語を選ぶ際には自動要約を用いる.発言文から重要でない単語を取り除くことで関連度の計算の精度を高めることが可能となる.
要約の手法としてはokapi BM25 \cite{okapiBM25}とLexRankを組み合わせた抽出的要約手法を用いる.
\begin{comment}
%======================================= 社会的背景
2013年頃からWeb上での大規模な議論活動が活発になり,大規模な人数での議論が期待されている.
大規模な議論では意見を共有することは可能であるが,議論を整理させることや収束させることは難しい.以上から大規模意見集約システムCOLLAGREEが開発された.本システムではWeb上で適切に大規模な議論を行うことができるように議論をマネジメントするファシリテーターを導入した\cite{collagreeTest}.
過去の実験ではファシリテーターの存在が議論の集約に大きな役割を果たしていることが認識されており,大規模な議論のためにファシリテータは必要である.しかし,議論の規模に伴って議論時間が長くなる傾向があり,同時にファシリテーターは常に議論の動向を見続ける必要がある.故に,議論の規模が大きくなればなるほどファシリテーターは長時間かつ大規模な議論の動向の監視によって大きな負担がかかる.大規模な議論が増加する傾向を踏まえるとファシリテーターにかかる負担を軽減する支援が必要である.\\
以上の問題を解決するため,話題の変化を追い,重要な話題の転換点をファシリテーターの代わりに検出することが有用であると考える.必要な時にだけファシリテーターが画面を見れば良いようにすることでファシリテーターの負担軽減が期待できる.
%========================================= 現行手法問題点背景
%議論支援に関する先行研究において,既存の手法は全てが文字列を文字列のまま扱う手法である.
%既存手法は殆どがパターンマッチングと重み付けの2つに区分することができる.
%パターンマッチングでは事前に単語を登録して,単語がマッチした場合に処理を行うが,処理それぞれに対して単語を登録しなければならず手間が膨大になってしまう.また,単語の意味が考慮されておらず,手作業で登録を行うので登録漏れがあった場合に単語の意味に関係なく処理を行うことが不可能となってしまう.
%重み付けは単語の出現頻度や文章の長さを使用して単語・文章に順位を付ける手法で必ずしも単語の登録が必要でないため多くの研究で使用されている.
%しかし,重み付けもまた単語を文字列のまま扱っており,意味までは考慮されていない.故に ,人間なら対応できる似た単語でも1文字違うだけで対処が困難となる.
議論支援に関する先行研究においてファシリテーターに対する支援を目的としたものは無く,殆どが議論の活性化や可視化を目的としている.
%=================================新手法
近年,自然言語処理の分野において分散表現が多くの研究で使われており.分散表現は文字列である単語を辞書データを使用して実数ベクトルへと変換する.辞書データにない単語には対応できないが,多様な処理を1つの辞書データで行うことができる.また,実数ベクトルの各数値が単語の意味を表現するものとなっており,数値を使用して処理を行うことができる.
分散表現を用いることで既存手法より人間の感覚に近しい処理を行うことができる.
%=================================
以上のような背景を踏まえて,分散表現を用いてファシリテーターの代わりに話題の変化を判定し,知らせることを目指す.
話題転換の検出は発言同士の近さ,すなわち発言に含まれる単語意味の近さと見ることができる.
分散表現ではベクトル同士の内積計算を行うことで単語同士の意味の近さを計算することができる.
また,分散表現を使用することで機械翻訳を始めとする複数の分野で精度の向上が確認されている.
\end{comment}
\section{研究の目的}
\label{intro:taget}
本論文では,分散表現を用いて議論中での発言に含まれる単語の関連度を計算し,話題の変化を観測する手法を提案する.

\section{本論文の構成}
本論文の構成を以下に示す.
\ref{relwork:chapter} 章では要約手法に関する研究と,分散表現に関する先行研究を紹介する.
次に,\ref{model:chapter}章では発言の要約手法の説明を行い,\ref{impl:chapter}章では分散表現を用いた単語集合間の関連度計算について説明する.
そして,\ref{exp:chapter}章では話題転換点の検出の評価実験について説明する.
最後に\ref{con:chapter}章で本論文のまとめと考察を示す.

 %-------------------------------------------------------------------------------
 \expandafter\ifx\csname MasterFile\endcsname\relax
	\def\BibFile{hoge}
	\expandafter\ifx\csname MasterFile\endcsname\relax
	\def\SubFile{hoge}
	\input{../thesis/thesis}
	\begin{document}
	\setcounter{chapter}{0}
	\fi
  %-------------------------------------------------------------------------------
\cleardoublepage
\chapter{序論}
\label{intro:chapter}
%本章では, 本研究を行なうに至った背景と目的について述べる.その後,本論文の構成について述べる.
\section{研究の背景}
\label{intro:background}
近年,Web上での大規模な議論活動が活発になっているが,現在一般的に使われている "2ちゃんねる" や "Twitter" といったシステムでは整理や収束を行うことが困難である.困難である原因として,議論の管理を行う者がいないことが挙げられる.
つまり,議論を整理・収束させるには議論のマネジメントを行う人物が必要である.
%
大規模意見集約システムCOLLAGREE\cite{collagreeTest}ではファシリテーターと呼ばれる人物が議論のマネジメントを行っている.
しかし,ファシリテーターは人間であり,長時間に渡って大人数での議論の動向をマネジメントし続けるのは困難である.
COLLAGREEで大規模な議論を収束させるためには,ファシリテーターが必要な時には画面を見るようにして,他の時は見なくても済むようにすることで画面に向き合う時間を減らす工夫があることが望ましい.ファシリテーターが画面を見るべきタイミングは議論の話題が変化したときである.以前の議論の内容から外れた発言がされた時,ファシリテーターが適切な発言をすることで,脱線や炎上を避けて議論を収束させることができる.
すなわち,ファシリテーターの代わりに自動的に議論中の話題の変化を観測することが求められている.
%
現在,COLLAGREE上で使用されている議論支援システムは「(1)投稿支援システム」と「(2)議論可視化システム」の2つに大別できる.
投稿支援システムはポイント機能やファシリテーションフレーズ簡易投稿機能のように,ユーザーが投稿をする際に何らかの補助やリアクションを行う.現行の機能では選択肢の提示に留まっており,作業量を減らすことには繋がりにくい.
一方,議論可視化システムは議論ツリーやキーワード抽出のように,ユーザーにスレッドとは異なる議論の見方を提供する.
\ref{Fig:argTree1}に議論ツリーの例を示す.
\begin{figure}[htbp]
 \begin{center}
  \includegraphics[width=\textwidth]{../images/2.Related_Work/argTree1.png}
  \caption{議論ツリー}
  \label{Fig:argTree1}
  \vspace{-10pt}
 \end{center}
\end{figure}
現行の機能では議論を見やすくすることに重点が置かれており,議論の把握の助けにはなるが画面に向き合う時間を減らすことにはなりにくい.むしろ,作業量を増やすことになり得ることもある.
従って,現行の支援機能ではファシリテーターの作業量の減少には繋がりにくい.
%
近年,自然言語処理の分野において分散表現が多くの研究で使われており,機械翻訳を始めとする単語の意味が重要となる分野で精度の向上が確認されている.分散表現を用いることで,人間に近い精度で話題の変化を観測することが可能となる.
%
以上のような背景を踏まえて,分散表現を用いて,話題の変化を観測し,話題の変化が確認された時にファシリテーターに伝えることが望ましい.
話題の変化の観測は,発言中に現れる単語の類似度の計算と見なすことができる.
分散表現を用いることで単語間の類似度を求めることができる,値が大きいほど単語がそれぞれ類似した実数ベクトルであることを表す.単語Aと単語Bの実数ベクトルが類似しているとは,単語Aと共に使われることの多い単語と単語Bと共に使われることの多い単語が多く共通していることを示す.故に,分散表現を使って単語の類似度を計算することができる.
%
発言文から単語を選ぶ際には自動要約を用いる.発言文から重要でない単語を取り除くことで関連度の計算の精度を高めることが可能となる.
要約の手法としてはokapi BM25 \cite{okapiBM25}とLexRankを組み合わせた抽出的要約手法を用いる.
\begin{comment}
%======================================= 社会的背景
2013年頃からWeb上での大規模な議論活動が活発になり,大規模な人数での議論が期待されている.
大規模な議論では意見を共有することは可能であるが,議論を整理させることや収束させることは難しい.以上から大規模意見集約システムCOLLAGREEが開発された.本システムではWeb上で適切に大規模な議論を行うことができるように議論をマネジメントするファシリテーターを導入した\cite{collagreeTest}.
過去の実験ではファシリテーターの存在が議論の集約に大きな役割を果たしていることが認識されており,大規模な議論のためにファシリテータは必要である.しかし,議論の規模に伴って議論時間が長くなる傾向があり,同時にファシリテーターは常に議論の動向を見続ける必要がある.故に,議論の規模が大きくなればなるほどファシリテーターは長時間かつ大規模な議論の動向の監視によって大きな負担がかかる.大規模な議論が増加する傾向を踏まえるとファシリテーターにかかる負担を軽減する支援が必要である.\\
以上の問題を解決するため,話題の変化を追い,重要な話題の転換点をファシリテーターの代わりに検出することが有用であると考える.必要な時にだけファシリテーターが画面を見れば良いようにすることでファシリテーターの負担軽減が期待できる.
%========================================= 現行手法問題点背景
%議論支援に関する先行研究において,既存の手法は全てが文字列を文字列のまま扱う手法である.
%既存手法は殆どがパターンマッチングと重み付けの2つに区分することができる.
%パターンマッチングでは事前に単語を登録して,単語がマッチした場合に処理を行うが,処理それぞれに対して単語を登録しなければならず手間が膨大になってしまう.また,単語の意味が考慮されておらず,手作業で登録を行うので登録漏れがあった場合に単語の意味に関係なく処理を行うことが不可能となってしまう.
%重み付けは単語の出現頻度や文章の長さを使用して単語・文章に順位を付ける手法で必ずしも単語の登録が必要でないため多くの研究で使用されている.
%しかし,重み付けもまた単語を文字列のまま扱っており,意味までは考慮されていない.故に ,人間なら対応できる似た単語でも1文字違うだけで対処が困難となる.
議論支援に関する先行研究においてファシリテーターに対する支援を目的としたものは無く,殆どが議論の活性化や可視化を目的としている.
%=================================新手法
近年,自然言語処理の分野において分散表現が多くの研究で使われており.分散表現は文字列である単語を辞書データを使用して実数ベクトルへと変換する.辞書データにない単語には対応できないが,多様な処理を1つの辞書データで行うことができる.また,実数ベクトルの各数値が単語の意味を表現するものとなっており,数値を使用して処理を行うことができる.
分散表現を用いることで既存手法より人間の感覚に近しい処理を行うことができる.
%=================================
以上のような背景を踏まえて,分散表現を用いてファシリテーターの代わりに話題の変化を判定し,知らせることを目指す.
話題転換の検出は発言同士の近さ,すなわち発言に含まれる単語意味の近さと見ることができる.
分散表現ではベクトル同士の内積計算を行うことで単語同士の意味の近さを計算することができる.
また,分散表現を使用することで機械翻訳を始めとする複数の分野で精度の向上が確認されている.
\end{comment}
\section{研究の目的}
\label{intro:taget}
本論文では,分散表現を用いて議論中での発言に含まれる単語の関連度を計算し,話題の変化を観測する手法を提案する.

\section{本論文の構成}
本論文の構成を以下に示す.
\ref{relwork:chapter} 章では要約手法に関する研究と,分散表現に関する先行研究を紹介する.
次に,\ref{model:chapter}章では発言の要約手法の説明を行い,\ref{impl:chapter}章では分散表現を用いた単語集合間の関連度計算について説明する.
そして,\ref{exp:chapter}章では話題転換点の検出の評価実験について説明する.
最後に\ref{con:chapter}章で本論文のまとめと考察を示す.

 %-------------------------------------------------------------------------------
 \expandafter\ifx\csname MasterFile\endcsname\relax
	\def\BibFile{hoge}
	\input{../Bibliography/chapter}
  \fi
  %-------------------------------------------------------------------------------
  \expandafter\ifx\csname MasterFile\endcsname\relax
  \end{document}
  \fi

  \fi
  %-------------------------------------------------------------------------------
  \expandafter\ifx\csname MasterFile\endcsname\relax
  \end{document}
  \fi
 % 投稿論文リスト
\expandafter\ifx\csname MasterFile\endcsname\relax
	\def\SubFile{hoge}
	\documentclass[a4j,12pt,twoside,openany]{jreport}
%\nofiles %tocファイルを更新させない
%\documentclass[12pt,a4j,twoside,openany]{jsbook}
\usepackage[dvipdfmx]{graphicx}
\usepackage{../dspc} % ベースラインスキップの指定
\usepackage{../slashbox} % 表に斜線を入れる
%\usepackage{../mediabb}
\usepackage{fancyvrb} % Verbatim環境
\usepackage{fancyhdr} % Headerの下線付き章見出し
\usepackage{here} % float[H]
\usepackage{multirow}
\usepackage{hhline} % 表の罫線の角を美しくする
\usepackage{amsmath} %コレがないとcasesが動かない
\usepackage{amsfonts} % 数学用フォント
\usepackage{bm} % 数式環境での bold
\usepackage{algorithm}
\usepackage{algorithmicx}
\usepackage[noend]{algpseudocode}
\usepackage[flushleft]{threeparttable} % 脚注付きテーブル
\usepackage{enumitem}
\usepackage{comment}
\usepackage{fancybox}
%\usepackage{csvsimple,booktabs,siunitx}
%\usepackage{filecontents}


\setlength{\evensidemargin}{5pt}
\setlength{\oddsidemargin}{40pt}
%\setlength{\headheight}{16.5pt}
%%\setlength{\headheight}{30pt}
\setcounter{secnumdepth}{3}
\setlist[description]{leftmargin=2\parindent,labelindent=\parindent}

\makeatletter
\def\@makechapterhead#1{%
	\vspace*{50\p@}%
	{
		\parindent \z@ \raggedright \normalfont
		\ifnum \c@secnumdepth >\m@ne
		% \if@mainmatter
			\huge\bfseries\@chapapp\thechapter\@chappos
			\par\nobreak
			\vskip 20\p@
		% \fi
		\fi
		\interlinepenalty\@M
		\Huge\bfseries #1\par\nobreak
		\vskip 40\p@
	}
}

%新しいコマンド定義
\newcounter{linenumber}
\newenvironment{listing}{%
  \begin{list}{%
    \small\arabic{linenumber}:}{%
      \usecounter{linenumber}%
      \setlength{\baselineskip}{18pt}%
      \setlength{\itemsep}{0pt}%
      \setlength{\parsep}{0pt}}}%
 {\end{list}}
\newcommand{\figcaption}[1]{\def\@captype{figure}\caption{#1}}
\newcommand{\tblcaption}[1]{\def\@captype{table}\caption{#1}}
\newcommand{\norm}[1]{\left\| #1 \right\|}
\newcommand{\cc}[1]{\multicolumn{1}{|c|}{#1}}
\newcommand{\circled}[1]{\raisebox{.5pt}{\textcircled{\raisebox{-.9pt} {#1}}}}
\newcommand{\specialcell}[2][c]{%
  \begin{tabular}[#1]{@{}c@{}}#2\end{tabular}}
\makeatother
%===============================================================================
\expandafter\ifx\csname SubFile\endcsname\relax
\begin{document}
\def\MasterFile{hoge}
%-------------------------------------------------------------------------------
%\maketitle
\thispagestyle{empty}
\input{../hyoushi/title}
%\addcontentsline{toc}{chapter}{表紙}
\thispagestyle{empty}
\mbox{}\newpage
%===============================================================================
%\frontmatter
%===============================================================================
%\mainmatter
%-------------------------------------------------------------------------------
\pagenumbering{arabic}
\cleardoublepage
\input{../0.Abstract/chapter}
%-------------------------------------------------------------------------------
\clearpage
\addcontentsline{toc}{chapter}{目次}
\tableofcontents

\clearpage
\addcontentsline{toc}{chapter}{図目次}
\listoffigures

\clearpage
\addcontentsline{toc}{chapter}{表目次}
\listoftables

%-------------------------------------------------------------------------------

%=====================
\pagestyle{fancy} % Headerをつける
\renewcommand{\sectionmark}[1]{\markright{\thesection\ \ \ #1}}
\renewcommand{\chaptermark}[1]{\markboth{#1}{}}
\lhead{}
\chead{}
\lfoot{}
\rfoot{}%-------------------------------------------------------------------------------
\input{../1.Introduction/chapter}
%-------------------------------------------------------------------------------
\input{../2.Related_Work/chapter}
%-------------------------------------------------------------------------------
\input{../3.The_Model/chapter}
%-------------------------------------------------------------------------------
\input{../4.Implementation/chapter}
%-------------------------------------------------------------------------------
\input{../5.Experiments/chapter}
%-------------------------------------------------------------------------------
\input{../6.Conclusion/chapter}

%===============================================================================
\pagestyle{plain}
%-------------------------------------------------------------------------------
\input{../7.Acknowledgement/chapter} %謝辞
%-------------------------------------------------------------------------------
\def\BibFile{../Bibliograhoy/database2}
\input{../Bibliography/chapter} %参考文献
% %===============================================================================
\appendix
\input{../A.Mypaper/chapter} % 投稿論文リスト
\input{../B.SIG-CCI2/chapter} %
\input{../C.IJCAI-16/chapter} %
%===============================================================================
\end{document}
\fi

	\begin{document}
	\setcounter{chapter}{0}
	\fi
  %-------------------------------------------------------------------------------
\cleardoublepage
\chapter{序論}
\label{intro:chapter}
%本章では, 本研究を行なうに至った背景と目的について述べる.その後,本論文の構成について述べる.
\section{研究の背景}
\label{intro:background}
近年,Web上での大規模な議論活動が活発になっているが,現在一般的に使われている "2ちゃんねる" や "Twitter" といったシステムでは整理や収束を行うことが困難である.困難である原因として,議論の管理を行う者がいないことが挙げられる.
つまり,議論を整理・収束させるには議論のマネジメントを行う人物が必要である.
%
大規模意見集約システムCOLLAGREE\cite{collagreeTest}ではファシリテーターと呼ばれる人物が議論のマネジメントを行っている.
しかし,ファシリテーターは人間であり,長時間に渡って大人数での議論の動向をマネジメントし続けるのは困難である.
COLLAGREEで大規模な議論を収束させるためには,ファシリテーターが必要な時には画面を見るようにして,他の時は見なくても済むようにすることで画面に向き合う時間を減らす工夫があることが望ましい.ファシリテーターが画面を見るべきタイミングは議論の話題が変化したときである.以前の議論の内容から外れた発言がされた時,ファシリテーターが適切な発言をすることで,脱線や炎上を避けて議論を収束させることができる.
すなわち,ファシリテーターの代わりに自動的に議論中の話題の変化を観測することが求められている.
%
現在,COLLAGREE上で使用されている議論支援システムは「(1)投稿支援システム」と「(2)議論可視化システム」の2つに大別できる.
投稿支援システムはポイント機能やファシリテーションフレーズ簡易投稿機能のように,ユーザーが投稿をする際に何らかの補助やリアクションを行う.現行の機能では選択肢の提示に留まっており,作業量を減らすことには繋がりにくい.
一方,議論可視化システムは議論ツリーやキーワード抽出のように,ユーザーにスレッドとは異なる議論の見方を提供する.
\ref{Fig:argTree1}に議論ツリーの例を示す.
\begin{figure}[htbp]
 \begin{center}
  \includegraphics[width=\textwidth]{../images/2.Related_Work/argTree1.png}
  \caption{議論ツリー}
  \label{Fig:argTree1}
  \vspace{-10pt}
 \end{center}
\end{figure}
現行の機能では議論を見やすくすることに重点が置かれており,議論の把握の助けにはなるが画面に向き合う時間を減らすことにはなりにくい.むしろ,作業量を増やすことになり得ることもある.
従って,現行の支援機能ではファシリテーターの作業量の減少には繋がりにくい.
%
近年,自然言語処理の分野において分散表現が多くの研究で使われており,機械翻訳を始めとする単語の意味が重要となる分野で精度の向上が確認されている.分散表現を用いることで,人間に近い精度で話題の変化を観測することが可能となる.
%
以上のような背景を踏まえて,分散表現を用いて,話題の変化を観測し,話題の変化が確認された時にファシリテーターに伝えることが望ましい.
話題の変化の観測は,発言中に現れる単語の類似度の計算と見なすことができる.
分散表現を用いることで単語間の類似度を求めることができる,値が大きいほど単語がそれぞれ類似した実数ベクトルであることを表す.単語Aと単語Bの実数ベクトルが類似しているとは,単語Aと共に使われることの多い単語と単語Bと共に使われることの多い単語が多く共通していることを示す.故に,分散表現を使って単語の類似度を計算することができる.
%
発言文から単語を選ぶ際には自動要約を用いる.発言文から重要でない単語を取り除くことで関連度の計算の精度を高めることが可能となる.
要約の手法としてはokapi BM25 \cite{okapiBM25}とLexRankを組み合わせた抽出的要約手法を用いる.
\begin{comment}
%======================================= 社会的背景
2013年頃からWeb上での大規模な議論活動が活発になり,大規模な人数での議論が期待されている.
大規模な議論では意見を共有することは可能であるが,議論を整理させることや収束させることは難しい.以上から大規模意見集約システムCOLLAGREEが開発された.本システムではWeb上で適切に大規模な議論を行うことができるように議論をマネジメントするファシリテーターを導入した\cite{collagreeTest}.
過去の実験ではファシリテーターの存在が議論の集約に大きな役割を果たしていることが認識されており,大規模な議論のためにファシリテータは必要である.しかし,議論の規模に伴って議論時間が長くなる傾向があり,同時にファシリテーターは常に議論の動向を見続ける必要がある.故に,議論の規模が大きくなればなるほどファシリテーターは長時間かつ大規模な議論の動向の監視によって大きな負担がかかる.大規模な議論が増加する傾向を踏まえるとファシリテーターにかかる負担を軽減する支援が必要である.\\
以上の問題を解決するため,話題の変化を追い,重要な話題の転換点をファシリテーターの代わりに検出することが有用であると考える.必要な時にだけファシリテーターが画面を見れば良いようにすることでファシリテーターの負担軽減が期待できる.
%========================================= 現行手法問題点背景
%議論支援に関する先行研究において,既存の手法は全てが文字列を文字列のまま扱う手法である.
%既存手法は殆どがパターンマッチングと重み付けの2つに区分することができる.
%パターンマッチングでは事前に単語を登録して,単語がマッチした場合に処理を行うが,処理それぞれに対して単語を登録しなければならず手間が膨大になってしまう.また,単語の意味が考慮されておらず,手作業で登録を行うので登録漏れがあった場合に単語の意味に関係なく処理を行うことが不可能となってしまう.
%重み付けは単語の出現頻度や文章の長さを使用して単語・文章に順位を付ける手法で必ずしも単語の登録が必要でないため多くの研究で使用されている.
%しかし,重み付けもまた単語を文字列のまま扱っており,意味までは考慮されていない.故に ,人間なら対応できる似た単語でも1文字違うだけで対処が困難となる.
議論支援に関する先行研究においてファシリテーターに対する支援を目的としたものは無く,殆どが議論の活性化や可視化を目的としている.
%=================================新手法
近年,自然言語処理の分野において分散表現が多くの研究で使われており.分散表現は文字列である単語を辞書データを使用して実数ベクトルへと変換する.辞書データにない単語には対応できないが,多様な処理を1つの辞書データで行うことができる.また,実数ベクトルの各数値が単語の意味を表現するものとなっており,数値を使用して処理を行うことができる.
分散表現を用いることで既存手法より人間の感覚に近しい処理を行うことができる.
%=================================
以上のような背景を踏まえて,分散表現を用いてファシリテーターの代わりに話題の変化を判定し,知らせることを目指す.
話題転換の検出は発言同士の近さ,すなわち発言に含まれる単語意味の近さと見ることができる.
分散表現ではベクトル同士の内積計算を行うことで単語同士の意味の近さを計算することができる.
また,分散表現を使用することで機械翻訳を始めとする複数の分野で精度の向上が確認されている.
\end{comment}
\section{研究の目的}
\label{intro:taget}
本論文では,分散表現を用いて議論中での発言に含まれる単語の関連度を計算し,話題の変化を観測する手法を提案する.

\section{本論文の構成}
本論文の構成を以下に示す.
\ref{relwork:chapter} 章では要約手法に関する研究と,分散表現に関する先行研究を紹介する.
次に,\ref{model:chapter}章では発言の要約手法の説明を行い,\ref{impl:chapter}章では分散表現を用いた単語集合間の関連度計算について説明する.
そして,\ref{exp:chapter}章では話題転換点の検出の評価実験について説明する.
最後に\ref{con:chapter}章で本論文のまとめと考察を示す.

 %-------------------------------------------------------------------------------
 \expandafter\ifx\csname MasterFile\endcsname\relax
	\def\BibFile{hoge}
	\expandafter\ifx\csname MasterFile\endcsname\relax
	\def\SubFile{hoge}
	\input{../thesis/thesis}
	\begin{document}
	\setcounter{chapter}{0}
	\fi
  %-------------------------------------------------------------------------------
\cleardoublepage
\chapter{序論}
\label{intro:chapter}
%本章では, 本研究を行なうに至った背景と目的について述べる.その後,本論文の構成について述べる.
\section{研究の背景}
\label{intro:background}
近年,Web上での大規模な議論活動が活発になっているが,現在一般的に使われている "2ちゃんねる" や "Twitter" といったシステムでは整理や収束を行うことが困難である.困難である原因として,議論の管理を行う者がいないことが挙げられる.
つまり,議論を整理・収束させるには議論のマネジメントを行う人物が必要である.
%
大規模意見集約システムCOLLAGREE\cite{collagreeTest}ではファシリテーターと呼ばれる人物が議論のマネジメントを行っている.
しかし,ファシリテーターは人間であり,長時間に渡って大人数での議論の動向をマネジメントし続けるのは困難である.
COLLAGREEで大規模な議論を収束させるためには,ファシリテーターが必要な時には画面を見るようにして,他の時は見なくても済むようにすることで画面に向き合う時間を減らす工夫があることが望ましい.ファシリテーターが画面を見るべきタイミングは議論の話題が変化したときである.以前の議論の内容から外れた発言がされた時,ファシリテーターが適切な発言をすることで,脱線や炎上を避けて議論を収束させることができる.
すなわち,ファシリテーターの代わりに自動的に議論中の話題の変化を観測することが求められている.
%
現在,COLLAGREE上で使用されている議論支援システムは「(1)投稿支援システム」と「(2)議論可視化システム」の2つに大別できる.
投稿支援システムはポイント機能やファシリテーションフレーズ簡易投稿機能のように,ユーザーが投稿をする際に何らかの補助やリアクションを行う.現行の機能では選択肢の提示に留まっており,作業量を減らすことには繋がりにくい.
一方,議論可視化システムは議論ツリーやキーワード抽出のように,ユーザーにスレッドとは異なる議論の見方を提供する.
\ref{Fig:argTree1}に議論ツリーの例を示す.
\begin{figure}[htbp]
 \begin{center}
  \includegraphics[width=\textwidth]{../images/2.Related_Work/argTree1.png}
  \caption{議論ツリー}
  \label{Fig:argTree1}
  \vspace{-10pt}
 \end{center}
\end{figure}
現行の機能では議論を見やすくすることに重点が置かれており,議論の把握の助けにはなるが画面に向き合う時間を減らすことにはなりにくい.むしろ,作業量を増やすことになり得ることもある.
従って,現行の支援機能ではファシリテーターの作業量の減少には繋がりにくい.
%
近年,自然言語処理の分野において分散表現が多くの研究で使われており,機械翻訳を始めとする単語の意味が重要となる分野で精度の向上が確認されている.分散表現を用いることで,人間に近い精度で話題の変化を観測することが可能となる.
%
以上のような背景を踏まえて,分散表現を用いて,話題の変化を観測し,話題の変化が確認された時にファシリテーターに伝えることが望ましい.
話題の変化の観測は,発言中に現れる単語の類似度の計算と見なすことができる.
分散表現を用いることで単語間の類似度を求めることができる,値が大きいほど単語がそれぞれ類似した実数ベクトルであることを表す.単語Aと単語Bの実数ベクトルが類似しているとは,単語Aと共に使われることの多い単語と単語Bと共に使われることの多い単語が多く共通していることを示す.故に,分散表現を使って単語の類似度を計算することができる.
%
発言文から単語を選ぶ際には自動要約を用いる.発言文から重要でない単語を取り除くことで関連度の計算の精度を高めることが可能となる.
要約の手法としてはokapi BM25 \cite{okapiBM25}とLexRankを組み合わせた抽出的要約手法を用いる.
\begin{comment}
%======================================= 社会的背景
2013年頃からWeb上での大規模な議論活動が活発になり,大規模な人数での議論が期待されている.
大規模な議論では意見を共有することは可能であるが,議論を整理させることや収束させることは難しい.以上から大規模意見集約システムCOLLAGREEが開発された.本システムではWeb上で適切に大規模な議論を行うことができるように議論をマネジメントするファシリテーターを導入した\cite{collagreeTest}.
過去の実験ではファシリテーターの存在が議論の集約に大きな役割を果たしていることが認識されており,大規模な議論のためにファシリテータは必要である.しかし,議論の規模に伴って議論時間が長くなる傾向があり,同時にファシリテーターは常に議論の動向を見続ける必要がある.故に,議論の規模が大きくなればなるほどファシリテーターは長時間かつ大規模な議論の動向の監視によって大きな負担がかかる.大規模な議論が増加する傾向を踏まえるとファシリテーターにかかる負担を軽減する支援が必要である.\\
以上の問題を解決するため,話題の変化を追い,重要な話題の転換点をファシリテーターの代わりに検出することが有用であると考える.必要な時にだけファシリテーターが画面を見れば良いようにすることでファシリテーターの負担軽減が期待できる.
%========================================= 現行手法問題点背景
%議論支援に関する先行研究において,既存の手法は全てが文字列を文字列のまま扱う手法である.
%既存手法は殆どがパターンマッチングと重み付けの2つに区分することができる.
%パターンマッチングでは事前に単語を登録して,単語がマッチした場合に処理を行うが,処理それぞれに対して単語を登録しなければならず手間が膨大になってしまう.また,単語の意味が考慮されておらず,手作業で登録を行うので登録漏れがあった場合に単語の意味に関係なく処理を行うことが不可能となってしまう.
%重み付けは単語の出現頻度や文章の長さを使用して単語・文章に順位を付ける手法で必ずしも単語の登録が必要でないため多くの研究で使用されている.
%しかし,重み付けもまた単語を文字列のまま扱っており,意味までは考慮されていない.故に ,人間なら対応できる似た単語でも1文字違うだけで対処が困難となる.
議論支援に関する先行研究においてファシリテーターに対する支援を目的としたものは無く,殆どが議論の活性化や可視化を目的としている.
%=================================新手法
近年,自然言語処理の分野において分散表現が多くの研究で使われており.分散表現は文字列である単語を辞書データを使用して実数ベクトルへと変換する.辞書データにない単語には対応できないが,多様な処理を1つの辞書データで行うことができる.また,実数ベクトルの各数値が単語の意味を表現するものとなっており,数値を使用して処理を行うことができる.
分散表現を用いることで既存手法より人間の感覚に近しい処理を行うことができる.
%=================================
以上のような背景を踏まえて,分散表現を用いてファシリテーターの代わりに話題の変化を判定し,知らせることを目指す.
話題転換の検出は発言同士の近さ,すなわち発言に含まれる単語意味の近さと見ることができる.
分散表現ではベクトル同士の内積計算を行うことで単語同士の意味の近さを計算することができる.
また,分散表現を使用することで機械翻訳を始めとする複数の分野で精度の向上が確認されている.
\end{comment}
\section{研究の目的}
\label{intro:taget}
本論文では,分散表現を用いて議論中での発言に含まれる単語の関連度を計算し,話題の変化を観測する手法を提案する.

\section{本論文の構成}
本論文の構成を以下に示す.
\ref{relwork:chapter} 章では要約手法に関する研究と,分散表現に関する先行研究を紹介する.
次に,\ref{model:chapter}章では発言の要約手法の説明を行い,\ref{impl:chapter}章では分散表現を用いた単語集合間の関連度計算について説明する.
そして,\ref{exp:chapter}章では話題転換点の検出の評価実験について説明する.
最後に\ref{con:chapter}章で本論文のまとめと考察を示す.

 %-------------------------------------------------------------------------------
 \expandafter\ifx\csname MasterFile\endcsname\relax
	\def\BibFile{hoge}
	\input{../Bibliography/chapter}
  \fi
  %-------------------------------------------------------------------------------
  \expandafter\ifx\csname MasterFile\endcsname\relax
  \end{document}
  \fi

  \fi
  %-------------------------------------------------------------------------------
  \expandafter\ifx\csname MasterFile\endcsname\relax
  \end{document}
  \fi
 %
\expandafter\ifx\csname MasterFile\endcsname\relax
	\def\SubFile{hoge}
	\documentclass[a4j,12pt,twoside,openany]{jreport}
%\nofiles %tocファイルを更新させない
%\documentclass[12pt,a4j,twoside,openany]{jsbook}
\usepackage[dvipdfmx]{graphicx}
\usepackage{../dspc} % ベースラインスキップの指定
\usepackage{../slashbox} % 表に斜線を入れる
%\usepackage{../mediabb}
\usepackage{fancyvrb} % Verbatim環境
\usepackage{fancyhdr} % Headerの下線付き章見出し
\usepackage{here} % float[H]
\usepackage{multirow}
\usepackage{hhline} % 表の罫線の角を美しくする
\usepackage{amsmath} %コレがないとcasesが動かない
\usepackage{amsfonts} % 数学用フォント
\usepackage{bm} % 数式環境での bold
\usepackage{algorithm}
\usepackage{algorithmicx}
\usepackage[noend]{algpseudocode}
\usepackage[flushleft]{threeparttable} % 脚注付きテーブル
\usepackage{enumitem}
\usepackage{comment}
\usepackage{fancybox}
%\usepackage{csvsimple,booktabs,siunitx}
%\usepackage{filecontents}


\setlength{\evensidemargin}{5pt}
\setlength{\oddsidemargin}{40pt}
%\setlength{\headheight}{16.5pt}
%%\setlength{\headheight}{30pt}
\setcounter{secnumdepth}{3}
\setlist[description]{leftmargin=2\parindent,labelindent=\parindent}

\makeatletter
\def\@makechapterhead#1{%
	\vspace*{50\p@}%
	{
		\parindent \z@ \raggedright \normalfont
		\ifnum \c@secnumdepth >\m@ne
		% \if@mainmatter
			\huge\bfseries\@chapapp\thechapter\@chappos
			\par\nobreak
			\vskip 20\p@
		% \fi
		\fi
		\interlinepenalty\@M
		\Huge\bfseries #1\par\nobreak
		\vskip 40\p@
	}
}

%新しいコマンド定義
\newcounter{linenumber}
\newenvironment{listing}{%
  \begin{list}{%
    \small\arabic{linenumber}:}{%
      \usecounter{linenumber}%
      \setlength{\baselineskip}{18pt}%
      \setlength{\itemsep}{0pt}%
      \setlength{\parsep}{0pt}}}%
 {\end{list}}
\newcommand{\figcaption}[1]{\def\@captype{figure}\caption{#1}}
\newcommand{\tblcaption}[1]{\def\@captype{table}\caption{#1}}
\newcommand{\norm}[1]{\left\| #1 \right\|}
\newcommand{\cc}[1]{\multicolumn{1}{|c|}{#1}}
\newcommand{\circled}[1]{\raisebox{.5pt}{\textcircled{\raisebox{-.9pt} {#1}}}}
\newcommand{\specialcell}[2][c]{%
  \begin{tabular}[#1]{@{}c@{}}#2\end{tabular}}
\makeatother
%===============================================================================
\expandafter\ifx\csname SubFile\endcsname\relax
\begin{document}
\def\MasterFile{hoge}
%-------------------------------------------------------------------------------
%\maketitle
\thispagestyle{empty}
\input{../hyoushi/title}
%\addcontentsline{toc}{chapter}{表紙}
\thispagestyle{empty}
\mbox{}\newpage
%===============================================================================
%\frontmatter
%===============================================================================
%\mainmatter
%-------------------------------------------------------------------------------
\pagenumbering{arabic}
\cleardoublepage
\input{../0.Abstract/chapter}
%-------------------------------------------------------------------------------
\clearpage
\addcontentsline{toc}{chapter}{目次}
\tableofcontents

\clearpage
\addcontentsline{toc}{chapter}{図目次}
\listoffigures

\clearpage
\addcontentsline{toc}{chapter}{表目次}
\listoftables

%-------------------------------------------------------------------------------

%=====================
\pagestyle{fancy} % Headerをつける
\renewcommand{\sectionmark}[1]{\markright{\thesection\ \ \ #1}}
\renewcommand{\chaptermark}[1]{\markboth{#1}{}}
\lhead{}
\chead{}
\lfoot{}
\rfoot{}%-------------------------------------------------------------------------------
\input{../1.Introduction/chapter}
%-------------------------------------------------------------------------------
\input{../2.Related_Work/chapter}
%-------------------------------------------------------------------------------
\input{../3.The_Model/chapter}
%-------------------------------------------------------------------------------
\input{../4.Implementation/chapter}
%-------------------------------------------------------------------------------
\input{../5.Experiments/chapter}
%-------------------------------------------------------------------------------
\input{../6.Conclusion/chapter}

%===============================================================================
\pagestyle{plain}
%-------------------------------------------------------------------------------
\input{../7.Acknowledgement/chapter} %謝辞
%-------------------------------------------------------------------------------
\def\BibFile{../Bibliograhoy/database2}
\input{../Bibliography/chapter} %参考文献
% %===============================================================================
\appendix
\input{../A.Mypaper/chapter} % 投稿論文リスト
\input{../B.SIG-CCI2/chapter} %
\input{../C.IJCAI-16/chapter} %
%===============================================================================
\end{document}
\fi

	\begin{document}
	\setcounter{chapter}{0}
	\fi
  %-------------------------------------------------------------------------------
\cleardoublepage
\chapter{序論}
\label{intro:chapter}
%本章では, 本研究を行なうに至った背景と目的について述べる.その後,本論文の構成について述べる.
\section{研究の背景}
\label{intro:background}
近年,Web上での大規模な議論活動が活発になっているが,現在一般的に使われている "2ちゃんねる" や "Twitter" といったシステムでは整理や収束を行うことが困難である.困難である原因として,議論の管理を行う者がいないことが挙げられる.
つまり,議論を整理・収束させるには議論のマネジメントを行う人物が必要である.
%
大規模意見集約システムCOLLAGREE\cite{collagreeTest}ではファシリテーターと呼ばれる人物が議論のマネジメントを行っている.
しかし,ファシリテーターは人間であり,長時間に渡って大人数での議論の動向をマネジメントし続けるのは困難である.
COLLAGREEで大規模な議論を収束させるためには,ファシリテーターが必要な時には画面を見るようにして,他の時は見なくても済むようにすることで画面に向き合う時間を減らす工夫があることが望ましい.ファシリテーターが画面を見るべきタイミングは議論の話題が変化したときである.以前の議論の内容から外れた発言がされた時,ファシリテーターが適切な発言をすることで,脱線や炎上を避けて議論を収束させることができる.
すなわち,ファシリテーターの代わりに自動的に議論中の話題の変化を観測することが求められている.
%
現在,COLLAGREE上で使用されている議論支援システムは「(1)投稿支援システム」と「(2)議論可視化システム」の2つに大別できる.
投稿支援システムはポイント機能やファシリテーションフレーズ簡易投稿機能のように,ユーザーが投稿をする際に何らかの補助やリアクションを行う.現行の機能では選択肢の提示に留まっており,作業量を減らすことには繋がりにくい.
一方,議論可視化システムは議論ツリーやキーワード抽出のように,ユーザーにスレッドとは異なる議論の見方を提供する.
\ref{Fig:argTree1}に議論ツリーの例を示す.
\begin{figure}[htbp]
 \begin{center}
  \includegraphics[width=\textwidth]{../images/2.Related_Work/argTree1.png}
  \caption{議論ツリー}
  \label{Fig:argTree1}
  \vspace{-10pt}
 \end{center}
\end{figure}
現行の機能では議論を見やすくすることに重点が置かれており,議論の把握の助けにはなるが画面に向き合う時間を減らすことにはなりにくい.むしろ,作業量を増やすことになり得ることもある.
従って,現行の支援機能ではファシリテーターの作業量の減少には繋がりにくい.
%
近年,自然言語処理の分野において分散表現が多くの研究で使われており,機械翻訳を始めとする単語の意味が重要となる分野で精度の向上が確認されている.分散表現を用いることで,人間に近い精度で話題の変化を観測することが可能となる.
%
以上のような背景を踏まえて,分散表現を用いて,話題の変化を観測し,話題の変化が確認された時にファシリテーターに伝えることが望ましい.
話題の変化の観測は,発言中に現れる単語の類似度の計算と見なすことができる.
分散表現を用いることで単語間の類似度を求めることができる,値が大きいほど単語がそれぞれ類似した実数ベクトルであることを表す.単語Aと単語Bの実数ベクトルが類似しているとは,単語Aと共に使われることの多い単語と単語Bと共に使われることの多い単語が多く共通していることを示す.故に,分散表現を使って単語の類似度を計算することができる.
%
発言文から単語を選ぶ際には自動要約を用いる.発言文から重要でない単語を取り除くことで関連度の計算の精度を高めることが可能となる.
要約の手法としてはokapi BM25 \cite{okapiBM25}とLexRankを組み合わせた抽出的要約手法を用いる.
\begin{comment}
%======================================= 社会的背景
2013年頃からWeb上での大規模な議論活動が活発になり,大規模な人数での議論が期待されている.
大規模な議論では意見を共有することは可能であるが,議論を整理させることや収束させることは難しい.以上から大規模意見集約システムCOLLAGREEが開発された.本システムではWeb上で適切に大規模な議論を行うことができるように議論をマネジメントするファシリテーターを導入した\cite{collagreeTest}.
過去の実験ではファシリテーターの存在が議論の集約に大きな役割を果たしていることが認識されており,大規模な議論のためにファシリテータは必要である.しかし,議論の規模に伴って議論時間が長くなる傾向があり,同時にファシリテーターは常に議論の動向を見続ける必要がある.故に,議論の規模が大きくなればなるほどファシリテーターは長時間かつ大規模な議論の動向の監視によって大きな負担がかかる.大規模な議論が増加する傾向を踏まえるとファシリテーターにかかる負担を軽減する支援が必要である.\\
以上の問題を解決するため,話題の変化を追い,重要な話題の転換点をファシリテーターの代わりに検出することが有用であると考える.必要な時にだけファシリテーターが画面を見れば良いようにすることでファシリテーターの負担軽減が期待できる.
%========================================= 現行手法問題点背景
%議論支援に関する先行研究において,既存の手法は全てが文字列を文字列のまま扱う手法である.
%既存手法は殆どがパターンマッチングと重み付けの2つに区分することができる.
%パターンマッチングでは事前に単語を登録して,単語がマッチした場合に処理を行うが,処理それぞれに対して単語を登録しなければならず手間が膨大になってしまう.また,単語の意味が考慮されておらず,手作業で登録を行うので登録漏れがあった場合に単語の意味に関係なく処理を行うことが不可能となってしまう.
%重み付けは単語の出現頻度や文章の長さを使用して単語・文章に順位を付ける手法で必ずしも単語の登録が必要でないため多くの研究で使用されている.
%しかし,重み付けもまた単語を文字列のまま扱っており,意味までは考慮されていない.故に ,人間なら対応できる似た単語でも1文字違うだけで対処が困難となる.
議論支援に関する先行研究においてファシリテーターに対する支援を目的としたものは無く,殆どが議論の活性化や可視化を目的としている.
%=================================新手法
近年,自然言語処理の分野において分散表現が多くの研究で使われており.分散表現は文字列である単語を辞書データを使用して実数ベクトルへと変換する.辞書データにない単語には対応できないが,多様な処理を1つの辞書データで行うことができる.また,実数ベクトルの各数値が単語の意味を表現するものとなっており,数値を使用して処理を行うことができる.
分散表現を用いることで既存手法より人間の感覚に近しい処理を行うことができる.
%=================================
以上のような背景を踏まえて,分散表現を用いてファシリテーターの代わりに話題の変化を判定し,知らせることを目指す.
話題転換の検出は発言同士の近さ,すなわち発言に含まれる単語意味の近さと見ることができる.
分散表現ではベクトル同士の内積計算を行うことで単語同士の意味の近さを計算することができる.
また,分散表現を使用することで機械翻訳を始めとする複数の分野で精度の向上が確認されている.
\end{comment}
\section{研究の目的}
\label{intro:taget}
本論文では,分散表現を用いて議論中での発言に含まれる単語の関連度を計算し,話題の変化を観測する手法を提案する.

\section{本論文の構成}
本論文の構成を以下に示す.
\ref{relwork:chapter} 章では要約手法に関する研究と,分散表現に関する先行研究を紹介する.
次に,\ref{model:chapter}章では発言の要約手法の説明を行い,\ref{impl:chapter}章では分散表現を用いた単語集合間の関連度計算について説明する.
そして,\ref{exp:chapter}章では話題転換点の検出の評価実験について説明する.
最後に\ref{con:chapter}章で本論文のまとめと考察を示す.

 %-------------------------------------------------------------------------------
 \expandafter\ifx\csname MasterFile\endcsname\relax
	\def\BibFile{hoge}
	\expandafter\ifx\csname MasterFile\endcsname\relax
	\def\SubFile{hoge}
	\input{../thesis/thesis}
	\begin{document}
	\setcounter{chapter}{0}
	\fi
  %-------------------------------------------------------------------------------
\cleardoublepage
\chapter{序論}
\label{intro:chapter}
%本章では, 本研究を行なうに至った背景と目的について述べる.その後,本論文の構成について述べる.
\section{研究の背景}
\label{intro:background}
近年,Web上での大規模な議論活動が活発になっているが,現在一般的に使われている "2ちゃんねる" や "Twitter" といったシステムでは整理や収束を行うことが困難である.困難である原因として,議論の管理を行う者がいないことが挙げられる.
つまり,議論を整理・収束させるには議論のマネジメントを行う人物が必要である.
%
大規模意見集約システムCOLLAGREE\cite{collagreeTest}ではファシリテーターと呼ばれる人物が議論のマネジメントを行っている.
しかし,ファシリテーターは人間であり,長時間に渡って大人数での議論の動向をマネジメントし続けるのは困難である.
COLLAGREEで大規模な議論を収束させるためには,ファシリテーターが必要な時には画面を見るようにして,他の時は見なくても済むようにすることで画面に向き合う時間を減らす工夫があることが望ましい.ファシリテーターが画面を見るべきタイミングは議論の話題が変化したときである.以前の議論の内容から外れた発言がされた時,ファシリテーターが適切な発言をすることで,脱線や炎上を避けて議論を収束させることができる.
すなわち,ファシリテーターの代わりに自動的に議論中の話題の変化を観測することが求められている.
%
現在,COLLAGREE上で使用されている議論支援システムは「(1)投稿支援システム」と「(2)議論可視化システム」の2つに大別できる.
投稿支援システムはポイント機能やファシリテーションフレーズ簡易投稿機能のように,ユーザーが投稿をする際に何らかの補助やリアクションを行う.現行の機能では選択肢の提示に留まっており,作業量を減らすことには繋がりにくい.
一方,議論可視化システムは議論ツリーやキーワード抽出のように,ユーザーにスレッドとは異なる議論の見方を提供する.
\ref{Fig:argTree1}に議論ツリーの例を示す.
\begin{figure}[htbp]
 \begin{center}
  \includegraphics[width=\textwidth]{../images/2.Related_Work/argTree1.png}
  \caption{議論ツリー}
  \label{Fig:argTree1}
  \vspace{-10pt}
 \end{center}
\end{figure}
現行の機能では議論を見やすくすることに重点が置かれており,議論の把握の助けにはなるが画面に向き合う時間を減らすことにはなりにくい.むしろ,作業量を増やすことになり得ることもある.
従って,現行の支援機能ではファシリテーターの作業量の減少には繋がりにくい.
%
近年,自然言語処理の分野において分散表現が多くの研究で使われており,機械翻訳を始めとする単語の意味が重要となる分野で精度の向上が確認されている.分散表現を用いることで,人間に近い精度で話題の変化を観測することが可能となる.
%
以上のような背景を踏まえて,分散表現を用いて,話題の変化を観測し,話題の変化が確認された時にファシリテーターに伝えることが望ましい.
話題の変化の観測は,発言中に現れる単語の類似度の計算と見なすことができる.
分散表現を用いることで単語間の類似度を求めることができる,値が大きいほど単語がそれぞれ類似した実数ベクトルであることを表す.単語Aと単語Bの実数ベクトルが類似しているとは,単語Aと共に使われることの多い単語と単語Bと共に使われることの多い単語が多く共通していることを示す.故に,分散表現を使って単語の類似度を計算することができる.
%
発言文から単語を選ぶ際には自動要約を用いる.発言文から重要でない単語を取り除くことで関連度の計算の精度を高めることが可能となる.
要約の手法としてはokapi BM25 \cite{okapiBM25}とLexRankを組み合わせた抽出的要約手法を用いる.
\begin{comment}
%======================================= 社会的背景
2013年頃からWeb上での大規模な議論活動が活発になり,大規模な人数での議論が期待されている.
大規模な議論では意見を共有することは可能であるが,議論を整理させることや収束させることは難しい.以上から大規模意見集約システムCOLLAGREEが開発された.本システムではWeb上で適切に大規模な議論を行うことができるように議論をマネジメントするファシリテーターを導入した\cite{collagreeTest}.
過去の実験ではファシリテーターの存在が議論の集約に大きな役割を果たしていることが認識されており,大規模な議論のためにファシリテータは必要である.しかし,議論の規模に伴って議論時間が長くなる傾向があり,同時にファシリテーターは常に議論の動向を見続ける必要がある.故に,議論の規模が大きくなればなるほどファシリテーターは長時間かつ大規模な議論の動向の監視によって大きな負担がかかる.大規模な議論が増加する傾向を踏まえるとファシリテーターにかかる負担を軽減する支援が必要である.\\
以上の問題を解決するため,話題の変化を追い,重要な話題の転換点をファシリテーターの代わりに検出することが有用であると考える.必要な時にだけファシリテーターが画面を見れば良いようにすることでファシリテーターの負担軽減が期待できる.
%========================================= 現行手法問題点背景
%議論支援に関する先行研究において,既存の手法は全てが文字列を文字列のまま扱う手法である.
%既存手法は殆どがパターンマッチングと重み付けの2つに区分することができる.
%パターンマッチングでは事前に単語を登録して,単語がマッチした場合に処理を行うが,処理それぞれに対して単語を登録しなければならず手間が膨大になってしまう.また,単語の意味が考慮されておらず,手作業で登録を行うので登録漏れがあった場合に単語の意味に関係なく処理を行うことが不可能となってしまう.
%重み付けは単語の出現頻度や文章の長さを使用して単語・文章に順位を付ける手法で必ずしも単語の登録が必要でないため多くの研究で使用されている.
%しかし,重み付けもまた単語を文字列のまま扱っており,意味までは考慮されていない.故に ,人間なら対応できる似た単語でも1文字違うだけで対処が困難となる.
議論支援に関する先行研究においてファシリテーターに対する支援を目的としたものは無く,殆どが議論の活性化や可視化を目的としている.
%=================================新手法
近年,自然言語処理の分野において分散表現が多くの研究で使われており.分散表現は文字列である単語を辞書データを使用して実数ベクトルへと変換する.辞書データにない単語には対応できないが,多様な処理を1つの辞書データで行うことができる.また,実数ベクトルの各数値が単語の意味を表現するものとなっており,数値を使用して処理を行うことができる.
分散表現を用いることで既存手法より人間の感覚に近しい処理を行うことができる.
%=================================
以上のような背景を踏まえて,分散表現を用いてファシリテーターの代わりに話題の変化を判定し,知らせることを目指す.
話題転換の検出は発言同士の近さ,すなわち発言に含まれる単語意味の近さと見ることができる.
分散表現ではベクトル同士の内積計算を行うことで単語同士の意味の近さを計算することができる.
また,分散表現を使用することで機械翻訳を始めとする複数の分野で精度の向上が確認されている.
\end{comment}
\section{研究の目的}
\label{intro:taget}
本論文では,分散表現を用いて議論中での発言に含まれる単語の関連度を計算し,話題の変化を観測する手法を提案する.

\section{本論文の構成}
本論文の構成を以下に示す.
\ref{relwork:chapter} 章では要約手法に関する研究と,分散表現に関する先行研究を紹介する.
次に,\ref{model:chapter}章では発言の要約手法の説明を行い,\ref{impl:chapter}章では分散表現を用いた単語集合間の関連度計算について説明する.
そして,\ref{exp:chapter}章では話題転換点の検出の評価実験について説明する.
最後に\ref{con:chapter}章で本論文のまとめと考察を示す.

 %-------------------------------------------------------------------------------
 \expandafter\ifx\csname MasterFile\endcsname\relax
	\def\BibFile{hoge}
	\input{../Bibliography/chapter}
  \fi
  %-------------------------------------------------------------------------------
  \expandafter\ifx\csname MasterFile\endcsname\relax
  \end{document}
  \fi

  \fi
  %-------------------------------------------------------------------------------
  \expandafter\ifx\csname MasterFile\endcsname\relax
  \end{document}
  \fi
 %
%===============================================================================
\end{document}
\fi

\begin{document}
\setcounter{chapter}{4}
\fi
%-------------------------------------------------------------------------------
\cleardoublepage
\chapter{評価実験}

\label{exp:chapter}

\section{序言}
\label{exp:introduction}
本章では,COLLAGREEで行われた議論のデータを対象にした提案手法の評価実験について述べる.評価実験では同じテーマのもと行われた複数の議論を用意し,提案手法の有用性を示す.%提案手法は\ref{}で述べた手法を用い,比較手法は\ref{}で述べた関連研究を用いる.
以下に本章の構成を述べる.
\clearpage
\section{対象データ}
\label{exp:data}
\subsection{議論データ}
\label{exp:data:discussion}
議論データはCOLLAGREE上で行われた別の実験での議論のものを使用する.データの概要を以下に示す.
\\
\noindent{\bf【実験概要】}
\begin{description}
\item [グループ人数]:2$\sim$3名
\item [議論時間]:90分前後
\item [議論テーマ]:外国人観光客向けの日本旅行プランの決定
\item [議論テーマ説明文]:みなさまに、外国人観光客向けの日本旅行プランを立てていただきます。 想定される旅行者の条件は以下の通りです。
\begin{itemize}
\item 英語は話せるが、日本語は話せない
\item 初めての日本旅行である
\item 日程は6泊7日
\item ホテルは自分たちで手配できる
\item 旅行のために貯金したので、金銭的には余裕があり、国内をいろいろとまわることが可能である
\item 来日、帰国の際の空港は、どこでもかまわない
\item 2つのプランを比較したいと考えている(プランは2つ用意してください)
\end{itemize}
\item [ファシリテータ]:あり
\end{description}

\subsection{評価データ}
\label{exp:data:evaluation}
 \ref{exp:data:discussion}節で説明した議論データに対し,次に述べる基準でアノテーションを行ってもらった.基準を満たすと思われる発言に"1"のタグを,満たすと思われない発言に"0"のタグを付ける.
\subsubsection*{\circled{1} それまで話題となっていた対象や事態とは異なる,新しい対象や事態への言及する発言}
話されている内容が,以前と全く異なる対象や事態へと移行する位置でデータを区切る.
\paragraph{例1:}\ \\
(今までの話題:パック旅行はなぜ安いのかについて)
\begin{itemize}
\item A:ホテルが宿泊費の一部を出しているから安いのかな?
\item B:おそらく。
\item A:なるほど。
\item B:\underline{沖縄行きも安いね。}(今まで沖縄の話はされておらず,この後“沖縄行きのパック旅行”に話題が変わる(かもしれない))
\end{itemize}
\paragraph{例2:}\ \\
(今までの話題:外国人のツアー旅行の行き先について)
\begin{itemize}
\item A:\underline{他は寄らなくてもよいですか?}(新しい行き先が出るように仕向けている)
\end{itemize}

 \subsubsection*{\circled{2} 既に言及された対象や事態の異なる側面への言及する発言}
既に話題として取り上げられることについて,以前とは異なる側面から言及がなされる位置で区切る.
\paragraph{例3:}\ \\
(今までの話題:外国人のツアー旅行の行き先について)
\begin{itemize}
\item A:広島、長崎はどう?
\item B:外国人観光客とか広島、長崎で見かけた覚えないな。
\item A:\underline{ツアーに英語を話せるスタッフとか付けられるかな?}(“ツアー旅行のスタッフ”に話題が変わる(かもしれない))
\end{itemize}

 \subsubsection*{\circled{3}議論のフェーズを移行させる(かもしれない)発言}
議論のフェーズを今までから移行させる(と思われる)発言の位置で区切る.
\paragraph{例4:}\ \\
(今までの話題:外国人のツアー旅行の行き先について)
\begin{itemize}
\item A:八坂神社や清水寺など有名どころがたくさんありますし、魅力的だと思います
\item B:\ulinej{京都周辺ツアー 清水寺、金閣寺、銀閣寺、伏見稲荷大社、嵐山、など日本の建物や食べ物など広島長崎ツアー 広島、長崎の戦争の地を見る事と、それぞれの場所で食べ物建造物を見るツアー}(地名を挙げる段階から,各地点を結ぶツアープランへの作成段階に話題が変わる(かもしれない).)
\end{itemize}

\paragraph{例5:}\ \\
(今までの話題:外国人のツアー旅行のプランについて)
\begin{itemize}
\item A:京都周辺ツアー
 清水寺、清水焼体験、抹茶・和菓子など体験、きもの体験、金閣寺、嵐山、伏見稲荷大社
 その中で乗れそうなら屋形船などはどうでしょうか?
\item B:屋形船、風情があって良いと思います。
\item (途中省略)
\item C: \ulinej{まとめると、
 ・京都周辺ツアー
 京都周辺(八坂神社、清水寺、金閣寺、銀閣寺、伏見稲荷大社、嵐山、有馬温泉)、おいしい料理(豆腐など)、温泉、6泊7日ツアー}
 \\
 \ulinej{
 ・広島長崎ツアー
 広島(3日):広島の原爆ドーム、平和記念公園、厳島神社、もみじまんじゅう、牡蠣、広島の筆(メイクや書道なので使用する)、お好み焼き、呉の戦艦、アナゴ
 (移動1日)
 長崎(3日):ハウステンボス、グラバー園、眼鏡橋、大浦天主堂、軍艦島、長崎ちゃんぽん、佐世保バーガー}
 \\
 \ulinej{
 この2プランで問題ないでしょうか?}(初めて,2つのツアーの内容をまとめ,議論の収束に近づけた.)
 \end{itemize}
また,ファシリテーターによる議論をコントロールするような発言も含む.
\paragraph{例6:}\ \\
\begin{itemize}
\item F:もし現在の旅先候補でよろしければ、具体的なプランづくりに移行したいと思います。
 よろしいでしょうか?
\item F:残り20分を切りました。
 皆様、いかがでしょうか?
\end{itemize}
以上の基準に沿ってタグを付けてもらい,"1"のタグが過半数以上付けられた発言を正解値=1,他を正解値=0とした.

\section{実験設定}
\subsection{パラメーター}
本実験ではパラメーターは次の通りに設定した.前処理にて用いるokapiBM25のパラメーターはk1=2, b=0.75とし,LexRankのパラメーターはn=50,threshold=0.7とした.
分散表現として用いるfastTextは次元数を100次元とし,学習データにはwikipediaダンプデータを用いた.
総合類似度の計算に用いるパラメーターはmaxTime=5400(90分),tWeight = 0.5とし,総合類似度の閾値は0.8とした.
表5.1 に実験の設定をまとめる.
\begin{table}[htbp]
\begin{center}
  \begin{tabular}{| c | c |  c |} \hline
      & k1 &  2 \\ \cline{2-3}
    okapiBM25 & b & 0.75 \\ \cline{2-3} \hline
     & n &  50 \\ \cline{2-3}
    LexRank & threshold & 0.7 \\ \cline{2-3} \hline
     & 次元 &  10 \\ \cline{2-3}
    fastText & 学習データ & wikipediaダンプデータ \\ \cline{2-3} \hline
     \multicolumn{2}{|c|}{maxTime} & 5400 \\  \hline
     \multicolumn{2}{|c|}{tWeight} & 0.5 \\  \hline
     \multicolumn{2}{|c|}{類似度閾値} & 0.8 \\  \hline
  \end{tabular}
  \caption{パラメーターの設定}
  \label{table:par}
  \end{center}
\end{table}

\subsection{比較手法}
\subsubsection*{\circled{1} 常時通知}
最も単純かつ分かりやすい比較手法として,発言の内容に関係なく常に通知を行う手法を用いる.
\subsubsection*{\circled{2} TF-IDFベクトル}
単語の意味は考慮せず出現頻度に基づく比較手法として,分散表現の代わりにTF-IDFで発言をベクトル化する手法を用いる.\textbf{Algorithm\ref{algo:combinedWeight}}の\ref{algo:combinedWight:bm25}行目でokapiBM25の代わりにTF-IDFを用いて連想配列を求め,重みのベクトルに変換する.発言内容の類似度計算は提案手法と同じでCosine類似度を用い,以降も同じである.
\subsection{評価指標}
本実験では評価指標として適合率(Precision),再現率(Recall),F値(F-measure)の3種類の指標を用いる.\\
適合率,再現率,F値はそれぞれ次のようにして求める.まず,発言の通知を行うと判定した時を予測値=1,通知を行わないと判定した時を予測値=0とおく.次に,予測値=1かつ正解値=1であるものの個数を$hits$(的中数),予測値=0かつ正解値=1であるものの個数を$misses$(見逃し数),予測値=1かつ正解値=0であるものの個数を$falseAlarms$(誤警報数)として数える.そして,式\ref{eq:Precision},式\ref{eq:Recall}及び式\ref{eq:F-measure}に従って適合率,再現率,F値を計算する.
\begin{equation}
\begin{aligned}
\label{eq:Precision}
Precision & = \frac{hits}{hits+falseAlarms}
\end{aligned}
\end{equation}
%
\begin{equation}
\begin{aligned}
\label{eq:Recall}
Recall & = \frac{hits}{hits+misses}
\end{aligned}
\end{equation}
%
\begin{equation}
\begin{aligned}
\label{eq:F-measure}
F-measure & = \frac{2 \cdot Precision \cdot Recall}{Precision+Recall}
\end{aligned}
\end{equation}

3つの値はどれも値が高いほど判定精度が高いことを示す.

\section{実験結果}
実験結果を表\ref{table:result}に示す.
\begin{table}[htb]
\begin{center}
  \begin{tabular}{| c | c | c | c |} \hline
    手法 & \multicolumn{3}{| c |}{平均評価指標} \\ \cline{2-4}
     & Pre & Rec & F \\ \hline \hline
    比較手法1 &0.3 & 0.5  & 0.4 \\ \hline
    比較手法2 &0.3 & 0.5  & 0.4 \\ \hline
    比較手法3 &0.3 & 0.5  & 0.4 \\ \hline
    \end{tabular}
    \caption{実験結果}
    \label{table:result}
\end{center}
\end{table}

\section{結言}
\label{exp:conclusion}
本章では本研究で提案する話題変化の判定手法が有用であることを実験により確認した.COLLAGREEにて行われた議論データに対して基準を満たすと思われるものにタグを付けてもらい評価データとした.評価実験では発言の内容に関係なく常に通知する手法と分散表現の代わりにTF-IDFを用いて発言をベクトル化する手法を比較手法として用いた.

実験の結果,提案手法は適合率と再現率の両方でバランス良く高い結果を示すことが分かり,比較手法よりも良い結果を出すことがわかった.

 %-------------------------------------------------------------------------------
 \expandafter\ifx\csname MasterFile\endcsname\relax
	\def\BibFile{hoge}
	\expandafter\ifx\csname MasterFile\endcsname\relax
	\def\SubFile{hoge}
	\documentclass[a4j,12pt,twoside,openany]{jreport}
%\nofiles %tocファイルを更新させない
%\documentclass[12pt,a4j,twoside,openany]{jsbook}
\usepackage[dvipdfmx]{graphicx}
\usepackage{../dspc} % ベースラインスキップの指定
\usepackage{../slashbox} % 表に斜線を入れる
%\usepackage{../mediabb}
\usepackage{fancyvrb} % Verbatim環境
\usepackage{fancyhdr} % Headerの下線付き章見出し
\usepackage{here} % float[H]
\usepackage{multirow}
\usepackage{hhline} % 表の罫線の角を美しくする
\usepackage{amsmath} %コレがないとcasesが動かない
\usepackage{amsfonts} % 数学用フォント
\usepackage{bm} % 数式環境での bold
\usepackage{algorithm}
\usepackage{algorithmicx}
\usepackage[noend]{algpseudocode}
\usepackage[flushleft]{threeparttable} % 脚注付きテーブル
\usepackage{enumitem}
\usepackage{comment}
\usepackage{fancybox}
%\usepackage{csvsimple,booktabs,siunitx}
%\usepackage{filecontents}


\setlength{\evensidemargin}{5pt}
\setlength{\oddsidemargin}{40pt}
%\setlength{\headheight}{16.5pt}
%%\setlength{\headheight}{30pt}
\setcounter{secnumdepth}{3}
\setlist[description]{leftmargin=2\parindent,labelindent=\parindent}

\makeatletter
\def\@makechapterhead#1{%
	\vspace*{50\p@}%
	{
		\parindent \z@ \raggedright \normalfont
		\ifnum \c@secnumdepth >\m@ne
		% \if@mainmatter
			\huge\bfseries\@chapapp\thechapter\@chappos
			\par\nobreak
			\vskip 20\p@
		% \fi
		\fi
		\interlinepenalty\@M
		\Huge\bfseries #1\par\nobreak
		\vskip 40\p@
	}
}

%新しいコマンド定義
\newcounter{linenumber}
\newenvironment{listing}{%
  \begin{list}{%
    \small\arabic{linenumber}:}{%
      \usecounter{linenumber}%
      \setlength{\baselineskip}{18pt}%
      \setlength{\itemsep}{0pt}%
      \setlength{\parsep}{0pt}}}%
 {\end{list}}
\newcommand{\figcaption}[1]{\def\@captype{figure}\caption{#1}}
\newcommand{\tblcaption}[1]{\def\@captype{table}\caption{#1}}
\newcommand{\norm}[1]{\left\| #1 \right\|}
\newcommand{\cc}[1]{\multicolumn{1}{|c|}{#1}}
\newcommand{\circled}[1]{\raisebox{.5pt}{\textcircled{\raisebox{-.9pt} {#1}}}}
\newcommand{\specialcell}[2][c]{%
  \begin{tabular}[#1]{@{}c@{}}#2\end{tabular}}
\makeatother
%===============================================================================
\expandafter\ifx\csname SubFile\endcsname\relax
\begin{document}
\def\MasterFile{hoge}
%-------------------------------------------------------------------------------
%\maketitle
\thispagestyle{empty}
\begin{titlepage}

% 題名
\def\title{分散表現を用いた\\話題変化判定}
% 補助題名
\def\subtitle{卒業論文}
% 著者
\def\author{芳野 魁}
% 入学年度(平成)
\def\year{29}
% 学籍番号
\def\number{26115162}
% 指導教官
\def\kyoukan{伊藤 孝行}
% 指導教官役職
\def\kyoukanrank{教授}
% 提出日
\def\teisyutubi{平成29年12月28日}

\pagestyle{empty}

\begin{center}

\vspace*{20mm}
{\Large\mc 平成29年度 \hspace{7mm} 卒 業 論 文}
\vspace{15mm}

%\setlength{\unitlength}{1mm}
\begin{picture}(100,60)
  \put(0,0){\makebox(100,60){\huge\bf\shortstack{\title}}}
\end{picture}
\\
%\begin{picture}(100,5)
%  \put(0,0){\makebox(100,5){\Large\bf\shortstack{\subtitle}}}
%\end{picture}
\end{center}
\vspace{10mm}
\begin{flushright}
\begin{tabular}{ll}
{\large 提出日} & {\large {\teisyutubi}} \\
{\large 所属}  & {\large 名古屋工業大学 情報工学科} \\
{\large 指導教員} & {\large {\kyoukan} {\kyoukanrank}} \\
 & \\
{\large 入学年度} & {\large 平成26年度入学}\\
{\large 学籍番号} &{\large {\number}} \\
 & \\
%{\large 氏名} & {\huge {\author}}
{\large 氏名} & {\huge\mc {\author}}
\end{tabular}
\end{flushright}

\end{titlepage}

%\addcontentsline{toc}{chapter}{表紙}
\thispagestyle{empty}
\mbox{}\newpage
%===============================================================================
%\frontmatter
%===============================================================================
%\mainmatter
%-------------------------------------------------------------------------------
\pagenumbering{arabic}
\cleardoublepage
\expandafter\ifx\csname MasterFile\endcsname\relax
	\def\SubFile{hoge}
	\input{../thesis/thesis}
	\begin{document}
	\setcounter{chapter}{0}
	\fi
  %-------------------------------------------------------------------------------
\cleardoublepage
\chapter{序論}
\label{intro:chapter}
%本章では, 本研究を行なうに至った背景と目的について述べる.その後,本論文の構成について述べる.
\section{研究の背景}
\label{intro:background}
近年,Web上での大規模な議論活動が活発になっているが,現在一般的に使われている "2ちゃんねる" や "Twitter" といったシステムでは整理や収束を行うことが困難である.困難である原因として,議論の管理を行う者がいないことが挙げられる.
つまり,議論を整理・収束させるには議論のマネジメントを行う人物が必要である.
%
大規模意見集約システムCOLLAGREE\cite{collagreeTest}ではファシリテーターと呼ばれる人物が議論のマネジメントを行っている.
しかし,ファシリテーターは人間であり,長時間に渡って大人数での議論の動向をマネジメントし続けるのは困難である.
COLLAGREEで大規模な議論を収束させるためには,ファシリテーターが必要な時には画面を見るようにして,他の時は見なくても済むようにすることで画面に向き合う時間を減らす工夫があることが望ましい.ファシリテーターが画面を見るべきタイミングは議論の話題が変化したときである.以前の議論の内容から外れた発言がされた時,ファシリテーターが適切な発言をすることで,脱線や炎上を避けて議論を収束させることができる.
すなわち,ファシリテーターの代わりに自動的に議論中の話題の変化を観測することが求められている.
%
現在,COLLAGREE上で使用されている議論支援システムは「(1)投稿支援システム」と「(2)議論可視化システム」の2つに大別できる.
投稿支援システムはポイント機能やファシリテーションフレーズ簡易投稿機能のように,ユーザーが投稿をする際に何らかの補助やリアクションを行う.現行の機能では選択肢の提示に留まっており,作業量を減らすことには繋がりにくい.
一方,議論可視化システムは議論ツリーやキーワード抽出のように,ユーザーにスレッドとは異なる議論の見方を提供する.
\ref{Fig:argTree1}に議論ツリーの例を示す.
\begin{figure}[htbp]
 \begin{center}
  \includegraphics[width=\textwidth]{../images/2.Related_Work/argTree1.png}
  \caption{議論ツリー}
  \label{Fig:argTree1}
  \vspace{-10pt}
 \end{center}
\end{figure}
現行の機能では議論を見やすくすることに重点が置かれており,議論の把握の助けにはなるが画面に向き合う時間を減らすことにはなりにくい.むしろ,作業量を増やすことになり得ることもある.
従って,現行の支援機能ではファシリテーターの作業量の減少には繋がりにくい.
%
近年,自然言語処理の分野において分散表現が多くの研究で使われており,機械翻訳を始めとする単語の意味が重要となる分野で精度の向上が確認されている.分散表現を用いることで,人間に近い精度で話題の変化を観測することが可能となる.
%
以上のような背景を踏まえて,分散表現を用いて,話題の変化を観測し,話題の変化が確認された時にファシリテーターに伝えることが望ましい.
話題の変化の観測は,発言中に現れる単語の類似度の計算と見なすことができる.
分散表現を用いることで単語間の類似度を求めることができる,値が大きいほど単語がそれぞれ類似した実数ベクトルであることを表す.単語Aと単語Bの実数ベクトルが類似しているとは,単語Aと共に使われることの多い単語と単語Bと共に使われることの多い単語が多く共通していることを示す.故に,分散表現を使って単語の類似度を計算することができる.
%
発言文から単語を選ぶ際には自動要約を用いる.発言文から重要でない単語を取り除くことで関連度の計算の精度を高めることが可能となる.
要約の手法としてはokapi BM25 \cite{okapiBM25}とLexRankを組み合わせた抽出的要約手法を用いる.
\begin{comment}
%======================================= 社会的背景
2013年頃からWeb上での大規模な議論活動が活発になり,大規模な人数での議論が期待されている.
大規模な議論では意見を共有することは可能であるが,議論を整理させることや収束させることは難しい.以上から大規模意見集約システムCOLLAGREEが開発された.本システムではWeb上で適切に大規模な議論を行うことができるように議論をマネジメントするファシリテーターを導入した\cite{collagreeTest}.
過去の実験ではファシリテーターの存在が議論の集約に大きな役割を果たしていることが認識されており,大規模な議論のためにファシリテータは必要である.しかし,議論の規模に伴って議論時間が長くなる傾向があり,同時にファシリテーターは常に議論の動向を見続ける必要がある.故に,議論の規模が大きくなればなるほどファシリテーターは長時間かつ大規模な議論の動向の監視によって大きな負担がかかる.大規模な議論が増加する傾向を踏まえるとファシリテーターにかかる負担を軽減する支援が必要である.\\
以上の問題を解決するため,話題の変化を追い,重要な話題の転換点をファシリテーターの代わりに検出することが有用であると考える.必要な時にだけファシリテーターが画面を見れば良いようにすることでファシリテーターの負担軽減が期待できる.
%========================================= 現行手法問題点背景
%議論支援に関する先行研究において,既存の手法は全てが文字列を文字列のまま扱う手法である.
%既存手法は殆どがパターンマッチングと重み付けの2つに区分することができる.
%パターンマッチングでは事前に単語を登録して,単語がマッチした場合に処理を行うが,処理それぞれに対して単語を登録しなければならず手間が膨大になってしまう.また,単語の意味が考慮されておらず,手作業で登録を行うので登録漏れがあった場合に単語の意味に関係なく処理を行うことが不可能となってしまう.
%重み付けは単語の出現頻度や文章の長さを使用して単語・文章に順位を付ける手法で必ずしも単語の登録が必要でないため多くの研究で使用されている.
%しかし,重み付けもまた単語を文字列のまま扱っており,意味までは考慮されていない.故に ,人間なら対応できる似た単語でも1文字違うだけで対処が困難となる.
議論支援に関する先行研究においてファシリテーターに対する支援を目的としたものは無く,殆どが議論の活性化や可視化を目的としている.
%=================================新手法
近年,自然言語処理の分野において分散表現が多くの研究で使われており.分散表現は文字列である単語を辞書データを使用して実数ベクトルへと変換する.辞書データにない単語には対応できないが,多様な処理を1つの辞書データで行うことができる.また,実数ベクトルの各数値が単語の意味を表現するものとなっており,数値を使用して処理を行うことができる.
分散表現を用いることで既存手法より人間の感覚に近しい処理を行うことができる.
%=================================
以上のような背景を踏まえて,分散表現を用いてファシリテーターの代わりに話題の変化を判定し,知らせることを目指す.
話題転換の検出は発言同士の近さ,すなわち発言に含まれる単語意味の近さと見ることができる.
分散表現ではベクトル同士の内積計算を行うことで単語同士の意味の近さを計算することができる.
また,分散表現を使用することで機械翻訳を始めとする複数の分野で精度の向上が確認されている.
\end{comment}
\section{研究の目的}
\label{intro:taget}
本論文では,分散表現を用いて議論中での発言に含まれる単語の関連度を計算し,話題の変化を観測する手法を提案する.

\section{本論文の構成}
本論文の構成を以下に示す.
\ref{relwork:chapter} 章では要約手法に関する研究と,分散表現に関する先行研究を紹介する.
次に,\ref{model:chapter}章では発言の要約手法の説明を行い,\ref{impl:chapter}章では分散表現を用いた単語集合間の関連度計算について説明する.
そして,\ref{exp:chapter}章では話題転換点の検出の評価実験について説明する.
最後に\ref{con:chapter}章で本論文のまとめと考察を示す.

 %-------------------------------------------------------------------------------
 \expandafter\ifx\csname MasterFile\endcsname\relax
	\def\BibFile{hoge}
	\input{../Bibliography/chapter}
  \fi
  %-------------------------------------------------------------------------------
  \expandafter\ifx\csname MasterFile\endcsname\relax
  \end{document}
  \fi

%-------------------------------------------------------------------------------
\clearpage
\addcontentsline{toc}{chapter}{目次}
\tableofcontents

\clearpage
\addcontentsline{toc}{chapter}{図目次}
\listoffigures

\clearpage
\addcontentsline{toc}{chapter}{表目次}
\listoftables

%-------------------------------------------------------------------------------

%=====================
\pagestyle{fancy} % Headerをつける
\renewcommand{\sectionmark}[1]{\markright{\thesection\ \ \ #1}}
\renewcommand{\chaptermark}[1]{\markboth{#1}{}}
\lhead{}
\chead{}
\lfoot{}
\rfoot{}%-------------------------------------------------------------------------------
\expandafter\ifx\csname MasterFile\endcsname\relax
	\def\SubFile{hoge}
	\input{../thesis/thesis}
	\begin{document}
	\setcounter{chapter}{0}
	\fi
  %-------------------------------------------------------------------------------
\cleardoublepage
\chapter{序論}
\label{intro:chapter}
%本章では, 本研究を行なうに至った背景と目的について述べる.その後,本論文の構成について述べる.
\section{研究の背景}
\label{intro:background}
近年,Web上での大規模な議論活動が活発になっているが,現在一般的に使われている "2ちゃんねる" や "Twitter" といったシステムでは整理や収束を行うことが困難である.困難である原因として,議論の管理を行う者がいないことが挙げられる.
つまり,議論を整理・収束させるには議論のマネジメントを行う人物が必要である.
%
大規模意見集約システムCOLLAGREE\cite{collagreeTest}ではファシリテーターと呼ばれる人物が議論のマネジメントを行っている.
しかし,ファシリテーターは人間であり,長時間に渡って大人数での議論の動向をマネジメントし続けるのは困難である.
COLLAGREEで大規模な議論を収束させるためには,ファシリテーターが必要な時には画面を見るようにして,他の時は見なくても済むようにすることで画面に向き合う時間を減らす工夫があることが望ましい.ファシリテーターが画面を見るべきタイミングは議論の話題が変化したときである.以前の議論の内容から外れた発言がされた時,ファシリテーターが適切な発言をすることで,脱線や炎上を避けて議論を収束させることができる.
すなわち,ファシリテーターの代わりに自動的に議論中の話題の変化を観測することが求められている.
%
現在,COLLAGREE上で使用されている議論支援システムは「(1)投稿支援システム」と「(2)議論可視化システム」の2つに大別できる.
投稿支援システムはポイント機能やファシリテーションフレーズ簡易投稿機能のように,ユーザーが投稿をする際に何らかの補助やリアクションを行う.現行の機能では選択肢の提示に留まっており,作業量を減らすことには繋がりにくい.
一方,議論可視化システムは議論ツリーやキーワード抽出のように,ユーザーにスレッドとは異なる議論の見方を提供する.
\ref{Fig:argTree1}に議論ツリーの例を示す.
\begin{figure}[htbp]
 \begin{center}
  \includegraphics[width=\textwidth]{../images/2.Related_Work/argTree1.png}
  \caption{議論ツリー}
  \label{Fig:argTree1}
  \vspace{-10pt}
 \end{center}
\end{figure}
現行の機能では議論を見やすくすることに重点が置かれており,議論の把握の助けにはなるが画面に向き合う時間を減らすことにはなりにくい.むしろ,作業量を増やすことになり得ることもある.
従って,現行の支援機能ではファシリテーターの作業量の減少には繋がりにくい.
%
近年,自然言語処理の分野において分散表現が多くの研究で使われており,機械翻訳を始めとする単語の意味が重要となる分野で精度の向上が確認されている.分散表現を用いることで,人間に近い精度で話題の変化を観測することが可能となる.
%
以上のような背景を踏まえて,分散表現を用いて,話題の変化を観測し,話題の変化が確認された時にファシリテーターに伝えることが望ましい.
話題の変化の観測は,発言中に現れる単語の類似度の計算と見なすことができる.
分散表現を用いることで単語間の類似度を求めることができる,値が大きいほど単語がそれぞれ類似した実数ベクトルであることを表す.単語Aと単語Bの実数ベクトルが類似しているとは,単語Aと共に使われることの多い単語と単語Bと共に使われることの多い単語が多く共通していることを示す.故に,分散表現を使って単語の類似度を計算することができる.
%
発言文から単語を選ぶ際には自動要約を用いる.発言文から重要でない単語を取り除くことで関連度の計算の精度を高めることが可能となる.
要約の手法としてはokapi BM25 \cite{okapiBM25}とLexRankを組み合わせた抽出的要約手法を用いる.
\begin{comment}
%======================================= 社会的背景
2013年頃からWeb上での大規模な議論活動が活発になり,大規模な人数での議論が期待されている.
大規模な議論では意見を共有することは可能であるが,議論を整理させることや収束させることは難しい.以上から大規模意見集約システムCOLLAGREEが開発された.本システムではWeb上で適切に大規模な議論を行うことができるように議論をマネジメントするファシリテーターを導入した\cite{collagreeTest}.
過去の実験ではファシリテーターの存在が議論の集約に大きな役割を果たしていることが認識されており,大規模な議論のためにファシリテータは必要である.しかし,議論の規模に伴って議論時間が長くなる傾向があり,同時にファシリテーターは常に議論の動向を見続ける必要がある.故に,議論の規模が大きくなればなるほどファシリテーターは長時間かつ大規模な議論の動向の監視によって大きな負担がかかる.大規模な議論が増加する傾向を踏まえるとファシリテーターにかかる負担を軽減する支援が必要である.\\
以上の問題を解決するため,話題の変化を追い,重要な話題の転換点をファシリテーターの代わりに検出することが有用であると考える.必要な時にだけファシリテーターが画面を見れば良いようにすることでファシリテーターの負担軽減が期待できる.
%========================================= 現行手法問題点背景
%議論支援に関する先行研究において,既存の手法は全てが文字列を文字列のまま扱う手法である.
%既存手法は殆どがパターンマッチングと重み付けの2つに区分することができる.
%パターンマッチングでは事前に単語を登録して,単語がマッチした場合に処理を行うが,処理それぞれに対して単語を登録しなければならず手間が膨大になってしまう.また,単語の意味が考慮されておらず,手作業で登録を行うので登録漏れがあった場合に単語の意味に関係なく処理を行うことが不可能となってしまう.
%重み付けは単語の出現頻度や文章の長さを使用して単語・文章に順位を付ける手法で必ずしも単語の登録が必要でないため多くの研究で使用されている.
%しかし,重み付けもまた単語を文字列のまま扱っており,意味までは考慮されていない.故に ,人間なら対応できる似た単語でも1文字違うだけで対処が困難となる.
議論支援に関する先行研究においてファシリテーターに対する支援を目的としたものは無く,殆どが議論の活性化や可視化を目的としている.
%=================================新手法
近年,自然言語処理の分野において分散表現が多くの研究で使われており.分散表現は文字列である単語を辞書データを使用して実数ベクトルへと変換する.辞書データにない単語には対応できないが,多様な処理を1つの辞書データで行うことができる.また,実数ベクトルの各数値が単語の意味を表現するものとなっており,数値を使用して処理を行うことができる.
分散表現を用いることで既存手法より人間の感覚に近しい処理を行うことができる.
%=================================
以上のような背景を踏まえて,分散表現を用いてファシリテーターの代わりに話題の変化を判定し,知らせることを目指す.
話題転換の検出は発言同士の近さ,すなわち発言に含まれる単語意味の近さと見ることができる.
分散表現ではベクトル同士の内積計算を行うことで単語同士の意味の近さを計算することができる.
また,分散表現を使用することで機械翻訳を始めとする複数の分野で精度の向上が確認されている.
\end{comment}
\section{研究の目的}
\label{intro:taget}
本論文では,分散表現を用いて議論中での発言に含まれる単語の関連度を計算し,話題の変化を観測する手法を提案する.

\section{本論文の構成}
本論文の構成を以下に示す.
\ref{relwork:chapter} 章では要約手法に関する研究と,分散表現に関する先行研究を紹介する.
次に,\ref{model:chapter}章では発言の要約手法の説明を行い,\ref{impl:chapter}章では分散表現を用いた単語集合間の関連度計算について説明する.
そして,\ref{exp:chapter}章では話題転換点の検出の評価実験について説明する.
最後に\ref{con:chapter}章で本論文のまとめと考察を示す.

 %-------------------------------------------------------------------------------
 \expandafter\ifx\csname MasterFile\endcsname\relax
	\def\BibFile{hoge}
	\input{../Bibliography/chapter}
  \fi
  %-------------------------------------------------------------------------------
  \expandafter\ifx\csname MasterFile\endcsname\relax
  \end{document}
  \fi

%-------------------------------------------------------------------------------
\expandafter\ifx\csname MasterFile\endcsname\relax
	\def\SubFile{hoge}
	\input{../thesis/thesis}
	\begin{document}
	\setcounter{chapter}{0}
	\fi
  %-------------------------------------------------------------------------------
\cleardoublepage
\chapter{序論}
\label{intro:chapter}
%本章では, 本研究を行なうに至った背景と目的について述べる.その後,本論文の構成について述べる.
\section{研究の背景}
\label{intro:background}
近年,Web上での大規模な議論活動が活発になっているが,現在一般的に使われている "2ちゃんねる" や "Twitter" といったシステムでは整理や収束を行うことが困難である.困難である原因として,議論の管理を行う者がいないことが挙げられる.
つまり,議論を整理・収束させるには議論のマネジメントを行う人物が必要である.
%
大規模意見集約システムCOLLAGREE\cite{collagreeTest}ではファシリテーターと呼ばれる人物が議論のマネジメントを行っている.
しかし,ファシリテーターは人間であり,長時間に渡って大人数での議論の動向をマネジメントし続けるのは困難である.
COLLAGREEで大規模な議論を収束させるためには,ファシリテーターが必要な時には画面を見るようにして,他の時は見なくても済むようにすることで画面に向き合う時間を減らす工夫があることが望ましい.ファシリテーターが画面を見るべきタイミングは議論の話題が変化したときである.以前の議論の内容から外れた発言がされた時,ファシリテーターが適切な発言をすることで,脱線や炎上を避けて議論を収束させることができる.
すなわち,ファシリテーターの代わりに自動的に議論中の話題の変化を観測することが求められている.
%
現在,COLLAGREE上で使用されている議論支援システムは「(1)投稿支援システム」と「(2)議論可視化システム」の2つに大別できる.
投稿支援システムはポイント機能やファシリテーションフレーズ簡易投稿機能のように,ユーザーが投稿をする際に何らかの補助やリアクションを行う.現行の機能では選択肢の提示に留まっており,作業量を減らすことには繋がりにくい.
一方,議論可視化システムは議論ツリーやキーワード抽出のように,ユーザーにスレッドとは異なる議論の見方を提供する.
\ref{Fig:argTree1}に議論ツリーの例を示す.
\begin{figure}[htbp]
 \begin{center}
  \includegraphics[width=\textwidth]{../images/2.Related_Work/argTree1.png}
  \caption{議論ツリー}
  \label{Fig:argTree1}
  \vspace{-10pt}
 \end{center}
\end{figure}
現行の機能では議論を見やすくすることに重点が置かれており,議論の把握の助けにはなるが画面に向き合う時間を減らすことにはなりにくい.むしろ,作業量を増やすことになり得ることもある.
従って,現行の支援機能ではファシリテーターの作業量の減少には繋がりにくい.
%
近年,自然言語処理の分野において分散表現が多くの研究で使われており,機械翻訳を始めとする単語の意味が重要となる分野で精度の向上が確認されている.分散表現を用いることで,人間に近い精度で話題の変化を観測することが可能となる.
%
以上のような背景を踏まえて,分散表現を用いて,話題の変化を観測し,話題の変化が確認された時にファシリテーターに伝えることが望ましい.
話題の変化の観測は,発言中に現れる単語の類似度の計算と見なすことができる.
分散表現を用いることで単語間の類似度を求めることができる,値が大きいほど単語がそれぞれ類似した実数ベクトルであることを表す.単語Aと単語Bの実数ベクトルが類似しているとは,単語Aと共に使われることの多い単語と単語Bと共に使われることの多い単語が多く共通していることを示す.故に,分散表現を使って単語の類似度を計算することができる.
%
発言文から単語を選ぶ際には自動要約を用いる.発言文から重要でない単語を取り除くことで関連度の計算の精度を高めることが可能となる.
要約の手法としてはokapi BM25 \cite{okapiBM25}とLexRankを組み合わせた抽出的要約手法を用いる.
\begin{comment}
%======================================= 社会的背景
2013年頃からWeb上での大規模な議論活動が活発になり,大規模な人数での議論が期待されている.
大規模な議論では意見を共有することは可能であるが,議論を整理させることや収束させることは難しい.以上から大規模意見集約システムCOLLAGREEが開発された.本システムではWeb上で適切に大規模な議論を行うことができるように議論をマネジメントするファシリテーターを導入した\cite{collagreeTest}.
過去の実験ではファシリテーターの存在が議論の集約に大きな役割を果たしていることが認識されており,大規模な議論のためにファシリテータは必要である.しかし,議論の規模に伴って議論時間が長くなる傾向があり,同時にファシリテーターは常に議論の動向を見続ける必要がある.故に,議論の規模が大きくなればなるほどファシリテーターは長時間かつ大規模な議論の動向の監視によって大きな負担がかかる.大規模な議論が増加する傾向を踏まえるとファシリテーターにかかる負担を軽減する支援が必要である.\\
以上の問題を解決するため,話題の変化を追い,重要な話題の転換点をファシリテーターの代わりに検出することが有用であると考える.必要な時にだけファシリテーターが画面を見れば良いようにすることでファシリテーターの負担軽減が期待できる.
%========================================= 現行手法問題点背景
%議論支援に関する先行研究において,既存の手法は全てが文字列を文字列のまま扱う手法である.
%既存手法は殆どがパターンマッチングと重み付けの2つに区分することができる.
%パターンマッチングでは事前に単語を登録して,単語がマッチした場合に処理を行うが,処理それぞれに対して単語を登録しなければならず手間が膨大になってしまう.また,単語の意味が考慮されておらず,手作業で登録を行うので登録漏れがあった場合に単語の意味に関係なく処理を行うことが不可能となってしまう.
%重み付けは単語の出現頻度や文章の長さを使用して単語・文章に順位を付ける手法で必ずしも単語の登録が必要でないため多くの研究で使用されている.
%しかし,重み付けもまた単語を文字列のまま扱っており,意味までは考慮されていない.故に ,人間なら対応できる似た単語でも1文字違うだけで対処が困難となる.
議論支援に関する先行研究においてファシリテーターに対する支援を目的としたものは無く,殆どが議論の活性化や可視化を目的としている.
%=================================新手法
近年,自然言語処理の分野において分散表現が多くの研究で使われており.分散表現は文字列である単語を辞書データを使用して実数ベクトルへと変換する.辞書データにない単語には対応できないが,多様な処理を1つの辞書データで行うことができる.また,実数ベクトルの各数値が単語の意味を表現するものとなっており,数値を使用して処理を行うことができる.
分散表現を用いることで既存手法より人間の感覚に近しい処理を行うことができる.
%=================================
以上のような背景を踏まえて,分散表現を用いてファシリテーターの代わりに話題の変化を判定し,知らせることを目指す.
話題転換の検出は発言同士の近さ,すなわち発言に含まれる単語意味の近さと見ることができる.
分散表現ではベクトル同士の内積計算を行うことで単語同士の意味の近さを計算することができる.
また,分散表現を使用することで機械翻訳を始めとする複数の分野で精度の向上が確認されている.
\end{comment}
\section{研究の目的}
\label{intro:taget}
本論文では,分散表現を用いて議論中での発言に含まれる単語の関連度を計算し,話題の変化を観測する手法を提案する.

\section{本論文の構成}
本論文の構成を以下に示す.
\ref{relwork:chapter} 章では要約手法に関する研究と,分散表現に関する先行研究を紹介する.
次に,\ref{model:chapter}章では発言の要約手法の説明を行い,\ref{impl:chapter}章では分散表現を用いた単語集合間の関連度計算について説明する.
そして,\ref{exp:chapter}章では話題転換点の検出の評価実験について説明する.
最後に\ref{con:chapter}章で本論文のまとめと考察を示す.

 %-------------------------------------------------------------------------------
 \expandafter\ifx\csname MasterFile\endcsname\relax
	\def\BibFile{hoge}
	\input{../Bibliography/chapter}
  \fi
  %-------------------------------------------------------------------------------
  \expandafter\ifx\csname MasterFile\endcsname\relax
  \end{document}
  \fi

%-------------------------------------------------------------------------------
\expandafter\ifx\csname MasterFile\endcsname\relax
	\def\SubFile{hoge}
	\input{../thesis/thesis}
	\begin{document}
	\setcounter{chapter}{0}
	\fi
  %-------------------------------------------------------------------------------
\cleardoublepage
\chapter{序論}
\label{intro:chapter}
%本章では, 本研究を行なうに至った背景と目的について述べる.その後,本論文の構成について述べる.
\section{研究の背景}
\label{intro:background}
近年,Web上での大規模な議論活動が活発になっているが,現在一般的に使われている "2ちゃんねる" や "Twitter" といったシステムでは整理や収束を行うことが困難である.困難である原因として,議論の管理を行う者がいないことが挙げられる.
つまり,議論を整理・収束させるには議論のマネジメントを行う人物が必要である.
%
大規模意見集約システムCOLLAGREE\cite{collagreeTest}ではファシリテーターと呼ばれる人物が議論のマネジメントを行っている.
しかし,ファシリテーターは人間であり,長時間に渡って大人数での議論の動向をマネジメントし続けるのは困難である.
COLLAGREEで大規模な議論を収束させるためには,ファシリテーターが必要な時には画面を見るようにして,他の時は見なくても済むようにすることで画面に向き合う時間を減らす工夫があることが望ましい.ファシリテーターが画面を見るべきタイミングは議論の話題が変化したときである.以前の議論の内容から外れた発言がされた時,ファシリテーターが適切な発言をすることで,脱線や炎上を避けて議論を収束させることができる.
すなわち,ファシリテーターの代わりに自動的に議論中の話題の変化を観測することが求められている.
%
現在,COLLAGREE上で使用されている議論支援システムは「(1)投稿支援システム」と「(2)議論可視化システム」の2つに大別できる.
投稿支援システムはポイント機能やファシリテーションフレーズ簡易投稿機能のように,ユーザーが投稿をする際に何らかの補助やリアクションを行う.現行の機能では選択肢の提示に留まっており,作業量を減らすことには繋がりにくい.
一方,議論可視化システムは議論ツリーやキーワード抽出のように,ユーザーにスレッドとは異なる議論の見方を提供する.
\ref{Fig:argTree1}に議論ツリーの例を示す.
\begin{figure}[htbp]
 \begin{center}
  \includegraphics[width=\textwidth]{../images/2.Related_Work/argTree1.png}
  \caption{議論ツリー}
  \label{Fig:argTree1}
  \vspace{-10pt}
 \end{center}
\end{figure}
現行の機能では議論を見やすくすることに重点が置かれており,議論の把握の助けにはなるが画面に向き合う時間を減らすことにはなりにくい.むしろ,作業量を増やすことになり得ることもある.
従って,現行の支援機能ではファシリテーターの作業量の減少には繋がりにくい.
%
近年,自然言語処理の分野において分散表現が多くの研究で使われており,機械翻訳を始めとする単語の意味が重要となる分野で精度の向上が確認されている.分散表現を用いることで,人間に近い精度で話題の変化を観測することが可能となる.
%
以上のような背景を踏まえて,分散表現を用いて,話題の変化を観測し,話題の変化が確認された時にファシリテーターに伝えることが望ましい.
話題の変化の観測は,発言中に現れる単語の類似度の計算と見なすことができる.
分散表現を用いることで単語間の類似度を求めることができる,値が大きいほど単語がそれぞれ類似した実数ベクトルであることを表す.単語Aと単語Bの実数ベクトルが類似しているとは,単語Aと共に使われることの多い単語と単語Bと共に使われることの多い単語が多く共通していることを示す.故に,分散表現を使って単語の類似度を計算することができる.
%
発言文から単語を選ぶ際には自動要約を用いる.発言文から重要でない単語を取り除くことで関連度の計算の精度を高めることが可能となる.
要約の手法としてはokapi BM25 \cite{okapiBM25}とLexRankを組み合わせた抽出的要約手法を用いる.
\begin{comment}
%======================================= 社会的背景
2013年頃からWeb上での大規模な議論活動が活発になり,大規模な人数での議論が期待されている.
大規模な議論では意見を共有することは可能であるが,議論を整理させることや収束させることは難しい.以上から大規模意見集約システムCOLLAGREEが開発された.本システムではWeb上で適切に大規模な議論を行うことができるように議論をマネジメントするファシリテーターを導入した\cite{collagreeTest}.
過去の実験ではファシリテーターの存在が議論の集約に大きな役割を果たしていることが認識されており,大規模な議論のためにファシリテータは必要である.しかし,議論の規模に伴って議論時間が長くなる傾向があり,同時にファシリテーターは常に議論の動向を見続ける必要がある.故に,議論の規模が大きくなればなるほどファシリテーターは長時間かつ大規模な議論の動向の監視によって大きな負担がかかる.大規模な議論が増加する傾向を踏まえるとファシリテーターにかかる負担を軽減する支援が必要である.\\
以上の問題を解決するため,話題の変化を追い,重要な話題の転換点をファシリテーターの代わりに検出することが有用であると考える.必要な時にだけファシリテーターが画面を見れば良いようにすることでファシリテーターの負担軽減が期待できる.
%========================================= 現行手法問題点背景
%議論支援に関する先行研究において,既存の手法は全てが文字列を文字列のまま扱う手法である.
%既存手法は殆どがパターンマッチングと重み付けの2つに区分することができる.
%パターンマッチングでは事前に単語を登録して,単語がマッチした場合に処理を行うが,処理それぞれに対して単語を登録しなければならず手間が膨大になってしまう.また,単語の意味が考慮されておらず,手作業で登録を行うので登録漏れがあった場合に単語の意味に関係なく処理を行うことが不可能となってしまう.
%重み付けは単語の出現頻度や文章の長さを使用して単語・文章に順位を付ける手法で必ずしも単語の登録が必要でないため多くの研究で使用されている.
%しかし,重み付けもまた単語を文字列のまま扱っており,意味までは考慮されていない.故に ,人間なら対応できる似た単語でも1文字違うだけで対処が困難となる.
議論支援に関する先行研究においてファシリテーターに対する支援を目的としたものは無く,殆どが議論の活性化や可視化を目的としている.
%=================================新手法
近年,自然言語処理の分野において分散表現が多くの研究で使われており.分散表現は文字列である単語を辞書データを使用して実数ベクトルへと変換する.辞書データにない単語には対応できないが,多様な処理を1つの辞書データで行うことができる.また,実数ベクトルの各数値が単語の意味を表現するものとなっており,数値を使用して処理を行うことができる.
分散表現を用いることで既存手法より人間の感覚に近しい処理を行うことができる.
%=================================
以上のような背景を踏まえて,分散表現を用いてファシリテーターの代わりに話題の変化を判定し,知らせることを目指す.
話題転換の検出は発言同士の近さ,すなわち発言に含まれる単語意味の近さと見ることができる.
分散表現ではベクトル同士の内積計算を行うことで単語同士の意味の近さを計算することができる.
また,分散表現を使用することで機械翻訳を始めとする複数の分野で精度の向上が確認されている.
\end{comment}
\section{研究の目的}
\label{intro:taget}
本論文では,分散表現を用いて議論中での発言に含まれる単語の関連度を計算し,話題の変化を観測する手法を提案する.

\section{本論文の構成}
本論文の構成を以下に示す.
\ref{relwork:chapter} 章では要約手法に関する研究と,分散表現に関する先行研究を紹介する.
次に,\ref{model:chapter}章では発言の要約手法の説明を行い,\ref{impl:chapter}章では分散表現を用いた単語集合間の関連度計算について説明する.
そして,\ref{exp:chapter}章では話題転換点の検出の評価実験について説明する.
最後に\ref{con:chapter}章で本論文のまとめと考察を示す.

 %-------------------------------------------------------------------------------
 \expandafter\ifx\csname MasterFile\endcsname\relax
	\def\BibFile{hoge}
	\input{../Bibliography/chapter}
  \fi
  %-------------------------------------------------------------------------------
  \expandafter\ifx\csname MasterFile\endcsname\relax
  \end{document}
  \fi

%-------------------------------------------------------------------------------
\expandafter\ifx\csname MasterFile\endcsname\relax
	\def\SubFile{hoge}
	\input{../thesis/thesis}
	\begin{document}
	\setcounter{chapter}{0}
	\fi
  %-------------------------------------------------------------------------------
\cleardoublepage
\chapter{序論}
\label{intro:chapter}
%本章では, 本研究を行なうに至った背景と目的について述べる.その後,本論文の構成について述べる.
\section{研究の背景}
\label{intro:background}
近年,Web上での大規模な議論活動が活発になっているが,現在一般的に使われている "2ちゃんねる" や "Twitter" といったシステムでは整理や収束を行うことが困難である.困難である原因として,議論の管理を行う者がいないことが挙げられる.
つまり,議論を整理・収束させるには議論のマネジメントを行う人物が必要である.
%
大規模意見集約システムCOLLAGREE\cite{collagreeTest}ではファシリテーターと呼ばれる人物が議論のマネジメントを行っている.
しかし,ファシリテーターは人間であり,長時間に渡って大人数での議論の動向をマネジメントし続けるのは困難である.
COLLAGREEで大規模な議論を収束させるためには,ファシリテーターが必要な時には画面を見るようにして,他の時は見なくても済むようにすることで画面に向き合う時間を減らす工夫があることが望ましい.ファシリテーターが画面を見るべきタイミングは議論の話題が変化したときである.以前の議論の内容から外れた発言がされた時,ファシリテーターが適切な発言をすることで,脱線や炎上を避けて議論を収束させることができる.
すなわち,ファシリテーターの代わりに自動的に議論中の話題の変化を観測することが求められている.
%
現在,COLLAGREE上で使用されている議論支援システムは「(1)投稿支援システム」と「(2)議論可視化システム」の2つに大別できる.
投稿支援システムはポイント機能やファシリテーションフレーズ簡易投稿機能のように,ユーザーが投稿をする際に何らかの補助やリアクションを行う.現行の機能では選択肢の提示に留まっており,作業量を減らすことには繋がりにくい.
一方,議論可視化システムは議論ツリーやキーワード抽出のように,ユーザーにスレッドとは異なる議論の見方を提供する.
\ref{Fig:argTree1}に議論ツリーの例を示す.
\begin{figure}[htbp]
 \begin{center}
  \includegraphics[width=\textwidth]{../images/2.Related_Work/argTree1.png}
  \caption{議論ツリー}
  \label{Fig:argTree1}
  \vspace{-10pt}
 \end{center}
\end{figure}
現行の機能では議論を見やすくすることに重点が置かれており,議論の把握の助けにはなるが画面に向き合う時間を減らすことにはなりにくい.むしろ,作業量を増やすことになり得ることもある.
従って,現行の支援機能ではファシリテーターの作業量の減少には繋がりにくい.
%
近年,自然言語処理の分野において分散表現が多くの研究で使われており,機械翻訳を始めとする単語の意味が重要となる分野で精度の向上が確認されている.分散表現を用いることで,人間に近い精度で話題の変化を観測することが可能となる.
%
以上のような背景を踏まえて,分散表現を用いて,話題の変化を観測し,話題の変化が確認された時にファシリテーターに伝えることが望ましい.
話題の変化の観測は,発言中に現れる単語の類似度の計算と見なすことができる.
分散表現を用いることで単語間の類似度を求めることができる,値が大きいほど単語がそれぞれ類似した実数ベクトルであることを表す.単語Aと単語Bの実数ベクトルが類似しているとは,単語Aと共に使われることの多い単語と単語Bと共に使われることの多い単語が多く共通していることを示す.故に,分散表現を使って単語の類似度を計算することができる.
%
発言文から単語を選ぶ際には自動要約を用いる.発言文から重要でない単語を取り除くことで関連度の計算の精度を高めることが可能となる.
要約の手法としてはokapi BM25 \cite{okapiBM25}とLexRankを組み合わせた抽出的要約手法を用いる.
\begin{comment}
%======================================= 社会的背景
2013年頃からWeb上での大規模な議論活動が活発になり,大規模な人数での議論が期待されている.
大規模な議論では意見を共有することは可能であるが,議論を整理させることや収束させることは難しい.以上から大規模意見集約システムCOLLAGREEが開発された.本システムではWeb上で適切に大規模な議論を行うことができるように議論をマネジメントするファシリテーターを導入した\cite{collagreeTest}.
過去の実験ではファシリテーターの存在が議論の集約に大きな役割を果たしていることが認識されており,大規模な議論のためにファシリテータは必要である.しかし,議論の規模に伴って議論時間が長くなる傾向があり,同時にファシリテーターは常に議論の動向を見続ける必要がある.故に,議論の規模が大きくなればなるほどファシリテーターは長時間かつ大規模な議論の動向の監視によって大きな負担がかかる.大規模な議論が増加する傾向を踏まえるとファシリテーターにかかる負担を軽減する支援が必要である.\\
以上の問題を解決するため,話題の変化を追い,重要な話題の転換点をファシリテーターの代わりに検出することが有用であると考える.必要な時にだけファシリテーターが画面を見れば良いようにすることでファシリテーターの負担軽減が期待できる.
%========================================= 現行手法問題点背景
%議論支援に関する先行研究において,既存の手法は全てが文字列を文字列のまま扱う手法である.
%既存手法は殆どがパターンマッチングと重み付けの2つに区分することができる.
%パターンマッチングでは事前に単語を登録して,単語がマッチした場合に処理を行うが,処理それぞれに対して単語を登録しなければならず手間が膨大になってしまう.また,単語の意味が考慮されておらず,手作業で登録を行うので登録漏れがあった場合に単語の意味に関係なく処理を行うことが不可能となってしまう.
%重み付けは単語の出現頻度や文章の長さを使用して単語・文章に順位を付ける手法で必ずしも単語の登録が必要でないため多くの研究で使用されている.
%しかし,重み付けもまた単語を文字列のまま扱っており,意味までは考慮されていない.故に ,人間なら対応できる似た単語でも1文字違うだけで対処が困難となる.
議論支援に関する先行研究においてファシリテーターに対する支援を目的としたものは無く,殆どが議論の活性化や可視化を目的としている.
%=================================新手法
近年,自然言語処理の分野において分散表現が多くの研究で使われており.分散表現は文字列である単語を辞書データを使用して実数ベクトルへと変換する.辞書データにない単語には対応できないが,多様な処理を1つの辞書データで行うことができる.また,実数ベクトルの各数値が単語の意味を表現するものとなっており,数値を使用して処理を行うことができる.
分散表現を用いることで既存手法より人間の感覚に近しい処理を行うことができる.
%=================================
以上のような背景を踏まえて,分散表現を用いてファシリテーターの代わりに話題の変化を判定し,知らせることを目指す.
話題転換の検出は発言同士の近さ,すなわち発言に含まれる単語意味の近さと見ることができる.
分散表現ではベクトル同士の内積計算を行うことで単語同士の意味の近さを計算することができる.
また,分散表現を使用することで機械翻訳を始めとする複数の分野で精度の向上が確認されている.
\end{comment}
\section{研究の目的}
\label{intro:taget}
本論文では,分散表現を用いて議論中での発言に含まれる単語の関連度を計算し,話題の変化を観測する手法を提案する.

\section{本論文の構成}
本論文の構成を以下に示す.
\ref{relwork:chapter} 章では要約手法に関する研究と,分散表現に関する先行研究を紹介する.
次に,\ref{model:chapter}章では発言の要約手法の説明を行い,\ref{impl:chapter}章では分散表現を用いた単語集合間の関連度計算について説明する.
そして,\ref{exp:chapter}章では話題転換点の検出の評価実験について説明する.
最後に\ref{con:chapter}章で本論文のまとめと考察を示す.

 %-------------------------------------------------------------------------------
 \expandafter\ifx\csname MasterFile\endcsname\relax
	\def\BibFile{hoge}
	\input{../Bibliography/chapter}
  \fi
  %-------------------------------------------------------------------------------
  \expandafter\ifx\csname MasterFile\endcsname\relax
  \end{document}
  \fi

%-------------------------------------------------------------------------------
\expandafter\ifx\csname MasterFile\endcsname\relax
	\def\SubFile{hoge}
	\input{../thesis/thesis}
	\begin{document}
	\setcounter{chapter}{0}
	\fi
  %-------------------------------------------------------------------------------
\cleardoublepage
\chapter{序論}
\label{intro:chapter}
%本章では, 本研究を行なうに至った背景と目的について述べる.その後,本論文の構成について述べる.
\section{研究の背景}
\label{intro:background}
近年,Web上での大規模な議論活動が活発になっているが,現在一般的に使われている "2ちゃんねる" や "Twitter" といったシステムでは整理や収束を行うことが困難である.困難である原因として,議論の管理を行う者がいないことが挙げられる.
つまり,議論を整理・収束させるには議論のマネジメントを行う人物が必要である.
%
大規模意見集約システムCOLLAGREE\cite{collagreeTest}ではファシリテーターと呼ばれる人物が議論のマネジメントを行っている.
しかし,ファシリテーターは人間であり,長時間に渡って大人数での議論の動向をマネジメントし続けるのは困難である.
COLLAGREEで大規模な議論を収束させるためには,ファシリテーターが必要な時には画面を見るようにして,他の時は見なくても済むようにすることで画面に向き合う時間を減らす工夫があることが望ましい.ファシリテーターが画面を見るべきタイミングは議論の話題が変化したときである.以前の議論の内容から外れた発言がされた時,ファシリテーターが適切な発言をすることで,脱線や炎上を避けて議論を収束させることができる.
すなわち,ファシリテーターの代わりに自動的に議論中の話題の変化を観測することが求められている.
%
現在,COLLAGREE上で使用されている議論支援システムは「(1)投稿支援システム」と「(2)議論可視化システム」の2つに大別できる.
投稿支援システムはポイント機能やファシリテーションフレーズ簡易投稿機能のように,ユーザーが投稿をする際に何らかの補助やリアクションを行う.現行の機能では選択肢の提示に留まっており,作業量を減らすことには繋がりにくい.
一方,議論可視化システムは議論ツリーやキーワード抽出のように,ユーザーにスレッドとは異なる議論の見方を提供する.
\ref{Fig:argTree1}に議論ツリーの例を示す.
\begin{figure}[htbp]
 \begin{center}
  \includegraphics[width=\textwidth]{../images/2.Related_Work/argTree1.png}
  \caption{議論ツリー}
  \label{Fig:argTree1}
  \vspace{-10pt}
 \end{center}
\end{figure}
現行の機能では議論を見やすくすることに重点が置かれており,議論の把握の助けにはなるが画面に向き合う時間を減らすことにはなりにくい.むしろ,作業量を増やすことになり得ることもある.
従って,現行の支援機能ではファシリテーターの作業量の減少には繋がりにくい.
%
近年,自然言語処理の分野において分散表現が多くの研究で使われており,機械翻訳を始めとする単語の意味が重要となる分野で精度の向上が確認されている.分散表現を用いることで,人間に近い精度で話題の変化を観測することが可能となる.
%
以上のような背景を踏まえて,分散表現を用いて,話題の変化を観測し,話題の変化が確認された時にファシリテーターに伝えることが望ましい.
話題の変化の観測は,発言中に現れる単語の類似度の計算と見なすことができる.
分散表現を用いることで単語間の類似度を求めることができる,値が大きいほど単語がそれぞれ類似した実数ベクトルであることを表す.単語Aと単語Bの実数ベクトルが類似しているとは,単語Aと共に使われることの多い単語と単語Bと共に使われることの多い単語が多く共通していることを示す.故に,分散表現を使って単語の類似度を計算することができる.
%
発言文から単語を選ぶ際には自動要約を用いる.発言文から重要でない単語を取り除くことで関連度の計算の精度を高めることが可能となる.
要約の手法としてはokapi BM25 \cite{okapiBM25}とLexRankを組み合わせた抽出的要約手法を用いる.
\begin{comment}
%======================================= 社会的背景
2013年頃からWeb上での大規模な議論活動が活発になり,大規模な人数での議論が期待されている.
大規模な議論では意見を共有することは可能であるが,議論を整理させることや収束させることは難しい.以上から大規模意見集約システムCOLLAGREEが開発された.本システムではWeb上で適切に大規模な議論を行うことができるように議論をマネジメントするファシリテーターを導入した\cite{collagreeTest}.
過去の実験ではファシリテーターの存在が議論の集約に大きな役割を果たしていることが認識されており,大規模な議論のためにファシリテータは必要である.しかし,議論の規模に伴って議論時間が長くなる傾向があり,同時にファシリテーターは常に議論の動向を見続ける必要がある.故に,議論の規模が大きくなればなるほどファシリテーターは長時間かつ大規模な議論の動向の監視によって大きな負担がかかる.大規模な議論が増加する傾向を踏まえるとファシリテーターにかかる負担を軽減する支援が必要である.\\
以上の問題を解決するため,話題の変化を追い,重要な話題の転換点をファシリテーターの代わりに検出することが有用であると考える.必要な時にだけファシリテーターが画面を見れば良いようにすることでファシリテーターの負担軽減が期待できる.
%========================================= 現行手法問題点背景
%議論支援に関する先行研究において,既存の手法は全てが文字列を文字列のまま扱う手法である.
%既存手法は殆どがパターンマッチングと重み付けの2つに区分することができる.
%パターンマッチングでは事前に単語を登録して,単語がマッチした場合に処理を行うが,処理それぞれに対して単語を登録しなければならず手間が膨大になってしまう.また,単語の意味が考慮されておらず,手作業で登録を行うので登録漏れがあった場合に単語の意味に関係なく処理を行うことが不可能となってしまう.
%重み付けは単語の出現頻度や文章の長さを使用して単語・文章に順位を付ける手法で必ずしも単語の登録が必要でないため多くの研究で使用されている.
%しかし,重み付けもまた単語を文字列のまま扱っており,意味までは考慮されていない.故に ,人間なら対応できる似た単語でも1文字違うだけで対処が困難となる.
議論支援に関する先行研究においてファシリテーターに対する支援を目的としたものは無く,殆どが議論の活性化や可視化を目的としている.
%=================================新手法
近年,自然言語処理の分野において分散表現が多くの研究で使われており.分散表現は文字列である単語を辞書データを使用して実数ベクトルへと変換する.辞書データにない単語には対応できないが,多様な処理を1つの辞書データで行うことができる.また,実数ベクトルの各数値が単語の意味を表現するものとなっており,数値を使用して処理を行うことができる.
分散表現を用いることで既存手法より人間の感覚に近しい処理を行うことができる.
%=================================
以上のような背景を踏まえて,分散表現を用いてファシリテーターの代わりに話題の変化を判定し,知らせることを目指す.
話題転換の検出は発言同士の近さ,すなわち発言に含まれる単語意味の近さと見ることができる.
分散表現ではベクトル同士の内積計算を行うことで単語同士の意味の近さを計算することができる.
また,分散表現を使用することで機械翻訳を始めとする複数の分野で精度の向上が確認されている.
\end{comment}
\section{研究の目的}
\label{intro:taget}
本論文では,分散表現を用いて議論中での発言に含まれる単語の関連度を計算し,話題の変化を観測する手法を提案する.

\section{本論文の構成}
本論文の構成を以下に示す.
\ref{relwork:chapter} 章では要約手法に関する研究と,分散表現に関する先行研究を紹介する.
次に,\ref{model:chapter}章では発言の要約手法の説明を行い,\ref{impl:chapter}章では分散表現を用いた単語集合間の関連度計算について説明する.
そして,\ref{exp:chapter}章では話題転換点の検出の評価実験について説明する.
最後に\ref{con:chapter}章で本論文のまとめと考察を示す.

 %-------------------------------------------------------------------------------
 \expandafter\ifx\csname MasterFile\endcsname\relax
	\def\BibFile{hoge}
	\input{../Bibliography/chapter}
  \fi
  %-------------------------------------------------------------------------------
  \expandafter\ifx\csname MasterFile\endcsname\relax
  \end{document}
  \fi

%-------------------------------------------------------------------------------
\expandafter\ifx\csname MasterFile\endcsname\relax
	\def\SubFile{hoge}
	\input{../thesis/thesis}
	\begin{document}
	\setcounter{chapter}{0}
	\fi
  %-------------------------------------------------------------------------------
\cleardoublepage
\chapter{序論}
\label{intro:chapter}
%本章では, 本研究を行なうに至った背景と目的について述べる.その後,本論文の構成について述べる.
\section{研究の背景}
\label{intro:background}
近年,Web上での大規模な議論活動が活発になっているが,現在一般的に使われている "2ちゃんねる" や "Twitter" といったシステムでは整理や収束を行うことが困難である.困難である原因として,議論の管理を行う者がいないことが挙げられる.
つまり,議論を整理・収束させるには議論のマネジメントを行う人物が必要である.
%
大規模意見集約システムCOLLAGREE\cite{collagreeTest}ではファシリテーターと呼ばれる人物が議論のマネジメントを行っている.
しかし,ファシリテーターは人間であり,長時間に渡って大人数での議論の動向をマネジメントし続けるのは困難である.
COLLAGREEで大規模な議論を収束させるためには,ファシリテーターが必要な時には画面を見るようにして,他の時は見なくても済むようにすることで画面に向き合う時間を減らす工夫があることが望ましい.ファシリテーターが画面を見るべきタイミングは議論の話題が変化したときである.以前の議論の内容から外れた発言がされた時,ファシリテーターが適切な発言をすることで,脱線や炎上を避けて議論を収束させることができる.
すなわち,ファシリテーターの代わりに自動的に議論中の話題の変化を観測することが求められている.
%
現在,COLLAGREE上で使用されている議論支援システムは「(1)投稿支援システム」と「(2)議論可視化システム」の2つに大別できる.
投稿支援システムはポイント機能やファシリテーションフレーズ簡易投稿機能のように,ユーザーが投稿をする際に何らかの補助やリアクションを行う.現行の機能では選択肢の提示に留まっており,作業量を減らすことには繋がりにくい.
一方,議論可視化システムは議論ツリーやキーワード抽出のように,ユーザーにスレッドとは異なる議論の見方を提供する.
\ref{Fig:argTree1}に議論ツリーの例を示す.
\begin{figure}[htbp]
 \begin{center}
  \includegraphics[width=\textwidth]{../images/2.Related_Work/argTree1.png}
  \caption{議論ツリー}
  \label{Fig:argTree1}
  \vspace{-10pt}
 \end{center}
\end{figure}
現行の機能では議論を見やすくすることに重点が置かれており,議論の把握の助けにはなるが画面に向き合う時間を減らすことにはなりにくい.むしろ,作業量を増やすことになり得ることもある.
従って,現行の支援機能ではファシリテーターの作業量の減少には繋がりにくい.
%
近年,自然言語処理の分野において分散表現が多くの研究で使われており,機械翻訳を始めとする単語の意味が重要となる分野で精度の向上が確認されている.分散表現を用いることで,人間に近い精度で話題の変化を観測することが可能となる.
%
以上のような背景を踏まえて,分散表現を用いて,話題の変化を観測し,話題の変化が確認された時にファシリテーターに伝えることが望ましい.
話題の変化の観測は,発言中に現れる単語の類似度の計算と見なすことができる.
分散表現を用いることで単語間の類似度を求めることができる,値が大きいほど単語がそれぞれ類似した実数ベクトルであることを表す.単語Aと単語Bの実数ベクトルが類似しているとは,単語Aと共に使われることの多い単語と単語Bと共に使われることの多い単語が多く共通していることを示す.故に,分散表現を使って単語の類似度を計算することができる.
%
発言文から単語を選ぶ際には自動要約を用いる.発言文から重要でない単語を取り除くことで関連度の計算の精度を高めることが可能となる.
要約の手法としてはokapi BM25 \cite{okapiBM25}とLexRankを組み合わせた抽出的要約手法を用いる.
\begin{comment}
%======================================= 社会的背景
2013年頃からWeb上での大規模な議論活動が活発になり,大規模な人数での議論が期待されている.
大規模な議論では意見を共有することは可能であるが,議論を整理させることや収束させることは難しい.以上から大規模意見集約システムCOLLAGREEが開発された.本システムではWeb上で適切に大規模な議論を行うことができるように議論をマネジメントするファシリテーターを導入した\cite{collagreeTest}.
過去の実験ではファシリテーターの存在が議論の集約に大きな役割を果たしていることが認識されており,大規模な議論のためにファシリテータは必要である.しかし,議論の規模に伴って議論時間が長くなる傾向があり,同時にファシリテーターは常に議論の動向を見続ける必要がある.故に,議論の規模が大きくなればなるほどファシリテーターは長時間かつ大規模な議論の動向の監視によって大きな負担がかかる.大規模な議論が増加する傾向を踏まえるとファシリテーターにかかる負担を軽減する支援が必要である.\\
以上の問題を解決するため,話題の変化を追い,重要な話題の転換点をファシリテーターの代わりに検出することが有用であると考える.必要な時にだけファシリテーターが画面を見れば良いようにすることでファシリテーターの負担軽減が期待できる.
%========================================= 現行手法問題点背景
%議論支援に関する先行研究において,既存の手法は全てが文字列を文字列のまま扱う手法である.
%既存手法は殆どがパターンマッチングと重み付けの2つに区分することができる.
%パターンマッチングでは事前に単語を登録して,単語がマッチした場合に処理を行うが,処理それぞれに対して単語を登録しなければならず手間が膨大になってしまう.また,単語の意味が考慮されておらず,手作業で登録を行うので登録漏れがあった場合に単語の意味に関係なく処理を行うことが不可能となってしまう.
%重み付けは単語の出現頻度や文章の長さを使用して単語・文章に順位を付ける手法で必ずしも単語の登録が必要でないため多くの研究で使用されている.
%しかし,重み付けもまた単語を文字列のまま扱っており,意味までは考慮されていない.故に ,人間なら対応できる似た単語でも1文字違うだけで対処が困難となる.
議論支援に関する先行研究においてファシリテーターに対する支援を目的としたものは無く,殆どが議論の活性化や可視化を目的としている.
%=================================新手法
近年,自然言語処理の分野において分散表現が多くの研究で使われており.分散表現は文字列である単語を辞書データを使用して実数ベクトルへと変換する.辞書データにない単語には対応できないが,多様な処理を1つの辞書データで行うことができる.また,実数ベクトルの各数値が単語の意味を表現するものとなっており,数値を使用して処理を行うことができる.
分散表現を用いることで既存手法より人間の感覚に近しい処理を行うことができる.
%=================================
以上のような背景を踏まえて,分散表現を用いてファシリテーターの代わりに話題の変化を判定し,知らせることを目指す.
話題転換の検出は発言同士の近さ,すなわち発言に含まれる単語意味の近さと見ることができる.
分散表現ではベクトル同士の内積計算を行うことで単語同士の意味の近さを計算することができる.
また,分散表現を使用することで機械翻訳を始めとする複数の分野で精度の向上が確認されている.
\end{comment}
\section{研究の目的}
\label{intro:taget}
本論文では,分散表現を用いて議論中での発言に含まれる単語の関連度を計算し,話題の変化を観測する手法を提案する.

\section{本論文の構成}
本論文の構成を以下に示す.
\ref{relwork:chapter} 章では要約手法に関する研究と,分散表現に関する先行研究を紹介する.
次に,\ref{model:chapter}章では発言の要約手法の説明を行い,\ref{impl:chapter}章では分散表現を用いた単語集合間の関連度計算について説明する.
そして,\ref{exp:chapter}章では話題転換点の検出の評価実験について説明する.
最後に\ref{con:chapter}章で本論文のまとめと考察を示す.

 %-------------------------------------------------------------------------------
 \expandafter\ifx\csname MasterFile\endcsname\relax
	\def\BibFile{hoge}
	\input{../Bibliography/chapter}
  \fi
  %-------------------------------------------------------------------------------
  \expandafter\ifx\csname MasterFile\endcsname\relax
  \end{document}
  \fi


%===============================================================================
\pagestyle{plain}
%-------------------------------------------------------------------------------
\expandafter\ifx\csname MasterFile\endcsname\relax
	\def\SubFile{hoge}
	\input{../thesis/thesis}
	\begin{document}
	\setcounter{chapter}{0}
	\fi
  %-------------------------------------------------------------------------------
\cleardoublepage
\chapter{序論}
\label{intro:chapter}
%本章では, 本研究を行なうに至った背景と目的について述べる.その後,本論文の構成について述べる.
\section{研究の背景}
\label{intro:background}
近年,Web上での大規模な議論活動が活発になっているが,現在一般的に使われている "2ちゃんねる" や "Twitter" といったシステムでは整理や収束を行うことが困難である.困難である原因として,議論の管理を行う者がいないことが挙げられる.
つまり,議論を整理・収束させるには議論のマネジメントを行う人物が必要である.
%
大規模意見集約システムCOLLAGREE\cite{collagreeTest}ではファシリテーターと呼ばれる人物が議論のマネジメントを行っている.
しかし,ファシリテーターは人間であり,長時間に渡って大人数での議論の動向をマネジメントし続けるのは困難である.
COLLAGREEで大規模な議論を収束させるためには,ファシリテーターが必要な時には画面を見るようにして,他の時は見なくても済むようにすることで画面に向き合う時間を減らす工夫があることが望ましい.ファシリテーターが画面を見るべきタイミングは議論の話題が変化したときである.以前の議論の内容から外れた発言がされた時,ファシリテーターが適切な発言をすることで,脱線や炎上を避けて議論を収束させることができる.
すなわち,ファシリテーターの代わりに自動的に議論中の話題の変化を観測することが求められている.
%
現在,COLLAGREE上で使用されている議論支援システムは「(1)投稿支援システム」と「(2)議論可視化システム」の2つに大別できる.
投稿支援システムはポイント機能やファシリテーションフレーズ簡易投稿機能のように,ユーザーが投稿をする際に何らかの補助やリアクションを行う.現行の機能では選択肢の提示に留まっており,作業量を減らすことには繋がりにくい.
一方,議論可視化システムは議論ツリーやキーワード抽出のように,ユーザーにスレッドとは異なる議論の見方を提供する.
\ref{Fig:argTree1}に議論ツリーの例を示す.
\begin{figure}[htbp]
 \begin{center}
  \includegraphics[width=\textwidth]{../images/2.Related_Work/argTree1.png}
  \caption{議論ツリー}
  \label{Fig:argTree1}
  \vspace{-10pt}
 \end{center}
\end{figure}
現行の機能では議論を見やすくすることに重点が置かれており,議論の把握の助けにはなるが画面に向き合う時間を減らすことにはなりにくい.むしろ,作業量を増やすことになり得ることもある.
従って,現行の支援機能ではファシリテーターの作業量の減少には繋がりにくい.
%
近年,自然言語処理の分野において分散表現が多くの研究で使われており,機械翻訳を始めとする単語の意味が重要となる分野で精度の向上が確認されている.分散表現を用いることで,人間に近い精度で話題の変化を観測することが可能となる.
%
以上のような背景を踏まえて,分散表現を用いて,話題の変化を観測し,話題の変化が確認された時にファシリテーターに伝えることが望ましい.
話題の変化の観測は,発言中に現れる単語の類似度の計算と見なすことができる.
分散表現を用いることで単語間の類似度を求めることができる,値が大きいほど単語がそれぞれ類似した実数ベクトルであることを表す.単語Aと単語Bの実数ベクトルが類似しているとは,単語Aと共に使われることの多い単語と単語Bと共に使われることの多い単語が多く共通していることを示す.故に,分散表現を使って単語の類似度を計算することができる.
%
発言文から単語を選ぶ際には自動要約を用いる.発言文から重要でない単語を取り除くことで関連度の計算の精度を高めることが可能となる.
要約の手法としてはokapi BM25 \cite{okapiBM25}とLexRankを組み合わせた抽出的要約手法を用いる.
\begin{comment}
%======================================= 社会的背景
2013年頃からWeb上での大規模な議論活動が活発になり,大規模な人数での議論が期待されている.
大規模な議論では意見を共有することは可能であるが,議論を整理させることや収束させることは難しい.以上から大規模意見集約システムCOLLAGREEが開発された.本システムではWeb上で適切に大規模な議論を行うことができるように議論をマネジメントするファシリテーターを導入した\cite{collagreeTest}.
過去の実験ではファシリテーターの存在が議論の集約に大きな役割を果たしていることが認識されており,大規模な議論のためにファシリテータは必要である.しかし,議論の規模に伴って議論時間が長くなる傾向があり,同時にファシリテーターは常に議論の動向を見続ける必要がある.故に,議論の規模が大きくなればなるほどファシリテーターは長時間かつ大規模な議論の動向の監視によって大きな負担がかかる.大規模な議論が増加する傾向を踏まえるとファシリテーターにかかる負担を軽減する支援が必要である.\\
以上の問題を解決するため,話題の変化を追い,重要な話題の転換点をファシリテーターの代わりに検出することが有用であると考える.必要な時にだけファシリテーターが画面を見れば良いようにすることでファシリテーターの負担軽減が期待できる.
%========================================= 現行手法問題点背景
%議論支援に関する先行研究において,既存の手法は全てが文字列を文字列のまま扱う手法である.
%既存手法は殆どがパターンマッチングと重み付けの2つに区分することができる.
%パターンマッチングでは事前に単語を登録して,単語がマッチした場合に処理を行うが,処理それぞれに対して単語を登録しなければならず手間が膨大になってしまう.また,単語の意味が考慮されておらず,手作業で登録を行うので登録漏れがあった場合に単語の意味に関係なく処理を行うことが不可能となってしまう.
%重み付けは単語の出現頻度や文章の長さを使用して単語・文章に順位を付ける手法で必ずしも単語の登録が必要でないため多くの研究で使用されている.
%しかし,重み付けもまた単語を文字列のまま扱っており,意味までは考慮されていない.故に ,人間なら対応できる似た単語でも1文字違うだけで対処が困難となる.
議論支援に関する先行研究においてファシリテーターに対する支援を目的としたものは無く,殆どが議論の活性化や可視化を目的としている.
%=================================新手法
近年,自然言語処理の分野において分散表現が多くの研究で使われており.分散表現は文字列である単語を辞書データを使用して実数ベクトルへと変換する.辞書データにない単語には対応できないが,多様な処理を1つの辞書データで行うことができる.また,実数ベクトルの各数値が単語の意味を表現するものとなっており,数値を使用して処理を行うことができる.
分散表現を用いることで既存手法より人間の感覚に近しい処理を行うことができる.
%=================================
以上のような背景を踏まえて,分散表現を用いてファシリテーターの代わりに話題の変化を判定し,知らせることを目指す.
話題転換の検出は発言同士の近さ,すなわち発言に含まれる単語意味の近さと見ることができる.
分散表現ではベクトル同士の内積計算を行うことで単語同士の意味の近さを計算することができる.
また,分散表現を使用することで機械翻訳を始めとする複数の分野で精度の向上が確認されている.
\end{comment}
\section{研究の目的}
\label{intro:taget}
本論文では,分散表現を用いて議論中での発言に含まれる単語の関連度を計算し,話題の変化を観測する手法を提案する.

\section{本論文の構成}
本論文の構成を以下に示す.
\ref{relwork:chapter} 章では要約手法に関する研究と,分散表現に関する先行研究を紹介する.
次に,\ref{model:chapter}章では発言の要約手法の説明を行い,\ref{impl:chapter}章では分散表現を用いた単語集合間の関連度計算について説明する.
そして,\ref{exp:chapter}章では話題転換点の検出の評価実験について説明する.
最後に\ref{con:chapter}章で本論文のまとめと考察を示す.

 %-------------------------------------------------------------------------------
 \expandafter\ifx\csname MasterFile\endcsname\relax
	\def\BibFile{hoge}
	\input{../Bibliography/chapter}
  \fi
  %-------------------------------------------------------------------------------
  \expandafter\ifx\csname MasterFile\endcsname\relax
  \end{document}
  \fi
 %謝辞
%-------------------------------------------------------------------------------
\def\BibFile{../Bibliograhoy/database2}
\expandafter\ifx\csname MasterFile\endcsname\relax
	\def\SubFile{hoge}
	\input{../thesis/thesis}
	\begin{document}
	\setcounter{chapter}{0}
	\fi
  %-------------------------------------------------------------------------------
\cleardoublepage
\chapter{序論}
\label{intro:chapter}
%本章では, 本研究を行なうに至った背景と目的について述べる.その後,本論文の構成について述べる.
\section{研究の背景}
\label{intro:background}
近年,Web上での大規模な議論活動が活発になっているが,現在一般的に使われている "2ちゃんねる" や "Twitter" といったシステムでは整理や収束を行うことが困難である.困難である原因として,議論の管理を行う者がいないことが挙げられる.
つまり,議論を整理・収束させるには議論のマネジメントを行う人物が必要である.
%
大規模意見集約システムCOLLAGREE\cite{collagreeTest}ではファシリテーターと呼ばれる人物が議論のマネジメントを行っている.
しかし,ファシリテーターは人間であり,長時間に渡って大人数での議論の動向をマネジメントし続けるのは困難である.
COLLAGREEで大規模な議論を収束させるためには,ファシリテーターが必要な時には画面を見るようにして,他の時は見なくても済むようにすることで画面に向き合う時間を減らす工夫があることが望ましい.ファシリテーターが画面を見るべきタイミングは議論の話題が変化したときである.以前の議論の内容から外れた発言がされた時,ファシリテーターが適切な発言をすることで,脱線や炎上を避けて議論を収束させることができる.
すなわち,ファシリテーターの代わりに自動的に議論中の話題の変化を観測することが求められている.
%
現在,COLLAGREE上で使用されている議論支援システムは「(1)投稿支援システム」と「(2)議論可視化システム」の2つに大別できる.
投稿支援システムはポイント機能やファシリテーションフレーズ簡易投稿機能のように,ユーザーが投稿をする際に何らかの補助やリアクションを行う.現行の機能では選択肢の提示に留まっており,作業量を減らすことには繋がりにくい.
一方,議論可視化システムは議論ツリーやキーワード抽出のように,ユーザーにスレッドとは異なる議論の見方を提供する.
\ref{Fig:argTree1}に議論ツリーの例を示す.
\begin{figure}[htbp]
 \begin{center}
  \includegraphics[width=\textwidth]{../images/2.Related_Work/argTree1.png}
  \caption{議論ツリー}
  \label{Fig:argTree1}
  \vspace{-10pt}
 \end{center}
\end{figure}
現行の機能では議論を見やすくすることに重点が置かれており,議論の把握の助けにはなるが画面に向き合う時間を減らすことにはなりにくい.むしろ,作業量を増やすことになり得ることもある.
従って,現行の支援機能ではファシリテーターの作業量の減少には繋がりにくい.
%
近年,自然言語処理の分野において分散表現が多くの研究で使われており,機械翻訳を始めとする単語の意味が重要となる分野で精度の向上が確認されている.分散表現を用いることで,人間に近い精度で話題の変化を観測することが可能となる.
%
以上のような背景を踏まえて,分散表現を用いて,話題の変化を観測し,話題の変化が確認された時にファシリテーターに伝えることが望ましい.
話題の変化の観測は,発言中に現れる単語の類似度の計算と見なすことができる.
分散表現を用いることで単語間の類似度を求めることができる,値が大きいほど単語がそれぞれ類似した実数ベクトルであることを表す.単語Aと単語Bの実数ベクトルが類似しているとは,単語Aと共に使われることの多い単語と単語Bと共に使われることの多い単語が多く共通していることを示す.故に,分散表現を使って単語の類似度を計算することができる.
%
発言文から単語を選ぶ際には自動要約を用いる.発言文から重要でない単語を取り除くことで関連度の計算の精度を高めることが可能となる.
要約の手法としてはokapi BM25 \cite{okapiBM25}とLexRankを組み合わせた抽出的要約手法を用いる.
\begin{comment}
%======================================= 社会的背景
2013年頃からWeb上での大規模な議論活動が活発になり,大規模な人数での議論が期待されている.
大規模な議論では意見を共有することは可能であるが,議論を整理させることや収束させることは難しい.以上から大規模意見集約システムCOLLAGREEが開発された.本システムではWeb上で適切に大規模な議論を行うことができるように議論をマネジメントするファシリテーターを導入した\cite{collagreeTest}.
過去の実験ではファシリテーターの存在が議論の集約に大きな役割を果たしていることが認識されており,大規模な議論のためにファシリテータは必要である.しかし,議論の規模に伴って議論時間が長くなる傾向があり,同時にファシリテーターは常に議論の動向を見続ける必要がある.故に,議論の規模が大きくなればなるほどファシリテーターは長時間かつ大規模な議論の動向の監視によって大きな負担がかかる.大規模な議論が増加する傾向を踏まえるとファシリテーターにかかる負担を軽減する支援が必要である.\\
以上の問題を解決するため,話題の変化を追い,重要な話題の転換点をファシリテーターの代わりに検出することが有用であると考える.必要な時にだけファシリテーターが画面を見れば良いようにすることでファシリテーターの負担軽減が期待できる.
%========================================= 現行手法問題点背景
%議論支援に関する先行研究において,既存の手法は全てが文字列を文字列のまま扱う手法である.
%既存手法は殆どがパターンマッチングと重み付けの2つに区分することができる.
%パターンマッチングでは事前に単語を登録して,単語がマッチした場合に処理を行うが,処理それぞれに対して単語を登録しなければならず手間が膨大になってしまう.また,単語の意味が考慮されておらず,手作業で登録を行うので登録漏れがあった場合に単語の意味に関係なく処理を行うことが不可能となってしまう.
%重み付けは単語の出現頻度や文章の長さを使用して単語・文章に順位を付ける手法で必ずしも単語の登録が必要でないため多くの研究で使用されている.
%しかし,重み付けもまた単語を文字列のまま扱っており,意味までは考慮されていない.故に ,人間なら対応できる似た単語でも1文字違うだけで対処が困難となる.
議論支援に関する先行研究においてファシリテーターに対する支援を目的としたものは無く,殆どが議論の活性化や可視化を目的としている.
%=================================新手法
近年,自然言語処理の分野において分散表現が多くの研究で使われており.分散表現は文字列である単語を辞書データを使用して実数ベクトルへと変換する.辞書データにない単語には対応できないが,多様な処理を1つの辞書データで行うことができる.また,実数ベクトルの各数値が単語の意味を表現するものとなっており,数値を使用して処理を行うことができる.
分散表現を用いることで既存手法より人間の感覚に近しい処理を行うことができる.
%=================================
以上のような背景を踏まえて,分散表現を用いてファシリテーターの代わりに話題の変化を判定し,知らせることを目指す.
話題転換の検出は発言同士の近さ,すなわち発言に含まれる単語意味の近さと見ることができる.
分散表現ではベクトル同士の内積計算を行うことで単語同士の意味の近さを計算することができる.
また,分散表現を使用することで機械翻訳を始めとする複数の分野で精度の向上が確認されている.
\end{comment}
\section{研究の目的}
\label{intro:taget}
本論文では,分散表現を用いて議論中での発言に含まれる単語の関連度を計算し,話題の変化を観測する手法を提案する.

\section{本論文の構成}
本論文の構成を以下に示す.
\ref{relwork:chapter} 章では要約手法に関する研究と,分散表現に関する先行研究を紹介する.
次に,\ref{model:chapter}章では発言の要約手法の説明を行い,\ref{impl:chapter}章では分散表現を用いた単語集合間の関連度計算について説明する.
そして,\ref{exp:chapter}章では話題転換点の検出の評価実験について説明する.
最後に\ref{con:chapter}章で本論文のまとめと考察を示す.

 %-------------------------------------------------------------------------------
 \expandafter\ifx\csname MasterFile\endcsname\relax
	\def\BibFile{hoge}
	\input{../Bibliography/chapter}
  \fi
  %-------------------------------------------------------------------------------
  \expandafter\ifx\csname MasterFile\endcsname\relax
  \end{document}
  \fi
 %参考文献
% %===============================================================================
\appendix
\expandafter\ifx\csname MasterFile\endcsname\relax
	\def\SubFile{hoge}
	\input{../thesis/thesis}
	\begin{document}
	\setcounter{chapter}{0}
	\fi
  %-------------------------------------------------------------------------------
\cleardoublepage
\chapter{序論}
\label{intro:chapter}
%本章では, 本研究を行なうに至った背景と目的について述べる.その後,本論文の構成について述べる.
\section{研究の背景}
\label{intro:background}
近年,Web上での大規模な議論活動が活発になっているが,現在一般的に使われている "2ちゃんねる" や "Twitter" といったシステムでは整理や収束を行うことが困難である.困難である原因として,議論の管理を行う者がいないことが挙げられる.
つまり,議論を整理・収束させるには議論のマネジメントを行う人物が必要である.
%
大規模意見集約システムCOLLAGREE\cite{collagreeTest}ではファシリテーターと呼ばれる人物が議論のマネジメントを行っている.
しかし,ファシリテーターは人間であり,長時間に渡って大人数での議論の動向をマネジメントし続けるのは困難である.
COLLAGREEで大規模な議論を収束させるためには,ファシリテーターが必要な時には画面を見るようにして,他の時は見なくても済むようにすることで画面に向き合う時間を減らす工夫があることが望ましい.ファシリテーターが画面を見るべきタイミングは議論の話題が変化したときである.以前の議論の内容から外れた発言がされた時,ファシリテーターが適切な発言をすることで,脱線や炎上を避けて議論を収束させることができる.
すなわち,ファシリテーターの代わりに自動的に議論中の話題の変化を観測することが求められている.
%
現在,COLLAGREE上で使用されている議論支援システムは「(1)投稿支援システム」と「(2)議論可視化システム」の2つに大別できる.
投稿支援システムはポイント機能やファシリテーションフレーズ簡易投稿機能のように,ユーザーが投稿をする際に何らかの補助やリアクションを行う.現行の機能では選択肢の提示に留まっており,作業量を減らすことには繋がりにくい.
一方,議論可視化システムは議論ツリーやキーワード抽出のように,ユーザーにスレッドとは異なる議論の見方を提供する.
\ref{Fig:argTree1}に議論ツリーの例を示す.
\begin{figure}[htbp]
 \begin{center}
  \includegraphics[width=\textwidth]{../images/2.Related_Work/argTree1.png}
  \caption{議論ツリー}
  \label{Fig:argTree1}
  \vspace{-10pt}
 \end{center}
\end{figure}
現行の機能では議論を見やすくすることに重点が置かれており,議論の把握の助けにはなるが画面に向き合う時間を減らすことにはなりにくい.むしろ,作業量を増やすことになり得ることもある.
従って,現行の支援機能ではファシリテーターの作業量の減少には繋がりにくい.
%
近年,自然言語処理の分野において分散表現が多くの研究で使われており,機械翻訳を始めとする単語の意味が重要となる分野で精度の向上が確認されている.分散表現を用いることで,人間に近い精度で話題の変化を観測することが可能となる.
%
以上のような背景を踏まえて,分散表現を用いて,話題の変化を観測し,話題の変化が確認された時にファシリテーターに伝えることが望ましい.
話題の変化の観測は,発言中に現れる単語の類似度の計算と見なすことができる.
分散表現を用いることで単語間の類似度を求めることができる,値が大きいほど単語がそれぞれ類似した実数ベクトルであることを表す.単語Aと単語Bの実数ベクトルが類似しているとは,単語Aと共に使われることの多い単語と単語Bと共に使われることの多い単語が多く共通していることを示す.故に,分散表現を使って単語の類似度を計算することができる.
%
発言文から単語を選ぶ際には自動要約を用いる.発言文から重要でない単語を取り除くことで関連度の計算の精度を高めることが可能となる.
要約の手法としてはokapi BM25 \cite{okapiBM25}とLexRankを組み合わせた抽出的要約手法を用いる.
\begin{comment}
%======================================= 社会的背景
2013年頃からWeb上での大規模な議論活動が活発になり,大規模な人数での議論が期待されている.
大規模な議論では意見を共有することは可能であるが,議論を整理させることや収束させることは難しい.以上から大規模意見集約システムCOLLAGREEが開発された.本システムではWeb上で適切に大規模な議論を行うことができるように議論をマネジメントするファシリテーターを導入した\cite{collagreeTest}.
過去の実験ではファシリテーターの存在が議論の集約に大きな役割を果たしていることが認識されており,大規模な議論のためにファシリテータは必要である.しかし,議論の規模に伴って議論時間が長くなる傾向があり,同時にファシリテーターは常に議論の動向を見続ける必要がある.故に,議論の規模が大きくなればなるほどファシリテーターは長時間かつ大規模な議論の動向の監視によって大きな負担がかかる.大規模な議論が増加する傾向を踏まえるとファシリテーターにかかる負担を軽減する支援が必要である.\\
以上の問題を解決するため,話題の変化を追い,重要な話題の転換点をファシリテーターの代わりに検出することが有用であると考える.必要な時にだけファシリテーターが画面を見れば良いようにすることでファシリテーターの負担軽減が期待できる.
%========================================= 現行手法問題点背景
%議論支援に関する先行研究において,既存の手法は全てが文字列を文字列のまま扱う手法である.
%既存手法は殆どがパターンマッチングと重み付けの2つに区分することができる.
%パターンマッチングでは事前に単語を登録して,単語がマッチした場合に処理を行うが,処理それぞれに対して単語を登録しなければならず手間が膨大になってしまう.また,単語の意味が考慮されておらず,手作業で登録を行うので登録漏れがあった場合に単語の意味に関係なく処理を行うことが不可能となってしまう.
%重み付けは単語の出現頻度や文章の長さを使用して単語・文章に順位を付ける手法で必ずしも単語の登録が必要でないため多くの研究で使用されている.
%しかし,重み付けもまた単語を文字列のまま扱っており,意味までは考慮されていない.故に ,人間なら対応できる似た単語でも1文字違うだけで対処が困難となる.
議論支援に関する先行研究においてファシリテーターに対する支援を目的としたものは無く,殆どが議論の活性化や可視化を目的としている.
%=================================新手法
近年,自然言語処理の分野において分散表現が多くの研究で使われており.分散表現は文字列である単語を辞書データを使用して実数ベクトルへと変換する.辞書データにない単語には対応できないが,多様な処理を1つの辞書データで行うことができる.また,実数ベクトルの各数値が単語の意味を表現するものとなっており,数値を使用して処理を行うことができる.
分散表現を用いることで既存手法より人間の感覚に近しい処理を行うことができる.
%=================================
以上のような背景を踏まえて,分散表現を用いてファシリテーターの代わりに話題の変化を判定し,知らせることを目指す.
話題転換の検出は発言同士の近さ,すなわち発言に含まれる単語意味の近さと見ることができる.
分散表現ではベクトル同士の内積計算を行うことで単語同士の意味の近さを計算することができる.
また,分散表現を使用することで機械翻訳を始めとする複数の分野で精度の向上が確認されている.
\end{comment}
\section{研究の目的}
\label{intro:taget}
本論文では,分散表現を用いて議論中での発言に含まれる単語の関連度を計算し,話題の変化を観測する手法を提案する.

\section{本論文の構成}
本論文の構成を以下に示す.
\ref{relwork:chapter} 章では要約手法に関する研究と,分散表現に関する先行研究を紹介する.
次に,\ref{model:chapter}章では発言の要約手法の説明を行い,\ref{impl:chapter}章では分散表現を用いた単語集合間の関連度計算について説明する.
そして,\ref{exp:chapter}章では話題転換点の検出の評価実験について説明する.
最後に\ref{con:chapter}章で本論文のまとめと考察を示す.

 %-------------------------------------------------------------------------------
 \expandafter\ifx\csname MasterFile\endcsname\relax
	\def\BibFile{hoge}
	\input{../Bibliography/chapter}
  \fi
  %-------------------------------------------------------------------------------
  \expandafter\ifx\csname MasterFile\endcsname\relax
  \end{document}
  \fi
 % 投稿論文リスト
\expandafter\ifx\csname MasterFile\endcsname\relax
	\def\SubFile{hoge}
	\input{../thesis/thesis}
	\begin{document}
	\setcounter{chapter}{0}
	\fi
  %-------------------------------------------------------------------------------
\cleardoublepage
\chapter{序論}
\label{intro:chapter}
%本章では, 本研究を行なうに至った背景と目的について述べる.その後,本論文の構成について述べる.
\section{研究の背景}
\label{intro:background}
近年,Web上での大規模な議論活動が活発になっているが,現在一般的に使われている "2ちゃんねる" や "Twitter" といったシステムでは整理や収束を行うことが困難である.困難である原因として,議論の管理を行う者がいないことが挙げられる.
つまり,議論を整理・収束させるには議論のマネジメントを行う人物が必要である.
%
大規模意見集約システムCOLLAGREE\cite{collagreeTest}ではファシリテーターと呼ばれる人物が議論のマネジメントを行っている.
しかし,ファシリテーターは人間であり,長時間に渡って大人数での議論の動向をマネジメントし続けるのは困難である.
COLLAGREEで大規模な議論を収束させるためには,ファシリテーターが必要な時には画面を見るようにして,他の時は見なくても済むようにすることで画面に向き合う時間を減らす工夫があることが望ましい.ファシリテーターが画面を見るべきタイミングは議論の話題が変化したときである.以前の議論の内容から外れた発言がされた時,ファシリテーターが適切な発言をすることで,脱線や炎上を避けて議論を収束させることができる.
すなわち,ファシリテーターの代わりに自動的に議論中の話題の変化を観測することが求められている.
%
現在,COLLAGREE上で使用されている議論支援システムは「(1)投稿支援システム」と「(2)議論可視化システム」の2つに大別できる.
投稿支援システムはポイント機能やファシリテーションフレーズ簡易投稿機能のように,ユーザーが投稿をする際に何らかの補助やリアクションを行う.現行の機能では選択肢の提示に留まっており,作業量を減らすことには繋がりにくい.
一方,議論可視化システムは議論ツリーやキーワード抽出のように,ユーザーにスレッドとは異なる議論の見方を提供する.
\ref{Fig:argTree1}に議論ツリーの例を示す.
\begin{figure}[htbp]
 \begin{center}
  \includegraphics[width=\textwidth]{../images/2.Related_Work/argTree1.png}
  \caption{議論ツリー}
  \label{Fig:argTree1}
  \vspace{-10pt}
 \end{center}
\end{figure}
現行の機能では議論を見やすくすることに重点が置かれており,議論の把握の助けにはなるが画面に向き合う時間を減らすことにはなりにくい.むしろ,作業量を増やすことになり得ることもある.
従って,現行の支援機能ではファシリテーターの作業量の減少には繋がりにくい.
%
近年,自然言語処理の分野において分散表現が多くの研究で使われており,機械翻訳を始めとする単語の意味が重要となる分野で精度の向上が確認されている.分散表現を用いることで,人間に近い精度で話題の変化を観測することが可能となる.
%
以上のような背景を踏まえて,分散表現を用いて,話題の変化を観測し,話題の変化が確認された時にファシリテーターに伝えることが望ましい.
話題の変化の観測は,発言中に現れる単語の類似度の計算と見なすことができる.
分散表現を用いることで単語間の類似度を求めることができる,値が大きいほど単語がそれぞれ類似した実数ベクトルであることを表す.単語Aと単語Bの実数ベクトルが類似しているとは,単語Aと共に使われることの多い単語と単語Bと共に使われることの多い単語が多く共通していることを示す.故に,分散表現を使って単語の類似度を計算することができる.
%
発言文から単語を選ぶ際には自動要約を用いる.発言文から重要でない単語を取り除くことで関連度の計算の精度を高めることが可能となる.
要約の手法としてはokapi BM25 \cite{okapiBM25}とLexRankを組み合わせた抽出的要約手法を用いる.
\begin{comment}
%======================================= 社会的背景
2013年頃からWeb上での大規模な議論活動が活発になり,大規模な人数での議論が期待されている.
大規模な議論では意見を共有することは可能であるが,議論を整理させることや収束させることは難しい.以上から大規模意見集約システムCOLLAGREEが開発された.本システムではWeb上で適切に大規模な議論を行うことができるように議論をマネジメントするファシリテーターを導入した\cite{collagreeTest}.
過去の実験ではファシリテーターの存在が議論の集約に大きな役割を果たしていることが認識されており,大規模な議論のためにファシリテータは必要である.しかし,議論の規模に伴って議論時間が長くなる傾向があり,同時にファシリテーターは常に議論の動向を見続ける必要がある.故に,議論の規模が大きくなればなるほどファシリテーターは長時間かつ大規模な議論の動向の監視によって大きな負担がかかる.大規模な議論が増加する傾向を踏まえるとファシリテーターにかかる負担を軽減する支援が必要である.\\
以上の問題を解決するため,話題の変化を追い,重要な話題の転換点をファシリテーターの代わりに検出することが有用であると考える.必要な時にだけファシリテーターが画面を見れば良いようにすることでファシリテーターの負担軽減が期待できる.
%========================================= 現行手法問題点背景
%議論支援に関する先行研究において,既存の手法は全てが文字列を文字列のまま扱う手法である.
%既存手法は殆どがパターンマッチングと重み付けの2つに区分することができる.
%パターンマッチングでは事前に単語を登録して,単語がマッチした場合に処理を行うが,処理それぞれに対して単語を登録しなければならず手間が膨大になってしまう.また,単語の意味が考慮されておらず,手作業で登録を行うので登録漏れがあった場合に単語の意味に関係なく処理を行うことが不可能となってしまう.
%重み付けは単語の出現頻度や文章の長さを使用して単語・文章に順位を付ける手法で必ずしも単語の登録が必要でないため多くの研究で使用されている.
%しかし,重み付けもまた単語を文字列のまま扱っており,意味までは考慮されていない.故に ,人間なら対応できる似た単語でも1文字違うだけで対処が困難となる.
議論支援に関する先行研究においてファシリテーターに対する支援を目的としたものは無く,殆どが議論の活性化や可視化を目的としている.
%=================================新手法
近年,自然言語処理の分野において分散表現が多くの研究で使われており.分散表現は文字列である単語を辞書データを使用して実数ベクトルへと変換する.辞書データにない単語には対応できないが,多様な処理を1つの辞書データで行うことができる.また,実数ベクトルの各数値が単語の意味を表現するものとなっており,数値を使用して処理を行うことができる.
分散表現を用いることで既存手法より人間の感覚に近しい処理を行うことができる.
%=================================
以上のような背景を踏まえて,分散表現を用いてファシリテーターの代わりに話題の変化を判定し,知らせることを目指す.
話題転換の検出は発言同士の近さ,すなわち発言に含まれる単語意味の近さと見ることができる.
分散表現ではベクトル同士の内積計算を行うことで単語同士の意味の近さを計算することができる.
また,分散表現を使用することで機械翻訳を始めとする複数の分野で精度の向上が確認されている.
\end{comment}
\section{研究の目的}
\label{intro:taget}
本論文では,分散表現を用いて議論中での発言に含まれる単語の関連度を計算し,話題の変化を観測する手法を提案する.

\section{本論文の構成}
本論文の構成を以下に示す.
\ref{relwork:chapter} 章では要約手法に関する研究と,分散表現に関する先行研究を紹介する.
次に,\ref{model:chapter}章では発言の要約手法の説明を行い,\ref{impl:chapter}章では分散表現を用いた単語集合間の関連度計算について説明する.
そして,\ref{exp:chapter}章では話題転換点の検出の評価実験について説明する.
最後に\ref{con:chapter}章で本論文のまとめと考察を示す.

 %-------------------------------------------------------------------------------
 \expandafter\ifx\csname MasterFile\endcsname\relax
	\def\BibFile{hoge}
	\input{../Bibliography/chapter}
  \fi
  %-------------------------------------------------------------------------------
  \expandafter\ifx\csname MasterFile\endcsname\relax
  \end{document}
  \fi
 %
\expandafter\ifx\csname MasterFile\endcsname\relax
	\def\SubFile{hoge}
	\input{../thesis/thesis}
	\begin{document}
	\setcounter{chapter}{0}
	\fi
  %-------------------------------------------------------------------------------
\cleardoublepage
\chapter{序論}
\label{intro:chapter}
%本章では, 本研究を行なうに至った背景と目的について述べる.その後,本論文の構成について述べる.
\section{研究の背景}
\label{intro:background}
近年,Web上での大規模な議論活動が活発になっているが,現在一般的に使われている "2ちゃんねる" や "Twitter" といったシステムでは整理や収束を行うことが困難である.困難である原因として,議論の管理を行う者がいないことが挙げられる.
つまり,議論を整理・収束させるには議論のマネジメントを行う人物が必要である.
%
大規模意見集約システムCOLLAGREE\cite{collagreeTest}ではファシリテーターと呼ばれる人物が議論のマネジメントを行っている.
しかし,ファシリテーターは人間であり,長時間に渡って大人数での議論の動向をマネジメントし続けるのは困難である.
COLLAGREEで大規模な議論を収束させるためには,ファシリテーターが必要な時には画面を見るようにして,他の時は見なくても済むようにすることで画面に向き合う時間を減らす工夫があることが望ましい.ファシリテーターが画面を見るべきタイミングは議論の話題が変化したときである.以前の議論の内容から外れた発言がされた時,ファシリテーターが適切な発言をすることで,脱線や炎上を避けて議論を収束させることができる.
すなわち,ファシリテーターの代わりに自動的に議論中の話題の変化を観測することが求められている.
%
現在,COLLAGREE上で使用されている議論支援システムは「(1)投稿支援システム」と「(2)議論可視化システム」の2つに大別できる.
投稿支援システムはポイント機能やファシリテーションフレーズ簡易投稿機能のように,ユーザーが投稿をする際に何らかの補助やリアクションを行う.現行の機能では選択肢の提示に留まっており,作業量を減らすことには繋がりにくい.
一方,議論可視化システムは議論ツリーやキーワード抽出のように,ユーザーにスレッドとは異なる議論の見方を提供する.
\ref{Fig:argTree1}に議論ツリーの例を示す.
\begin{figure}[htbp]
 \begin{center}
  \includegraphics[width=\textwidth]{../images/2.Related_Work/argTree1.png}
  \caption{議論ツリー}
  \label{Fig:argTree1}
  \vspace{-10pt}
 \end{center}
\end{figure}
現行の機能では議論を見やすくすることに重点が置かれており,議論の把握の助けにはなるが画面に向き合う時間を減らすことにはなりにくい.むしろ,作業量を増やすことになり得ることもある.
従って,現行の支援機能ではファシリテーターの作業量の減少には繋がりにくい.
%
近年,自然言語処理の分野において分散表現が多くの研究で使われており,機械翻訳を始めとする単語の意味が重要となる分野で精度の向上が確認されている.分散表現を用いることで,人間に近い精度で話題の変化を観測することが可能となる.
%
以上のような背景を踏まえて,分散表現を用いて,話題の変化を観測し,話題の変化が確認された時にファシリテーターに伝えることが望ましい.
話題の変化の観測は,発言中に現れる単語の類似度の計算と見なすことができる.
分散表現を用いることで単語間の類似度を求めることができる,値が大きいほど単語がそれぞれ類似した実数ベクトルであることを表す.単語Aと単語Bの実数ベクトルが類似しているとは,単語Aと共に使われることの多い単語と単語Bと共に使われることの多い単語が多く共通していることを示す.故に,分散表現を使って単語の類似度を計算することができる.
%
発言文から単語を選ぶ際には自動要約を用いる.発言文から重要でない単語を取り除くことで関連度の計算の精度を高めることが可能となる.
要約の手法としてはokapi BM25 \cite{okapiBM25}とLexRankを組み合わせた抽出的要約手法を用いる.
\begin{comment}
%======================================= 社会的背景
2013年頃からWeb上での大規模な議論活動が活発になり,大規模な人数での議論が期待されている.
大規模な議論では意見を共有することは可能であるが,議論を整理させることや収束させることは難しい.以上から大規模意見集約システムCOLLAGREEが開発された.本システムではWeb上で適切に大規模な議論を行うことができるように議論をマネジメントするファシリテーターを導入した\cite{collagreeTest}.
過去の実験ではファシリテーターの存在が議論の集約に大きな役割を果たしていることが認識されており,大規模な議論のためにファシリテータは必要である.しかし,議論の規模に伴って議論時間が長くなる傾向があり,同時にファシリテーターは常に議論の動向を見続ける必要がある.故に,議論の規模が大きくなればなるほどファシリテーターは長時間かつ大規模な議論の動向の監視によって大きな負担がかかる.大規模な議論が増加する傾向を踏まえるとファシリテーターにかかる負担を軽減する支援が必要である.\\
以上の問題を解決するため,話題の変化を追い,重要な話題の転換点をファシリテーターの代わりに検出することが有用であると考える.必要な時にだけファシリテーターが画面を見れば良いようにすることでファシリテーターの負担軽減が期待できる.
%========================================= 現行手法問題点背景
%議論支援に関する先行研究において,既存の手法は全てが文字列を文字列のまま扱う手法である.
%既存手法は殆どがパターンマッチングと重み付けの2つに区分することができる.
%パターンマッチングでは事前に単語を登録して,単語がマッチした場合に処理を行うが,処理それぞれに対して単語を登録しなければならず手間が膨大になってしまう.また,単語の意味が考慮されておらず,手作業で登録を行うので登録漏れがあった場合に単語の意味に関係なく処理を行うことが不可能となってしまう.
%重み付けは単語の出現頻度や文章の長さを使用して単語・文章に順位を付ける手法で必ずしも単語の登録が必要でないため多くの研究で使用されている.
%しかし,重み付けもまた単語を文字列のまま扱っており,意味までは考慮されていない.故に ,人間なら対応できる似た単語でも1文字違うだけで対処が困難となる.
議論支援に関する先行研究においてファシリテーターに対する支援を目的としたものは無く,殆どが議論の活性化や可視化を目的としている.
%=================================新手法
近年,自然言語処理の分野において分散表現が多くの研究で使われており.分散表現は文字列である単語を辞書データを使用して実数ベクトルへと変換する.辞書データにない単語には対応できないが,多様な処理を1つの辞書データで行うことができる.また,実数ベクトルの各数値が単語の意味を表現するものとなっており,数値を使用して処理を行うことができる.
分散表現を用いることで既存手法より人間の感覚に近しい処理を行うことができる.
%=================================
以上のような背景を踏まえて,分散表現を用いてファシリテーターの代わりに話題の変化を判定し,知らせることを目指す.
話題転換の検出は発言同士の近さ,すなわち発言に含まれる単語意味の近さと見ることができる.
分散表現ではベクトル同士の内積計算を行うことで単語同士の意味の近さを計算することができる.
また,分散表現を使用することで機械翻訳を始めとする複数の分野で精度の向上が確認されている.
\end{comment}
\section{研究の目的}
\label{intro:taget}
本論文では,分散表現を用いて議論中での発言に含まれる単語の関連度を計算し,話題の変化を観測する手法を提案する.

\section{本論文の構成}
本論文の構成を以下に示す.
\ref{relwork:chapter} 章では要約手法に関する研究と,分散表現に関する先行研究を紹介する.
次に,\ref{model:chapter}章では発言の要約手法の説明を行い,\ref{impl:chapter}章では分散表現を用いた単語集合間の関連度計算について説明する.
そして,\ref{exp:chapter}章では話題転換点の検出の評価実験について説明する.
最後に\ref{con:chapter}章で本論文のまとめと考察を示す.

 %-------------------------------------------------------------------------------
 \expandafter\ifx\csname MasterFile\endcsname\relax
	\def\BibFile{hoge}
	\input{../Bibliography/chapter}
  \fi
  %-------------------------------------------------------------------------------
  \expandafter\ifx\csname MasterFile\endcsname\relax
  \end{document}
  \fi
 %
%===============================================================================
\end{document}
\fi

	\begin{document}
	\setcounter{chapter}{0}
	\fi
  %-------------------------------------------------------------------------------
\cleardoublepage
\chapter{序論}
\label{intro:chapter}
%本章では, 本研究を行なうに至った背景と目的について述べる.その後,本論文の構成について述べる.
\section{研究の背景}
\label{intro:background}
近年,Web上での大規模な議論活動が活発になっているが,現在一般的に使われている "2ちゃんねる" や "Twitter" といったシステムでは整理や収束を行うことが困難である.困難である原因として,議論の管理を行う者がいないことが挙げられる.
つまり,議論を整理・収束させるには議論のマネジメントを行う人物が必要である.
%
大規模意見集約システムCOLLAGREE\cite{collagreeTest}ではファシリテーターと呼ばれる人物が議論のマネジメントを行っている.
しかし,ファシリテーターは人間であり,長時間に渡って大人数での議論の動向をマネジメントし続けるのは困難である.
COLLAGREEで大規模な議論を収束させるためには,ファシリテーターが必要な時には画面を見るようにして,他の時は見なくても済むようにすることで画面に向き合う時間を減らす工夫があることが望ましい.ファシリテーターが画面を見るべきタイミングは議論の話題が変化したときである.以前の議論の内容から外れた発言がされた時,ファシリテーターが適切な発言をすることで,脱線や炎上を避けて議論を収束させることができる.
すなわち,ファシリテーターの代わりに自動的に議論中の話題の変化を観測することが求められている.
%
現在,COLLAGREE上で使用されている議論支援システムは「(1)投稿支援システム」と「(2)議論可視化システム」の2つに大別できる.
投稿支援システムはポイント機能やファシリテーションフレーズ簡易投稿機能のように,ユーザーが投稿をする際に何らかの補助やリアクションを行う.現行の機能では選択肢の提示に留まっており,作業量を減らすことには繋がりにくい.
一方,議論可視化システムは議論ツリーやキーワード抽出のように,ユーザーにスレッドとは異なる議論の見方を提供する.
\ref{Fig:argTree1}に議論ツリーの例を示す.
\begin{figure}[htbp]
 \begin{center}
  \includegraphics[width=\textwidth]{../images/2.Related_Work/argTree1.png}
  \caption{議論ツリー}
  \label{Fig:argTree1}
  \vspace{-10pt}
 \end{center}
\end{figure}
現行の機能では議論を見やすくすることに重点が置かれており,議論の把握の助けにはなるが画面に向き合う時間を減らすことにはなりにくい.むしろ,作業量を増やすことになり得ることもある.
従って,現行の支援機能ではファシリテーターの作業量の減少には繋がりにくい.
%
近年,自然言語処理の分野において分散表現が多くの研究で使われており,機械翻訳を始めとする単語の意味が重要となる分野で精度の向上が確認されている.分散表現を用いることで,人間に近い精度で話題の変化を観測することが可能となる.
%
以上のような背景を踏まえて,分散表現を用いて,話題の変化を観測し,話題の変化が確認された時にファシリテーターに伝えることが望ましい.
話題の変化の観測は,発言中に現れる単語の類似度の計算と見なすことができる.
分散表現を用いることで単語間の類似度を求めることができる,値が大きいほど単語がそれぞれ類似した実数ベクトルであることを表す.単語Aと単語Bの実数ベクトルが類似しているとは,単語Aと共に使われることの多い単語と単語Bと共に使われることの多い単語が多く共通していることを示す.故に,分散表現を使って単語の類似度を計算することができる.
%
発言文から単語を選ぶ際には自動要約を用いる.発言文から重要でない単語を取り除くことで関連度の計算の精度を高めることが可能となる.
要約の手法としてはokapi BM25 \cite{okapiBM25}とLexRankを組み合わせた抽出的要約手法を用いる.
\begin{comment}
%======================================= 社会的背景
2013年頃からWeb上での大規模な議論活動が活発になり,大規模な人数での議論が期待されている.
大規模な議論では意見を共有することは可能であるが,議論を整理させることや収束させることは難しい.以上から大規模意見集約システムCOLLAGREEが開発された.本システムではWeb上で適切に大規模な議論を行うことができるように議論をマネジメントするファシリテーターを導入した\cite{collagreeTest}.
過去の実験ではファシリテーターの存在が議論の集約に大きな役割を果たしていることが認識されており,大規模な議論のためにファシリテータは必要である.しかし,議論の規模に伴って議論時間が長くなる傾向があり,同時にファシリテーターは常に議論の動向を見続ける必要がある.故に,議論の規模が大きくなればなるほどファシリテーターは長時間かつ大規模な議論の動向の監視によって大きな負担がかかる.大規模な議論が増加する傾向を踏まえるとファシリテーターにかかる負担を軽減する支援が必要である.\\
以上の問題を解決するため,話題の変化を追い,重要な話題の転換点をファシリテーターの代わりに検出することが有用であると考える.必要な時にだけファシリテーターが画面を見れば良いようにすることでファシリテーターの負担軽減が期待できる.
%========================================= 現行手法問題点背景
%議論支援に関する先行研究において,既存の手法は全てが文字列を文字列のまま扱う手法である.
%既存手法は殆どがパターンマッチングと重み付けの2つに区分することができる.
%パターンマッチングでは事前に単語を登録して,単語がマッチした場合に処理を行うが,処理それぞれに対して単語を登録しなければならず手間が膨大になってしまう.また,単語の意味が考慮されておらず,手作業で登録を行うので登録漏れがあった場合に単語の意味に関係なく処理を行うことが不可能となってしまう.
%重み付けは単語の出現頻度や文章の長さを使用して単語・文章に順位を付ける手法で必ずしも単語の登録が必要でないため多くの研究で使用されている.
%しかし,重み付けもまた単語を文字列のまま扱っており,意味までは考慮されていない.故に ,人間なら対応できる似た単語でも1文字違うだけで対処が困難となる.
議論支援に関する先行研究においてファシリテーターに対する支援を目的としたものは無く,殆どが議論の活性化や可視化を目的としている.
%=================================新手法
近年,自然言語処理の分野において分散表現が多くの研究で使われており.分散表現は文字列である単語を辞書データを使用して実数ベクトルへと変換する.辞書データにない単語には対応できないが,多様な処理を1つの辞書データで行うことができる.また,実数ベクトルの各数値が単語の意味を表現するものとなっており,数値を使用して処理を行うことができる.
分散表現を用いることで既存手法より人間の感覚に近しい処理を行うことができる.
%=================================
以上のような背景を踏まえて,分散表現を用いてファシリテーターの代わりに話題の変化を判定し,知らせることを目指す.
話題転換の検出は発言同士の近さ,すなわち発言に含まれる単語意味の近さと見ることができる.
分散表現ではベクトル同士の内積計算を行うことで単語同士の意味の近さを計算することができる.
また,分散表現を使用することで機械翻訳を始めとする複数の分野で精度の向上が確認されている.
\end{comment}
\section{研究の目的}
\label{intro:taget}
本論文では,分散表現を用いて議論中での発言に含まれる単語の関連度を計算し,話題の変化を観測する手法を提案する.

\section{本論文の構成}
本論文の構成を以下に示す.
\ref{relwork:chapter} 章では要約手法に関する研究と,分散表現に関する先行研究を紹介する.
次に,\ref{model:chapter}章では発言の要約手法の説明を行い,\ref{impl:chapter}章では分散表現を用いた単語集合間の関連度計算について説明する.
そして,\ref{exp:chapter}章では話題転換点の検出の評価実験について説明する.
最後に\ref{con:chapter}章で本論文のまとめと考察を示す.

 %-------------------------------------------------------------------------------
 \expandafter\ifx\csname MasterFile\endcsname\relax
	\def\BibFile{hoge}
	\expandafter\ifx\csname MasterFile\endcsname\relax
	\def\SubFile{hoge}
	\documentclass[a4j,12pt,twoside,openany]{jreport}
%\nofiles %tocファイルを更新させない
%\documentclass[12pt,a4j,twoside,openany]{jsbook}
\usepackage[dvipdfmx]{graphicx}
\usepackage{../dspc} % ベースラインスキップの指定
\usepackage{../slashbox} % 表に斜線を入れる
%\usepackage{../mediabb}
\usepackage{fancyvrb} % Verbatim環境
\usepackage{fancyhdr} % Headerの下線付き章見出し
\usepackage{here} % float[H]
\usepackage{multirow}
\usepackage{hhline} % 表の罫線の角を美しくする
\usepackage{amsmath} %コレがないとcasesが動かない
\usepackage{amsfonts} % 数学用フォント
\usepackage{bm} % 数式環境での bold
\usepackage{algorithm}
\usepackage{algorithmicx}
\usepackage[noend]{algpseudocode}
\usepackage[flushleft]{threeparttable} % 脚注付きテーブル
\usepackage{enumitem}
\usepackage{comment}
\usepackage{fancybox}
%\usepackage{csvsimple,booktabs,siunitx}
%\usepackage{filecontents}


\setlength{\evensidemargin}{5pt}
\setlength{\oddsidemargin}{40pt}
%\setlength{\headheight}{16.5pt}
%%\setlength{\headheight}{30pt}
\setcounter{secnumdepth}{3}
\setlist[description]{leftmargin=2\parindent,labelindent=\parindent}

\makeatletter
\def\@makechapterhead#1{%
	\vspace*{50\p@}%
	{
		\parindent \z@ \raggedright \normalfont
		\ifnum \c@secnumdepth >\m@ne
		% \if@mainmatter
			\huge\bfseries\@chapapp\thechapter\@chappos
			\par\nobreak
			\vskip 20\p@
		% \fi
		\fi
		\interlinepenalty\@M
		\Huge\bfseries #1\par\nobreak
		\vskip 40\p@
	}
}

%新しいコマンド定義
\newcounter{linenumber}
\newenvironment{listing}{%
  \begin{list}{%
    \small\arabic{linenumber}:}{%
      \usecounter{linenumber}%
      \setlength{\baselineskip}{18pt}%
      \setlength{\itemsep}{0pt}%
      \setlength{\parsep}{0pt}}}%
 {\end{list}}
\newcommand{\figcaption}[1]{\def\@captype{figure}\caption{#1}}
\newcommand{\tblcaption}[1]{\def\@captype{table}\caption{#1}}
\newcommand{\norm}[1]{\left\| #1 \right\|}
\newcommand{\cc}[1]{\multicolumn{1}{|c|}{#1}}
\newcommand{\circled}[1]{\raisebox{.5pt}{\textcircled{\raisebox{-.9pt} {#1}}}}
\newcommand{\specialcell}[2][c]{%
  \begin{tabular}[#1]{@{}c@{}}#2\end{tabular}}
\makeatother
%===============================================================================
\expandafter\ifx\csname SubFile\endcsname\relax
\begin{document}
\def\MasterFile{hoge}
%-------------------------------------------------------------------------------
%\maketitle
\thispagestyle{empty}
\input{../hyoushi/title}
%\addcontentsline{toc}{chapter}{表紙}
\thispagestyle{empty}
\mbox{}\newpage
%===============================================================================
%\frontmatter
%===============================================================================
%\mainmatter
%-------------------------------------------------------------------------------
\pagenumbering{arabic}
\cleardoublepage
\input{../0.Abstract/chapter}
%-------------------------------------------------------------------------------
\clearpage
\addcontentsline{toc}{chapter}{目次}
\tableofcontents

\clearpage
\addcontentsline{toc}{chapter}{図目次}
\listoffigures

\clearpage
\addcontentsline{toc}{chapter}{表目次}
\listoftables

%-------------------------------------------------------------------------------

%=====================
\pagestyle{fancy} % Headerをつける
\renewcommand{\sectionmark}[1]{\markright{\thesection\ \ \ #1}}
\renewcommand{\chaptermark}[1]{\markboth{#1}{}}
\lhead{}
\chead{}
\lfoot{}
\rfoot{}%-------------------------------------------------------------------------------
\input{../1.Introduction/chapter}
%-------------------------------------------------------------------------------
\input{../2.Related_Work/chapter}
%-------------------------------------------------------------------------------
\input{../3.The_Model/chapter}
%-------------------------------------------------------------------------------
\input{../4.Implementation/chapter}
%-------------------------------------------------------------------------------
\input{../5.Experiments/chapter}
%-------------------------------------------------------------------------------
\input{../6.Conclusion/chapter}

%===============================================================================
\pagestyle{plain}
%-------------------------------------------------------------------------------
\input{../7.Acknowledgement/chapter} %謝辞
%-------------------------------------------------------------------------------
\def\BibFile{../Bibliograhoy/database2}
\input{../Bibliography/chapter} %参考文献
% %===============================================================================
\appendix
\input{../A.Mypaper/chapter} % 投稿論文リスト
\input{../B.SIG-CCI2/chapter} %
\input{../C.IJCAI-16/chapter} %
%===============================================================================
\end{document}
\fi

	\begin{document}
	\setcounter{chapter}{0}
	\fi
  %-------------------------------------------------------------------------------
\cleardoublepage
\chapter{序論}
\label{intro:chapter}
%本章では, 本研究を行なうに至った背景と目的について述べる.その後,本論文の構成について述べる.
\section{研究の背景}
\label{intro:background}
近年,Web上での大規模な議論活動が活発になっているが,現在一般的に使われている "2ちゃんねる" や "Twitter" といったシステムでは整理や収束を行うことが困難である.困難である原因として,議論の管理を行う者がいないことが挙げられる.
つまり,議論を整理・収束させるには議論のマネジメントを行う人物が必要である.
%
大規模意見集約システムCOLLAGREE\cite{collagreeTest}ではファシリテーターと呼ばれる人物が議論のマネジメントを行っている.
しかし,ファシリテーターは人間であり,長時間に渡って大人数での議論の動向をマネジメントし続けるのは困難である.
COLLAGREEで大規模な議論を収束させるためには,ファシリテーターが必要な時には画面を見るようにして,他の時は見なくても済むようにすることで画面に向き合う時間を減らす工夫があることが望ましい.ファシリテーターが画面を見るべきタイミングは議論の話題が変化したときである.以前の議論の内容から外れた発言がされた時,ファシリテーターが適切な発言をすることで,脱線や炎上を避けて議論を収束させることができる.
すなわち,ファシリテーターの代わりに自動的に議論中の話題の変化を観測することが求められている.
%
現在,COLLAGREE上で使用されている議論支援システムは「(1)投稿支援システム」と「(2)議論可視化システム」の2つに大別できる.
投稿支援システムはポイント機能やファシリテーションフレーズ簡易投稿機能のように,ユーザーが投稿をする際に何らかの補助やリアクションを行う.現行の機能では選択肢の提示に留まっており,作業量を減らすことには繋がりにくい.
一方,議論可視化システムは議論ツリーやキーワード抽出のように,ユーザーにスレッドとは異なる議論の見方を提供する.
\ref{Fig:argTree1}に議論ツリーの例を示す.
\begin{figure}[htbp]
 \begin{center}
  \includegraphics[width=\textwidth]{../images/2.Related_Work/argTree1.png}
  \caption{議論ツリー}
  \label{Fig:argTree1}
  \vspace{-10pt}
 \end{center}
\end{figure}
現行の機能では議論を見やすくすることに重点が置かれており,議論の把握の助けにはなるが画面に向き合う時間を減らすことにはなりにくい.むしろ,作業量を増やすことになり得ることもある.
従って,現行の支援機能ではファシリテーターの作業量の減少には繋がりにくい.
%
近年,自然言語処理の分野において分散表現が多くの研究で使われており,機械翻訳を始めとする単語の意味が重要となる分野で精度の向上が確認されている.分散表現を用いることで,人間に近い精度で話題の変化を観測することが可能となる.
%
以上のような背景を踏まえて,分散表現を用いて,話題の変化を観測し,話題の変化が確認された時にファシリテーターに伝えることが望ましい.
話題の変化の観測は,発言中に現れる単語の類似度の計算と見なすことができる.
分散表現を用いることで単語間の類似度を求めることができる,値が大きいほど単語がそれぞれ類似した実数ベクトルであることを表す.単語Aと単語Bの実数ベクトルが類似しているとは,単語Aと共に使われることの多い単語と単語Bと共に使われることの多い単語が多く共通していることを示す.故に,分散表現を使って単語の類似度を計算することができる.
%
発言文から単語を選ぶ際には自動要約を用いる.発言文から重要でない単語を取り除くことで関連度の計算の精度を高めることが可能となる.
要約の手法としてはokapi BM25 \cite{okapiBM25}とLexRankを組み合わせた抽出的要約手法を用いる.
\begin{comment}
%======================================= 社会的背景
2013年頃からWeb上での大規模な議論活動が活発になり,大規模な人数での議論が期待されている.
大規模な議論では意見を共有することは可能であるが,議論を整理させることや収束させることは難しい.以上から大規模意見集約システムCOLLAGREEが開発された.本システムではWeb上で適切に大規模な議論を行うことができるように議論をマネジメントするファシリテーターを導入した\cite{collagreeTest}.
過去の実験ではファシリテーターの存在が議論の集約に大きな役割を果たしていることが認識されており,大規模な議論のためにファシリテータは必要である.しかし,議論の規模に伴って議論時間が長くなる傾向があり,同時にファシリテーターは常に議論の動向を見続ける必要がある.故に,議論の規模が大きくなればなるほどファシリテーターは長時間かつ大規模な議論の動向の監視によって大きな負担がかかる.大規模な議論が増加する傾向を踏まえるとファシリテーターにかかる負担を軽減する支援が必要である.\\
以上の問題を解決するため,話題の変化を追い,重要な話題の転換点をファシリテーターの代わりに検出することが有用であると考える.必要な時にだけファシリテーターが画面を見れば良いようにすることでファシリテーターの負担軽減が期待できる.
%========================================= 現行手法問題点背景
%議論支援に関する先行研究において,既存の手法は全てが文字列を文字列のまま扱う手法である.
%既存手法は殆どがパターンマッチングと重み付けの2つに区分することができる.
%パターンマッチングでは事前に単語を登録して,単語がマッチした場合に処理を行うが,処理それぞれに対して単語を登録しなければならず手間が膨大になってしまう.また,単語の意味が考慮されておらず,手作業で登録を行うので登録漏れがあった場合に単語の意味に関係なく処理を行うことが不可能となってしまう.
%重み付けは単語の出現頻度や文章の長さを使用して単語・文章に順位を付ける手法で必ずしも単語の登録が必要でないため多くの研究で使用されている.
%しかし,重み付けもまた単語を文字列のまま扱っており,意味までは考慮されていない.故に ,人間なら対応できる似た単語でも1文字違うだけで対処が困難となる.
議論支援に関する先行研究においてファシリテーターに対する支援を目的としたものは無く,殆どが議論の活性化や可視化を目的としている.
%=================================新手法
近年,自然言語処理の分野において分散表現が多くの研究で使われており.分散表現は文字列である単語を辞書データを使用して実数ベクトルへと変換する.辞書データにない単語には対応できないが,多様な処理を1つの辞書データで行うことができる.また,実数ベクトルの各数値が単語の意味を表現するものとなっており,数値を使用して処理を行うことができる.
分散表現を用いることで既存手法より人間の感覚に近しい処理を行うことができる.
%=================================
以上のような背景を踏まえて,分散表現を用いてファシリテーターの代わりに話題の変化を判定し,知らせることを目指す.
話題転換の検出は発言同士の近さ,すなわち発言に含まれる単語意味の近さと見ることができる.
分散表現ではベクトル同士の内積計算を行うことで単語同士の意味の近さを計算することができる.
また,分散表現を使用することで機械翻訳を始めとする複数の分野で精度の向上が確認されている.
\end{comment}
\section{研究の目的}
\label{intro:taget}
本論文では,分散表現を用いて議論中での発言に含まれる単語の関連度を計算し,話題の変化を観測する手法を提案する.

\section{本論文の構成}
本論文の構成を以下に示す.
\ref{relwork:chapter} 章では要約手法に関する研究と,分散表現に関する先行研究を紹介する.
次に,\ref{model:chapter}章では発言の要約手法の説明を行い,\ref{impl:chapter}章では分散表現を用いた単語集合間の関連度計算について説明する.
そして,\ref{exp:chapter}章では話題転換点の検出の評価実験について説明する.
最後に\ref{con:chapter}章で本論文のまとめと考察を示す.

 %-------------------------------------------------------------------------------
 \expandafter\ifx\csname MasterFile\endcsname\relax
	\def\BibFile{hoge}
	\expandafter\ifx\csname MasterFile\endcsname\relax
	\def\SubFile{hoge}
	\input{../thesis/thesis}
	\begin{document}
	\setcounter{chapter}{0}
	\fi
  %-------------------------------------------------------------------------------
\cleardoublepage
\chapter{序論}
\label{intro:chapter}
%本章では, 本研究を行なうに至った背景と目的について述べる.その後,本論文の構成について述べる.
\section{研究の背景}
\label{intro:background}
近年,Web上での大規模な議論活動が活発になっているが,現在一般的に使われている "2ちゃんねる" や "Twitter" といったシステムでは整理や収束を行うことが困難である.困難である原因として,議論の管理を行う者がいないことが挙げられる.
つまり,議論を整理・収束させるには議論のマネジメントを行う人物が必要である.
%
大規模意見集約システムCOLLAGREE\cite{collagreeTest}ではファシリテーターと呼ばれる人物が議論のマネジメントを行っている.
しかし,ファシリテーターは人間であり,長時間に渡って大人数での議論の動向をマネジメントし続けるのは困難である.
COLLAGREEで大規模な議論を収束させるためには,ファシリテーターが必要な時には画面を見るようにして,他の時は見なくても済むようにすることで画面に向き合う時間を減らす工夫があることが望ましい.ファシリテーターが画面を見るべきタイミングは議論の話題が変化したときである.以前の議論の内容から外れた発言がされた時,ファシリテーターが適切な発言をすることで,脱線や炎上を避けて議論を収束させることができる.
すなわち,ファシリテーターの代わりに自動的に議論中の話題の変化を観測することが求められている.
%
現在,COLLAGREE上で使用されている議論支援システムは「(1)投稿支援システム」と「(2)議論可視化システム」の2つに大別できる.
投稿支援システムはポイント機能やファシリテーションフレーズ簡易投稿機能のように,ユーザーが投稿をする際に何らかの補助やリアクションを行う.現行の機能では選択肢の提示に留まっており,作業量を減らすことには繋がりにくい.
一方,議論可視化システムは議論ツリーやキーワード抽出のように,ユーザーにスレッドとは異なる議論の見方を提供する.
\ref{Fig:argTree1}に議論ツリーの例を示す.
\begin{figure}[htbp]
 \begin{center}
  \includegraphics[width=\textwidth]{../images/2.Related_Work/argTree1.png}
  \caption{議論ツリー}
  \label{Fig:argTree1}
  \vspace{-10pt}
 \end{center}
\end{figure}
現行の機能では議論を見やすくすることに重点が置かれており,議論の把握の助けにはなるが画面に向き合う時間を減らすことにはなりにくい.むしろ,作業量を増やすことになり得ることもある.
従って,現行の支援機能ではファシリテーターの作業量の減少には繋がりにくい.
%
近年,自然言語処理の分野において分散表現が多くの研究で使われており,機械翻訳を始めとする単語の意味が重要となる分野で精度の向上が確認されている.分散表現を用いることで,人間に近い精度で話題の変化を観測することが可能となる.
%
以上のような背景を踏まえて,分散表現を用いて,話題の変化を観測し,話題の変化が確認された時にファシリテーターに伝えることが望ましい.
話題の変化の観測は,発言中に現れる単語の類似度の計算と見なすことができる.
分散表現を用いることで単語間の類似度を求めることができる,値が大きいほど単語がそれぞれ類似した実数ベクトルであることを表す.単語Aと単語Bの実数ベクトルが類似しているとは,単語Aと共に使われることの多い単語と単語Bと共に使われることの多い単語が多く共通していることを示す.故に,分散表現を使って単語の類似度を計算することができる.
%
発言文から単語を選ぶ際には自動要約を用いる.発言文から重要でない単語を取り除くことで関連度の計算の精度を高めることが可能となる.
要約の手法としてはokapi BM25 \cite{okapiBM25}とLexRankを組み合わせた抽出的要約手法を用いる.
\begin{comment}
%======================================= 社会的背景
2013年頃からWeb上での大規模な議論活動が活発になり,大規模な人数での議論が期待されている.
大規模な議論では意見を共有することは可能であるが,議論を整理させることや収束させることは難しい.以上から大規模意見集約システムCOLLAGREEが開発された.本システムではWeb上で適切に大規模な議論を行うことができるように議論をマネジメントするファシリテーターを導入した\cite{collagreeTest}.
過去の実験ではファシリテーターの存在が議論の集約に大きな役割を果たしていることが認識されており,大規模な議論のためにファシリテータは必要である.しかし,議論の規模に伴って議論時間が長くなる傾向があり,同時にファシリテーターは常に議論の動向を見続ける必要がある.故に,議論の規模が大きくなればなるほどファシリテーターは長時間かつ大規模な議論の動向の監視によって大きな負担がかかる.大規模な議論が増加する傾向を踏まえるとファシリテーターにかかる負担を軽減する支援が必要である.\\
以上の問題を解決するため,話題の変化を追い,重要な話題の転換点をファシリテーターの代わりに検出することが有用であると考える.必要な時にだけファシリテーターが画面を見れば良いようにすることでファシリテーターの負担軽減が期待できる.
%========================================= 現行手法問題点背景
%議論支援に関する先行研究において,既存の手法は全てが文字列を文字列のまま扱う手法である.
%既存手法は殆どがパターンマッチングと重み付けの2つに区分することができる.
%パターンマッチングでは事前に単語を登録して,単語がマッチした場合に処理を行うが,処理それぞれに対して単語を登録しなければならず手間が膨大になってしまう.また,単語の意味が考慮されておらず,手作業で登録を行うので登録漏れがあった場合に単語の意味に関係なく処理を行うことが不可能となってしまう.
%重み付けは単語の出現頻度や文章の長さを使用して単語・文章に順位を付ける手法で必ずしも単語の登録が必要でないため多くの研究で使用されている.
%しかし,重み付けもまた単語を文字列のまま扱っており,意味までは考慮されていない.故に ,人間なら対応できる似た単語でも1文字違うだけで対処が困難となる.
議論支援に関する先行研究においてファシリテーターに対する支援を目的としたものは無く,殆どが議論の活性化や可視化を目的としている.
%=================================新手法
近年,自然言語処理の分野において分散表現が多くの研究で使われており.分散表現は文字列である単語を辞書データを使用して実数ベクトルへと変換する.辞書データにない単語には対応できないが,多様な処理を1つの辞書データで行うことができる.また,実数ベクトルの各数値が単語の意味を表現するものとなっており,数値を使用して処理を行うことができる.
分散表現を用いることで既存手法より人間の感覚に近しい処理を行うことができる.
%=================================
以上のような背景を踏まえて,分散表現を用いてファシリテーターの代わりに話題の変化を判定し,知らせることを目指す.
話題転換の検出は発言同士の近さ,すなわち発言に含まれる単語意味の近さと見ることができる.
分散表現ではベクトル同士の内積計算を行うことで単語同士の意味の近さを計算することができる.
また,分散表現を使用することで機械翻訳を始めとする複数の分野で精度の向上が確認されている.
\end{comment}
\section{研究の目的}
\label{intro:taget}
本論文では,分散表現を用いて議論中での発言に含まれる単語の関連度を計算し,話題の変化を観測する手法を提案する.

\section{本論文の構成}
本論文の構成を以下に示す.
\ref{relwork:chapter} 章では要約手法に関する研究と,分散表現に関する先行研究を紹介する.
次に,\ref{model:chapter}章では発言の要約手法の説明を行い,\ref{impl:chapter}章では分散表現を用いた単語集合間の関連度計算について説明する.
そして,\ref{exp:chapter}章では話題転換点の検出の評価実験について説明する.
最後に\ref{con:chapter}章で本論文のまとめと考察を示す.

 %-------------------------------------------------------------------------------
 \expandafter\ifx\csname MasterFile\endcsname\relax
	\def\BibFile{hoge}
	\input{../Bibliography/chapter}
  \fi
  %-------------------------------------------------------------------------------
  \expandafter\ifx\csname MasterFile\endcsname\relax
  \end{document}
  \fi

  \fi
  %-------------------------------------------------------------------------------
  \expandafter\ifx\csname MasterFile\endcsname\relax
  \end{document}
  \fi

  \fi
  %-------------------------------------------------------------------------------
  \expandafter\ifx\csname MasterFile\endcsname\relax
  \end{document}
  \fi

  \fi
  %-------------------------------------------------------------------------------
  \expandafter\ifx\csname MasterFile\endcsname\relax
\end{document}
\fi
