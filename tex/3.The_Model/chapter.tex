\expandafter\ifx\csname MasterFile\endcsname\relax
	\def\SubFile{hoge}
  \documentclass[a4j,12pt,twoside,openany]{jreport}
%\nofiles %tocファイルを更新させない
%\documentclass[12pt,a4j,twoside,openany]{jsbook}
\usepackage[dvipdfmx]{graphicx}
\usepackage{../dspc} % ベースラインスキップの指定
\usepackage{../slashbox} % 表に斜線を入れる
%\usepackage{../mediabb}
\usepackage{fancyvrb} % Verbatim環境
\usepackage{fancyhdr} % Headerの下線付き章見出し
\usepackage{here} % float[H]
\usepackage{multirow}
\usepackage{hhline} % 表の罫線の角を美しくする
\usepackage{amsmath} %コレがないとcasesが動かない
\usepackage{amsfonts} % 数学用フォント
\usepackage{bm} % 数式環境での bold
\usepackage{algorithm}
\usepackage{algorithmicx}
\usepackage[noend]{algpseudocode}
\usepackage[flushleft]{threeparttable} % 脚注付きテーブル
\usepackage{enumitem}
\usepackage{comment}
\usepackage{fancybox}
%\usepackage{csvsimple,booktabs,siunitx}
%\usepackage{filecontents}


\setlength{\evensidemargin}{5pt}
\setlength{\oddsidemargin}{40pt}
%\setlength{\headheight}{16.5pt}
%%\setlength{\headheight}{30pt}
\setcounter{secnumdepth}{3}
\setlist[description]{leftmargin=2\parindent,labelindent=\parindent}

\makeatletter
\def\@makechapterhead#1{%
	\vspace*{50\p@}%
	{
		\parindent \z@ \raggedright \normalfont
		\ifnum \c@secnumdepth >\m@ne
		% \if@mainmatter
			\huge\bfseries\@chapapp\thechapter\@chappos
			\par\nobreak
			\vskip 20\p@
		% \fi
		\fi
		\interlinepenalty\@M
		\Huge\bfseries #1\par\nobreak
		\vskip 40\p@
	}
}

%新しいコマンド定義
\newcounter{linenumber}
\newenvironment{listing}{%
  \begin{list}{%
    \small\arabic{linenumber}:}{%
      \usecounter{linenumber}%
      \setlength{\baselineskip}{18pt}%
      \setlength{\itemsep}{0pt}%
      \setlength{\parsep}{0pt}}}%
 {\end{list}}
\newcommand{\figcaption}[1]{\def\@captype{figure}\caption{#1}}
\newcommand{\tblcaption}[1]{\def\@captype{table}\caption{#1}}
\newcommand{\norm}[1]{\left\| #1 \right\|}
\newcommand{\cc}[1]{\multicolumn{1}{|c|}{#1}}
\newcommand{\circled}[1]{\raisebox{.5pt}{\textcircled{\raisebox{-.9pt} {#1}}}}
\newcommand{\specialcell}[2][c]{%
  \begin{tabular}[#1]{@{}c@{}}#2\end{tabular}}
\makeatother
%===============================================================================
\expandafter\ifx\csname SubFile\endcsname\relax
\begin{document}
\def\MasterFile{hoge}
%-------------------------------------------------------------------------------
%\maketitle
\thispagestyle{empty}
\begin{titlepage}

% 題名
\def\title{分散表現を用いた\\話題変化判定}
% 補助題名
\def\subtitle{卒業論文}
% 著者
\def\author{芳野 魁}
% 入学年度(平成)
\def\year{29}
% 学籍番号
\def\number{26115162}
% 指導教官
\def\kyoukan{伊藤 孝行}
% 指導教官役職
\def\kyoukanrank{教授}
% 提出日
\def\teisyutubi{平成29年12月28日}

\pagestyle{empty}

\begin{center}

\vspace*{20mm}
{\Large\mc 平成29年度 \hspace{7mm} 卒 業 論 文}
\vspace{15mm}

%\setlength{\unitlength}{1mm}
\begin{picture}(100,60)
  \put(0,0){\makebox(100,60){\huge\bf\shortstack{\title}}}
\end{picture}
\\
%\begin{picture}(100,5)
%  \put(0,0){\makebox(100,5){\Large\bf\shortstack{\subtitle}}}
%\end{picture}
\end{center}
\vspace{10mm}
\begin{flushright}
\begin{tabular}{ll}
{\large 提出日} & {\large {\teisyutubi}} \\
{\large 所属}  & {\large 名古屋工業大学 情報工学科} \\
{\large 指導教員} & {\large {\kyoukan} {\kyoukanrank}} \\
 & \\
{\large 入学年度} & {\large 平成26年度入学}\\
{\large 学籍番号} &{\large {\number}} \\
 & \\
%{\large 氏名} & {\huge {\author}}
{\large 氏名} & {\huge\mc {\author}}
\end{tabular}
\end{flushright}

\end{titlepage}

%\addcontentsline{toc}{chapter}{表紙}
\thispagestyle{empty}
\mbox{}\newpage
%===============================================================================
%\frontmatter
%===============================================================================
%\mainmatter
%-------------------------------------------------------------------------------
\pagenumbering{arabic}
\cleardoublepage
\expandafter\ifx\csname MasterFile\endcsname\relax
	\def\SubFile{hoge}
	\documentclass[a4j,12pt,twoside,openany]{jreport}
%\nofiles %tocファイルを更新させない
%\documentclass[12pt,a4j,twoside,openany]{jsbook}
\usepackage[dvipdfmx]{graphicx}
\usepackage{../dspc} % ベースラインスキップの指定
\usepackage{../slashbox} % 表に斜線を入れる
%\usepackage{../mediabb}
\usepackage{fancyvrb} % Verbatim環境
\usepackage{fancyhdr} % Headerの下線付き章見出し
\usepackage{here} % float[H]
\usepackage{multirow}
\usepackage{hhline} % 表の罫線の角を美しくする
\usepackage{amsmath} %コレがないとcasesが動かない
\usepackage{amsfonts} % 数学用フォント
\usepackage{bm} % 数式環境での bold
\usepackage{algorithm}
\usepackage{algorithmicx}
\usepackage[noend]{algpseudocode}
\usepackage[flushleft]{threeparttable} % 脚注付きテーブル
\usepackage{enumitem}
\usepackage{comment}
\usepackage{fancybox}
%\usepackage{csvsimple,booktabs,siunitx}
%\usepackage{filecontents}


\setlength{\evensidemargin}{5pt}
\setlength{\oddsidemargin}{40pt}
%\setlength{\headheight}{16.5pt}
%%\setlength{\headheight}{30pt}
\setcounter{secnumdepth}{3}
\setlist[description]{leftmargin=2\parindent,labelindent=\parindent}

\makeatletter
\def\@makechapterhead#1{%
	\vspace*{50\p@}%
	{
		\parindent \z@ \raggedright \normalfont
		\ifnum \c@secnumdepth >\m@ne
		% \if@mainmatter
			\huge\bfseries\@chapapp\thechapter\@chappos
			\par\nobreak
			\vskip 20\p@
		% \fi
		\fi
		\interlinepenalty\@M
		\Huge\bfseries #1\par\nobreak
		\vskip 40\p@
	}
}

%新しいコマンド定義
\newcounter{linenumber}
\newenvironment{listing}{%
  \begin{list}{%
    \small\arabic{linenumber}:}{%
      \usecounter{linenumber}%
      \setlength{\baselineskip}{18pt}%
      \setlength{\itemsep}{0pt}%
      \setlength{\parsep}{0pt}}}%
 {\end{list}}
\newcommand{\figcaption}[1]{\def\@captype{figure}\caption{#1}}
\newcommand{\tblcaption}[1]{\def\@captype{table}\caption{#1}}
\newcommand{\norm}[1]{\left\| #1 \right\|}
\newcommand{\cc}[1]{\multicolumn{1}{|c|}{#1}}
\newcommand{\circled}[1]{\raisebox{.5pt}{\textcircled{\raisebox{-.9pt} {#1}}}}
\newcommand{\specialcell}[2][c]{%
  \begin{tabular}[#1]{@{}c@{}}#2\end{tabular}}
\makeatother
%===============================================================================
\expandafter\ifx\csname SubFile\endcsname\relax
\begin{document}
\def\MasterFile{hoge}
%-------------------------------------------------------------------------------
%\maketitle
\thispagestyle{empty}
\input{../hyoushi/title}
%\addcontentsline{toc}{chapter}{表紙}
\thispagestyle{empty}
\mbox{}\newpage
%===============================================================================
%\frontmatter
%===============================================================================
%\mainmatter
%-------------------------------------------------------------------------------
\pagenumbering{arabic}
\cleardoublepage
\input{../0.Abstract/chapter}
%-------------------------------------------------------------------------------
\clearpage
\addcontentsline{toc}{chapter}{目次}
\tableofcontents

\clearpage
\addcontentsline{toc}{chapter}{図目次}
\listoffigures

\clearpage
\addcontentsline{toc}{chapter}{表目次}
\listoftables

%-------------------------------------------------------------------------------

%=====================
\pagestyle{fancy} % Headerをつける
\renewcommand{\sectionmark}[1]{\markright{\thesection\ \ \ #1}}
\renewcommand{\chaptermark}[1]{\markboth{#1}{}}
\lhead{}
\chead{}
\lfoot{}
\rfoot{}%-------------------------------------------------------------------------------
\input{../1.Introduction/chapter}
%-------------------------------------------------------------------------------
\input{../2.Related_Work/chapter}
%-------------------------------------------------------------------------------
\input{../3.The_Model/chapter}
%-------------------------------------------------------------------------------
\input{../4.Implementation/chapter}
%-------------------------------------------------------------------------------
\input{../5.Experiments/chapter}
%-------------------------------------------------------------------------------
\input{../6.Conclusion/chapter}

%===============================================================================
\pagestyle{plain}
%-------------------------------------------------------------------------------
\input{../7.Acknowledgement/chapter} %謝辞
%-------------------------------------------------------------------------------
\def\BibFile{../Bibliograhoy/database2}
\input{../Bibliography/chapter} %参考文献
% %===============================================================================
\appendix
\input{../A.Mypaper/chapter} % 投稿論文リスト
\input{../B.SIG-CCI2/chapter} %
\input{../C.IJCAI-16/chapter} %
%===============================================================================
\end{document}
\fi

	\begin{document}
	\setcounter{chapter}{0}
	\fi
  %-------------------------------------------------------------------------------
\cleardoublepage
\chapter{序論}
\label{intro:chapter}
%本章では, 本研究を行なうに至った背景と目的について述べる.その後,本論文の構成について述べる.
\section{研究の背景}
\label{intro:background}
近年,Web上での大規模な議論活動が活発になっているが,現在一般的に使われている "2ちゃんねる" や "Twitter" といったシステムでは整理や収束を行うことが困難である.困難である原因として,議論の管理を行う者がいないことが挙げられる.
つまり,議論を整理・収束させるには議論のマネジメントを行う人物が必要である.
%
大規模意見集約システムCOLLAGREE\cite{collagreeTest}ではファシリテーターと呼ばれる人物が議論のマネジメントを行っている.
しかし,ファシリテーターは人間であり,長時間に渡って大人数での議論の動向をマネジメントし続けるのは困難である.
COLLAGREEで大規模な議論を収束させるためには,ファシリテーターが必要な時には画面を見るようにして,他の時は見なくても済むようにすることで画面に向き合う時間を減らす工夫があることが望ましい.ファシリテーターが画面を見るべきタイミングは議論の話題が変化したときである.以前の議論の内容から外れた発言がされた時,ファシリテーターが適切な発言をすることで,脱線や炎上を避けて議論を収束させることができる.
すなわち,ファシリテーターの代わりに自動的に議論中の話題の変化を観測することが求められている.
%
現在,COLLAGREE上で使用されている議論支援システムは「(1)投稿支援システム」と「(2)議論可視化システム」の2つに大別できる.
投稿支援システムはポイント機能やファシリテーションフレーズ簡易投稿機能のように,ユーザーが投稿をする際に何らかの補助やリアクションを行う.現行の機能では選択肢の提示に留まっており,作業量を減らすことには繋がりにくい.
一方,議論可視化システムは議論ツリーやキーワード抽出のように,ユーザーにスレッドとは異なる議論の見方を提供する.
\ref{Fig:argTree1}に議論ツリーの例を示す.
\begin{figure}[htbp]
 \begin{center}
  \includegraphics[width=\textwidth]{../images/2.Related_Work/argTree1.png}
  \caption{議論ツリー}
  \label{Fig:argTree1}
  \vspace{-10pt}
 \end{center}
\end{figure}
現行の機能では議論を見やすくすることに重点が置かれており,議論の把握の助けにはなるが画面に向き合う時間を減らすことにはなりにくい.むしろ,作業量を増やすことになり得ることもある.
従って,現行の支援機能ではファシリテーターの作業量の減少には繋がりにくい.
%
近年,自然言語処理の分野において分散表現が多くの研究で使われており,機械翻訳を始めとする単語の意味が重要となる分野で精度の向上が確認されている.分散表現を用いることで,人間に近い精度で話題の変化を観測することが可能となる.
%
以上のような背景を踏まえて,分散表現を用いて,話題の変化を観測し,話題の変化が確認された時にファシリテーターに伝えることが望ましい.
話題の変化の観測は,発言中に現れる単語の類似度の計算と見なすことができる.
分散表現を用いることで単語間の類似度を求めることができる,値が大きいほど単語がそれぞれ類似した実数ベクトルであることを表す.単語Aと単語Bの実数ベクトルが類似しているとは,単語Aと共に使われることの多い単語と単語Bと共に使われることの多い単語が多く共通していることを示す.故に,分散表現を使って単語の類似度を計算することができる.
%
発言文から単語を選ぶ際には自動要約を用いる.発言文から重要でない単語を取り除くことで関連度の計算の精度を高めることが可能となる.
要約の手法としてはokapi BM25 \cite{okapiBM25}とLexRankを組み合わせた抽出的要約手法を用いる.
\begin{comment}
%======================================= 社会的背景
2013年頃からWeb上での大規模な議論活動が活発になり,大規模な人数での議論が期待されている.
大規模な議論では意見を共有することは可能であるが,議論を整理させることや収束させることは難しい.以上から大規模意見集約システムCOLLAGREEが開発された.本システムではWeb上で適切に大規模な議論を行うことができるように議論をマネジメントするファシリテーターを導入した\cite{collagreeTest}.
過去の実験ではファシリテーターの存在が議論の集約に大きな役割を果たしていることが認識されており,大規模な議論のためにファシリテータは必要である.しかし,議論の規模に伴って議論時間が長くなる傾向があり,同時にファシリテーターは常に議論の動向を見続ける必要がある.故に,議論の規模が大きくなればなるほどファシリテーターは長時間かつ大規模な議論の動向の監視によって大きな負担がかかる.大規模な議論が増加する傾向を踏まえるとファシリテーターにかかる負担を軽減する支援が必要である.\\
以上の問題を解決するため,話題の変化を追い,重要な話題の転換点をファシリテーターの代わりに検出することが有用であると考える.必要な時にだけファシリテーターが画面を見れば良いようにすることでファシリテーターの負担軽減が期待できる.
%========================================= 現行手法問題点背景
%議論支援に関する先行研究において,既存の手法は全てが文字列を文字列のまま扱う手法である.
%既存手法は殆どがパターンマッチングと重み付けの2つに区分することができる.
%パターンマッチングでは事前に単語を登録して,単語がマッチした場合に処理を行うが,処理それぞれに対して単語を登録しなければならず手間が膨大になってしまう.また,単語の意味が考慮されておらず,手作業で登録を行うので登録漏れがあった場合に単語の意味に関係なく処理を行うことが不可能となってしまう.
%重み付けは単語の出現頻度や文章の長さを使用して単語・文章に順位を付ける手法で必ずしも単語の登録が必要でないため多くの研究で使用されている.
%しかし,重み付けもまた単語を文字列のまま扱っており,意味までは考慮されていない.故に ,人間なら対応できる似た単語でも1文字違うだけで対処が困難となる.
議論支援に関する先行研究においてファシリテーターに対する支援を目的としたものは無く,殆どが議論の活性化や可視化を目的としている.
%=================================新手法
近年,自然言語処理の分野において分散表現が多くの研究で使われており.分散表現は文字列である単語を辞書データを使用して実数ベクトルへと変換する.辞書データにない単語には対応できないが,多様な処理を1つの辞書データで行うことができる.また,実数ベクトルの各数値が単語の意味を表現するものとなっており,数値を使用して処理を行うことができる.
分散表現を用いることで既存手法より人間の感覚に近しい処理を行うことができる.
%=================================
以上のような背景を踏まえて,分散表現を用いてファシリテーターの代わりに話題の変化を判定し,知らせることを目指す.
話題転換の検出は発言同士の近さ,すなわち発言に含まれる単語意味の近さと見ることができる.
分散表現ではベクトル同士の内積計算を行うことで単語同士の意味の近さを計算することができる.
また,分散表現を使用することで機械翻訳を始めとする複数の分野で精度の向上が確認されている.
\end{comment}
\section{研究の目的}
\label{intro:taget}
本論文では,分散表現を用いて議論中での発言に含まれる単語の関連度を計算し,話題の変化を観測する手法を提案する.

\section{本論文の構成}
本論文の構成を以下に示す.
\ref{relwork:chapter} 章では要約手法に関する研究と,分散表現に関する先行研究を紹介する.
次に,\ref{model:chapter}章では発言の要約手法の説明を行い,\ref{impl:chapter}章では分散表現を用いた単語集合間の関連度計算について説明する.
そして,\ref{exp:chapter}章では話題転換点の検出の評価実験について説明する.
最後に\ref{con:chapter}章で本論文のまとめと考察を示す.

 %-------------------------------------------------------------------------------
 \expandafter\ifx\csname MasterFile\endcsname\relax
	\def\BibFile{hoge}
	\expandafter\ifx\csname MasterFile\endcsname\relax
	\def\SubFile{hoge}
	\input{../thesis/thesis}
	\begin{document}
	\setcounter{chapter}{0}
	\fi
  %-------------------------------------------------------------------------------
\cleardoublepage
\chapter{序論}
\label{intro:chapter}
%本章では, 本研究を行なうに至った背景と目的について述べる.その後,本論文の構成について述べる.
\section{研究の背景}
\label{intro:background}
近年,Web上での大規模な議論活動が活発になっているが,現在一般的に使われている "2ちゃんねる" や "Twitter" といったシステムでは整理や収束を行うことが困難である.困難である原因として,議論の管理を行う者がいないことが挙げられる.
つまり,議論を整理・収束させるには議論のマネジメントを行う人物が必要である.
%
大規模意見集約システムCOLLAGREE\cite{collagreeTest}ではファシリテーターと呼ばれる人物が議論のマネジメントを行っている.
しかし,ファシリテーターは人間であり,長時間に渡って大人数での議論の動向をマネジメントし続けるのは困難である.
COLLAGREEで大規模な議論を収束させるためには,ファシリテーターが必要な時には画面を見るようにして,他の時は見なくても済むようにすることで画面に向き合う時間を減らす工夫があることが望ましい.ファシリテーターが画面を見るべきタイミングは議論の話題が変化したときである.以前の議論の内容から外れた発言がされた時,ファシリテーターが適切な発言をすることで,脱線や炎上を避けて議論を収束させることができる.
すなわち,ファシリテーターの代わりに自動的に議論中の話題の変化を観測することが求められている.
%
現在,COLLAGREE上で使用されている議論支援システムは「(1)投稿支援システム」と「(2)議論可視化システム」の2つに大別できる.
投稿支援システムはポイント機能やファシリテーションフレーズ簡易投稿機能のように,ユーザーが投稿をする際に何らかの補助やリアクションを行う.現行の機能では選択肢の提示に留まっており,作業量を減らすことには繋がりにくい.
一方,議論可視化システムは議論ツリーやキーワード抽出のように,ユーザーにスレッドとは異なる議論の見方を提供する.
\ref{Fig:argTree1}に議論ツリーの例を示す.
\begin{figure}[htbp]
 \begin{center}
  \includegraphics[width=\textwidth]{../images/2.Related_Work/argTree1.png}
  \caption{議論ツリー}
  \label{Fig:argTree1}
  \vspace{-10pt}
 \end{center}
\end{figure}
現行の機能では議論を見やすくすることに重点が置かれており,議論の把握の助けにはなるが画面に向き合う時間を減らすことにはなりにくい.むしろ,作業量を増やすことになり得ることもある.
従って,現行の支援機能ではファシリテーターの作業量の減少には繋がりにくい.
%
近年,自然言語処理の分野において分散表現が多くの研究で使われており,機械翻訳を始めとする単語の意味が重要となる分野で精度の向上が確認されている.分散表現を用いることで,人間に近い精度で話題の変化を観測することが可能となる.
%
以上のような背景を踏まえて,分散表現を用いて,話題の変化を観測し,話題の変化が確認された時にファシリテーターに伝えることが望ましい.
話題の変化の観測は,発言中に現れる単語の類似度の計算と見なすことができる.
分散表現を用いることで単語間の類似度を求めることができる,値が大きいほど単語がそれぞれ類似した実数ベクトルであることを表す.単語Aと単語Bの実数ベクトルが類似しているとは,単語Aと共に使われることの多い単語と単語Bと共に使われることの多い単語が多く共通していることを示す.故に,分散表現を使って単語の類似度を計算することができる.
%
発言文から単語を選ぶ際には自動要約を用いる.発言文から重要でない単語を取り除くことで関連度の計算の精度を高めることが可能となる.
要約の手法としてはokapi BM25 \cite{okapiBM25}とLexRankを組み合わせた抽出的要約手法を用いる.
\begin{comment}
%======================================= 社会的背景
2013年頃からWeb上での大規模な議論活動が活発になり,大規模な人数での議論が期待されている.
大規模な議論では意見を共有することは可能であるが,議論を整理させることや収束させることは難しい.以上から大規模意見集約システムCOLLAGREEが開発された.本システムではWeb上で適切に大規模な議論を行うことができるように議論をマネジメントするファシリテーターを導入した\cite{collagreeTest}.
過去の実験ではファシリテーターの存在が議論の集約に大きな役割を果たしていることが認識されており,大規模な議論のためにファシリテータは必要である.しかし,議論の規模に伴って議論時間が長くなる傾向があり,同時にファシリテーターは常に議論の動向を見続ける必要がある.故に,議論の規模が大きくなればなるほどファシリテーターは長時間かつ大規模な議論の動向の監視によって大きな負担がかかる.大規模な議論が増加する傾向を踏まえるとファシリテーターにかかる負担を軽減する支援が必要である.\\
以上の問題を解決するため,話題の変化を追い,重要な話題の転換点をファシリテーターの代わりに検出することが有用であると考える.必要な時にだけファシリテーターが画面を見れば良いようにすることでファシリテーターの負担軽減が期待できる.
%========================================= 現行手法問題点背景
%議論支援に関する先行研究において,既存の手法は全てが文字列を文字列のまま扱う手法である.
%既存手法は殆どがパターンマッチングと重み付けの2つに区分することができる.
%パターンマッチングでは事前に単語を登録して,単語がマッチした場合に処理を行うが,処理それぞれに対して単語を登録しなければならず手間が膨大になってしまう.また,単語の意味が考慮されておらず,手作業で登録を行うので登録漏れがあった場合に単語の意味に関係なく処理を行うことが不可能となってしまう.
%重み付けは単語の出現頻度や文章の長さを使用して単語・文章に順位を付ける手法で必ずしも単語の登録が必要でないため多くの研究で使用されている.
%しかし,重み付けもまた単語を文字列のまま扱っており,意味までは考慮されていない.故に ,人間なら対応できる似た単語でも1文字違うだけで対処が困難となる.
議論支援に関する先行研究においてファシリテーターに対する支援を目的としたものは無く,殆どが議論の活性化や可視化を目的としている.
%=================================新手法
近年,自然言語処理の分野において分散表現が多くの研究で使われており.分散表現は文字列である単語を辞書データを使用して実数ベクトルへと変換する.辞書データにない単語には対応できないが,多様な処理を1つの辞書データで行うことができる.また,実数ベクトルの各数値が単語の意味を表現するものとなっており,数値を使用して処理を行うことができる.
分散表現を用いることで既存手法より人間の感覚に近しい処理を行うことができる.
%=================================
以上のような背景を踏まえて,分散表現を用いてファシリテーターの代わりに話題の変化を判定し,知らせることを目指す.
話題転換の検出は発言同士の近さ,すなわち発言に含まれる単語意味の近さと見ることができる.
分散表現ではベクトル同士の内積計算を行うことで単語同士の意味の近さを計算することができる.
また,分散表現を使用することで機械翻訳を始めとする複数の分野で精度の向上が確認されている.
\end{comment}
\section{研究の目的}
\label{intro:taget}
本論文では,分散表現を用いて議論中での発言に含まれる単語の関連度を計算し,話題の変化を観測する手法を提案する.

\section{本論文の構成}
本論文の構成を以下に示す.
\ref{relwork:chapter} 章では要約手法に関する研究と,分散表現に関する先行研究を紹介する.
次に,\ref{model:chapter}章では発言の要約手法の説明を行い,\ref{impl:chapter}章では分散表現を用いた単語集合間の関連度計算について説明する.
そして,\ref{exp:chapter}章では話題転換点の検出の評価実験について説明する.
最後に\ref{con:chapter}章で本論文のまとめと考察を示す.

 %-------------------------------------------------------------------------------
 \expandafter\ifx\csname MasterFile\endcsname\relax
	\def\BibFile{hoge}
	\input{../Bibliography/chapter}
  \fi
  %-------------------------------------------------------------------------------
  \expandafter\ifx\csname MasterFile\endcsname\relax
  \end{document}
  \fi

  \fi
  %-------------------------------------------------------------------------------
  \expandafter\ifx\csname MasterFile\endcsname\relax
  \end{document}
  \fi

%-------------------------------------------------------------------------------
\clearpage
\addcontentsline{toc}{chapter}{目次}
\tableofcontents

\clearpage
\addcontentsline{toc}{chapter}{図目次}
\listoffigures

\clearpage
\addcontentsline{toc}{chapter}{表目次}
\listoftables

%-------------------------------------------------------------------------------

%=====================
\pagestyle{fancy} % Headerをつける
\renewcommand{\sectionmark}[1]{\markright{\thesection\ \ \ #1}}
\renewcommand{\chaptermark}[1]{\markboth{#1}{}}
\lhead{}
\chead{}
\lfoot{}
\rfoot{}%-------------------------------------------------------------------------------
\expandafter\ifx\csname MasterFile\endcsname\relax
	\def\SubFile{hoge}
	\documentclass[a4j,12pt,twoside,openany]{jreport}
%\nofiles %tocファイルを更新させない
%\documentclass[12pt,a4j,twoside,openany]{jsbook}
\usepackage[dvipdfmx]{graphicx}
\usepackage{../dspc} % ベースラインスキップの指定
\usepackage{../slashbox} % 表に斜線を入れる
%\usepackage{../mediabb}
\usepackage{fancyvrb} % Verbatim環境
\usepackage{fancyhdr} % Headerの下線付き章見出し
\usepackage{here} % float[H]
\usepackage{multirow}
\usepackage{hhline} % 表の罫線の角を美しくする
\usepackage{amsmath} %コレがないとcasesが動かない
\usepackage{amsfonts} % 数学用フォント
\usepackage{bm} % 数式環境での bold
\usepackage{algorithm}
\usepackage{algorithmicx}
\usepackage[noend]{algpseudocode}
\usepackage[flushleft]{threeparttable} % 脚注付きテーブル
\usepackage{enumitem}
\usepackage{comment}
\usepackage{fancybox}
%\usepackage{csvsimple,booktabs,siunitx}
%\usepackage{filecontents}


\setlength{\evensidemargin}{5pt}
\setlength{\oddsidemargin}{40pt}
%\setlength{\headheight}{16.5pt}
%%\setlength{\headheight}{30pt}
\setcounter{secnumdepth}{3}
\setlist[description]{leftmargin=2\parindent,labelindent=\parindent}

\makeatletter
\def\@makechapterhead#1{%
	\vspace*{50\p@}%
	{
		\parindent \z@ \raggedright \normalfont
		\ifnum \c@secnumdepth >\m@ne
		% \if@mainmatter
			\huge\bfseries\@chapapp\thechapter\@chappos
			\par\nobreak
			\vskip 20\p@
		% \fi
		\fi
		\interlinepenalty\@M
		\Huge\bfseries #1\par\nobreak
		\vskip 40\p@
	}
}

%新しいコマンド定義
\newcounter{linenumber}
\newenvironment{listing}{%
  \begin{list}{%
    \small\arabic{linenumber}:}{%
      \usecounter{linenumber}%
      \setlength{\baselineskip}{18pt}%
      \setlength{\itemsep}{0pt}%
      \setlength{\parsep}{0pt}}}%
 {\end{list}}
\newcommand{\figcaption}[1]{\def\@captype{figure}\caption{#1}}
\newcommand{\tblcaption}[1]{\def\@captype{table}\caption{#1}}
\newcommand{\norm}[1]{\left\| #1 \right\|}
\newcommand{\cc}[1]{\multicolumn{1}{|c|}{#1}}
\newcommand{\circled}[1]{\raisebox{.5pt}{\textcircled{\raisebox{-.9pt} {#1}}}}
\newcommand{\specialcell}[2][c]{%
  \begin{tabular}[#1]{@{}c@{}}#2\end{tabular}}
\makeatother
%===============================================================================
\expandafter\ifx\csname SubFile\endcsname\relax
\begin{document}
\def\MasterFile{hoge}
%-------------------------------------------------------------------------------
%\maketitle
\thispagestyle{empty}
\input{../hyoushi/title}
%\addcontentsline{toc}{chapter}{表紙}
\thispagestyle{empty}
\mbox{}\newpage
%===============================================================================
%\frontmatter
%===============================================================================
%\mainmatter
%-------------------------------------------------------------------------------
\pagenumbering{arabic}
\cleardoublepage
\input{../0.Abstract/chapter}
%-------------------------------------------------------------------------------
\clearpage
\addcontentsline{toc}{chapter}{目次}
\tableofcontents

\clearpage
\addcontentsline{toc}{chapter}{図目次}
\listoffigures

\clearpage
\addcontentsline{toc}{chapter}{表目次}
\listoftables

%-------------------------------------------------------------------------------

%=====================
\pagestyle{fancy} % Headerをつける
\renewcommand{\sectionmark}[1]{\markright{\thesection\ \ \ #1}}
\renewcommand{\chaptermark}[1]{\markboth{#1}{}}
\lhead{}
\chead{}
\lfoot{}
\rfoot{}%-------------------------------------------------------------------------------
\input{../1.Introduction/chapter}
%-------------------------------------------------------------------------------
\input{../2.Related_Work/chapter}
%-------------------------------------------------------------------------------
\input{../3.The_Model/chapter}
%-------------------------------------------------------------------------------
\input{../4.Implementation/chapter}
%-------------------------------------------------------------------------------
\input{../5.Experiments/chapter}
%-------------------------------------------------------------------------------
\input{../6.Conclusion/chapter}

%===============================================================================
\pagestyle{plain}
%-------------------------------------------------------------------------------
\input{../7.Acknowledgement/chapter} %謝辞
%-------------------------------------------------------------------------------
\def\BibFile{../Bibliograhoy/database2}
\input{../Bibliography/chapter} %参考文献
% %===============================================================================
\appendix
\input{../A.Mypaper/chapter} % 投稿論文リスト
\input{../B.SIG-CCI2/chapter} %
\input{../C.IJCAI-16/chapter} %
%===============================================================================
\end{document}
\fi

	\begin{document}
	\setcounter{chapter}{0}
	\fi
  %-------------------------------------------------------------------------------
\cleardoublepage
\chapter{序論}
\label{intro:chapter}
%本章では, 本研究を行なうに至った背景と目的について述べる.その後,本論文の構成について述べる.
\section{研究の背景}
\label{intro:background}
近年,Web上での大規模な議論活動が活発になっているが,現在一般的に使われている "2ちゃんねる" や "Twitter" といったシステムでは整理や収束を行うことが困難である.困難である原因として,議論の管理を行う者がいないことが挙げられる.
つまり,議論を整理・収束させるには議論のマネジメントを行う人物が必要である.
%
大規模意見集約システムCOLLAGREE\cite{collagreeTest}ではファシリテーターと呼ばれる人物が議論のマネジメントを行っている.
しかし,ファシリテーターは人間であり,長時間に渡って大人数での議論の動向をマネジメントし続けるのは困難である.
COLLAGREEで大規模な議論を収束させるためには,ファシリテーターが必要な時には画面を見るようにして,他の時は見なくても済むようにすることで画面に向き合う時間を減らす工夫があることが望ましい.ファシリテーターが画面を見るべきタイミングは議論の話題が変化したときである.以前の議論の内容から外れた発言がされた時,ファシリテーターが適切な発言をすることで,脱線や炎上を避けて議論を収束させることができる.
すなわち,ファシリテーターの代わりに自動的に議論中の話題の変化を観測することが求められている.
%
現在,COLLAGREE上で使用されている議論支援システムは「(1)投稿支援システム」と「(2)議論可視化システム」の2つに大別できる.
投稿支援システムはポイント機能やファシリテーションフレーズ簡易投稿機能のように,ユーザーが投稿をする際に何らかの補助やリアクションを行う.現行の機能では選択肢の提示に留まっており,作業量を減らすことには繋がりにくい.
一方,議論可視化システムは議論ツリーやキーワード抽出のように,ユーザーにスレッドとは異なる議論の見方を提供する.
\ref{Fig:argTree1}に議論ツリーの例を示す.
\begin{figure}[htbp]
 \begin{center}
  \includegraphics[width=\textwidth]{../images/2.Related_Work/argTree1.png}
  \caption{議論ツリー}
  \label{Fig:argTree1}
  \vspace{-10pt}
 \end{center}
\end{figure}
現行の機能では議論を見やすくすることに重点が置かれており,議論の把握の助けにはなるが画面に向き合う時間を減らすことにはなりにくい.むしろ,作業量を増やすことになり得ることもある.
従って,現行の支援機能ではファシリテーターの作業量の減少には繋がりにくい.
%
近年,自然言語処理の分野において分散表現が多くの研究で使われており,機械翻訳を始めとする単語の意味が重要となる分野で精度の向上が確認されている.分散表現を用いることで,人間に近い精度で話題の変化を観測することが可能となる.
%
以上のような背景を踏まえて,分散表現を用いて,話題の変化を観測し,話題の変化が確認された時にファシリテーターに伝えることが望ましい.
話題の変化の観測は,発言中に現れる単語の類似度の計算と見なすことができる.
分散表現を用いることで単語間の類似度を求めることができる,値が大きいほど単語がそれぞれ類似した実数ベクトルであることを表す.単語Aと単語Bの実数ベクトルが類似しているとは,単語Aと共に使われることの多い単語と単語Bと共に使われることの多い単語が多く共通していることを示す.故に,分散表現を使って単語の類似度を計算することができる.
%
発言文から単語を選ぶ際には自動要約を用いる.発言文から重要でない単語を取り除くことで関連度の計算の精度を高めることが可能となる.
要約の手法としてはokapi BM25 \cite{okapiBM25}とLexRankを組み合わせた抽出的要約手法を用いる.
\begin{comment}
%======================================= 社会的背景
2013年頃からWeb上での大規模な議論活動が活発になり,大規模な人数での議論が期待されている.
大規模な議論では意見を共有することは可能であるが,議論を整理させることや収束させることは難しい.以上から大規模意見集約システムCOLLAGREEが開発された.本システムではWeb上で適切に大規模な議論を行うことができるように議論をマネジメントするファシリテーターを導入した\cite{collagreeTest}.
過去の実験ではファシリテーターの存在が議論の集約に大きな役割を果たしていることが認識されており,大規模な議論のためにファシリテータは必要である.しかし,議論の規模に伴って議論時間が長くなる傾向があり,同時にファシリテーターは常に議論の動向を見続ける必要がある.故に,議論の規模が大きくなればなるほどファシリテーターは長時間かつ大規模な議論の動向の監視によって大きな負担がかかる.大規模な議論が増加する傾向を踏まえるとファシリテーターにかかる負担を軽減する支援が必要である.\\
以上の問題を解決するため,話題の変化を追い,重要な話題の転換点をファシリテーターの代わりに検出することが有用であると考える.必要な時にだけファシリテーターが画面を見れば良いようにすることでファシリテーターの負担軽減が期待できる.
%========================================= 現行手法問題点背景
%議論支援に関する先行研究において,既存の手法は全てが文字列を文字列のまま扱う手法である.
%既存手法は殆どがパターンマッチングと重み付けの2つに区分することができる.
%パターンマッチングでは事前に単語を登録して,単語がマッチした場合に処理を行うが,処理それぞれに対して単語を登録しなければならず手間が膨大になってしまう.また,単語の意味が考慮されておらず,手作業で登録を行うので登録漏れがあった場合に単語の意味に関係なく処理を行うことが不可能となってしまう.
%重み付けは単語の出現頻度や文章の長さを使用して単語・文章に順位を付ける手法で必ずしも単語の登録が必要でないため多くの研究で使用されている.
%しかし,重み付けもまた単語を文字列のまま扱っており,意味までは考慮されていない.故に ,人間なら対応できる似た単語でも1文字違うだけで対処が困難となる.
議論支援に関する先行研究においてファシリテーターに対する支援を目的としたものは無く,殆どが議論の活性化や可視化を目的としている.
%=================================新手法
近年,自然言語処理の分野において分散表現が多くの研究で使われており.分散表現は文字列である単語を辞書データを使用して実数ベクトルへと変換する.辞書データにない単語には対応できないが,多様な処理を1つの辞書データで行うことができる.また,実数ベクトルの各数値が単語の意味を表現するものとなっており,数値を使用して処理を行うことができる.
分散表現を用いることで既存手法より人間の感覚に近しい処理を行うことができる.
%=================================
以上のような背景を踏まえて,分散表現を用いてファシリテーターの代わりに話題の変化を判定し,知らせることを目指す.
話題転換の検出は発言同士の近さ,すなわち発言に含まれる単語意味の近さと見ることができる.
分散表現ではベクトル同士の内積計算を行うことで単語同士の意味の近さを計算することができる.
また,分散表現を使用することで機械翻訳を始めとする複数の分野で精度の向上が確認されている.
\end{comment}
\section{研究の目的}
\label{intro:taget}
本論文では,分散表現を用いて議論中での発言に含まれる単語の関連度を計算し,話題の変化を観測する手法を提案する.

\section{本論文の構成}
本論文の構成を以下に示す.
\ref{relwork:chapter} 章では要約手法に関する研究と,分散表現に関する先行研究を紹介する.
次に,\ref{model:chapter}章では発言の要約手法の説明を行い,\ref{impl:chapter}章では分散表現を用いた単語集合間の関連度計算について説明する.
そして,\ref{exp:chapter}章では話題転換点の検出の評価実験について説明する.
最後に\ref{con:chapter}章で本論文のまとめと考察を示す.

 %-------------------------------------------------------------------------------
 \expandafter\ifx\csname MasterFile\endcsname\relax
	\def\BibFile{hoge}
	\expandafter\ifx\csname MasterFile\endcsname\relax
	\def\SubFile{hoge}
	\input{../thesis/thesis}
	\begin{document}
	\setcounter{chapter}{0}
	\fi
  %-------------------------------------------------------------------------------
\cleardoublepage
\chapter{序論}
\label{intro:chapter}
%本章では, 本研究を行なうに至った背景と目的について述べる.その後,本論文の構成について述べる.
\section{研究の背景}
\label{intro:background}
近年,Web上での大規模な議論活動が活発になっているが,現在一般的に使われている "2ちゃんねる" や "Twitter" といったシステムでは整理や収束を行うことが困難である.困難である原因として,議論の管理を行う者がいないことが挙げられる.
つまり,議論を整理・収束させるには議論のマネジメントを行う人物が必要である.
%
大規模意見集約システムCOLLAGREE\cite{collagreeTest}ではファシリテーターと呼ばれる人物が議論のマネジメントを行っている.
しかし,ファシリテーターは人間であり,長時間に渡って大人数での議論の動向をマネジメントし続けるのは困難である.
COLLAGREEで大規模な議論を収束させるためには,ファシリテーターが必要な時には画面を見るようにして,他の時は見なくても済むようにすることで画面に向き合う時間を減らす工夫があることが望ましい.ファシリテーターが画面を見るべきタイミングは議論の話題が変化したときである.以前の議論の内容から外れた発言がされた時,ファシリテーターが適切な発言をすることで,脱線や炎上を避けて議論を収束させることができる.
すなわち,ファシリテーターの代わりに自動的に議論中の話題の変化を観測することが求められている.
%
現在,COLLAGREE上で使用されている議論支援システムは「(1)投稿支援システム」と「(2)議論可視化システム」の2つに大別できる.
投稿支援システムはポイント機能やファシリテーションフレーズ簡易投稿機能のように,ユーザーが投稿をする際に何らかの補助やリアクションを行う.現行の機能では選択肢の提示に留まっており,作業量を減らすことには繋がりにくい.
一方,議論可視化システムは議論ツリーやキーワード抽出のように,ユーザーにスレッドとは異なる議論の見方を提供する.
\ref{Fig:argTree1}に議論ツリーの例を示す.
\begin{figure}[htbp]
 \begin{center}
  \includegraphics[width=\textwidth]{../images/2.Related_Work/argTree1.png}
  \caption{議論ツリー}
  \label{Fig:argTree1}
  \vspace{-10pt}
 \end{center}
\end{figure}
現行の機能では議論を見やすくすることに重点が置かれており,議論の把握の助けにはなるが画面に向き合う時間を減らすことにはなりにくい.むしろ,作業量を増やすことになり得ることもある.
従って,現行の支援機能ではファシリテーターの作業量の減少には繋がりにくい.
%
近年,自然言語処理の分野において分散表現が多くの研究で使われており,機械翻訳を始めとする単語の意味が重要となる分野で精度の向上が確認されている.分散表現を用いることで,人間に近い精度で話題の変化を観測することが可能となる.
%
以上のような背景を踏まえて,分散表現を用いて,話題の変化を観測し,話題の変化が確認された時にファシリテーターに伝えることが望ましい.
話題の変化の観測は,発言中に現れる単語の類似度の計算と見なすことができる.
分散表現を用いることで単語間の類似度を求めることができる,値が大きいほど単語がそれぞれ類似した実数ベクトルであることを表す.単語Aと単語Bの実数ベクトルが類似しているとは,単語Aと共に使われることの多い単語と単語Bと共に使われることの多い単語が多く共通していることを示す.故に,分散表現を使って単語の類似度を計算することができる.
%
発言文から単語を選ぶ際には自動要約を用いる.発言文から重要でない単語を取り除くことで関連度の計算の精度を高めることが可能となる.
要約の手法としてはokapi BM25 \cite{okapiBM25}とLexRankを組み合わせた抽出的要約手法を用いる.
\begin{comment}
%======================================= 社会的背景
2013年頃からWeb上での大規模な議論活動が活発になり,大規模な人数での議論が期待されている.
大規模な議論では意見を共有することは可能であるが,議論を整理させることや収束させることは難しい.以上から大規模意見集約システムCOLLAGREEが開発された.本システムではWeb上で適切に大規模な議論を行うことができるように議論をマネジメントするファシリテーターを導入した\cite{collagreeTest}.
過去の実験ではファシリテーターの存在が議論の集約に大きな役割を果たしていることが認識されており,大規模な議論のためにファシリテータは必要である.しかし,議論の規模に伴って議論時間が長くなる傾向があり,同時にファシリテーターは常に議論の動向を見続ける必要がある.故に,議論の規模が大きくなればなるほどファシリテーターは長時間かつ大規模な議論の動向の監視によって大きな負担がかかる.大規模な議論が増加する傾向を踏まえるとファシリテーターにかかる負担を軽減する支援が必要である.\\
以上の問題を解決するため,話題の変化を追い,重要な話題の転換点をファシリテーターの代わりに検出することが有用であると考える.必要な時にだけファシリテーターが画面を見れば良いようにすることでファシリテーターの負担軽減が期待できる.
%========================================= 現行手法問題点背景
%議論支援に関する先行研究において,既存の手法は全てが文字列を文字列のまま扱う手法である.
%既存手法は殆どがパターンマッチングと重み付けの2つに区分することができる.
%パターンマッチングでは事前に単語を登録して,単語がマッチした場合に処理を行うが,処理それぞれに対して単語を登録しなければならず手間が膨大になってしまう.また,単語の意味が考慮されておらず,手作業で登録を行うので登録漏れがあった場合に単語の意味に関係なく処理を行うことが不可能となってしまう.
%重み付けは単語の出現頻度や文章の長さを使用して単語・文章に順位を付ける手法で必ずしも単語の登録が必要でないため多くの研究で使用されている.
%しかし,重み付けもまた単語を文字列のまま扱っており,意味までは考慮されていない.故に ,人間なら対応できる似た単語でも1文字違うだけで対処が困難となる.
議論支援に関する先行研究においてファシリテーターに対する支援を目的としたものは無く,殆どが議論の活性化や可視化を目的としている.
%=================================新手法
近年,自然言語処理の分野において分散表現が多くの研究で使われており.分散表現は文字列である単語を辞書データを使用して実数ベクトルへと変換する.辞書データにない単語には対応できないが,多様な処理を1つの辞書データで行うことができる.また,実数ベクトルの各数値が単語の意味を表現するものとなっており,数値を使用して処理を行うことができる.
分散表現を用いることで既存手法より人間の感覚に近しい処理を行うことができる.
%=================================
以上のような背景を踏まえて,分散表現を用いてファシリテーターの代わりに話題の変化を判定し,知らせることを目指す.
話題転換の検出は発言同士の近さ,すなわち発言に含まれる単語意味の近さと見ることができる.
分散表現ではベクトル同士の内積計算を行うことで単語同士の意味の近さを計算することができる.
また,分散表現を使用することで機械翻訳を始めとする複数の分野で精度の向上が確認されている.
\end{comment}
\section{研究の目的}
\label{intro:taget}
本論文では,分散表現を用いて議論中での発言に含まれる単語の関連度を計算し,話題の変化を観測する手法を提案する.

\section{本論文の構成}
本論文の構成を以下に示す.
\ref{relwork:chapter} 章では要約手法に関する研究と,分散表現に関する先行研究を紹介する.
次に,\ref{model:chapter}章では発言の要約手法の説明を行い,\ref{impl:chapter}章では分散表現を用いた単語集合間の関連度計算について説明する.
そして,\ref{exp:chapter}章では話題転換点の検出の評価実験について説明する.
最後に\ref{con:chapter}章で本論文のまとめと考察を示す.

 %-------------------------------------------------------------------------------
 \expandafter\ifx\csname MasterFile\endcsname\relax
	\def\BibFile{hoge}
	\input{../Bibliography/chapter}
  \fi
  %-------------------------------------------------------------------------------
  \expandafter\ifx\csname MasterFile\endcsname\relax
  \end{document}
  \fi

  \fi
  %-------------------------------------------------------------------------------
  \expandafter\ifx\csname MasterFile\endcsname\relax
  \end{document}
  \fi

%-------------------------------------------------------------------------------
\expandafter\ifx\csname MasterFile\endcsname\relax
	\def\SubFile{hoge}
	\documentclass[a4j,12pt,twoside,openany]{jreport}
%\nofiles %tocファイルを更新させない
%\documentclass[12pt,a4j,twoside,openany]{jsbook}
\usepackage[dvipdfmx]{graphicx}
\usepackage{../dspc} % ベースラインスキップの指定
\usepackage{../slashbox} % 表に斜線を入れる
%\usepackage{../mediabb}
\usepackage{fancyvrb} % Verbatim環境
\usepackage{fancyhdr} % Headerの下線付き章見出し
\usepackage{here} % float[H]
\usepackage{multirow}
\usepackage{hhline} % 表の罫線の角を美しくする
\usepackage{amsmath} %コレがないとcasesが動かない
\usepackage{amsfonts} % 数学用フォント
\usepackage{bm} % 数式環境での bold
\usepackage{algorithm}
\usepackage{algorithmicx}
\usepackage[noend]{algpseudocode}
\usepackage[flushleft]{threeparttable} % 脚注付きテーブル
\usepackage{enumitem}
\usepackage{comment}
\usepackage{fancybox}
%\usepackage{csvsimple,booktabs,siunitx}
%\usepackage{filecontents}


\setlength{\evensidemargin}{5pt}
\setlength{\oddsidemargin}{40pt}
%\setlength{\headheight}{16.5pt}
%%\setlength{\headheight}{30pt}
\setcounter{secnumdepth}{3}
\setlist[description]{leftmargin=2\parindent,labelindent=\parindent}

\makeatletter
\def\@makechapterhead#1{%
	\vspace*{50\p@}%
	{
		\parindent \z@ \raggedright \normalfont
		\ifnum \c@secnumdepth >\m@ne
		% \if@mainmatter
			\huge\bfseries\@chapapp\thechapter\@chappos
			\par\nobreak
			\vskip 20\p@
		% \fi
		\fi
		\interlinepenalty\@M
		\Huge\bfseries #1\par\nobreak
		\vskip 40\p@
	}
}

%新しいコマンド定義
\newcounter{linenumber}
\newenvironment{listing}{%
  \begin{list}{%
    \small\arabic{linenumber}:}{%
      \usecounter{linenumber}%
      \setlength{\baselineskip}{18pt}%
      \setlength{\itemsep}{0pt}%
      \setlength{\parsep}{0pt}}}%
 {\end{list}}
\newcommand{\figcaption}[1]{\def\@captype{figure}\caption{#1}}
\newcommand{\tblcaption}[1]{\def\@captype{table}\caption{#1}}
\newcommand{\norm}[1]{\left\| #1 \right\|}
\newcommand{\cc}[1]{\multicolumn{1}{|c|}{#1}}
\newcommand{\circled}[1]{\raisebox{.5pt}{\textcircled{\raisebox{-.9pt} {#1}}}}
\newcommand{\specialcell}[2][c]{%
  \begin{tabular}[#1]{@{}c@{}}#2\end{tabular}}
\makeatother
%===============================================================================
\expandafter\ifx\csname SubFile\endcsname\relax
\begin{document}
\def\MasterFile{hoge}
%-------------------------------------------------------------------------------
%\maketitle
\thispagestyle{empty}
\input{../hyoushi/title}
%\addcontentsline{toc}{chapter}{表紙}
\thispagestyle{empty}
\mbox{}\newpage
%===============================================================================
%\frontmatter
%===============================================================================
%\mainmatter
%-------------------------------------------------------------------------------
\pagenumbering{arabic}
\cleardoublepage
\input{../0.Abstract/chapter}
%-------------------------------------------------------------------------------
\clearpage
\addcontentsline{toc}{chapter}{目次}
\tableofcontents

\clearpage
\addcontentsline{toc}{chapter}{図目次}
\listoffigures

\clearpage
\addcontentsline{toc}{chapter}{表目次}
\listoftables

%-------------------------------------------------------------------------------

%=====================
\pagestyle{fancy} % Headerをつける
\renewcommand{\sectionmark}[1]{\markright{\thesection\ \ \ #1}}
\renewcommand{\chaptermark}[1]{\markboth{#1}{}}
\lhead{}
\chead{}
\lfoot{}
\rfoot{}%-------------------------------------------------------------------------------
\input{../1.Introduction/chapter}
%-------------------------------------------------------------------------------
\input{../2.Related_Work/chapter}
%-------------------------------------------------------------------------------
\input{../3.The_Model/chapter}
%-------------------------------------------------------------------------------
\input{../4.Implementation/chapter}
%-------------------------------------------------------------------------------
\input{../5.Experiments/chapter}
%-------------------------------------------------------------------------------
\input{../6.Conclusion/chapter}

%===============================================================================
\pagestyle{plain}
%-------------------------------------------------------------------------------
\input{../7.Acknowledgement/chapter} %謝辞
%-------------------------------------------------------------------------------
\def\BibFile{../Bibliograhoy/database2}
\input{../Bibliography/chapter} %参考文献
% %===============================================================================
\appendix
\input{../A.Mypaper/chapter} % 投稿論文リスト
\input{../B.SIG-CCI2/chapter} %
\input{../C.IJCAI-16/chapter} %
%===============================================================================
\end{document}
\fi

	\begin{document}
	\setcounter{chapter}{0}
	\fi
  %-------------------------------------------------------------------------------
\cleardoublepage
\chapter{序論}
\label{intro:chapter}
%本章では, 本研究を行なうに至った背景と目的について述べる.その後,本論文の構成について述べる.
\section{研究の背景}
\label{intro:background}
近年,Web上での大規模な議論活動が活発になっているが,現在一般的に使われている "2ちゃんねる" や "Twitter" といったシステムでは整理や収束を行うことが困難である.困難である原因として,議論の管理を行う者がいないことが挙げられる.
つまり,議論を整理・収束させるには議論のマネジメントを行う人物が必要である.
%
大規模意見集約システムCOLLAGREE\cite{collagreeTest}ではファシリテーターと呼ばれる人物が議論のマネジメントを行っている.
しかし,ファシリテーターは人間であり,長時間に渡って大人数での議論の動向をマネジメントし続けるのは困難である.
COLLAGREEで大規模な議論を収束させるためには,ファシリテーターが必要な時には画面を見るようにして,他の時は見なくても済むようにすることで画面に向き合う時間を減らす工夫があることが望ましい.ファシリテーターが画面を見るべきタイミングは議論の話題が変化したときである.以前の議論の内容から外れた発言がされた時,ファシリテーターが適切な発言をすることで,脱線や炎上を避けて議論を収束させることができる.
すなわち,ファシリテーターの代わりに自動的に議論中の話題の変化を観測することが求められている.
%
現在,COLLAGREE上で使用されている議論支援システムは「(1)投稿支援システム」と「(2)議論可視化システム」の2つに大別できる.
投稿支援システムはポイント機能やファシリテーションフレーズ簡易投稿機能のように,ユーザーが投稿をする際に何らかの補助やリアクションを行う.現行の機能では選択肢の提示に留まっており,作業量を減らすことには繋がりにくい.
一方,議論可視化システムは議論ツリーやキーワード抽出のように,ユーザーにスレッドとは異なる議論の見方を提供する.
\ref{Fig:argTree1}に議論ツリーの例を示す.
\begin{figure}[htbp]
 \begin{center}
  \includegraphics[width=\textwidth]{../images/2.Related_Work/argTree1.png}
  \caption{議論ツリー}
  \label{Fig:argTree1}
  \vspace{-10pt}
 \end{center}
\end{figure}
現行の機能では議論を見やすくすることに重点が置かれており,議論の把握の助けにはなるが画面に向き合う時間を減らすことにはなりにくい.むしろ,作業量を増やすことになり得ることもある.
従って,現行の支援機能ではファシリテーターの作業量の減少には繋がりにくい.
%
近年,自然言語処理の分野において分散表現が多くの研究で使われており,機械翻訳を始めとする単語の意味が重要となる分野で精度の向上が確認されている.分散表現を用いることで,人間に近い精度で話題の変化を観測することが可能となる.
%
以上のような背景を踏まえて,分散表現を用いて,話題の変化を観測し,話題の変化が確認された時にファシリテーターに伝えることが望ましい.
話題の変化の観測は,発言中に現れる単語の類似度の計算と見なすことができる.
分散表現を用いることで単語間の類似度を求めることができる,値が大きいほど単語がそれぞれ類似した実数ベクトルであることを表す.単語Aと単語Bの実数ベクトルが類似しているとは,単語Aと共に使われることの多い単語と単語Bと共に使われることの多い単語が多く共通していることを示す.故に,分散表現を使って単語の類似度を計算することができる.
%
発言文から単語を選ぶ際には自動要約を用いる.発言文から重要でない単語を取り除くことで関連度の計算の精度を高めることが可能となる.
要約の手法としてはokapi BM25 \cite{okapiBM25}とLexRankを組み合わせた抽出的要約手法を用いる.
\begin{comment}
%======================================= 社会的背景
2013年頃からWeb上での大規模な議論活動が活発になり,大規模な人数での議論が期待されている.
大規模な議論では意見を共有することは可能であるが,議論を整理させることや収束させることは難しい.以上から大規模意見集約システムCOLLAGREEが開発された.本システムではWeb上で適切に大規模な議論を行うことができるように議論をマネジメントするファシリテーターを導入した\cite{collagreeTest}.
過去の実験ではファシリテーターの存在が議論の集約に大きな役割を果たしていることが認識されており,大規模な議論のためにファシリテータは必要である.しかし,議論の規模に伴って議論時間が長くなる傾向があり,同時にファシリテーターは常に議論の動向を見続ける必要がある.故に,議論の規模が大きくなればなるほどファシリテーターは長時間かつ大規模な議論の動向の監視によって大きな負担がかかる.大規模な議論が増加する傾向を踏まえるとファシリテーターにかかる負担を軽減する支援が必要である.\\
以上の問題を解決するため,話題の変化を追い,重要な話題の転換点をファシリテーターの代わりに検出することが有用であると考える.必要な時にだけファシリテーターが画面を見れば良いようにすることでファシリテーターの負担軽減が期待できる.
%========================================= 現行手法問題点背景
%議論支援に関する先行研究において,既存の手法は全てが文字列を文字列のまま扱う手法である.
%既存手法は殆どがパターンマッチングと重み付けの2つに区分することができる.
%パターンマッチングでは事前に単語を登録して,単語がマッチした場合に処理を行うが,処理それぞれに対して単語を登録しなければならず手間が膨大になってしまう.また,単語の意味が考慮されておらず,手作業で登録を行うので登録漏れがあった場合に単語の意味に関係なく処理を行うことが不可能となってしまう.
%重み付けは単語の出現頻度や文章の長さを使用して単語・文章に順位を付ける手法で必ずしも単語の登録が必要でないため多くの研究で使用されている.
%しかし,重み付けもまた単語を文字列のまま扱っており,意味までは考慮されていない.故に ,人間なら対応できる似た単語でも1文字違うだけで対処が困難となる.
議論支援に関する先行研究においてファシリテーターに対する支援を目的としたものは無く,殆どが議論の活性化や可視化を目的としている.
%=================================新手法
近年,自然言語処理の分野において分散表現が多くの研究で使われており.分散表現は文字列である単語を辞書データを使用して実数ベクトルへと変換する.辞書データにない単語には対応できないが,多様な処理を1つの辞書データで行うことができる.また,実数ベクトルの各数値が単語の意味を表現するものとなっており,数値を使用して処理を行うことができる.
分散表現を用いることで既存手法より人間の感覚に近しい処理を行うことができる.
%=================================
以上のような背景を踏まえて,分散表現を用いてファシリテーターの代わりに話題の変化を判定し,知らせることを目指す.
話題転換の検出は発言同士の近さ,すなわち発言に含まれる単語意味の近さと見ることができる.
分散表現ではベクトル同士の内積計算を行うことで単語同士の意味の近さを計算することができる.
また,分散表現を使用することで機械翻訳を始めとする複数の分野で精度の向上が確認されている.
\end{comment}
\section{研究の目的}
\label{intro:taget}
本論文では,分散表現を用いて議論中での発言に含まれる単語の関連度を計算し,話題の変化を観測する手法を提案する.

\section{本論文の構成}
本論文の構成を以下に示す.
\ref{relwork:chapter} 章では要約手法に関する研究と,分散表現に関する先行研究を紹介する.
次に,\ref{model:chapter}章では発言の要約手法の説明を行い,\ref{impl:chapter}章では分散表現を用いた単語集合間の関連度計算について説明する.
そして,\ref{exp:chapter}章では話題転換点の検出の評価実験について説明する.
最後に\ref{con:chapter}章で本論文のまとめと考察を示す.

 %-------------------------------------------------------------------------------
 \expandafter\ifx\csname MasterFile\endcsname\relax
	\def\BibFile{hoge}
	\expandafter\ifx\csname MasterFile\endcsname\relax
	\def\SubFile{hoge}
	\input{../thesis/thesis}
	\begin{document}
	\setcounter{chapter}{0}
	\fi
  %-------------------------------------------------------------------------------
\cleardoublepage
\chapter{序論}
\label{intro:chapter}
%本章では, 本研究を行なうに至った背景と目的について述べる.その後,本論文の構成について述べる.
\section{研究の背景}
\label{intro:background}
近年,Web上での大規模な議論活動が活発になっているが,現在一般的に使われている "2ちゃんねる" や "Twitter" といったシステムでは整理や収束を行うことが困難である.困難である原因として,議論の管理を行う者がいないことが挙げられる.
つまり,議論を整理・収束させるには議論のマネジメントを行う人物が必要である.
%
大規模意見集約システムCOLLAGREE\cite{collagreeTest}ではファシリテーターと呼ばれる人物が議論のマネジメントを行っている.
しかし,ファシリテーターは人間であり,長時間に渡って大人数での議論の動向をマネジメントし続けるのは困難である.
COLLAGREEで大規模な議論を収束させるためには,ファシリテーターが必要な時には画面を見るようにして,他の時は見なくても済むようにすることで画面に向き合う時間を減らす工夫があることが望ましい.ファシリテーターが画面を見るべきタイミングは議論の話題が変化したときである.以前の議論の内容から外れた発言がされた時,ファシリテーターが適切な発言をすることで,脱線や炎上を避けて議論を収束させることができる.
すなわち,ファシリテーターの代わりに自動的に議論中の話題の変化を観測することが求められている.
%
現在,COLLAGREE上で使用されている議論支援システムは「(1)投稿支援システム」と「(2)議論可視化システム」の2つに大別できる.
投稿支援システムはポイント機能やファシリテーションフレーズ簡易投稿機能のように,ユーザーが投稿をする際に何らかの補助やリアクションを行う.現行の機能では選択肢の提示に留まっており,作業量を減らすことには繋がりにくい.
一方,議論可視化システムは議論ツリーやキーワード抽出のように,ユーザーにスレッドとは異なる議論の見方を提供する.
\ref{Fig:argTree1}に議論ツリーの例を示す.
\begin{figure}[htbp]
 \begin{center}
  \includegraphics[width=\textwidth]{../images/2.Related_Work/argTree1.png}
  \caption{議論ツリー}
  \label{Fig:argTree1}
  \vspace{-10pt}
 \end{center}
\end{figure}
現行の機能では議論を見やすくすることに重点が置かれており,議論の把握の助けにはなるが画面に向き合う時間を減らすことにはなりにくい.むしろ,作業量を増やすことになり得ることもある.
従って,現行の支援機能ではファシリテーターの作業量の減少には繋がりにくい.
%
近年,自然言語処理の分野において分散表現が多くの研究で使われており,機械翻訳を始めとする単語の意味が重要となる分野で精度の向上が確認されている.分散表現を用いることで,人間に近い精度で話題の変化を観測することが可能となる.
%
以上のような背景を踏まえて,分散表現を用いて,話題の変化を観測し,話題の変化が確認された時にファシリテーターに伝えることが望ましい.
話題の変化の観測は,発言中に現れる単語の類似度の計算と見なすことができる.
分散表現を用いることで単語間の類似度を求めることができる,値が大きいほど単語がそれぞれ類似した実数ベクトルであることを表す.単語Aと単語Bの実数ベクトルが類似しているとは,単語Aと共に使われることの多い単語と単語Bと共に使われることの多い単語が多く共通していることを示す.故に,分散表現を使って単語の類似度を計算することができる.
%
発言文から単語を選ぶ際には自動要約を用いる.発言文から重要でない単語を取り除くことで関連度の計算の精度を高めることが可能となる.
要約の手法としてはokapi BM25 \cite{okapiBM25}とLexRankを組み合わせた抽出的要約手法を用いる.
\begin{comment}
%======================================= 社会的背景
2013年頃からWeb上での大規模な議論活動が活発になり,大規模な人数での議論が期待されている.
大規模な議論では意見を共有することは可能であるが,議論を整理させることや収束させることは難しい.以上から大規模意見集約システムCOLLAGREEが開発された.本システムではWeb上で適切に大規模な議論を行うことができるように議論をマネジメントするファシリテーターを導入した\cite{collagreeTest}.
過去の実験ではファシリテーターの存在が議論の集約に大きな役割を果たしていることが認識されており,大規模な議論のためにファシリテータは必要である.しかし,議論の規模に伴って議論時間が長くなる傾向があり,同時にファシリテーターは常に議論の動向を見続ける必要がある.故に,議論の規模が大きくなればなるほどファシリテーターは長時間かつ大規模な議論の動向の監視によって大きな負担がかかる.大規模な議論が増加する傾向を踏まえるとファシリテーターにかかる負担を軽減する支援が必要である.\\
以上の問題を解決するため,話題の変化を追い,重要な話題の転換点をファシリテーターの代わりに検出することが有用であると考える.必要な時にだけファシリテーターが画面を見れば良いようにすることでファシリテーターの負担軽減が期待できる.
%========================================= 現行手法問題点背景
%議論支援に関する先行研究において,既存の手法は全てが文字列を文字列のまま扱う手法である.
%既存手法は殆どがパターンマッチングと重み付けの2つに区分することができる.
%パターンマッチングでは事前に単語を登録して,単語がマッチした場合に処理を行うが,処理それぞれに対して単語を登録しなければならず手間が膨大になってしまう.また,単語の意味が考慮されておらず,手作業で登録を行うので登録漏れがあった場合に単語の意味に関係なく処理を行うことが不可能となってしまう.
%重み付けは単語の出現頻度や文章の長さを使用して単語・文章に順位を付ける手法で必ずしも単語の登録が必要でないため多くの研究で使用されている.
%しかし,重み付けもまた単語を文字列のまま扱っており,意味までは考慮されていない.故に ,人間なら対応できる似た単語でも1文字違うだけで対処が困難となる.
議論支援に関する先行研究においてファシリテーターに対する支援を目的としたものは無く,殆どが議論の活性化や可視化を目的としている.
%=================================新手法
近年,自然言語処理の分野において分散表現が多くの研究で使われており.分散表現は文字列である単語を辞書データを使用して実数ベクトルへと変換する.辞書データにない単語には対応できないが,多様な処理を1つの辞書データで行うことができる.また,実数ベクトルの各数値が単語の意味を表現するものとなっており,数値を使用して処理を行うことができる.
分散表現を用いることで既存手法より人間の感覚に近しい処理を行うことができる.
%=================================
以上のような背景を踏まえて,分散表現を用いてファシリテーターの代わりに話題の変化を判定し,知らせることを目指す.
話題転換の検出は発言同士の近さ,すなわち発言に含まれる単語意味の近さと見ることができる.
分散表現ではベクトル同士の内積計算を行うことで単語同士の意味の近さを計算することができる.
また,分散表現を使用することで機械翻訳を始めとする複数の分野で精度の向上が確認されている.
\end{comment}
\section{研究の目的}
\label{intro:taget}
本論文では,分散表現を用いて議論中での発言に含まれる単語の関連度を計算し,話題の変化を観測する手法を提案する.

\section{本論文の構成}
本論文の構成を以下に示す.
\ref{relwork:chapter} 章では要約手法に関する研究と,分散表現に関する先行研究を紹介する.
次に,\ref{model:chapter}章では発言の要約手法の説明を行い,\ref{impl:chapter}章では分散表現を用いた単語集合間の関連度計算について説明する.
そして,\ref{exp:chapter}章では話題転換点の検出の評価実験について説明する.
最後に\ref{con:chapter}章で本論文のまとめと考察を示す.

 %-------------------------------------------------------------------------------
 \expandafter\ifx\csname MasterFile\endcsname\relax
	\def\BibFile{hoge}
	\input{../Bibliography/chapter}
  \fi
  %-------------------------------------------------------------------------------
  \expandafter\ifx\csname MasterFile\endcsname\relax
  \end{document}
  \fi

  \fi
  %-------------------------------------------------------------------------------
  \expandafter\ifx\csname MasterFile\endcsname\relax
  \end{document}
  \fi

%-------------------------------------------------------------------------------
\expandafter\ifx\csname MasterFile\endcsname\relax
	\def\SubFile{hoge}
	\documentclass[a4j,12pt,twoside,openany]{jreport}
%\nofiles %tocファイルを更新させない
%\documentclass[12pt,a4j,twoside,openany]{jsbook}
\usepackage[dvipdfmx]{graphicx}
\usepackage{../dspc} % ベースラインスキップの指定
\usepackage{../slashbox} % 表に斜線を入れる
%\usepackage{../mediabb}
\usepackage{fancyvrb} % Verbatim環境
\usepackage{fancyhdr} % Headerの下線付き章見出し
\usepackage{here} % float[H]
\usepackage{multirow}
\usepackage{hhline} % 表の罫線の角を美しくする
\usepackage{amsmath} %コレがないとcasesが動かない
\usepackage{amsfonts} % 数学用フォント
\usepackage{bm} % 数式環境での bold
\usepackage{algorithm}
\usepackage{algorithmicx}
\usepackage[noend]{algpseudocode}
\usepackage[flushleft]{threeparttable} % 脚注付きテーブル
\usepackage{enumitem}
\usepackage{comment}
\usepackage{fancybox}
%\usepackage{csvsimple,booktabs,siunitx}
%\usepackage{filecontents}


\setlength{\evensidemargin}{5pt}
\setlength{\oddsidemargin}{40pt}
%\setlength{\headheight}{16.5pt}
%%\setlength{\headheight}{30pt}
\setcounter{secnumdepth}{3}
\setlist[description]{leftmargin=2\parindent,labelindent=\parindent}

\makeatletter
\def\@makechapterhead#1{%
	\vspace*{50\p@}%
	{
		\parindent \z@ \raggedright \normalfont
		\ifnum \c@secnumdepth >\m@ne
		% \if@mainmatter
			\huge\bfseries\@chapapp\thechapter\@chappos
			\par\nobreak
			\vskip 20\p@
		% \fi
		\fi
		\interlinepenalty\@M
		\Huge\bfseries #1\par\nobreak
		\vskip 40\p@
	}
}

%新しいコマンド定義
\newcounter{linenumber}
\newenvironment{listing}{%
  \begin{list}{%
    \small\arabic{linenumber}:}{%
      \usecounter{linenumber}%
      \setlength{\baselineskip}{18pt}%
      \setlength{\itemsep}{0pt}%
      \setlength{\parsep}{0pt}}}%
 {\end{list}}
\newcommand{\figcaption}[1]{\def\@captype{figure}\caption{#1}}
\newcommand{\tblcaption}[1]{\def\@captype{table}\caption{#1}}
\newcommand{\norm}[1]{\left\| #1 \right\|}
\newcommand{\cc}[1]{\multicolumn{1}{|c|}{#1}}
\newcommand{\circled}[1]{\raisebox{.5pt}{\textcircled{\raisebox{-.9pt} {#1}}}}
\newcommand{\specialcell}[2][c]{%
  \begin{tabular}[#1]{@{}c@{}}#2\end{tabular}}
\makeatother
%===============================================================================
\expandafter\ifx\csname SubFile\endcsname\relax
\begin{document}
\def\MasterFile{hoge}
%-------------------------------------------------------------------------------
%\maketitle
\thispagestyle{empty}
\input{../hyoushi/title}
%\addcontentsline{toc}{chapter}{表紙}
\thispagestyle{empty}
\mbox{}\newpage
%===============================================================================
%\frontmatter
%===============================================================================
%\mainmatter
%-------------------------------------------------------------------------------
\pagenumbering{arabic}
\cleardoublepage
\input{../0.Abstract/chapter}
%-------------------------------------------------------------------------------
\clearpage
\addcontentsline{toc}{chapter}{目次}
\tableofcontents

\clearpage
\addcontentsline{toc}{chapter}{図目次}
\listoffigures

\clearpage
\addcontentsline{toc}{chapter}{表目次}
\listoftables

%-------------------------------------------------------------------------------

%=====================
\pagestyle{fancy} % Headerをつける
\renewcommand{\sectionmark}[1]{\markright{\thesection\ \ \ #1}}
\renewcommand{\chaptermark}[1]{\markboth{#1}{}}
\lhead{}
\chead{}
\lfoot{}
\rfoot{}%-------------------------------------------------------------------------------
\input{../1.Introduction/chapter}
%-------------------------------------------------------------------------------
\input{../2.Related_Work/chapter}
%-------------------------------------------------------------------------------
\input{../3.The_Model/chapter}
%-------------------------------------------------------------------------------
\input{../4.Implementation/chapter}
%-------------------------------------------------------------------------------
\input{../5.Experiments/chapter}
%-------------------------------------------------------------------------------
\input{../6.Conclusion/chapter}

%===============================================================================
\pagestyle{plain}
%-------------------------------------------------------------------------------
\input{../7.Acknowledgement/chapter} %謝辞
%-------------------------------------------------------------------------------
\def\BibFile{../Bibliograhoy/database2}
\input{../Bibliography/chapter} %参考文献
% %===============================================================================
\appendix
\input{../A.Mypaper/chapter} % 投稿論文リスト
\input{../B.SIG-CCI2/chapter} %
\input{../C.IJCAI-16/chapter} %
%===============================================================================
\end{document}
\fi

	\begin{document}
	\setcounter{chapter}{0}
	\fi
  %-------------------------------------------------------------------------------
\cleardoublepage
\chapter{序論}
\label{intro:chapter}
%本章では, 本研究を行なうに至った背景と目的について述べる.その後,本論文の構成について述べる.
\section{研究の背景}
\label{intro:background}
近年,Web上での大規模な議論活動が活発になっているが,現在一般的に使われている "2ちゃんねる" や "Twitter" といったシステムでは整理や収束を行うことが困難である.困難である原因として,議論の管理を行う者がいないことが挙げられる.
つまり,議論を整理・収束させるには議論のマネジメントを行う人物が必要である.
%
大規模意見集約システムCOLLAGREE\cite{collagreeTest}ではファシリテーターと呼ばれる人物が議論のマネジメントを行っている.
しかし,ファシリテーターは人間であり,長時間に渡って大人数での議論の動向をマネジメントし続けるのは困難である.
COLLAGREEで大規模な議論を収束させるためには,ファシリテーターが必要な時には画面を見るようにして,他の時は見なくても済むようにすることで画面に向き合う時間を減らす工夫があることが望ましい.ファシリテーターが画面を見るべきタイミングは議論の話題が変化したときである.以前の議論の内容から外れた発言がされた時,ファシリテーターが適切な発言をすることで,脱線や炎上を避けて議論を収束させることができる.
すなわち,ファシリテーターの代わりに自動的に議論中の話題の変化を観測することが求められている.
%
現在,COLLAGREE上で使用されている議論支援システムは「(1)投稿支援システム」と「(2)議論可視化システム」の2つに大別できる.
投稿支援システムはポイント機能やファシリテーションフレーズ簡易投稿機能のように,ユーザーが投稿をする際に何らかの補助やリアクションを行う.現行の機能では選択肢の提示に留まっており,作業量を減らすことには繋がりにくい.
一方,議論可視化システムは議論ツリーやキーワード抽出のように,ユーザーにスレッドとは異なる議論の見方を提供する.
\ref{Fig:argTree1}に議論ツリーの例を示す.
\begin{figure}[htbp]
 \begin{center}
  \includegraphics[width=\textwidth]{../images/2.Related_Work/argTree1.png}
  \caption{議論ツリー}
  \label{Fig:argTree1}
  \vspace{-10pt}
 \end{center}
\end{figure}
現行の機能では議論を見やすくすることに重点が置かれており,議論の把握の助けにはなるが画面に向き合う時間を減らすことにはなりにくい.むしろ,作業量を増やすことになり得ることもある.
従って,現行の支援機能ではファシリテーターの作業量の減少には繋がりにくい.
%
近年,自然言語処理の分野において分散表現が多くの研究で使われており,機械翻訳を始めとする単語の意味が重要となる分野で精度の向上が確認されている.分散表現を用いることで,人間に近い精度で話題の変化を観測することが可能となる.
%
以上のような背景を踏まえて,分散表現を用いて,話題の変化を観測し,話題の変化が確認された時にファシリテーターに伝えることが望ましい.
話題の変化の観測は,発言中に現れる単語の類似度の計算と見なすことができる.
分散表現を用いることで単語間の類似度を求めることができる,値が大きいほど単語がそれぞれ類似した実数ベクトルであることを表す.単語Aと単語Bの実数ベクトルが類似しているとは,単語Aと共に使われることの多い単語と単語Bと共に使われることの多い単語が多く共通していることを示す.故に,分散表現を使って単語の類似度を計算することができる.
%
発言文から単語を選ぶ際には自動要約を用いる.発言文から重要でない単語を取り除くことで関連度の計算の精度を高めることが可能となる.
要約の手法としてはokapi BM25 \cite{okapiBM25}とLexRankを組み合わせた抽出的要約手法を用いる.
\begin{comment}
%======================================= 社会的背景
2013年頃からWeb上での大規模な議論活動が活発になり,大規模な人数での議論が期待されている.
大規模な議論では意見を共有することは可能であるが,議論を整理させることや収束させることは難しい.以上から大規模意見集約システムCOLLAGREEが開発された.本システムではWeb上で適切に大規模な議論を行うことができるように議論をマネジメントするファシリテーターを導入した\cite{collagreeTest}.
過去の実験ではファシリテーターの存在が議論の集約に大きな役割を果たしていることが認識されており,大規模な議論のためにファシリテータは必要である.しかし,議論の規模に伴って議論時間が長くなる傾向があり,同時にファシリテーターは常に議論の動向を見続ける必要がある.故に,議論の規模が大きくなればなるほどファシリテーターは長時間かつ大規模な議論の動向の監視によって大きな負担がかかる.大規模な議論が増加する傾向を踏まえるとファシリテーターにかかる負担を軽減する支援が必要である.\\
以上の問題を解決するため,話題の変化を追い,重要な話題の転換点をファシリテーターの代わりに検出することが有用であると考える.必要な時にだけファシリテーターが画面を見れば良いようにすることでファシリテーターの負担軽減が期待できる.
%========================================= 現行手法問題点背景
%議論支援に関する先行研究において,既存の手法は全てが文字列を文字列のまま扱う手法である.
%既存手法は殆どがパターンマッチングと重み付けの2つに区分することができる.
%パターンマッチングでは事前に単語を登録して,単語がマッチした場合に処理を行うが,処理それぞれに対して単語を登録しなければならず手間が膨大になってしまう.また,単語の意味が考慮されておらず,手作業で登録を行うので登録漏れがあった場合に単語の意味に関係なく処理を行うことが不可能となってしまう.
%重み付けは単語の出現頻度や文章の長さを使用して単語・文章に順位を付ける手法で必ずしも単語の登録が必要でないため多くの研究で使用されている.
%しかし,重み付けもまた単語を文字列のまま扱っており,意味までは考慮されていない.故に ,人間なら対応できる似た単語でも1文字違うだけで対処が困難となる.
議論支援に関する先行研究においてファシリテーターに対する支援を目的としたものは無く,殆どが議論の活性化や可視化を目的としている.
%=================================新手法
近年,自然言語処理の分野において分散表現が多くの研究で使われており.分散表現は文字列である単語を辞書データを使用して実数ベクトルへと変換する.辞書データにない単語には対応できないが,多様な処理を1つの辞書データで行うことができる.また,実数ベクトルの各数値が単語の意味を表現するものとなっており,数値を使用して処理を行うことができる.
分散表現を用いることで既存手法より人間の感覚に近しい処理を行うことができる.
%=================================
以上のような背景を踏まえて,分散表現を用いてファシリテーターの代わりに話題の変化を判定し,知らせることを目指す.
話題転換の検出は発言同士の近さ,すなわち発言に含まれる単語意味の近さと見ることができる.
分散表現ではベクトル同士の内積計算を行うことで単語同士の意味の近さを計算することができる.
また,分散表現を使用することで機械翻訳を始めとする複数の分野で精度の向上が確認されている.
\end{comment}
\section{研究の目的}
\label{intro:taget}
本論文では,分散表現を用いて議論中での発言に含まれる単語の関連度を計算し,話題の変化を観測する手法を提案する.

\section{本論文の構成}
本論文の構成を以下に示す.
\ref{relwork:chapter} 章では要約手法に関する研究と,分散表現に関する先行研究を紹介する.
次に,\ref{model:chapter}章では発言の要約手法の説明を行い,\ref{impl:chapter}章では分散表現を用いた単語集合間の関連度計算について説明する.
そして,\ref{exp:chapter}章では話題転換点の検出の評価実験について説明する.
最後に\ref{con:chapter}章で本論文のまとめと考察を示す.

 %-------------------------------------------------------------------------------
 \expandafter\ifx\csname MasterFile\endcsname\relax
	\def\BibFile{hoge}
	\expandafter\ifx\csname MasterFile\endcsname\relax
	\def\SubFile{hoge}
	\input{../thesis/thesis}
	\begin{document}
	\setcounter{chapter}{0}
	\fi
  %-------------------------------------------------------------------------------
\cleardoublepage
\chapter{序論}
\label{intro:chapter}
%本章では, 本研究を行なうに至った背景と目的について述べる.その後,本論文の構成について述べる.
\section{研究の背景}
\label{intro:background}
近年,Web上での大規模な議論活動が活発になっているが,現在一般的に使われている "2ちゃんねる" や "Twitter" といったシステムでは整理や収束を行うことが困難である.困難である原因として,議論の管理を行う者がいないことが挙げられる.
つまり,議論を整理・収束させるには議論のマネジメントを行う人物が必要である.
%
大規模意見集約システムCOLLAGREE\cite{collagreeTest}ではファシリテーターと呼ばれる人物が議論のマネジメントを行っている.
しかし,ファシリテーターは人間であり,長時間に渡って大人数での議論の動向をマネジメントし続けるのは困難である.
COLLAGREEで大規模な議論を収束させるためには,ファシリテーターが必要な時には画面を見るようにして,他の時は見なくても済むようにすることで画面に向き合う時間を減らす工夫があることが望ましい.ファシリテーターが画面を見るべきタイミングは議論の話題が変化したときである.以前の議論の内容から外れた発言がされた時,ファシリテーターが適切な発言をすることで,脱線や炎上を避けて議論を収束させることができる.
すなわち,ファシリテーターの代わりに自動的に議論中の話題の変化を観測することが求められている.
%
現在,COLLAGREE上で使用されている議論支援システムは「(1)投稿支援システム」と「(2)議論可視化システム」の2つに大別できる.
投稿支援システムはポイント機能やファシリテーションフレーズ簡易投稿機能のように,ユーザーが投稿をする際に何らかの補助やリアクションを行う.現行の機能では選択肢の提示に留まっており,作業量を減らすことには繋がりにくい.
一方,議論可視化システムは議論ツリーやキーワード抽出のように,ユーザーにスレッドとは異なる議論の見方を提供する.
\ref{Fig:argTree1}に議論ツリーの例を示す.
\begin{figure}[htbp]
 \begin{center}
  \includegraphics[width=\textwidth]{../images/2.Related_Work/argTree1.png}
  \caption{議論ツリー}
  \label{Fig:argTree1}
  \vspace{-10pt}
 \end{center}
\end{figure}
現行の機能では議論を見やすくすることに重点が置かれており,議論の把握の助けにはなるが画面に向き合う時間を減らすことにはなりにくい.むしろ,作業量を増やすことになり得ることもある.
従って,現行の支援機能ではファシリテーターの作業量の減少には繋がりにくい.
%
近年,自然言語処理の分野において分散表現が多くの研究で使われており,機械翻訳を始めとする単語の意味が重要となる分野で精度の向上が確認されている.分散表現を用いることで,人間に近い精度で話題の変化を観測することが可能となる.
%
以上のような背景を踏まえて,分散表現を用いて,話題の変化を観測し,話題の変化が確認された時にファシリテーターに伝えることが望ましい.
話題の変化の観測は,発言中に現れる単語の類似度の計算と見なすことができる.
分散表現を用いることで単語間の類似度を求めることができる,値が大きいほど単語がそれぞれ類似した実数ベクトルであることを表す.単語Aと単語Bの実数ベクトルが類似しているとは,単語Aと共に使われることの多い単語と単語Bと共に使われることの多い単語が多く共通していることを示す.故に,分散表現を使って単語の類似度を計算することができる.
%
発言文から単語を選ぶ際には自動要約を用いる.発言文から重要でない単語を取り除くことで関連度の計算の精度を高めることが可能となる.
要約の手法としてはokapi BM25 \cite{okapiBM25}とLexRankを組み合わせた抽出的要約手法を用いる.
\begin{comment}
%======================================= 社会的背景
2013年頃からWeb上での大規模な議論活動が活発になり,大規模な人数での議論が期待されている.
大規模な議論では意見を共有することは可能であるが,議論を整理させることや収束させることは難しい.以上から大規模意見集約システムCOLLAGREEが開発された.本システムではWeb上で適切に大規模な議論を行うことができるように議論をマネジメントするファシリテーターを導入した\cite{collagreeTest}.
過去の実験ではファシリテーターの存在が議論の集約に大きな役割を果たしていることが認識されており,大規模な議論のためにファシリテータは必要である.しかし,議論の規模に伴って議論時間が長くなる傾向があり,同時にファシリテーターは常に議論の動向を見続ける必要がある.故に,議論の規模が大きくなればなるほどファシリテーターは長時間かつ大規模な議論の動向の監視によって大きな負担がかかる.大規模な議論が増加する傾向を踏まえるとファシリテーターにかかる負担を軽減する支援が必要である.\\
以上の問題を解決するため,話題の変化を追い,重要な話題の転換点をファシリテーターの代わりに検出することが有用であると考える.必要な時にだけファシリテーターが画面を見れば良いようにすることでファシリテーターの負担軽減が期待できる.
%========================================= 現行手法問題点背景
%議論支援に関する先行研究において,既存の手法は全てが文字列を文字列のまま扱う手法である.
%既存手法は殆どがパターンマッチングと重み付けの2つに区分することができる.
%パターンマッチングでは事前に単語を登録して,単語がマッチした場合に処理を行うが,処理それぞれに対して単語を登録しなければならず手間が膨大になってしまう.また,単語の意味が考慮されておらず,手作業で登録を行うので登録漏れがあった場合に単語の意味に関係なく処理を行うことが不可能となってしまう.
%重み付けは単語の出現頻度や文章の長さを使用して単語・文章に順位を付ける手法で必ずしも単語の登録が必要でないため多くの研究で使用されている.
%しかし,重み付けもまた単語を文字列のまま扱っており,意味までは考慮されていない.故に ,人間なら対応できる似た単語でも1文字違うだけで対処が困難となる.
議論支援に関する先行研究においてファシリテーターに対する支援を目的としたものは無く,殆どが議論の活性化や可視化を目的としている.
%=================================新手法
近年,自然言語処理の分野において分散表現が多くの研究で使われており.分散表現は文字列である単語を辞書データを使用して実数ベクトルへと変換する.辞書データにない単語には対応できないが,多様な処理を1つの辞書データで行うことができる.また,実数ベクトルの各数値が単語の意味を表現するものとなっており,数値を使用して処理を行うことができる.
分散表現を用いることで既存手法より人間の感覚に近しい処理を行うことができる.
%=================================
以上のような背景を踏まえて,分散表現を用いてファシリテーターの代わりに話題の変化を判定し,知らせることを目指す.
話題転換の検出は発言同士の近さ,すなわち発言に含まれる単語意味の近さと見ることができる.
分散表現ではベクトル同士の内積計算を行うことで単語同士の意味の近さを計算することができる.
また,分散表現を使用することで機械翻訳を始めとする複数の分野で精度の向上が確認されている.
\end{comment}
\section{研究の目的}
\label{intro:taget}
本論文では,分散表現を用いて議論中での発言に含まれる単語の関連度を計算し,話題の変化を観測する手法を提案する.

\section{本論文の構成}
本論文の構成を以下に示す.
\ref{relwork:chapter} 章では要約手法に関する研究と,分散表現に関する先行研究を紹介する.
次に,\ref{model:chapter}章では発言の要約手法の説明を行い,\ref{impl:chapter}章では分散表現を用いた単語集合間の関連度計算について説明する.
そして,\ref{exp:chapter}章では話題転換点の検出の評価実験について説明する.
最後に\ref{con:chapter}章で本論文のまとめと考察を示す.

 %-------------------------------------------------------------------------------
 \expandafter\ifx\csname MasterFile\endcsname\relax
	\def\BibFile{hoge}
	\input{../Bibliography/chapter}
  \fi
  %-------------------------------------------------------------------------------
  \expandafter\ifx\csname MasterFile\endcsname\relax
  \end{document}
  \fi

  \fi
  %-------------------------------------------------------------------------------
  \expandafter\ifx\csname MasterFile\endcsname\relax
  \end{document}
  \fi

%-------------------------------------------------------------------------------
\expandafter\ifx\csname MasterFile\endcsname\relax
	\def\SubFile{hoge}
	\documentclass[a4j,12pt,twoside,openany]{jreport}
%\nofiles %tocファイルを更新させない
%\documentclass[12pt,a4j,twoside,openany]{jsbook}
\usepackage[dvipdfmx]{graphicx}
\usepackage{../dspc} % ベースラインスキップの指定
\usepackage{../slashbox} % 表に斜線を入れる
%\usepackage{../mediabb}
\usepackage{fancyvrb} % Verbatim環境
\usepackage{fancyhdr} % Headerの下線付き章見出し
\usepackage{here} % float[H]
\usepackage{multirow}
\usepackage{hhline} % 表の罫線の角を美しくする
\usepackage{amsmath} %コレがないとcasesが動かない
\usepackage{amsfonts} % 数学用フォント
\usepackage{bm} % 数式環境での bold
\usepackage{algorithm}
\usepackage{algorithmicx}
\usepackage[noend]{algpseudocode}
\usepackage[flushleft]{threeparttable} % 脚注付きテーブル
\usepackage{enumitem}
\usepackage{comment}
\usepackage{fancybox}
%\usepackage{csvsimple,booktabs,siunitx}
%\usepackage{filecontents}


\setlength{\evensidemargin}{5pt}
\setlength{\oddsidemargin}{40pt}
%\setlength{\headheight}{16.5pt}
%%\setlength{\headheight}{30pt}
\setcounter{secnumdepth}{3}
\setlist[description]{leftmargin=2\parindent,labelindent=\parindent}

\makeatletter
\def\@makechapterhead#1{%
	\vspace*{50\p@}%
	{
		\parindent \z@ \raggedright \normalfont
		\ifnum \c@secnumdepth >\m@ne
		% \if@mainmatter
			\huge\bfseries\@chapapp\thechapter\@chappos
			\par\nobreak
			\vskip 20\p@
		% \fi
		\fi
		\interlinepenalty\@M
		\Huge\bfseries #1\par\nobreak
		\vskip 40\p@
	}
}

%新しいコマンド定義
\newcounter{linenumber}
\newenvironment{listing}{%
  \begin{list}{%
    \small\arabic{linenumber}:}{%
      \usecounter{linenumber}%
      \setlength{\baselineskip}{18pt}%
      \setlength{\itemsep}{0pt}%
      \setlength{\parsep}{0pt}}}%
 {\end{list}}
\newcommand{\figcaption}[1]{\def\@captype{figure}\caption{#1}}
\newcommand{\tblcaption}[1]{\def\@captype{table}\caption{#1}}
\newcommand{\norm}[1]{\left\| #1 \right\|}
\newcommand{\cc}[1]{\multicolumn{1}{|c|}{#1}}
\newcommand{\circled}[1]{\raisebox{.5pt}{\textcircled{\raisebox{-.9pt} {#1}}}}
\newcommand{\specialcell}[2][c]{%
  \begin{tabular}[#1]{@{}c@{}}#2\end{tabular}}
\makeatother
%===============================================================================
\expandafter\ifx\csname SubFile\endcsname\relax
\begin{document}
\def\MasterFile{hoge}
%-------------------------------------------------------------------------------
%\maketitle
\thispagestyle{empty}
\input{../hyoushi/title}
%\addcontentsline{toc}{chapter}{表紙}
\thispagestyle{empty}
\mbox{}\newpage
%===============================================================================
%\frontmatter
%===============================================================================
%\mainmatter
%-------------------------------------------------------------------------------
\pagenumbering{arabic}
\cleardoublepage
\input{../0.Abstract/chapter}
%-------------------------------------------------------------------------------
\clearpage
\addcontentsline{toc}{chapter}{目次}
\tableofcontents

\clearpage
\addcontentsline{toc}{chapter}{図目次}
\listoffigures

\clearpage
\addcontentsline{toc}{chapter}{表目次}
\listoftables

%-------------------------------------------------------------------------------

%=====================
\pagestyle{fancy} % Headerをつける
\renewcommand{\sectionmark}[1]{\markright{\thesection\ \ \ #1}}
\renewcommand{\chaptermark}[1]{\markboth{#1}{}}
\lhead{}
\chead{}
\lfoot{}
\rfoot{}%-------------------------------------------------------------------------------
\input{../1.Introduction/chapter}
%-------------------------------------------------------------------------------
\input{../2.Related_Work/chapter}
%-------------------------------------------------------------------------------
\input{../3.The_Model/chapter}
%-------------------------------------------------------------------------------
\input{../4.Implementation/chapter}
%-------------------------------------------------------------------------------
\input{../5.Experiments/chapter}
%-------------------------------------------------------------------------------
\input{../6.Conclusion/chapter}

%===============================================================================
\pagestyle{plain}
%-------------------------------------------------------------------------------
\input{../7.Acknowledgement/chapter} %謝辞
%-------------------------------------------------------------------------------
\def\BibFile{../Bibliograhoy/database2}
\input{../Bibliography/chapter} %参考文献
% %===============================================================================
\appendix
\input{../A.Mypaper/chapter} % 投稿論文リスト
\input{../B.SIG-CCI2/chapter} %
\input{../C.IJCAI-16/chapter} %
%===============================================================================
\end{document}
\fi

	\begin{document}
	\setcounter{chapter}{0}
	\fi
  %-------------------------------------------------------------------------------
\cleardoublepage
\chapter{序論}
\label{intro:chapter}
%本章では, 本研究を行なうに至った背景と目的について述べる.その後,本論文の構成について述べる.
\section{研究の背景}
\label{intro:background}
近年,Web上での大規模な議論活動が活発になっているが,現在一般的に使われている "2ちゃんねる" や "Twitter" といったシステムでは整理や収束を行うことが困難である.困難である原因として,議論の管理を行う者がいないことが挙げられる.
つまり,議論を整理・収束させるには議論のマネジメントを行う人物が必要である.
%
大規模意見集約システムCOLLAGREE\cite{collagreeTest}ではファシリテーターと呼ばれる人物が議論のマネジメントを行っている.
しかし,ファシリテーターは人間であり,長時間に渡って大人数での議論の動向をマネジメントし続けるのは困難である.
COLLAGREEで大規模な議論を収束させるためには,ファシリテーターが必要な時には画面を見るようにして,他の時は見なくても済むようにすることで画面に向き合う時間を減らす工夫があることが望ましい.ファシリテーターが画面を見るべきタイミングは議論の話題が変化したときである.以前の議論の内容から外れた発言がされた時,ファシリテーターが適切な発言をすることで,脱線や炎上を避けて議論を収束させることができる.
すなわち,ファシリテーターの代わりに自動的に議論中の話題の変化を観測することが求められている.
%
現在,COLLAGREE上で使用されている議論支援システムは「(1)投稿支援システム」と「(2)議論可視化システム」の2つに大別できる.
投稿支援システムはポイント機能やファシリテーションフレーズ簡易投稿機能のように,ユーザーが投稿をする際に何らかの補助やリアクションを行う.現行の機能では選択肢の提示に留まっており,作業量を減らすことには繋がりにくい.
一方,議論可視化システムは議論ツリーやキーワード抽出のように,ユーザーにスレッドとは異なる議論の見方を提供する.
\ref{Fig:argTree1}に議論ツリーの例を示す.
\begin{figure}[htbp]
 \begin{center}
  \includegraphics[width=\textwidth]{../images/2.Related_Work/argTree1.png}
  \caption{議論ツリー}
  \label{Fig:argTree1}
  \vspace{-10pt}
 \end{center}
\end{figure}
現行の機能では議論を見やすくすることに重点が置かれており,議論の把握の助けにはなるが画面に向き合う時間を減らすことにはなりにくい.むしろ,作業量を増やすことになり得ることもある.
従って,現行の支援機能ではファシリテーターの作業量の減少には繋がりにくい.
%
近年,自然言語処理の分野において分散表現が多くの研究で使われており,機械翻訳を始めとする単語の意味が重要となる分野で精度の向上が確認されている.分散表現を用いることで,人間に近い精度で話題の変化を観測することが可能となる.
%
以上のような背景を踏まえて,分散表現を用いて,話題の変化を観測し,話題の変化が確認された時にファシリテーターに伝えることが望ましい.
話題の変化の観測は,発言中に現れる単語の類似度の計算と見なすことができる.
分散表現を用いることで単語間の類似度を求めることができる,値が大きいほど単語がそれぞれ類似した実数ベクトルであることを表す.単語Aと単語Bの実数ベクトルが類似しているとは,単語Aと共に使われることの多い単語と単語Bと共に使われることの多い単語が多く共通していることを示す.故に,分散表現を使って単語の類似度を計算することができる.
%
発言文から単語を選ぶ際には自動要約を用いる.発言文から重要でない単語を取り除くことで関連度の計算の精度を高めることが可能となる.
要約の手法としてはokapi BM25 \cite{okapiBM25}とLexRankを組み合わせた抽出的要約手法を用いる.
\begin{comment}
%======================================= 社会的背景
2013年頃からWeb上での大規模な議論活動が活発になり,大規模な人数での議論が期待されている.
大規模な議論では意見を共有することは可能であるが,議論を整理させることや収束させることは難しい.以上から大規模意見集約システムCOLLAGREEが開発された.本システムではWeb上で適切に大規模な議論を行うことができるように議論をマネジメントするファシリテーターを導入した\cite{collagreeTest}.
過去の実験ではファシリテーターの存在が議論の集約に大きな役割を果たしていることが認識されており,大規模な議論のためにファシリテータは必要である.しかし,議論の規模に伴って議論時間が長くなる傾向があり,同時にファシリテーターは常に議論の動向を見続ける必要がある.故に,議論の規模が大きくなればなるほどファシリテーターは長時間かつ大規模な議論の動向の監視によって大きな負担がかかる.大規模な議論が増加する傾向を踏まえるとファシリテーターにかかる負担を軽減する支援が必要である.\\
以上の問題を解決するため,話題の変化を追い,重要な話題の転換点をファシリテーターの代わりに検出することが有用であると考える.必要な時にだけファシリテーターが画面を見れば良いようにすることでファシリテーターの負担軽減が期待できる.
%========================================= 現行手法問題点背景
%議論支援に関する先行研究において,既存の手法は全てが文字列を文字列のまま扱う手法である.
%既存手法は殆どがパターンマッチングと重み付けの2つに区分することができる.
%パターンマッチングでは事前に単語を登録して,単語がマッチした場合に処理を行うが,処理それぞれに対して単語を登録しなければならず手間が膨大になってしまう.また,単語の意味が考慮されておらず,手作業で登録を行うので登録漏れがあった場合に単語の意味に関係なく処理を行うことが不可能となってしまう.
%重み付けは単語の出現頻度や文章の長さを使用して単語・文章に順位を付ける手法で必ずしも単語の登録が必要でないため多くの研究で使用されている.
%しかし,重み付けもまた単語を文字列のまま扱っており,意味までは考慮されていない.故に ,人間なら対応できる似た単語でも1文字違うだけで対処が困難となる.
議論支援に関する先行研究においてファシリテーターに対する支援を目的としたものは無く,殆どが議論の活性化や可視化を目的としている.
%=================================新手法
近年,自然言語処理の分野において分散表現が多くの研究で使われており.分散表現は文字列である単語を辞書データを使用して実数ベクトルへと変換する.辞書データにない単語には対応できないが,多様な処理を1つの辞書データで行うことができる.また,実数ベクトルの各数値が単語の意味を表現するものとなっており,数値を使用して処理を行うことができる.
分散表現を用いることで既存手法より人間の感覚に近しい処理を行うことができる.
%=================================
以上のような背景を踏まえて,分散表現を用いてファシリテーターの代わりに話題の変化を判定し,知らせることを目指す.
話題転換の検出は発言同士の近さ,すなわち発言に含まれる単語意味の近さと見ることができる.
分散表現ではベクトル同士の内積計算を行うことで単語同士の意味の近さを計算することができる.
また,分散表現を使用することで機械翻訳を始めとする複数の分野で精度の向上が確認されている.
\end{comment}
\section{研究の目的}
\label{intro:taget}
本論文では,分散表現を用いて議論中での発言に含まれる単語の関連度を計算し,話題の変化を観測する手法を提案する.

\section{本論文の構成}
本論文の構成を以下に示す.
\ref{relwork:chapter} 章では要約手法に関する研究と,分散表現に関する先行研究を紹介する.
次に,\ref{model:chapter}章では発言の要約手法の説明を行い,\ref{impl:chapter}章では分散表現を用いた単語集合間の関連度計算について説明する.
そして,\ref{exp:chapter}章では話題転換点の検出の評価実験について説明する.
最後に\ref{con:chapter}章で本論文のまとめと考察を示す.

 %-------------------------------------------------------------------------------
 \expandafter\ifx\csname MasterFile\endcsname\relax
	\def\BibFile{hoge}
	\expandafter\ifx\csname MasterFile\endcsname\relax
	\def\SubFile{hoge}
	\input{../thesis/thesis}
	\begin{document}
	\setcounter{chapter}{0}
	\fi
  %-------------------------------------------------------------------------------
\cleardoublepage
\chapter{序論}
\label{intro:chapter}
%本章では, 本研究を行なうに至った背景と目的について述べる.その後,本論文の構成について述べる.
\section{研究の背景}
\label{intro:background}
近年,Web上での大規模な議論活動が活発になっているが,現在一般的に使われている "2ちゃんねる" や "Twitter" といったシステムでは整理や収束を行うことが困難である.困難である原因として,議論の管理を行う者がいないことが挙げられる.
つまり,議論を整理・収束させるには議論のマネジメントを行う人物が必要である.
%
大規模意見集約システムCOLLAGREE\cite{collagreeTest}ではファシリテーターと呼ばれる人物が議論のマネジメントを行っている.
しかし,ファシリテーターは人間であり,長時間に渡って大人数での議論の動向をマネジメントし続けるのは困難である.
COLLAGREEで大規模な議論を収束させるためには,ファシリテーターが必要な時には画面を見るようにして,他の時は見なくても済むようにすることで画面に向き合う時間を減らす工夫があることが望ましい.ファシリテーターが画面を見るべきタイミングは議論の話題が変化したときである.以前の議論の内容から外れた発言がされた時,ファシリテーターが適切な発言をすることで,脱線や炎上を避けて議論を収束させることができる.
すなわち,ファシリテーターの代わりに自動的に議論中の話題の変化を観測することが求められている.
%
現在,COLLAGREE上で使用されている議論支援システムは「(1)投稿支援システム」と「(2)議論可視化システム」の2つに大別できる.
投稿支援システムはポイント機能やファシリテーションフレーズ簡易投稿機能のように,ユーザーが投稿をする際に何らかの補助やリアクションを行う.現行の機能では選択肢の提示に留まっており,作業量を減らすことには繋がりにくい.
一方,議論可視化システムは議論ツリーやキーワード抽出のように,ユーザーにスレッドとは異なる議論の見方を提供する.
\ref{Fig:argTree1}に議論ツリーの例を示す.
\begin{figure}[htbp]
 \begin{center}
  \includegraphics[width=\textwidth]{../images/2.Related_Work/argTree1.png}
  \caption{議論ツリー}
  \label{Fig:argTree1}
  \vspace{-10pt}
 \end{center}
\end{figure}
現行の機能では議論を見やすくすることに重点が置かれており,議論の把握の助けにはなるが画面に向き合う時間を減らすことにはなりにくい.むしろ,作業量を増やすことになり得ることもある.
従って,現行の支援機能ではファシリテーターの作業量の減少には繋がりにくい.
%
近年,自然言語処理の分野において分散表現が多くの研究で使われており,機械翻訳を始めとする単語の意味が重要となる分野で精度の向上が確認されている.分散表現を用いることで,人間に近い精度で話題の変化を観測することが可能となる.
%
以上のような背景を踏まえて,分散表現を用いて,話題の変化を観測し,話題の変化が確認された時にファシリテーターに伝えることが望ましい.
話題の変化の観測は,発言中に現れる単語の類似度の計算と見なすことができる.
分散表現を用いることで単語間の類似度を求めることができる,値が大きいほど単語がそれぞれ類似した実数ベクトルであることを表す.単語Aと単語Bの実数ベクトルが類似しているとは,単語Aと共に使われることの多い単語と単語Bと共に使われることの多い単語が多く共通していることを示す.故に,分散表現を使って単語の類似度を計算することができる.
%
発言文から単語を選ぶ際には自動要約を用いる.発言文から重要でない単語を取り除くことで関連度の計算の精度を高めることが可能となる.
要約の手法としてはokapi BM25 \cite{okapiBM25}とLexRankを組み合わせた抽出的要約手法を用いる.
\begin{comment}
%======================================= 社会的背景
2013年頃からWeb上での大規模な議論活動が活発になり,大規模な人数での議論が期待されている.
大規模な議論では意見を共有することは可能であるが,議論を整理させることや収束させることは難しい.以上から大規模意見集約システムCOLLAGREEが開発された.本システムではWeb上で適切に大規模な議論を行うことができるように議論をマネジメントするファシリテーターを導入した\cite{collagreeTest}.
過去の実験ではファシリテーターの存在が議論の集約に大きな役割を果たしていることが認識されており,大規模な議論のためにファシリテータは必要である.しかし,議論の規模に伴って議論時間が長くなる傾向があり,同時にファシリテーターは常に議論の動向を見続ける必要がある.故に,議論の規模が大きくなればなるほどファシリテーターは長時間かつ大規模な議論の動向の監視によって大きな負担がかかる.大規模な議論が増加する傾向を踏まえるとファシリテーターにかかる負担を軽減する支援が必要である.\\
以上の問題を解決するため,話題の変化を追い,重要な話題の転換点をファシリテーターの代わりに検出することが有用であると考える.必要な時にだけファシリテーターが画面を見れば良いようにすることでファシリテーターの負担軽減が期待できる.
%========================================= 現行手法問題点背景
%議論支援に関する先行研究において,既存の手法は全てが文字列を文字列のまま扱う手法である.
%既存手法は殆どがパターンマッチングと重み付けの2つに区分することができる.
%パターンマッチングでは事前に単語を登録して,単語がマッチした場合に処理を行うが,処理それぞれに対して単語を登録しなければならず手間が膨大になってしまう.また,単語の意味が考慮されておらず,手作業で登録を行うので登録漏れがあった場合に単語の意味に関係なく処理を行うことが不可能となってしまう.
%重み付けは単語の出現頻度や文章の長さを使用して単語・文章に順位を付ける手法で必ずしも単語の登録が必要でないため多くの研究で使用されている.
%しかし,重み付けもまた単語を文字列のまま扱っており,意味までは考慮されていない.故に ,人間なら対応できる似た単語でも1文字違うだけで対処が困難となる.
議論支援に関する先行研究においてファシリテーターに対する支援を目的としたものは無く,殆どが議論の活性化や可視化を目的としている.
%=================================新手法
近年,自然言語処理の分野において分散表現が多くの研究で使われており.分散表現は文字列である単語を辞書データを使用して実数ベクトルへと変換する.辞書データにない単語には対応できないが,多様な処理を1つの辞書データで行うことができる.また,実数ベクトルの各数値が単語の意味を表現するものとなっており,数値を使用して処理を行うことができる.
分散表現を用いることで既存手法より人間の感覚に近しい処理を行うことができる.
%=================================
以上のような背景を踏まえて,分散表現を用いてファシリテーターの代わりに話題の変化を判定し,知らせることを目指す.
話題転換の検出は発言同士の近さ,すなわち発言に含まれる単語意味の近さと見ることができる.
分散表現ではベクトル同士の内積計算を行うことで単語同士の意味の近さを計算することができる.
また,分散表現を使用することで機械翻訳を始めとする複数の分野で精度の向上が確認されている.
\end{comment}
\section{研究の目的}
\label{intro:taget}
本論文では,分散表現を用いて議論中での発言に含まれる単語の関連度を計算し,話題の変化を観測する手法を提案する.

\section{本論文の構成}
本論文の構成を以下に示す.
\ref{relwork:chapter} 章では要約手法に関する研究と,分散表現に関する先行研究を紹介する.
次に,\ref{model:chapter}章では発言の要約手法の説明を行い,\ref{impl:chapter}章では分散表現を用いた単語集合間の関連度計算について説明する.
そして,\ref{exp:chapter}章では話題転換点の検出の評価実験について説明する.
最後に\ref{con:chapter}章で本論文のまとめと考察を示す.

 %-------------------------------------------------------------------------------
 \expandafter\ifx\csname MasterFile\endcsname\relax
	\def\BibFile{hoge}
	\input{../Bibliography/chapter}
  \fi
  %-------------------------------------------------------------------------------
  \expandafter\ifx\csname MasterFile\endcsname\relax
  \end{document}
  \fi

  \fi
  %-------------------------------------------------------------------------------
  \expandafter\ifx\csname MasterFile\endcsname\relax
  \end{document}
  \fi

%-------------------------------------------------------------------------------
\expandafter\ifx\csname MasterFile\endcsname\relax
	\def\SubFile{hoge}
	\documentclass[a4j,12pt,twoside,openany]{jreport}
%\nofiles %tocファイルを更新させない
%\documentclass[12pt,a4j,twoside,openany]{jsbook}
\usepackage[dvipdfmx]{graphicx}
\usepackage{../dspc} % ベースラインスキップの指定
\usepackage{../slashbox} % 表に斜線を入れる
%\usepackage{../mediabb}
\usepackage{fancyvrb} % Verbatim環境
\usepackage{fancyhdr} % Headerの下線付き章見出し
\usepackage{here} % float[H]
\usepackage{multirow}
\usepackage{hhline} % 表の罫線の角を美しくする
\usepackage{amsmath} %コレがないとcasesが動かない
\usepackage{amsfonts} % 数学用フォント
\usepackage{bm} % 数式環境での bold
\usepackage{algorithm}
\usepackage{algorithmicx}
\usepackage[noend]{algpseudocode}
\usepackage[flushleft]{threeparttable} % 脚注付きテーブル
\usepackage{enumitem}
\usepackage{comment}
\usepackage{fancybox}
%\usepackage{csvsimple,booktabs,siunitx}
%\usepackage{filecontents}


\setlength{\evensidemargin}{5pt}
\setlength{\oddsidemargin}{40pt}
%\setlength{\headheight}{16.5pt}
%%\setlength{\headheight}{30pt}
\setcounter{secnumdepth}{3}
\setlist[description]{leftmargin=2\parindent,labelindent=\parindent}

\makeatletter
\def\@makechapterhead#1{%
	\vspace*{50\p@}%
	{
		\parindent \z@ \raggedright \normalfont
		\ifnum \c@secnumdepth >\m@ne
		% \if@mainmatter
			\huge\bfseries\@chapapp\thechapter\@chappos
			\par\nobreak
			\vskip 20\p@
		% \fi
		\fi
		\interlinepenalty\@M
		\Huge\bfseries #1\par\nobreak
		\vskip 40\p@
	}
}

%新しいコマンド定義
\newcounter{linenumber}
\newenvironment{listing}{%
  \begin{list}{%
    \small\arabic{linenumber}:}{%
      \usecounter{linenumber}%
      \setlength{\baselineskip}{18pt}%
      \setlength{\itemsep}{0pt}%
      \setlength{\parsep}{0pt}}}%
 {\end{list}}
\newcommand{\figcaption}[1]{\def\@captype{figure}\caption{#1}}
\newcommand{\tblcaption}[1]{\def\@captype{table}\caption{#1}}
\newcommand{\norm}[1]{\left\| #1 \right\|}
\newcommand{\cc}[1]{\multicolumn{1}{|c|}{#1}}
\newcommand{\circled}[1]{\raisebox{.5pt}{\textcircled{\raisebox{-.9pt} {#1}}}}
\newcommand{\specialcell}[2][c]{%
  \begin{tabular}[#1]{@{}c@{}}#2\end{tabular}}
\makeatother
%===============================================================================
\expandafter\ifx\csname SubFile\endcsname\relax
\begin{document}
\def\MasterFile{hoge}
%-------------------------------------------------------------------------------
%\maketitle
\thispagestyle{empty}
\input{../hyoushi/title}
%\addcontentsline{toc}{chapter}{表紙}
\thispagestyle{empty}
\mbox{}\newpage
%===============================================================================
%\frontmatter
%===============================================================================
%\mainmatter
%-------------------------------------------------------------------------------
\pagenumbering{arabic}
\cleardoublepage
\input{../0.Abstract/chapter}
%-------------------------------------------------------------------------------
\clearpage
\addcontentsline{toc}{chapter}{目次}
\tableofcontents

\clearpage
\addcontentsline{toc}{chapter}{図目次}
\listoffigures

\clearpage
\addcontentsline{toc}{chapter}{表目次}
\listoftables

%-------------------------------------------------------------------------------

%=====================
\pagestyle{fancy} % Headerをつける
\renewcommand{\sectionmark}[1]{\markright{\thesection\ \ \ #1}}
\renewcommand{\chaptermark}[1]{\markboth{#1}{}}
\lhead{}
\chead{}
\lfoot{}
\rfoot{}%-------------------------------------------------------------------------------
\input{../1.Introduction/chapter}
%-------------------------------------------------------------------------------
\input{../2.Related_Work/chapter}
%-------------------------------------------------------------------------------
\input{../3.The_Model/chapter}
%-------------------------------------------------------------------------------
\input{../4.Implementation/chapter}
%-------------------------------------------------------------------------------
\input{../5.Experiments/chapter}
%-------------------------------------------------------------------------------
\input{../6.Conclusion/chapter}

%===============================================================================
\pagestyle{plain}
%-------------------------------------------------------------------------------
\input{../7.Acknowledgement/chapter} %謝辞
%-------------------------------------------------------------------------------
\def\BibFile{../Bibliograhoy/database2}
\input{../Bibliography/chapter} %参考文献
% %===============================================================================
\appendix
\input{../A.Mypaper/chapter} % 投稿論文リスト
\input{../B.SIG-CCI2/chapter} %
\input{../C.IJCAI-16/chapter} %
%===============================================================================
\end{document}
\fi

	\begin{document}
	\setcounter{chapter}{0}
	\fi
  %-------------------------------------------------------------------------------
\cleardoublepage
\chapter{序論}
\label{intro:chapter}
%本章では, 本研究を行なうに至った背景と目的について述べる.その後,本論文の構成について述べる.
\section{研究の背景}
\label{intro:background}
近年,Web上での大規模な議論活動が活発になっているが,現在一般的に使われている "2ちゃんねる" や "Twitter" といったシステムでは整理や収束を行うことが困難である.困難である原因として,議論の管理を行う者がいないことが挙げられる.
つまり,議論を整理・収束させるには議論のマネジメントを行う人物が必要である.
%
大規模意見集約システムCOLLAGREE\cite{collagreeTest}ではファシリテーターと呼ばれる人物が議論のマネジメントを行っている.
しかし,ファシリテーターは人間であり,長時間に渡って大人数での議論の動向をマネジメントし続けるのは困難である.
COLLAGREEで大規模な議論を収束させるためには,ファシリテーターが必要な時には画面を見るようにして,他の時は見なくても済むようにすることで画面に向き合う時間を減らす工夫があることが望ましい.ファシリテーターが画面を見るべきタイミングは議論の話題が変化したときである.以前の議論の内容から外れた発言がされた時,ファシリテーターが適切な発言をすることで,脱線や炎上を避けて議論を収束させることができる.
すなわち,ファシリテーターの代わりに自動的に議論中の話題の変化を観測することが求められている.
%
現在,COLLAGREE上で使用されている議論支援システムは「(1)投稿支援システム」と「(2)議論可視化システム」の2つに大別できる.
投稿支援システムはポイント機能やファシリテーションフレーズ簡易投稿機能のように,ユーザーが投稿をする際に何らかの補助やリアクションを行う.現行の機能では選択肢の提示に留まっており,作業量を減らすことには繋がりにくい.
一方,議論可視化システムは議論ツリーやキーワード抽出のように,ユーザーにスレッドとは異なる議論の見方を提供する.
\ref{Fig:argTree1}に議論ツリーの例を示す.
\begin{figure}[htbp]
 \begin{center}
  \includegraphics[width=\textwidth]{../images/2.Related_Work/argTree1.png}
  \caption{議論ツリー}
  \label{Fig:argTree1}
  \vspace{-10pt}
 \end{center}
\end{figure}
現行の機能では議論を見やすくすることに重点が置かれており,議論の把握の助けにはなるが画面に向き合う時間を減らすことにはなりにくい.むしろ,作業量を増やすことになり得ることもある.
従って,現行の支援機能ではファシリテーターの作業量の減少には繋がりにくい.
%
近年,自然言語処理の分野において分散表現が多くの研究で使われており,機械翻訳を始めとする単語の意味が重要となる分野で精度の向上が確認されている.分散表現を用いることで,人間に近い精度で話題の変化を観測することが可能となる.
%
以上のような背景を踏まえて,分散表現を用いて,話題の変化を観測し,話題の変化が確認された時にファシリテーターに伝えることが望ましい.
話題の変化の観測は,発言中に現れる単語の類似度の計算と見なすことができる.
分散表現を用いることで単語間の類似度を求めることができる,値が大きいほど単語がそれぞれ類似した実数ベクトルであることを表す.単語Aと単語Bの実数ベクトルが類似しているとは,単語Aと共に使われることの多い単語と単語Bと共に使われることの多い単語が多く共通していることを示す.故に,分散表現を使って単語の類似度を計算することができる.
%
発言文から単語を選ぶ際には自動要約を用いる.発言文から重要でない単語を取り除くことで関連度の計算の精度を高めることが可能となる.
要約の手法としてはokapi BM25 \cite{okapiBM25}とLexRankを組み合わせた抽出的要約手法を用いる.
\begin{comment}
%======================================= 社会的背景
2013年頃からWeb上での大規模な議論活動が活発になり,大規模な人数での議論が期待されている.
大規模な議論では意見を共有することは可能であるが,議論を整理させることや収束させることは難しい.以上から大規模意見集約システムCOLLAGREEが開発された.本システムではWeb上で適切に大規模な議論を行うことができるように議論をマネジメントするファシリテーターを導入した\cite{collagreeTest}.
過去の実験ではファシリテーターの存在が議論の集約に大きな役割を果たしていることが認識されており,大規模な議論のためにファシリテータは必要である.しかし,議論の規模に伴って議論時間が長くなる傾向があり,同時にファシリテーターは常に議論の動向を見続ける必要がある.故に,議論の規模が大きくなればなるほどファシリテーターは長時間かつ大規模な議論の動向の監視によって大きな負担がかかる.大規模な議論が増加する傾向を踏まえるとファシリテーターにかかる負担を軽減する支援が必要である.\\
以上の問題を解決するため,話題の変化を追い,重要な話題の転換点をファシリテーターの代わりに検出することが有用であると考える.必要な時にだけファシリテーターが画面を見れば良いようにすることでファシリテーターの負担軽減が期待できる.
%========================================= 現行手法問題点背景
%議論支援に関する先行研究において,既存の手法は全てが文字列を文字列のまま扱う手法である.
%既存手法は殆どがパターンマッチングと重み付けの2つに区分することができる.
%パターンマッチングでは事前に単語を登録して,単語がマッチした場合に処理を行うが,処理それぞれに対して単語を登録しなければならず手間が膨大になってしまう.また,単語の意味が考慮されておらず,手作業で登録を行うので登録漏れがあった場合に単語の意味に関係なく処理を行うことが不可能となってしまう.
%重み付けは単語の出現頻度や文章の長さを使用して単語・文章に順位を付ける手法で必ずしも単語の登録が必要でないため多くの研究で使用されている.
%しかし,重み付けもまた単語を文字列のまま扱っており,意味までは考慮されていない.故に ,人間なら対応できる似た単語でも1文字違うだけで対処が困難となる.
議論支援に関する先行研究においてファシリテーターに対する支援を目的としたものは無く,殆どが議論の活性化や可視化を目的としている.
%=================================新手法
近年,自然言語処理の分野において分散表現が多くの研究で使われており.分散表現は文字列である単語を辞書データを使用して実数ベクトルへと変換する.辞書データにない単語には対応できないが,多様な処理を1つの辞書データで行うことができる.また,実数ベクトルの各数値が単語の意味を表現するものとなっており,数値を使用して処理を行うことができる.
分散表現を用いることで既存手法より人間の感覚に近しい処理を行うことができる.
%=================================
以上のような背景を踏まえて,分散表現を用いてファシリテーターの代わりに話題の変化を判定し,知らせることを目指す.
話題転換の検出は発言同士の近さ,すなわち発言に含まれる単語意味の近さと見ることができる.
分散表現ではベクトル同士の内積計算を行うことで単語同士の意味の近さを計算することができる.
また,分散表現を使用することで機械翻訳を始めとする複数の分野で精度の向上が確認されている.
\end{comment}
\section{研究の目的}
\label{intro:taget}
本論文では,分散表現を用いて議論中での発言に含まれる単語の関連度を計算し,話題の変化を観測する手法を提案する.

\section{本論文の構成}
本論文の構成を以下に示す.
\ref{relwork:chapter} 章では要約手法に関する研究と,分散表現に関する先行研究を紹介する.
次に,\ref{model:chapter}章では発言の要約手法の説明を行い,\ref{impl:chapter}章では分散表現を用いた単語集合間の関連度計算について説明する.
そして,\ref{exp:chapter}章では話題転換点の検出の評価実験について説明する.
最後に\ref{con:chapter}章で本論文のまとめと考察を示す.

 %-------------------------------------------------------------------------------
 \expandafter\ifx\csname MasterFile\endcsname\relax
	\def\BibFile{hoge}
	\expandafter\ifx\csname MasterFile\endcsname\relax
	\def\SubFile{hoge}
	\input{../thesis/thesis}
	\begin{document}
	\setcounter{chapter}{0}
	\fi
  %-------------------------------------------------------------------------------
\cleardoublepage
\chapter{序論}
\label{intro:chapter}
%本章では, 本研究を行なうに至った背景と目的について述べる.その後,本論文の構成について述べる.
\section{研究の背景}
\label{intro:background}
近年,Web上での大規模な議論活動が活発になっているが,現在一般的に使われている "2ちゃんねる" や "Twitter" といったシステムでは整理や収束を行うことが困難である.困難である原因として,議論の管理を行う者がいないことが挙げられる.
つまり,議論を整理・収束させるには議論のマネジメントを行う人物が必要である.
%
大規模意見集約システムCOLLAGREE\cite{collagreeTest}ではファシリテーターと呼ばれる人物が議論のマネジメントを行っている.
しかし,ファシリテーターは人間であり,長時間に渡って大人数での議論の動向をマネジメントし続けるのは困難である.
COLLAGREEで大規模な議論を収束させるためには,ファシリテーターが必要な時には画面を見るようにして,他の時は見なくても済むようにすることで画面に向き合う時間を減らす工夫があることが望ましい.ファシリテーターが画面を見るべきタイミングは議論の話題が変化したときである.以前の議論の内容から外れた発言がされた時,ファシリテーターが適切な発言をすることで,脱線や炎上を避けて議論を収束させることができる.
すなわち,ファシリテーターの代わりに自動的に議論中の話題の変化を観測することが求められている.
%
現在,COLLAGREE上で使用されている議論支援システムは「(1)投稿支援システム」と「(2)議論可視化システム」の2つに大別できる.
投稿支援システムはポイント機能やファシリテーションフレーズ簡易投稿機能のように,ユーザーが投稿をする際に何らかの補助やリアクションを行う.現行の機能では選択肢の提示に留まっており,作業量を減らすことには繋がりにくい.
一方,議論可視化システムは議論ツリーやキーワード抽出のように,ユーザーにスレッドとは異なる議論の見方を提供する.
\ref{Fig:argTree1}に議論ツリーの例を示す.
\begin{figure}[htbp]
 \begin{center}
  \includegraphics[width=\textwidth]{../images/2.Related_Work/argTree1.png}
  \caption{議論ツリー}
  \label{Fig:argTree1}
  \vspace{-10pt}
 \end{center}
\end{figure}
現行の機能では議論を見やすくすることに重点が置かれており,議論の把握の助けにはなるが画面に向き合う時間を減らすことにはなりにくい.むしろ,作業量を増やすことになり得ることもある.
従って,現行の支援機能ではファシリテーターの作業量の減少には繋がりにくい.
%
近年,自然言語処理の分野において分散表現が多くの研究で使われており,機械翻訳を始めとする単語の意味が重要となる分野で精度の向上が確認されている.分散表現を用いることで,人間に近い精度で話題の変化を観測することが可能となる.
%
以上のような背景を踏まえて,分散表現を用いて,話題の変化を観測し,話題の変化が確認された時にファシリテーターに伝えることが望ましい.
話題の変化の観測は,発言中に現れる単語の類似度の計算と見なすことができる.
分散表現を用いることで単語間の類似度を求めることができる,値が大きいほど単語がそれぞれ類似した実数ベクトルであることを表す.単語Aと単語Bの実数ベクトルが類似しているとは,単語Aと共に使われることの多い単語と単語Bと共に使われることの多い単語が多く共通していることを示す.故に,分散表現を使って単語の類似度を計算することができる.
%
発言文から単語を選ぶ際には自動要約を用いる.発言文から重要でない単語を取り除くことで関連度の計算の精度を高めることが可能となる.
要約の手法としてはokapi BM25 \cite{okapiBM25}とLexRankを組み合わせた抽出的要約手法を用いる.
\begin{comment}
%======================================= 社会的背景
2013年頃からWeb上での大規模な議論活動が活発になり,大規模な人数での議論が期待されている.
大規模な議論では意見を共有することは可能であるが,議論を整理させることや収束させることは難しい.以上から大規模意見集約システムCOLLAGREEが開発された.本システムではWeb上で適切に大規模な議論を行うことができるように議論をマネジメントするファシリテーターを導入した\cite{collagreeTest}.
過去の実験ではファシリテーターの存在が議論の集約に大きな役割を果たしていることが認識されており,大規模な議論のためにファシリテータは必要である.しかし,議論の規模に伴って議論時間が長くなる傾向があり,同時にファシリテーターは常に議論の動向を見続ける必要がある.故に,議論の規模が大きくなればなるほどファシリテーターは長時間かつ大規模な議論の動向の監視によって大きな負担がかかる.大規模な議論が増加する傾向を踏まえるとファシリテーターにかかる負担を軽減する支援が必要である.\\
以上の問題を解決するため,話題の変化を追い,重要な話題の転換点をファシリテーターの代わりに検出することが有用であると考える.必要な時にだけファシリテーターが画面を見れば良いようにすることでファシリテーターの負担軽減が期待できる.
%========================================= 現行手法問題点背景
%議論支援に関する先行研究において,既存の手法は全てが文字列を文字列のまま扱う手法である.
%既存手法は殆どがパターンマッチングと重み付けの2つに区分することができる.
%パターンマッチングでは事前に単語を登録して,単語がマッチした場合に処理を行うが,処理それぞれに対して単語を登録しなければならず手間が膨大になってしまう.また,単語の意味が考慮されておらず,手作業で登録を行うので登録漏れがあった場合に単語の意味に関係なく処理を行うことが不可能となってしまう.
%重み付けは単語の出現頻度や文章の長さを使用して単語・文章に順位を付ける手法で必ずしも単語の登録が必要でないため多くの研究で使用されている.
%しかし,重み付けもまた単語を文字列のまま扱っており,意味までは考慮されていない.故に ,人間なら対応できる似た単語でも1文字違うだけで対処が困難となる.
議論支援に関する先行研究においてファシリテーターに対する支援を目的としたものは無く,殆どが議論の活性化や可視化を目的としている.
%=================================新手法
近年,自然言語処理の分野において分散表現が多くの研究で使われており.分散表現は文字列である単語を辞書データを使用して実数ベクトルへと変換する.辞書データにない単語には対応できないが,多様な処理を1つの辞書データで行うことができる.また,実数ベクトルの各数値が単語の意味を表現するものとなっており,数値を使用して処理を行うことができる.
分散表現を用いることで既存手法より人間の感覚に近しい処理を行うことができる.
%=================================
以上のような背景を踏まえて,分散表現を用いてファシリテーターの代わりに話題の変化を判定し,知らせることを目指す.
話題転換の検出は発言同士の近さ,すなわち発言に含まれる単語意味の近さと見ることができる.
分散表現ではベクトル同士の内積計算を行うことで単語同士の意味の近さを計算することができる.
また,分散表現を使用することで機械翻訳を始めとする複数の分野で精度の向上が確認されている.
\end{comment}
\section{研究の目的}
\label{intro:taget}
本論文では,分散表現を用いて議論中での発言に含まれる単語の関連度を計算し,話題の変化を観測する手法を提案する.

\section{本論文の構成}
本論文の構成を以下に示す.
\ref{relwork:chapter} 章では要約手法に関する研究と,分散表現に関する先行研究を紹介する.
次に,\ref{model:chapter}章では発言の要約手法の説明を行い,\ref{impl:chapter}章では分散表現を用いた単語集合間の関連度計算について説明する.
そして,\ref{exp:chapter}章では話題転換点の検出の評価実験について説明する.
最後に\ref{con:chapter}章で本論文のまとめと考察を示す.

 %-------------------------------------------------------------------------------
 \expandafter\ifx\csname MasterFile\endcsname\relax
	\def\BibFile{hoge}
	\input{../Bibliography/chapter}
  \fi
  %-------------------------------------------------------------------------------
  \expandafter\ifx\csname MasterFile\endcsname\relax
  \end{document}
  \fi

  \fi
  %-------------------------------------------------------------------------------
  \expandafter\ifx\csname MasterFile\endcsname\relax
  \end{document}
  \fi

%-------------------------------------------------------------------------------
\expandafter\ifx\csname MasterFile\endcsname\relax
	\def\SubFile{hoge}
	\documentclass[a4j,12pt,twoside,openany]{jreport}
%\nofiles %tocファイルを更新させない
%\documentclass[12pt,a4j,twoside,openany]{jsbook}
\usepackage[dvipdfmx]{graphicx}
\usepackage{../dspc} % ベースラインスキップの指定
\usepackage{../slashbox} % 表に斜線を入れる
%\usepackage{../mediabb}
\usepackage{fancyvrb} % Verbatim環境
\usepackage{fancyhdr} % Headerの下線付き章見出し
\usepackage{here} % float[H]
\usepackage{multirow}
\usepackage{hhline} % 表の罫線の角を美しくする
\usepackage{amsmath} %コレがないとcasesが動かない
\usepackage{amsfonts} % 数学用フォント
\usepackage{bm} % 数式環境での bold
\usepackage{algorithm}
\usepackage{algorithmicx}
\usepackage[noend]{algpseudocode}
\usepackage[flushleft]{threeparttable} % 脚注付きテーブル
\usepackage{enumitem}
\usepackage{comment}
\usepackage{fancybox}
%\usepackage{csvsimple,booktabs,siunitx}
%\usepackage{filecontents}


\setlength{\evensidemargin}{5pt}
\setlength{\oddsidemargin}{40pt}
%\setlength{\headheight}{16.5pt}
%%\setlength{\headheight}{30pt}
\setcounter{secnumdepth}{3}
\setlist[description]{leftmargin=2\parindent,labelindent=\parindent}

\makeatletter
\def\@makechapterhead#1{%
	\vspace*{50\p@}%
	{
		\parindent \z@ \raggedright \normalfont
		\ifnum \c@secnumdepth >\m@ne
		% \if@mainmatter
			\huge\bfseries\@chapapp\thechapter\@chappos
			\par\nobreak
			\vskip 20\p@
		% \fi
		\fi
		\interlinepenalty\@M
		\Huge\bfseries #1\par\nobreak
		\vskip 40\p@
	}
}

%新しいコマンド定義
\newcounter{linenumber}
\newenvironment{listing}{%
  \begin{list}{%
    \small\arabic{linenumber}:}{%
      \usecounter{linenumber}%
      \setlength{\baselineskip}{18pt}%
      \setlength{\itemsep}{0pt}%
      \setlength{\parsep}{0pt}}}%
 {\end{list}}
\newcommand{\figcaption}[1]{\def\@captype{figure}\caption{#1}}
\newcommand{\tblcaption}[1]{\def\@captype{table}\caption{#1}}
\newcommand{\norm}[1]{\left\| #1 \right\|}
\newcommand{\cc}[1]{\multicolumn{1}{|c|}{#1}}
\newcommand{\circled}[1]{\raisebox{.5pt}{\textcircled{\raisebox{-.9pt} {#1}}}}
\newcommand{\specialcell}[2][c]{%
  \begin{tabular}[#1]{@{}c@{}}#2\end{tabular}}
\makeatother
%===============================================================================
\expandafter\ifx\csname SubFile\endcsname\relax
\begin{document}
\def\MasterFile{hoge}
%-------------------------------------------------------------------------------
%\maketitle
\thispagestyle{empty}
\input{../hyoushi/title}
%\addcontentsline{toc}{chapter}{表紙}
\thispagestyle{empty}
\mbox{}\newpage
%===============================================================================
%\frontmatter
%===============================================================================
%\mainmatter
%-------------------------------------------------------------------------------
\pagenumbering{arabic}
\cleardoublepage
\input{../0.Abstract/chapter}
%-------------------------------------------------------------------------------
\clearpage
\addcontentsline{toc}{chapter}{目次}
\tableofcontents

\clearpage
\addcontentsline{toc}{chapter}{図目次}
\listoffigures

\clearpage
\addcontentsline{toc}{chapter}{表目次}
\listoftables

%-------------------------------------------------------------------------------

%=====================
\pagestyle{fancy} % Headerをつける
\renewcommand{\sectionmark}[1]{\markright{\thesection\ \ \ #1}}
\renewcommand{\chaptermark}[1]{\markboth{#1}{}}
\lhead{}
\chead{}
\lfoot{}
\rfoot{}%-------------------------------------------------------------------------------
\input{../1.Introduction/chapter}
%-------------------------------------------------------------------------------
\input{../2.Related_Work/chapter}
%-------------------------------------------------------------------------------
\input{../3.The_Model/chapter}
%-------------------------------------------------------------------------------
\input{../4.Implementation/chapter}
%-------------------------------------------------------------------------------
\input{../5.Experiments/chapter}
%-------------------------------------------------------------------------------
\input{../6.Conclusion/chapter}

%===============================================================================
\pagestyle{plain}
%-------------------------------------------------------------------------------
\input{../7.Acknowledgement/chapter} %謝辞
%-------------------------------------------------------------------------------
\def\BibFile{../Bibliograhoy/database2}
\input{../Bibliography/chapter} %参考文献
% %===============================================================================
\appendix
\input{../A.Mypaper/chapter} % 投稿論文リスト
\input{../B.SIG-CCI2/chapter} %
\input{../C.IJCAI-16/chapter} %
%===============================================================================
\end{document}
\fi

	\begin{document}
	\setcounter{chapter}{0}
	\fi
  %-------------------------------------------------------------------------------
\cleardoublepage
\chapter{序論}
\label{intro:chapter}
%本章では, 本研究を行なうに至った背景と目的について述べる.その後,本論文の構成について述べる.
\section{研究の背景}
\label{intro:background}
近年,Web上での大規模な議論活動が活発になっているが,現在一般的に使われている "2ちゃんねる" や "Twitter" といったシステムでは整理や収束を行うことが困難である.困難である原因として,議論の管理を行う者がいないことが挙げられる.
つまり,議論を整理・収束させるには議論のマネジメントを行う人物が必要である.
%
大規模意見集約システムCOLLAGREE\cite{collagreeTest}ではファシリテーターと呼ばれる人物が議論のマネジメントを行っている.
しかし,ファシリテーターは人間であり,長時間に渡って大人数での議論の動向をマネジメントし続けるのは困難である.
COLLAGREEで大規模な議論を収束させるためには,ファシリテーターが必要な時には画面を見るようにして,他の時は見なくても済むようにすることで画面に向き合う時間を減らす工夫があることが望ましい.ファシリテーターが画面を見るべきタイミングは議論の話題が変化したときである.以前の議論の内容から外れた発言がされた時,ファシリテーターが適切な発言をすることで,脱線や炎上を避けて議論を収束させることができる.
すなわち,ファシリテーターの代わりに自動的に議論中の話題の変化を観測することが求められている.
%
現在,COLLAGREE上で使用されている議論支援システムは「(1)投稿支援システム」と「(2)議論可視化システム」の2つに大別できる.
投稿支援システムはポイント機能やファシリテーションフレーズ簡易投稿機能のように,ユーザーが投稿をする際に何らかの補助やリアクションを行う.現行の機能では選択肢の提示に留まっており,作業量を減らすことには繋がりにくい.
一方,議論可視化システムは議論ツリーやキーワード抽出のように,ユーザーにスレッドとは異なる議論の見方を提供する.
\ref{Fig:argTree1}に議論ツリーの例を示す.
\begin{figure}[htbp]
 \begin{center}
  \includegraphics[width=\textwidth]{../images/2.Related_Work/argTree1.png}
  \caption{議論ツリー}
  \label{Fig:argTree1}
  \vspace{-10pt}
 \end{center}
\end{figure}
現行の機能では議論を見やすくすることに重点が置かれており,議論の把握の助けにはなるが画面に向き合う時間を減らすことにはなりにくい.むしろ,作業量を増やすことになり得ることもある.
従って,現行の支援機能ではファシリテーターの作業量の減少には繋がりにくい.
%
近年,自然言語処理の分野において分散表現が多くの研究で使われており,機械翻訳を始めとする単語の意味が重要となる分野で精度の向上が確認されている.分散表現を用いることで,人間に近い精度で話題の変化を観測することが可能となる.
%
以上のような背景を踏まえて,分散表現を用いて,話題の変化を観測し,話題の変化が確認された時にファシリテーターに伝えることが望ましい.
話題の変化の観測は,発言中に現れる単語の類似度の計算と見なすことができる.
分散表現を用いることで単語間の類似度を求めることができる,値が大きいほど単語がそれぞれ類似した実数ベクトルであることを表す.単語Aと単語Bの実数ベクトルが類似しているとは,単語Aと共に使われることの多い単語と単語Bと共に使われることの多い単語が多く共通していることを示す.故に,分散表現を使って単語の類似度を計算することができる.
%
発言文から単語を選ぶ際には自動要約を用いる.発言文から重要でない単語を取り除くことで関連度の計算の精度を高めることが可能となる.
要約の手法としてはokapi BM25 \cite{okapiBM25}とLexRankを組み合わせた抽出的要約手法を用いる.
\begin{comment}
%======================================= 社会的背景
2013年頃からWeb上での大規模な議論活動が活発になり,大規模な人数での議論が期待されている.
大規模な議論では意見を共有することは可能であるが,議論を整理させることや収束させることは難しい.以上から大規模意見集約システムCOLLAGREEが開発された.本システムではWeb上で適切に大規模な議論を行うことができるように議論をマネジメントするファシリテーターを導入した\cite{collagreeTest}.
過去の実験ではファシリテーターの存在が議論の集約に大きな役割を果たしていることが認識されており,大規模な議論のためにファシリテータは必要である.しかし,議論の規模に伴って議論時間が長くなる傾向があり,同時にファシリテーターは常に議論の動向を見続ける必要がある.故に,議論の規模が大きくなればなるほどファシリテーターは長時間かつ大規模な議論の動向の監視によって大きな負担がかかる.大規模な議論が増加する傾向を踏まえるとファシリテーターにかかる負担を軽減する支援が必要である.\\
以上の問題を解決するため,話題の変化を追い,重要な話題の転換点をファシリテーターの代わりに検出することが有用であると考える.必要な時にだけファシリテーターが画面を見れば良いようにすることでファシリテーターの負担軽減が期待できる.
%========================================= 現行手法問題点背景
%議論支援に関する先行研究において,既存の手法は全てが文字列を文字列のまま扱う手法である.
%既存手法は殆どがパターンマッチングと重み付けの2つに区分することができる.
%パターンマッチングでは事前に単語を登録して,単語がマッチした場合に処理を行うが,処理それぞれに対して単語を登録しなければならず手間が膨大になってしまう.また,単語の意味が考慮されておらず,手作業で登録を行うので登録漏れがあった場合に単語の意味に関係なく処理を行うことが不可能となってしまう.
%重み付けは単語の出現頻度や文章の長さを使用して単語・文章に順位を付ける手法で必ずしも単語の登録が必要でないため多くの研究で使用されている.
%しかし,重み付けもまた単語を文字列のまま扱っており,意味までは考慮されていない.故に ,人間なら対応できる似た単語でも1文字違うだけで対処が困難となる.
議論支援に関する先行研究においてファシリテーターに対する支援を目的としたものは無く,殆どが議論の活性化や可視化を目的としている.
%=================================新手法
近年,自然言語処理の分野において分散表現が多くの研究で使われており.分散表現は文字列である単語を辞書データを使用して実数ベクトルへと変換する.辞書データにない単語には対応できないが,多様な処理を1つの辞書データで行うことができる.また,実数ベクトルの各数値が単語の意味を表現するものとなっており,数値を使用して処理を行うことができる.
分散表現を用いることで既存手法より人間の感覚に近しい処理を行うことができる.
%=================================
以上のような背景を踏まえて,分散表現を用いてファシリテーターの代わりに話題の変化を判定し,知らせることを目指す.
話題転換の検出は発言同士の近さ,すなわち発言に含まれる単語意味の近さと見ることができる.
分散表現ではベクトル同士の内積計算を行うことで単語同士の意味の近さを計算することができる.
また,分散表現を使用することで機械翻訳を始めとする複数の分野で精度の向上が確認されている.
\end{comment}
\section{研究の目的}
\label{intro:taget}
本論文では,分散表現を用いて議論中での発言に含まれる単語の関連度を計算し,話題の変化を観測する手法を提案する.

\section{本論文の構成}
本論文の構成を以下に示す.
\ref{relwork:chapter} 章では要約手法に関する研究と,分散表現に関する先行研究を紹介する.
次に,\ref{model:chapter}章では発言の要約手法の説明を行い,\ref{impl:chapter}章では分散表現を用いた単語集合間の関連度計算について説明する.
そして,\ref{exp:chapter}章では話題転換点の検出の評価実験について説明する.
最後に\ref{con:chapter}章で本論文のまとめと考察を示す.

 %-------------------------------------------------------------------------------
 \expandafter\ifx\csname MasterFile\endcsname\relax
	\def\BibFile{hoge}
	\expandafter\ifx\csname MasterFile\endcsname\relax
	\def\SubFile{hoge}
	\input{../thesis/thesis}
	\begin{document}
	\setcounter{chapter}{0}
	\fi
  %-------------------------------------------------------------------------------
\cleardoublepage
\chapter{序論}
\label{intro:chapter}
%本章では, 本研究を行なうに至った背景と目的について述べる.その後,本論文の構成について述べる.
\section{研究の背景}
\label{intro:background}
近年,Web上での大規模な議論活動が活発になっているが,現在一般的に使われている "2ちゃんねる" や "Twitter" といったシステムでは整理や収束を行うことが困難である.困難である原因として,議論の管理を行う者がいないことが挙げられる.
つまり,議論を整理・収束させるには議論のマネジメントを行う人物が必要である.
%
大規模意見集約システムCOLLAGREE\cite{collagreeTest}ではファシリテーターと呼ばれる人物が議論のマネジメントを行っている.
しかし,ファシリテーターは人間であり,長時間に渡って大人数での議論の動向をマネジメントし続けるのは困難である.
COLLAGREEで大規模な議論を収束させるためには,ファシリテーターが必要な時には画面を見るようにして,他の時は見なくても済むようにすることで画面に向き合う時間を減らす工夫があることが望ましい.ファシリテーターが画面を見るべきタイミングは議論の話題が変化したときである.以前の議論の内容から外れた発言がされた時,ファシリテーターが適切な発言をすることで,脱線や炎上を避けて議論を収束させることができる.
すなわち,ファシリテーターの代わりに自動的に議論中の話題の変化を観測することが求められている.
%
現在,COLLAGREE上で使用されている議論支援システムは「(1)投稿支援システム」と「(2)議論可視化システム」の2つに大別できる.
投稿支援システムはポイント機能やファシリテーションフレーズ簡易投稿機能のように,ユーザーが投稿をする際に何らかの補助やリアクションを行う.現行の機能では選択肢の提示に留まっており,作業量を減らすことには繋がりにくい.
一方,議論可視化システムは議論ツリーやキーワード抽出のように,ユーザーにスレッドとは異なる議論の見方を提供する.
\ref{Fig:argTree1}に議論ツリーの例を示す.
\begin{figure}[htbp]
 \begin{center}
  \includegraphics[width=\textwidth]{../images/2.Related_Work/argTree1.png}
  \caption{議論ツリー}
  \label{Fig:argTree1}
  \vspace{-10pt}
 \end{center}
\end{figure}
現行の機能では議論を見やすくすることに重点が置かれており,議論の把握の助けにはなるが画面に向き合う時間を減らすことにはなりにくい.むしろ,作業量を増やすことになり得ることもある.
従って,現行の支援機能ではファシリテーターの作業量の減少には繋がりにくい.
%
近年,自然言語処理の分野において分散表現が多くの研究で使われており,機械翻訳を始めとする単語の意味が重要となる分野で精度の向上が確認されている.分散表現を用いることで,人間に近い精度で話題の変化を観測することが可能となる.
%
以上のような背景を踏まえて,分散表現を用いて,話題の変化を観測し,話題の変化が確認された時にファシリテーターに伝えることが望ましい.
話題の変化の観測は,発言中に現れる単語の類似度の計算と見なすことができる.
分散表現を用いることで単語間の類似度を求めることができる,値が大きいほど単語がそれぞれ類似した実数ベクトルであることを表す.単語Aと単語Bの実数ベクトルが類似しているとは,単語Aと共に使われることの多い単語と単語Bと共に使われることの多い単語が多く共通していることを示す.故に,分散表現を使って単語の類似度を計算することができる.
%
発言文から単語を選ぶ際には自動要約を用いる.発言文から重要でない単語を取り除くことで関連度の計算の精度を高めることが可能となる.
要約の手法としてはokapi BM25 \cite{okapiBM25}とLexRankを組み合わせた抽出的要約手法を用いる.
\begin{comment}
%======================================= 社会的背景
2013年頃からWeb上での大規模な議論活動が活発になり,大規模な人数での議論が期待されている.
大規模な議論では意見を共有することは可能であるが,議論を整理させることや収束させることは難しい.以上から大規模意見集約システムCOLLAGREEが開発された.本システムではWeb上で適切に大規模な議論を行うことができるように議論をマネジメントするファシリテーターを導入した\cite{collagreeTest}.
過去の実験ではファシリテーターの存在が議論の集約に大きな役割を果たしていることが認識されており,大規模な議論のためにファシリテータは必要である.しかし,議論の規模に伴って議論時間が長くなる傾向があり,同時にファシリテーターは常に議論の動向を見続ける必要がある.故に,議論の規模が大きくなればなるほどファシリテーターは長時間かつ大規模な議論の動向の監視によって大きな負担がかかる.大規模な議論が増加する傾向を踏まえるとファシリテーターにかかる負担を軽減する支援が必要である.\\
以上の問題を解決するため,話題の変化を追い,重要な話題の転換点をファシリテーターの代わりに検出することが有用であると考える.必要な時にだけファシリテーターが画面を見れば良いようにすることでファシリテーターの負担軽減が期待できる.
%========================================= 現行手法問題点背景
%議論支援に関する先行研究において,既存の手法は全てが文字列を文字列のまま扱う手法である.
%既存手法は殆どがパターンマッチングと重み付けの2つに区分することができる.
%パターンマッチングでは事前に単語を登録して,単語がマッチした場合に処理を行うが,処理それぞれに対して単語を登録しなければならず手間が膨大になってしまう.また,単語の意味が考慮されておらず,手作業で登録を行うので登録漏れがあった場合に単語の意味に関係なく処理を行うことが不可能となってしまう.
%重み付けは単語の出現頻度や文章の長さを使用して単語・文章に順位を付ける手法で必ずしも単語の登録が必要でないため多くの研究で使用されている.
%しかし,重み付けもまた単語を文字列のまま扱っており,意味までは考慮されていない.故に ,人間なら対応できる似た単語でも1文字違うだけで対処が困難となる.
議論支援に関する先行研究においてファシリテーターに対する支援を目的としたものは無く,殆どが議論の活性化や可視化を目的としている.
%=================================新手法
近年,自然言語処理の分野において分散表現が多くの研究で使われており.分散表現は文字列である単語を辞書データを使用して実数ベクトルへと変換する.辞書データにない単語には対応できないが,多様な処理を1つの辞書データで行うことができる.また,実数ベクトルの各数値が単語の意味を表現するものとなっており,数値を使用して処理を行うことができる.
分散表現を用いることで既存手法より人間の感覚に近しい処理を行うことができる.
%=================================
以上のような背景を踏まえて,分散表現を用いてファシリテーターの代わりに話題の変化を判定し,知らせることを目指す.
話題転換の検出は発言同士の近さ,すなわち発言に含まれる単語意味の近さと見ることができる.
分散表現ではベクトル同士の内積計算を行うことで単語同士の意味の近さを計算することができる.
また,分散表現を使用することで機械翻訳を始めとする複数の分野で精度の向上が確認されている.
\end{comment}
\section{研究の目的}
\label{intro:taget}
本論文では,分散表現を用いて議論中での発言に含まれる単語の関連度を計算し,話題の変化を観測する手法を提案する.

\section{本論文の構成}
本論文の構成を以下に示す.
\ref{relwork:chapter} 章では要約手法に関する研究と,分散表現に関する先行研究を紹介する.
次に,\ref{model:chapter}章では発言の要約手法の説明を行い,\ref{impl:chapter}章では分散表現を用いた単語集合間の関連度計算について説明する.
そして,\ref{exp:chapter}章では話題転換点の検出の評価実験について説明する.
最後に\ref{con:chapter}章で本論文のまとめと考察を示す.

 %-------------------------------------------------------------------------------
 \expandafter\ifx\csname MasterFile\endcsname\relax
	\def\BibFile{hoge}
	\input{../Bibliography/chapter}
  \fi
  %-------------------------------------------------------------------------------
  \expandafter\ifx\csname MasterFile\endcsname\relax
  \end{document}
  \fi

  \fi
  %-------------------------------------------------------------------------------
  \expandafter\ifx\csname MasterFile\endcsname\relax
  \end{document}
  \fi


%===============================================================================
\pagestyle{plain}
%-------------------------------------------------------------------------------
\expandafter\ifx\csname MasterFile\endcsname\relax
	\def\SubFile{hoge}
	\documentclass[a4j,12pt,twoside,openany]{jreport}
%\nofiles %tocファイルを更新させない
%\documentclass[12pt,a4j,twoside,openany]{jsbook}
\usepackage[dvipdfmx]{graphicx}
\usepackage{../dspc} % ベースラインスキップの指定
\usepackage{../slashbox} % 表に斜線を入れる
%\usepackage{../mediabb}
\usepackage{fancyvrb} % Verbatim環境
\usepackage{fancyhdr} % Headerの下線付き章見出し
\usepackage{here} % float[H]
\usepackage{multirow}
\usepackage{hhline} % 表の罫線の角を美しくする
\usepackage{amsmath} %コレがないとcasesが動かない
\usepackage{amsfonts} % 数学用フォント
\usepackage{bm} % 数式環境での bold
\usepackage{algorithm}
\usepackage{algorithmicx}
\usepackage[noend]{algpseudocode}
\usepackage[flushleft]{threeparttable} % 脚注付きテーブル
\usepackage{enumitem}
\usepackage{comment}
\usepackage{fancybox}
%\usepackage{csvsimple,booktabs,siunitx}
%\usepackage{filecontents}


\setlength{\evensidemargin}{5pt}
\setlength{\oddsidemargin}{40pt}
%\setlength{\headheight}{16.5pt}
%%\setlength{\headheight}{30pt}
\setcounter{secnumdepth}{3}
\setlist[description]{leftmargin=2\parindent,labelindent=\parindent}

\makeatletter
\def\@makechapterhead#1{%
	\vspace*{50\p@}%
	{
		\parindent \z@ \raggedright \normalfont
		\ifnum \c@secnumdepth >\m@ne
		% \if@mainmatter
			\huge\bfseries\@chapapp\thechapter\@chappos
			\par\nobreak
			\vskip 20\p@
		% \fi
		\fi
		\interlinepenalty\@M
		\Huge\bfseries #1\par\nobreak
		\vskip 40\p@
	}
}

%新しいコマンド定義
\newcounter{linenumber}
\newenvironment{listing}{%
  \begin{list}{%
    \small\arabic{linenumber}:}{%
      \usecounter{linenumber}%
      \setlength{\baselineskip}{18pt}%
      \setlength{\itemsep}{0pt}%
      \setlength{\parsep}{0pt}}}%
 {\end{list}}
\newcommand{\figcaption}[1]{\def\@captype{figure}\caption{#1}}
\newcommand{\tblcaption}[1]{\def\@captype{table}\caption{#1}}
\newcommand{\norm}[1]{\left\| #1 \right\|}
\newcommand{\cc}[1]{\multicolumn{1}{|c|}{#1}}
\newcommand{\circled}[1]{\raisebox{.5pt}{\textcircled{\raisebox{-.9pt} {#1}}}}
\newcommand{\specialcell}[2][c]{%
  \begin{tabular}[#1]{@{}c@{}}#2\end{tabular}}
\makeatother
%===============================================================================
\expandafter\ifx\csname SubFile\endcsname\relax
\begin{document}
\def\MasterFile{hoge}
%-------------------------------------------------------------------------------
%\maketitle
\thispagestyle{empty}
\input{../hyoushi/title}
%\addcontentsline{toc}{chapter}{表紙}
\thispagestyle{empty}
\mbox{}\newpage
%===============================================================================
%\frontmatter
%===============================================================================
%\mainmatter
%-------------------------------------------------------------------------------
\pagenumbering{arabic}
\cleardoublepage
\input{../0.Abstract/chapter}
%-------------------------------------------------------------------------------
\clearpage
\addcontentsline{toc}{chapter}{目次}
\tableofcontents

\clearpage
\addcontentsline{toc}{chapter}{図目次}
\listoffigures

\clearpage
\addcontentsline{toc}{chapter}{表目次}
\listoftables

%-------------------------------------------------------------------------------

%=====================
\pagestyle{fancy} % Headerをつける
\renewcommand{\sectionmark}[1]{\markright{\thesection\ \ \ #1}}
\renewcommand{\chaptermark}[1]{\markboth{#1}{}}
\lhead{}
\chead{}
\lfoot{}
\rfoot{}%-------------------------------------------------------------------------------
\input{../1.Introduction/chapter}
%-------------------------------------------------------------------------------
\input{../2.Related_Work/chapter}
%-------------------------------------------------------------------------------
\input{../3.The_Model/chapter}
%-------------------------------------------------------------------------------
\input{../4.Implementation/chapter}
%-------------------------------------------------------------------------------
\input{../5.Experiments/chapter}
%-------------------------------------------------------------------------------
\input{../6.Conclusion/chapter}

%===============================================================================
\pagestyle{plain}
%-------------------------------------------------------------------------------
\input{../7.Acknowledgement/chapter} %謝辞
%-------------------------------------------------------------------------------
\def\BibFile{../Bibliograhoy/database2}
\input{../Bibliography/chapter} %参考文献
% %===============================================================================
\appendix
\input{../A.Mypaper/chapter} % 投稿論文リスト
\input{../B.SIG-CCI2/chapter} %
\input{../C.IJCAI-16/chapter} %
%===============================================================================
\end{document}
\fi

	\begin{document}
	\setcounter{chapter}{0}
	\fi
  %-------------------------------------------------------------------------------
\cleardoublepage
\chapter{序論}
\label{intro:chapter}
%本章では, 本研究を行なうに至った背景と目的について述べる.その後,本論文の構成について述べる.
\section{研究の背景}
\label{intro:background}
近年,Web上での大規模な議論活動が活発になっているが,現在一般的に使われている "2ちゃんねる" や "Twitter" といったシステムでは整理や収束を行うことが困難である.困難である原因として,議論の管理を行う者がいないことが挙げられる.
つまり,議論を整理・収束させるには議論のマネジメントを行う人物が必要である.
%
大規模意見集約システムCOLLAGREE\cite{collagreeTest}ではファシリテーターと呼ばれる人物が議論のマネジメントを行っている.
しかし,ファシリテーターは人間であり,長時間に渡って大人数での議論の動向をマネジメントし続けるのは困難である.
COLLAGREEで大規模な議論を収束させるためには,ファシリテーターが必要な時には画面を見るようにして,他の時は見なくても済むようにすることで画面に向き合う時間を減らす工夫があることが望ましい.ファシリテーターが画面を見るべきタイミングは議論の話題が変化したときである.以前の議論の内容から外れた発言がされた時,ファシリテーターが適切な発言をすることで,脱線や炎上を避けて議論を収束させることができる.
すなわち,ファシリテーターの代わりに自動的に議論中の話題の変化を観測することが求められている.
%
現在,COLLAGREE上で使用されている議論支援システムは「(1)投稿支援システム」と「(2)議論可視化システム」の2つに大別できる.
投稿支援システムはポイント機能やファシリテーションフレーズ簡易投稿機能のように,ユーザーが投稿をする際に何らかの補助やリアクションを行う.現行の機能では選択肢の提示に留まっており,作業量を減らすことには繋がりにくい.
一方,議論可視化システムは議論ツリーやキーワード抽出のように,ユーザーにスレッドとは異なる議論の見方を提供する.
\ref{Fig:argTree1}に議論ツリーの例を示す.
\begin{figure}[htbp]
 \begin{center}
  \includegraphics[width=\textwidth]{../images/2.Related_Work/argTree1.png}
  \caption{議論ツリー}
  \label{Fig:argTree1}
  \vspace{-10pt}
 \end{center}
\end{figure}
現行の機能では議論を見やすくすることに重点が置かれており,議論の把握の助けにはなるが画面に向き合う時間を減らすことにはなりにくい.むしろ,作業量を増やすことになり得ることもある.
従って,現行の支援機能ではファシリテーターの作業量の減少には繋がりにくい.
%
近年,自然言語処理の分野において分散表現が多くの研究で使われており,機械翻訳を始めとする単語の意味が重要となる分野で精度の向上が確認されている.分散表現を用いることで,人間に近い精度で話題の変化を観測することが可能となる.
%
以上のような背景を踏まえて,分散表現を用いて,話題の変化を観測し,話題の変化が確認された時にファシリテーターに伝えることが望ましい.
話題の変化の観測は,発言中に現れる単語の類似度の計算と見なすことができる.
分散表現を用いることで単語間の類似度を求めることができる,値が大きいほど単語がそれぞれ類似した実数ベクトルであることを表す.単語Aと単語Bの実数ベクトルが類似しているとは,単語Aと共に使われることの多い単語と単語Bと共に使われることの多い単語が多く共通していることを示す.故に,分散表現を使って単語の類似度を計算することができる.
%
発言文から単語を選ぶ際には自動要約を用いる.発言文から重要でない単語を取り除くことで関連度の計算の精度を高めることが可能となる.
要約の手法としてはokapi BM25 \cite{okapiBM25}とLexRankを組み合わせた抽出的要約手法を用いる.
\begin{comment}
%======================================= 社会的背景
2013年頃からWeb上での大規模な議論活動が活発になり,大規模な人数での議論が期待されている.
大規模な議論では意見を共有することは可能であるが,議論を整理させることや収束させることは難しい.以上から大規模意見集約システムCOLLAGREEが開発された.本システムではWeb上で適切に大規模な議論を行うことができるように議論をマネジメントするファシリテーターを導入した\cite{collagreeTest}.
過去の実験ではファシリテーターの存在が議論の集約に大きな役割を果たしていることが認識されており,大規模な議論のためにファシリテータは必要である.しかし,議論の規模に伴って議論時間が長くなる傾向があり,同時にファシリテーターは常に議論の動向を見続ける必要がある.故に,議論の規模が大きくなればなるほどファシリテーターは長時間かつ大規模な議論の動向の監視によって大きな負担がかかる.大規模な議論が増加する傾向を踏まえるとファシリテーターにかかる負担を軽減する支援が必要である.\\
以上の問題を解決するため,話題の変化を追い,重要な話題の転換点をファシリテーターの代わりに検出することが有用であると考える.必要な時にだけファシリテーターが画面を見れば良いようにすることでファシリテーターの負担軽減が期待できる.
%========================================= 現行手法問題点背景
%議論支援に関する先行研究において,既存の手法は全てが文字列を文字列のまま扱う手法である.
%既存手法は殆どがパターンマッチングと重み付けの2つに区分することができる.
%パターンマッチングでは事前に単語を登録して,単語がマッチした場合に処理を行うが,処理それぞれに対して単語を登録しなければならず手間が膨大になってしまう.また,単語の意味が考慮されておらず,手作業で登録を行うので登録漏れがあった場合に単語の意味に関係なく処理を行うことが不可能となってしまう.
%重み付けは単語の出現頻度や文章の長さを使用して単語・文章に順位を付ける手法で必ずしも単語の登録が必要でないため多くの研究で使用されている.
%しかし,重み付けもまた単語を文字列のまま扱っており,意味までは考慮されていない.故に ,人間なら対応できる似た単語でも1文字違うだけで対処が困難となる.
議論支援に関する先行研究においてファシリテーターに対する支援を目的としたものは無く,殆どが議論の活性化や可視化を目的としている.
%=================================新手法
近年,自然言語処理の分野において分散表現が多くの研究で使われており.分散表現は文字列である単語を辞書データを使用して実数ベクトルへと変換する.辞書データにない単語には対応できないが,多様な処理を1つの辞書データで行うことができる.また,実数ベクトルの各数値が単語の意味を表現するものとなっており,数値を使用して処理を行うことができる.
分散表現を用いることで既存手法より人間の感覚に近しい処理を行うことができる.
%=================================
以上のような背景を踏まえて,分散表現を用いてファシリテーターの代わりに話題の変化を判定し,知らせることを目指す.
話題転換の検出は発言同士の近さ,すなわち発言に含まれる単語意味の近さと見ることができる.
分散表現ではベクトル同士の内積計算を行うことで単語同士の意味の近さを計算することができる.
また,分散表現を使用することで機械翻訳を始めとする複数の分野で精度の向上が確認されている.
\end{comment}
\section{研究の目的}
\label{intro:taget}
本論文では,分散表現を用いて議論中での発言に含まれる単語の関連度を計算し,話題の変化を観測する手法を提案する.

\section{本論文の構成}
本論文の構成を以下に示す.
\ref{relwork:chapter} 章では要約手法に関する研究と,分散表現に関する先行研究を紹介する.
次に,\ref{model:chapter}章では発言の要約手法の説明を行い,\ref{impl:chapter}章では分散表現を用いた単語集合間の関連度計算について説明する.
そして,\ref{exp:chapter}章では話題転換点の検出の評価実験について説明する.
最後に\ref{con:chapter}章で本論文のまとめと考察を示す.

 %-------------------------------------------------------------------------------
 \expandafter\ifx\csname MasterFile\endcsname\relax
	\def\BibFile{hoge}
	\expandafter\ifx\csname MasterFile\endcsname\relax
	\def\SubFile{hoge}
	\input{../thesis/thesis}
	\begin{document}
	\setcounter{chapter}{0}
	\fi
  %-------------------------------------------------------------------------------
\cleardoublepage
\chapter{序論}
\label{intro:chapter}
%本章では, 本研究を行なうに至った背景と目的について述べる.その後,本論文の構成について述べる.
\section{研究の背景}
\label{intro:background}
近年,Web上での大規模な議論活動が活発になっているが,現在一般的に使われている "2ちゃんねる" や "Twitter" といったシステムでは整理や収束を行うことが困難である.困難である原因として,議論の管理を行う者がいないことが挙げられる.
つまり,議論を整理・収束させるには議論のマネジメントを行う人物が必要である.
%
大規模意見集約システムCOLLAGREE\cite{collagreeTest}ではファシリテーターと呼ばれる人物が議論のマネジメントを行っている.
しかし,ファシリテーターは人間であり,長時間に渡って大人数での議論の動向をマネジメントし続けるのは困難である.
COLLAGREEで大規模な議論を収束させるためには,ファシリテーターが必要な時には画面を見るようにして,他の時は見なくても済むようにすることで画面に向き合う時間を減らす工夫があることが望ましい.ファシリテーターが画面を見るべきタイミングは議論の話題が変化したときである.以前の議論の内容から外れた発言がされた時,ファシリテーターが適切な発言をすることで,脱線や炎上を避けて議論を収束させることができる.
すなわち,ファシリテーターの代わりに自動的に議論中の話題の変化を観測することが求められている.
%
現在,COLLAGREE上で使用されている議論支援システムは「(1)投稿支援システム」と「(2)議論可視化システム」の2つに大別できる.
投稿支援システムはポイント機能やファシリテーションフレーズ簡易投稿機能のように,ユーザーが投稿をする際に何らかの補助やリアクションを行う.現行の機能では選択肢の提示に留まっており,作業量を減らすことには繋がりにくい.
一方,議論可視化システムは議論ツリーやキーワード抽出のように,ユーザーにスレッドとは異なる議論の見方を提供する.
\ref{Fig:argTree1}に議論ツリーの例を示す.
\begin{figure}[htbp]
 \begin{center}
  \includegraphics[width=\textwidth]{../images/2.Related_Work/argTree1.png}
  \caption{議論ツリー}
  \label{Fig:argTree1}
  \vspace{-10pt}
 \end{center}
\end{figure}
現行の機能では議論を見やすくすることに重点が置かれており,議論の把握の助けにはなるが画面に向き合う時間を減らすことにはなりにくい.むしろ,作業量を増やすことになり得ることもある.
従って,現行の支援機能ではファシリテーターの作業量の減少には繋がりにくい.
%
近年,自然言語処理の分野において分散表現が多くの研究で使われており,機械翻訳を始めとする単語の意味が重要となる分野で精度の向上が確認されている.分散表現を用いることで,人間に近い精度で話題の変化を観測することが可能となる.
%
以上のような背景を踏まえて,分散表現を用いて,話題の変化を観測し,話題の変化が確認された時にファシリテーターに伝えることが望ましい.
話題の変化の観測は,発言中に現れる単語の類似度の計算と見なすことができる.
分散表現を用いることで単語間の類似度を求めることができる,値が大きいほど単語がそれぞれ類似した実数ベクトルであることを表す.単語Aと単語Bの実数ベクトルが類似しているとは,単語Aと共に使われることの多い単語と単語Bと共に使われることの多い単語が多く共通していることを示す.故に,分散表現を使って単語の類似度を計算することができる.
%
発言文から単語を選ぶ際には自動要約を用いる.発言文から重要でない単語を取り除くことで関連度の計算の精度を高めることが可能となる.
要約の手法としてはokapi BM25 \cite{okapiBM25}とLexRankを組み合わせた抽出的要約手法を用いる.
\begin{comment}
%======================================= 社会的背景
2013年頃からWeb上での大規模な議論活動が活発になり,大規模な人数での議論が期待されている.
大規模な議論では意見を共有することは可能であるが,議論を整理させることや収束させることは難しい.以上から大規模意見集約システムCOLLAGREEが開発された.本システムではWeb上で適切に大規模な議論を行うことができるように議論をマネジメントするファシリテーターを導入した\cite{collagreeTest}.
過去の実験ではファシリテーターの存在が議論の集約に大きな役割を果たしていることが認識されており,大規模な議論のためにファシリテータは必要である.しかし,議論の規模に伴って議論時間が長くなる傾向があり,同時にファシリテーターは常に議論の動向を見続ける必要がある.故に,議論の規模が大きくなればなるほどファシリテーターは長時間かつ大規模な議論の動向の監視によって大きな負担がかかる.大規模な議論が増加する傾向を踏まえるとファシリテーターにかかる負担を軽減する支援が必要である.\\
以上の問題を解決するため,話題の変化を追い,重要な話題の転換点をファシリテーターの代わりに検出することが有用であると考える.必要な時にだけファシリテーターが画面を見れば良いようにすることでファシリテーターの負担軽減が期待できる.
%========================================= 現行手法問題点背景
%議論支援に関する先行研究において,既存の手法は全てが文字列を文字列のまま扱う手法である.
%既存手法は殆どがパターンマッチングと重み付けの2つに区分することができる.
%パターンマッチングでは事前に単語を登録して,単語がマッチした場合に処理を行うが,処理それぞれに対して単語を登録しなければならず手間が膨大になってしまう.また,単語の意味が考慮されておらず,手作業で登録を行うので登録漏れがあった場合に単語の意味に関係なく処理を行うことが不可能となってしまう.
%重み付けは単語の出現頻度や文章の長さを使用して単語・文章に順位を付ける手法で必ずしも単語の登録が必要でないため多くの研究で使用されている.
%しかし,重み付けもまた単語を文字列のまま扱っており,意味までは考慮されていない.故に ,人間なら対応できる似た単語でも1文字違うだけで対処が困難となる.
議論支援に関する先行研究においてファシリテーターに対する支援を目的としたものは無く,殆どが議論の活性化や可視化を目的としている.
%=================================新手法
近年,自然言語処理の分野において分散表現が多くの研究で使われており.分散表現は文字列である単語を辞書データを使用して実数ベクトルへと変換する.辞書データにない単語には対応できないが,多様な処理を1つの辞書データで行うことができる.また,実数ベクトルの各数値が単語の意味を表現するものとなっており,数値を使用して処理を行うことができる.
分散表現を用いることで既存手法より人間の感覚に近しい処理を行うことができる.
%=================================
以上のような背景を踏まえて,分散表現を用いてファシリテーターの代わりに話題の変化を判定し,知らせることを目指す.
話題転換の検出は発言同士の近さ,すなわち発言に含まれる単語意味の近さと見ることができる.
分散表現ではベクトル同士の内積計算を行うことで単語同士の意味の近さを計算することができる.
また,分散表現を使用することで機械翻訳を始めとする複数の分野で精度の向上が確認されている.
\end{comment}
\section{研究の目的}
\label{intro:taget}
本論文では,分散表現を用いて議論中での発言に含まれる単語の関連度を計算し,話題の変化を観測する手法を提案する.

\section{本論文の構成}
本論文の構成を以下に示す.
\ref{relwork:chapter} 章では要約手法に関する研究と,分散表現に関する先行研究を紹介する.
次に,\ref{model:chapter}章では発言の要約手法の説明を行い,\ref{impl:chapter}章では分散表現を用いた単語集合間の関連度計算について説明する.
そして,\ref{exp:chapter}章では話題転換点の検出の評価実験について説明する.
最後に\ref{con:chapter}章で本論文のまとめと考察を示す.

 %-------------------------------------------------------------------------------
 \expandafter\ifx\csname MasterFile\endcsname\relax
	\def\BibFile{hoge}
	\input{../Bibliography/chapter}
  \fi
  %-------------------------------------------------------------------------------
  \expandafter\ifx\csname MasterFile\endcsname\relax
  \end{document}
  \fi

  \fi
  %-------------------------------------------------------------------------------
  \expandafter\ifx\csname MasterFile\endcsname\relax
  \end{document}
  \fi
 %謝辞
%-------------------------------------------------------------------------------
\def\BibFile{../Bibliograhoy/database2}
\expandafter\ifx\csname MasterFile\endcsname\relax
	\def\SubFile{hoge}
	\documentclass[a4j,12pt,twoside,openany]{jreport}
%\nofiles %tocファイルを更新させない
%\documentclass[12pt,a4j,twoside,openany]{jsbook}
\usepackage[dvipdfmx]{graphicx}
\usepackage{../dspc} % ベースラインスキップの指定
\usepackage{../slashbox} % 表に斜線を入れる
%\usepackage{../mediabb}
\usepackage{fancyvrb} % Verbatim環境
\usepackage{fancyhdr} % Headerの下線付き章見出し
\usepackage{here} % float[H]
\usepackage{multirow}
\usepackage{hhline} % 表の罫線の角を美しくする
\usepackage{amsmath} %コレがないとcasesが動かない
\usepackage{amsfonts} % 数学用フォント
\usepackage{bm} % 数式環境での bold
\usepackage{algorithm}
\usepackage{algorithmicx}
\usepackage[noend]{algpseudocode}
\usepackage[flushleft]{threeparttable} % 脚注付きテーブル
\usepackage{enumitem}
\usepackage{comment}
\usepackage{fancybox}
%\usepackage{csvsimple,booktabs,siunitx}
%\usepackage{filecontents}


\setlength{\evensidemargin}{5pt}
\setlength{\oddsidemargin}{40pt}
%\setlength{\headheight}{16.5pt}
%%\setlength{\headheight}{30pt}
\setcounter{secnumdepth}{3}
\setlist[description]{leftmargin=2\parindent,labelindent=\parindent}

\makeatletter
\def\@makechapterhead#1{%
	\vspace*{50\p@}%
	{
		\parindent \z@ \raggedright \normalfont
		\ifnum \c@secnumdepth >\m@ne
		% \if@mainmatter
			\huge\bfseries\@chapapp\thechapter\@chappos
			\par\nobreak
			\vskip 20\p@
		% \fi
		\fi
		\interlinepenalty\@M
		\Huge\bfseries #1\par\nobreak
		\vskip 40\p@
	}
}

%新しいコマンド定義
\newcounter{linenumber}
\newenvironment{listing}{%
  \begin{list}{%
    \small\arabic{linenumber}:}{%
      \usecounter{linenumber}%
      \setlength{\baselineskip}{18pt}%
      \setlength{\itemsep}{0pt}%
      \setlength{\parsep}{0pt}}}%
 {\end{list}}
\newcommand{\figcaption}[1]{\def\@captype{figure}\caption{#1}}
\newcommand{\tblcaption}[1]{\def\@captype{table}\caption{#1}}
\newcommand{\norm}[1]{\left\| #1 \right\|}
\newcommand{\cc}[1]{\multicolumn{1}{|c|}{#1}}
\newcommand{\circled}[1]{\raisebox{.5pt}{\textcircled{\raisebox{-.9pt} {#1}}}}
\newcommand{\specialcell}[2][c]{%
  \begin{tabular}[#1]{@{}c@{}}#2\end{tabular}}
\makeatother
%===============================================================================
\expandafter\ifx\csname SubFile\endcsname\relax
\begin{document}
\def\MasterFile{hoge}
%-------------------------------------------------------------------------------
%\maketitle
\thispagestyle{empty}
\input{../hyoushi/title}
%\addcontentsline{toc}{chapter}{表紙}
\thispagestyle{empty}
\mbox{}\newpage
%===============================================================================
%\frontmatter
%===============================================================================
%\mainmatter
%-------------------------------------------------------------------------------
\pagenumbering{arabic}
\cleardoublepage
\input{../0.Abstract/chapter}
%-------------------------------------------------------------------------------
\clearpage
\addcontentsline{toc}{chapter}{目次}
\tableofcontents

\clearpage
\addcontentsline{toc}{chapter}{図目次}
\listoffigures

\clearpage
\addcontentsline{toc}{chapter}{表目次}
\listoftables

%-------------------------------------------------------------------------------

%=====================
\pagestyle{fancy} % Headerをつける
\renewcommand{\sectionmark}[1]{\markright{\thesection\ \ \ #1}}
\renewcommand{\chaptermark}[1]{\markboth{#1}{}}
\lhead{}
\chead{}
\lfoot{}
\rfoot{}%-------------------------------------------------------------------------------
\input{../1.Introduction/chapter}
%-------------------------------------------------------------------------------
\input{../2.Related_Work/chapter}
%-------------------------------------------------------------------------------
\input{../3.The_Model/chapter}
%-------------------------------------------------------------------------------
\input{../4.Implementation/chapter}
%-------------------------------------------------------------------------------
\input{../5.Experiments/chapter}
%-------------------------------------------------------------------------------
\input{../6.Conclusion/chapter}

%===============================================================================
\pagestyle{plain}
%-------------------------------------------------------------------------------
\input{../7.Acknowledgement/chapter} %謝辞
%-------------------------------------------------------------------------------
\def\BibFile{../Bibliograhoy/database2}
\input{../Bibliography/chapter} %参考文献
% %===============================================================================
\appendix
\input{../A.Mypaper/chapter} % 投稿論文リスト
\input{../B.SIG-CCI2/chapter} %
\input{../C.IJCAI-16/chapter} %
%===============================================================================
\end{document}
\fi

	\begin{document}
	\setcounter{chapter}{0}
	\fi
  %-------------------------------------------------------------------------------
\cleardoublepage
\chapter{序論}
\label{intro:chapter}
%本章では, 本研究を行なうに至った背景と目的について述べる.その後,本論文の構成について述べる.
\section{研究の背景}
\label{intro:background}
近年,Web上での大規模な議論活動が活発になっているが,現在一般的に使われている "2ちゃんねる" や "Twitter" といったシステムでは整理や収束を行うことが困難である.困難である原因として,議論の管理を行う者がいないことが挙げられる.
つまり,議論を整理・収束させるには議論のマネジメントを行う人物が必要である.
%
大規模意見集約システムCOLLAGREE\cite{collagreeTest}ではファシリテーターと呼ばれる人物が議論のマネジメントを行っている.
しかし,ファシリテーターは人間であり,長時間に渡って大人数での議論の動向をマネジメントし続けるのは困難である.
COLLAGREEで大規模な議論を収束させるためには,ファシリテーターが必要な時には画面を見るようにして,他の時は見なくても済むようにすることで画面に向き合う時間を減らす工夫があることが望ましい.ファシリテーターが画面を見るべきタイミングは議論の話題が変化したときである.以前の議論の内容から外れた発言がされた時,ファシリテーターが適切な発言をすることで,脱線や炎上を避けて議論を収束させることができる.
すなわち,ファシリテーターの代わりに自動的に議論中の話題の変化を観測することが求められている.
%
現在,COLLAGREE上で使用されている議論支援システムは「(1)投稿支援システム」と「(2)議論可視化システム」の2つに大別できる.
投稿支援システムはポイント機能やファシリテーションフレーズ簡易投稿機能のように,ユーザーが投稿をする際に何らかの補助やリアクションを行う.現行の機能では選択肢の提示に留まっており,作業量を減らすことには繋がりにくい.
一方,議論可視化システムは議論ツリーやキーワード抽出のように,ユーザーにスレッドとは異なる議論の見方を提供する.
\ref{Fig:argTree1}に議論ツリーの例を示す.
\begin{figure}[htbp]
 \begin{center}
  \includegraphics[width=\textwidth]{../images/2.Related_Work/argTree1.png}
  \caption{議論ツリー}
  \label{Fig:argTree1}
  \vspace{-10pt}
 \end{center}
\end{figure}
現行の機能では議論を見やすくすることに重点が置かれており,議論の把握の助けにはなるが画面に向き合う時間を減らすことにはなりにくい.むしろ,作業量を増やすことになり得ることもある.
従って,現行の支援機能ではファシリテーターの作業量の減少には繋がりにくい.
%
近年,自然言語処理の分野において分散表現が多くの研究で使われており,機械翻訳を始めとする単語の意味が重要となる分野で精度の向上が確認されている.分散表現を用いることで,人間に近い精度で話題の変化を観測することが可能となる.
%
以上のような背景を踏まえて,分散表現を用いて,話題の変化を観測し,話題の変化が確認された時にファシリテーターに伝えることが望ましい.
話題の変化の観測は,発言中に現れる単語の類似度の計算と見なすことができる.
分散表現を用いることで単語間の類似度を求めることができる,値が大きいほど単語がそれぞれ類似した実数ベクトルであることを表す.単語Aと単語Bの実数ベクトルが類似しているとは,単語Aと共に使われることの多い単語と単語Bと共に使われることの多い単語が多く共通していることを示す.故に,分散表現を使って単語の類似度を計算することができる.
%
発言文から単語を選ぶ際には自動要約を用いる.発言文から重要でない単語を取り除くことで関連度の計算の精度を高めることが可能となる.
要約の手法としてはokapi BM25 \cite{okapiBM25}とLexRankを組み合わせた抽出的要約手法を用いる.
\begin{comment}
%======================================= 社会的背景
2013年頃からWeb上での大規模な議論活動が活発になり,大規模な人数での議論が期待されている.
大規模な議論では意見を共有することは可能であるが,議論を整理させることや収束させることは難しい.以上から大規模意見集約システムCOLLAGREEが開発された.本システムではWeb上で適切に大規模な議論を行うことができるように議論をマネジメントするファシリテーターを導入した\cite{collagreeTest}.
過去の実験ではファシリテーターの存在が議論の集約に大きな役割を果たしていることが認識されており,大規模な議論のためにファシリテータは必要である.しかし,議論の規模に伴って議論時間が長くなる傾向があり,同時にファシリテーターは常に議論の動向を見続ける必要がある.故に,議論の規模が大きくなればなるほどファシリテーターは長時間かつ大規模な議論の動向の監視によって大きな負担がかかる.大規模な議論が増加する傾向を踏まえるとファシリテーターにかかる負担を軽減する支援が必要である.\\
以上の問題を解決するため,話題の変化を追い,重要な話題の転換点をファシリテーターの代わりに検出することが有用であると考える.必要な時にだけファシリテーターが画面を見れば良いようにすることでファシリテーターの負担軽減が期待できる.
%========================================= 現行手法問題点背景
%議論支援に関する先行研究において,既存の手法は全てが文字列を文字列のまま扱う手法である.
%既存手法は殆どがパターンマッチングと重み付けの2つに区分することができる.
%パターンマッチングでは事前に単語を登録して,単語がマッチした場合に処理を行うが,処理それぞれに対して単語を登録しなければならず手間が膨大になってしまう.また,単語の意味が考慮されておらず,手作業で登録を行うので登録漏れがあった場合に単語の意味に関係なく処理を行うことが不可能となってしまう.
%重み付けは単語の出現頻度や文章の長さを使用して単語・文章に順位を付ける手法で必ずしも単語の登録が必要でないため多くの研究で使用されている.
%しかし,重み付けもまた単語を文字列のまま扱っており,意味までは考慮されていない.故に ,人間なら対応できる似た単語でも1文字違うだけで対処が困難となる.
議論支援に関する先行研究においてファシリテーターに対する支援を目的としたものは無く,殆どが議論の活性化や可視化を目的としている.
%=================================新手法
近年,自然言語処理の分野において分散表現が多くの研究で使われており.分散表現は文字列である単語を辞書データを使用して実数ベクトルへと変換する.辞書データにない単語には対応できないが,多様な処理を1つの辞書データで行うことができる.また,実数ベクトルの各数値が単語の意味を表現するものとなっており,数値を使用して処理を行うことができる.
分散表現を用いることで既存手法より人間の感覚に近しい処理を行うことができる.
%=================================
以上のような背景を踏まえて,分散表現を用いてファシリテーターの代わりに話題の変化を判定し,知らせることを目指す.
話題転換の検出は発言同士の近さ,すなわち発言に含まれる単語意味の近さと見ることができる.
分散表現ではベクトル同士の内積計算を行うことで単語同士の意味の近さを計算することができる.
また,分散表現を使用することで機械翻訳を始めとする複数の分野で精度の向上が確認されている.
\end{comment}
\section{研究の目的}
\label{intro:taget}
本論文では,分散表現を用いて議論中での発言に含まれる単語の関連度を計算し,話題の変化を観測する手法を提案する.

\section{本論文の構成}
本論文の構成を以下に示す.
\ref{relwork:chapter} 章では要約手法に関する研究と,分散表現に関する先行研究を紹介する.
次に,\ref{model:chapter}章では発言の要約手法の説明を行い,\ref{impl:chapter}章では分散表現を用いた単語集合間の関連度計算について説明する.
そして,\ref{exp:chapter}章では話題転換点の検出の評価実験について説明する.
最後に\ref{con:chapter}章で本論文のまとめと考察を示す.

 %-------------------------------------------------------------------------------
 \expandafter\ifx\csname MasterFile\endcsname\relax
	\def\BibFile{hoge}
	\expandafter\ifx\csname MasterFile\endcsname\relax
	\def\SubFile{hoge}
	\input{../thesis/thesis}
	\begin{document}
	\setcounter{chapter}{0}
	\fi
  %-------------------------------------------------------------------------------
\cleardoublepage
\chapter{序論}
\label{intro:chapter}
%本章では, 本研究を行なうに至った背景と目的について述べる.その後,本論文の構成について述べる.
\section{研究の背景}
\label{intro:background}
近年,Web上での大規模な議論活動が活発になっているが,現在一般的に使われている "2ちゃんねる" や "Twitter" といったシステムでは整理や収束を行うことが困難である.困難である原因として,議論の管理を行う者がいないことが挙げられる.
つまり,議論を整理・収束させるには議論のマネジメントを行う人物が必要である.
%
大規模意見集約システムCOLLAGREE\cite{collagreeTest}ではファシリテーターと呼ばれる人物が議論のマネジメントを行っている.
しかし,ファシリテーターは人間であり,長時間に渡って大人数での議論の動向をマネジメントし続けるのは困難である.
COLLAGREEで大規模な議論を収束させるためには,ファシリテーターが必要な時には画面を見るようにして,他の時は見なくても済むようにすることで画面に向き合う時間を減らす工夫があることが望ましい.ファシリテーターが画面を見るべきタイミングは議論の話題が変化したときである.以前の議論の内容から外れた発言がされた時,ファシリテーターが適切な発言をすることで,脱線や炎上を避けて議論を収束させることができる.
すなわち,ファシリテーターの代わりに自動的に議論中の話題の変化を観測することが求められている.
%
現在,COLLAGREE上で使用されている議論支援システムは「(1)投稿支援システム」と「(2)議論可視化システム」の2つに大別できる.
投稿支援システムはポイント機能やファシリテーションフレーズ簡易投稿機能のように,ユーザーが投稿をする際に何らかの補助やリアクションを行う.現行の機能では選択肢の提示に留まっており,作業量を減らすことには繋がりにくい.
一方,議論可視化システムは議論ツリーやキーワード抽出のように,ユーザーにスレッドとは異なる議論の見方を提供する.
\ref{Fig:argTree1}に議論ツリーの例を示す.
\begin{figure}[htbp]
 \begin{center}
  \includegraphics[width=\textwidth]{../images/2.Related_Work/argTree1.png}
  \caption{議論ツリー}
  \label{Fig:argTree1}
  \vspace{-10pt}
 \end{center}
\end{figure}
現行の機能では議論を見やすくすることに重点が置かれており,議論の把握の助けにはなるが画面に向き合う時間を減らすことにはなりにくい.むしろ,作業量を増やすことになり得ることもある.
従って,現行の支援機能ではファシリテーターの作業量の減少には繋がりにくい.
%
近年,自然言語処理の分野において分散表現が多くの研究で使われており,機械翻訳を始めとする単語の意味が重要となる分野で精度の向上が確認されている.分散表現を用いることで,人間に近い精度で話題の変化を観測することが可能となる.
%
以上のような背景を踏まえて,分散表現を用いて,話題の変化を観測し,話題の変化が確認された時にファシリテーターに伝えることが望ましい.
話題の変化の観測は,発言中に現れる単語の類似度の計算と見なすことができる.
分散表現を用いることで単語間の類似度を求めることができる,値が大きいほど単語がそれぞれ類似した実数ベクトルであることを表す.単語Aと単語Bの実数ベクトルが類似しているとは,単語Aと共に使われることの多い単語と単語Bと共に使われることの多い単語が多く共通していることを示す.故に,分散表現を使って単語の類似度を計算することができる.
%
発言文から単語を選ぶ際には自動要約を用いる.発言文から重要でない単語を取り除くことで関連度の計算の精度を高めることが可能となる.
要約の手法としてはokapi BM25 \cite{okapiBM25}とLexRankを組み合わせた抽出的要約手法を用いる.
\begin{comment}
%======================================= 社会的背景
2013年頃からWeb上での大規模な議論活動が活発になり,大規模な人数での議論が期待されている.
大規模な議論では意見を共有することは可能であるが,議論を整理させることや収束させることは難しい.以上から大規模意見集約システムCOLLAGREEが開発された.本システムではWeb上で適切に大規模な議論を行うことができるように議論をマネジメントするファシリテーターを導入した\cite{collagreeTest}.
過去の実験ではファシリテーターの存在が議論の集約に大きな役割を果たしていることが認識されており,大規模な議論のためにファシリテータは必要である.しかし,議論の規模に伴って議論時間が長くなる傾向があり,同時にファシリテーターは常に議論の動向を見続ける必要がある.故に,議論の規模が大きくなればなるほどファシリテーターは長時間かつ大規模な議論の動向の監視によって大きな負担がかかる.大規模な議論が増加する傾向を踏まえるとファシリテーターにかかる負担を軽減する支援が必要である.\\
以上の問題を解決するため,話題の変化を追い,重要な話題の転換点をファシリテーターの代わりに検出することが有用であると考える.必要な時にだけファシリテーターが画面を見れば良いようにすることでファシリテーターの負担軽減が期待できる.
%========================================= 現行手法問題点背景
%議論支援に関する先行研究において,既存の手法は全てが文字列を文字列のまま扱う手法である.
%既存手法は殆どがパターンマッチングと重み付けの2つに区分することができる.
%パターンマッチングでは事前に単語を登録して,単語がマッチした場合に処理を行うが,処理それぞれに対して単語を登録しなければならず手間が膨大になってしまう.また,単語の意味が考慮されておらず,手作業で登録を行うので登録漏れがあった場合に単語の意味に関係なく処理を行うことが不可能となってしまう.
%重み付けは単語の出現頻度や文章の長さを使用して単語・文章に順位を付ける手法で必ずしも単語の登録が必要でないため多くの研究で使用されている.
%しかし,重み付けもまた単語を文字列のまま扱っており,意味までは考慮されていない.故に ,人間なら対応できる似た単語でも1文字違うだけで対処が困難となる.
議論支援に関する先行研究においてファシリテーターに対する支援を目的としたものは無く,殆どが議論の活性化や可視化を目的としている.
%=================================新手法
近年,自然言語処理の分野において分散表現が多くの研究で使われており.分散表現は文字列である単語を辞書データを使用して実数ベクトルへと変換する.辞書データにない単語には対応できないが,多様な処理を1つの辞書データで行うことができる.また,実数ベクトルの各数値が単語の意味を表現するものとなっており,数値を使用して処理を行うことができる.
分散表現を用いることで既存手法より人間の感覚に近しい処理を行うことができる.
%=================================
以上のような背景を踏まえて,分散表現を用いてファシリテーターの代わりに話題の変化を判定し,知らせることを目指す.
話題転換の検出は発言同士の近さ,すなわち発言に含まれる単語意味の近さと見ることができる.
分散表現ではベクトル同士の内積計算を行うことで単語同士の意味の近さを計算することができる.
また,分散表現を使用することで機械翻訳を始めとする複数の分野で精度の向上が確認されている.
\end{comment}
\section{研究の目的}
\label{intro:taget}
本論文では,分散表現を用いて議論中での発言に含まれる単語の関連度を計算し,話題の変化を観測する手法を提案する.

\section{本論文の構成}
本論文の構成を以下に示す.
\ref{relwork:chapter} 章では要約手法に関する研究と,分散表現に関する先行研究を紹介する.
次に,\ref{model:chapter}章では発言の要約手法の説明を行い,\ref{impl:chapter}章では分散表現を用いた単語集合間の関連度計算について説明する.
そして,\ref{exp:chapter}章では話題転換点の検出の評価実験について説明する.
最後に\ref{con:chapter}章で本論文のまとめと考察を示す.

 %-------------------------------------------------------------------------------
 \expandafter\ifx\csname MasterFile\endcsname\relax
	\def\BibFile{hoge}
	\input{../Bibliography/chapter}
  \fi
  %-------------------------------------------------------------------------------
  \expandafter\ifx\csname MasterFile\endcsname\relax
  \end{document}
  \fi

  \fi
  %-------------------------------------------------------------------------------
  \expandafter\ifx\csname MasterFile\endcsname\relax
  \end{document}
  \fi
 %参考文献
% %===============================================================================
\appendix
\expandafter\ifx\csname MasterFile\endcsname\relax
	\def\SubFile{hoge}
	\documentclass[a4j,12pt,twoside,openany]{jreport}
%\nofiles %tocファイルを更新させない
%\documentclass[12pt,a4j,twoside,openany]{jsbook}
\usepackage[dvipdfmx]{graphicx}
\usepackage{../dspc} % ベースラインスキップの指定
\usepackage{../slashbox} % 表に斜線を入れる
%\usepackage{../mediabb}
\usepackage{fancyvrb} % Verbatim環境
\usepackage{fancyhdr} % Headerの下線付き章見出し
\usepackage{here} % float[H]
\usepackage{multirow}
\usepackage{hhline} % 表の罫線の角を美しくする
\usepackage{amsmath} %コレがないとcasesが動かない
\usepackage{amsfonts} % 数学用フォント
\usepackage{bm} % 数式環境での bold
\usepackage{algorithm}
\usepackage{algorithmicx}
\usepackage[noend]{algpseudocode}
\usepackage[flushleft]{threeparttable} % 脚注付きテーブル
\usepackage{enumitem}
\usepackage{comment}
\usepackage{fancybox}
%\usepackage{csvsimple,booktabs,siunitx}
%\usepackage{filecontents}


\setlength{\evensidemargin}{5pt}
\setlength{\oddsidemargin}{40pt}
%\setlength{\headheight}{16.5pt}
%%\setlength{\headheight}{30pt}
\setcounter{secnumdepth}{3}
\setlist[description]{leftmargin=2\parindent,labelindent=\parindent}

\makeatletter
\def\@makechapterhead#1{%
	\vspace*{50\p@}%
	{
		\parindent \z@ \raggedright \normalfont
		\ifnum \c@secnumdepth >\m@ne
		% \if@mainmatter
			\huge\bfseries\@chapapp\thechapter\@chappos
			\par\nobreak
			\vskip 20\p@
		% \fi
		\fi
		\interlinepenalty\@M
		\Huge\bfseries #1\par\nobreak
		\vskip 40\p@
	}
}

%新しいコマンド定義
\newcounter{linenumber}
\newenvironment{listing}{%
  \begin{list}{%
    \small\arabic{linenumber}:}{%
      \usecounter{linenumber}%
      \setlength{\baselineskip}{18pt}%
      \setlength{\itemsep}{0pt}%
      \setlength{\parsep}{0pt}}}%
 {\end{list}}
\newcommand{\figcaption}[1]{\def\@captype{figure}\caption{#1}}
\newcommand{\tblcaption}[1]{\def\@captype{table}\caption{#1}}
\newcommand{\norm}[1]{\left\| #1 \right\|}
\newcommand{\cc}[1]{\multicolumn{1}{|c|}{#1}}
\newcommand{\circled}[1]{\raisebox{.5pt}{\textcircled{\raisebox{-.9pt} {#1}}}}
\newcommand{\specialcell}[2][c]{%
  \begin{tabular}[#1]{@{}c@{}}#2\end{tabular}}
\makeatother
%===============================================================================
\expandafter\ifx\csname SubFile\endcsname\relax
\begin{document}
\def\MasterFile{hoge}
%-------------------------------------------------------------------------------
%\maketitle
\thispagestyle{empty}
\input{../hyoushi/title}
%\addcontentsline{toc}{chapter}{表紙}
\thispagestyle{empty}
\mbox{}\newpage
%===============================================================================
%\frontmatter
%===============================================================================
%\mainmatter
%-------------------------------------------------------------------------------
\pagenumbering{arabic}
\cleardoublepage
\input{../0.Abstract/chapter}
%-------------------------------------------------------------------------------
\clearpage
\addcontentsline{toc}{chapter}{目次}
\tableofcontents

\clearpage
\addcontentsline{toc}{chapter}{図目次}
\listoffigures

\clearpage
\addcontentsline{toc}{chapter}{表目次}
\listoftables

%-------------------------------------------------------------------------------

%=====================
\pagestyle{fancy} % Headerをつける
\renewcommand{\sectionmark}[1]{\markright{\thesection\ \ \ #1}}
\renewcommand{\chaptermark}[1]{\markboth{#1}{}}
\lhead{}
\chead{}
\lfoot{}
\rfoot{}%-------------------------------------------------------------------------------
\input{../1.Introduction/chapter}
%-------------------------------------------------------------------------------
\input{../2.Related_Work/chapter}
%-------------------------------------------------------------------------------
\input{../3.The_Model/chapter}
%-------------------------------------------------------------------------------
\input{../4.Implementation/chapter}
%-------------------------------------------------------------------------------
\input{../5.Experiments/chapter}
%-------------------------------------------------------------------------------
\input{../6.Conclusion/chapter}

%===============================================================================
\pagestyle{plain}
%-------------------------------------------------------------------------------
\input{../7.Acknowledgement/chapter} %謝辞
%-------------------------------------------------------------------------------
\def\BibFile{../Bibliograhoy/database2}
\input{../Bibliography/chapter} %参考文献
% %===============================================================================
\appendix
\input{../A.Mypaper/chapter} % 投稿論文リスト
\input{../B.SIG-CCI2/chapter} %
\input{../C.IJCAI-16/chapter} %
%===============================================================================
\end{document}
\fi

	\begin{document}
	\setcounter{chapter}{0}
	\fi
  %-------------------------------------------------------------------------------
\cleardoublepage
\chapter{序論}
\label{intro:chapter}
%本章では, 本研究を行なうに至った背景と目的について述べる.その後,本論文の構成について述べる.
\section{研究の背景}
\label{intro:background}
近年,Web上での大規模な議論活動が活発になっているが,現在一般的に使われている "2ちゃんねる" や "Twitter" といったシステムでは整理や収束を行うことが困難である.困難である原因として,議論の管理を行う者がいないことが挙げられる.
つまり,議論を整理・収束させるには議論のマネジメントを行う人物が必要である.
%
大規模意見集約システムCOLLAGREE\cite{collagreeTest}ではファシリテーターと呼ばれる人物が議論のマネジメントを行っている.
しかし,ファシリテーターは人間であり,長時間に渡って大人数での議論の動向をマネジメントし続けるのは困難である.
COLLAGREEで大規模な議論を収束させるためには,ファシリテーターが必要な時には画面を見るようにして,他の時は見なくても済むようにすることで画面に向き合う時間を減らす工夫があることが望ましい.ファシリテーターが画面を見るべきタイミングは議論の話題が変化したときである.以前の議論の内容から外れた発言がされた時,ファシリテーターが適切な発言をすることで,脱線や炎上を避けて議論を収束させることができる.
すなわち,ファシリテーターの代わりに自動的に議論中の話題の変化を観測することが求められている.
%
現在,COLLAGREE上で使用されている議論支援システムは「(1)投稿支援システム」と「(2)議論可視化システム」の2つに大別できる.
投稿支援システムはポイント機能やファシリテーションフレーズ簡易投稿機能のように,ユーザーが投稿をする際に何らかの補助やリアクションを行う.現行の機能では選択肢の提示に留まっており,作業量を減らすことには繋がりにくい.
一方,議論可視化システムは議論ツリーやキーワード抽出のように,ユーザーにスレッドとは異なる議論の見方を提供する.
\ref{Fig:argTree1}に議論ツリーの例を示す.
\begin{figure}[htbp]
 \begin{center}
  \includegraphics[width=\textwidth]{../images/2.Related_Work/argTree1.png}
  \caption{議論ツリー}
  \label{Fig:argTree1}
  \vspace{-10pt}
 \end{center}
\end{figure}
現行の機能では議論を見やすくすることに重点が置かれており,議論の把握の助けにはなるが画面に向き合う時間を減らすことにはなりにくい.むしろ,作業量を増やすことになり得ることもある.
従って,現行の支援機能ではファシリテーターの作業量の減少には繋がりにくい.
%
近年,自然言語処理の分野において分散表現が多くの研究で使われており,機械翻訳を始めとする単語の意味が重要となる分野で精度の向上が確認されている.分散表現を用いることで,人間に近い精度で話題の変化を観測することが可能となる.
%
以上のような背景を踏まえて,分散表現を用いて,話題の変化を観測し,話題の変化が確認された時にファシリテーターに伝えることが望ましい.
話題の変化の観測は,発言中に現れる単語の類似度の計算と見なすことができる.
分散表現を用いることで単語間の類似度を求めることができる,値が大きいほど単語がそれぞれ類似した実数ベクトルであることを表す.単語Aと単語Bの実数ベクトルが類似しているとは,単語Aと共に使われることの多い単語と単語Bと共に使われることの多い単語が多く共通していることを示す.故に,分散表現を使って単語の類似度を計算することができる.
%
発言文から単語を選ぶ際には自動要約を用いる.発言文から重要でない単語を取り除くことで関連度の計算の精度を高めることが可能となる.
要約の手法としてはokapi BM25 \cite{okapiBM25}とLexRankを組み合わせた抽出的要約手法を用いる.
\begin{comment}
%======================================= 社会的背景
2013年頃からWeb上での大規模な議論活動が活発になり,大規模な人数での議論が期待されている.
大規模な議論では意見を共有することは可能であるが,議論を整理させることや収束させることは難しい.以上から大規模意見集約システムCOLLAGREEが開発された.本システムではWeb上で適切に大規模な議論を行うことができるように議論をマネジメントするファシリテーターを導入した\cite{collagreeTest}.
過去の実験ではファシリテーターの存在が議論の集約に大きな役割を果たしていることが認識されており,大規模な議論のためにファシリテータは必要である.しかし,議論の規模に伴って議論時間が長くなる傾向があり,同時にファシリテーターは常に議論の動向を見続ける必要がある.故に,議論の規模が大きくなればなるほどファシリテーターは長時間かつ大規模な議論の動向の監視によって大きな負担がかかる.大規模な議論が増加する傾向を踏まえるとファシリテーターにかかる負担を軽減する支援が必要である.\\
以上の問題を解決するため,話題の変化を追い,重要な話題の転換点をファシリテーターの代わりに検出することが有用であると考える.必要な時にだけファシリテーターが画面を見れば良いようにすることでファシリテーターの負担軽減が期待できる.
%========================================= 現行手法問題点背景
%議論支援に関する先行研究において,既存の手法は全てが文字列を文字列のまま扱う手法である.
%既存手法は殆どがパターンマッチングと重み付けの2つに区分することができる.
%パターンマッチングでは事前に単語を登録して,単語がマッチした場合に処理を行うが,処理それぞれに対して単語を登録しなければならず手間が膨大になってしまう.また,単語の意味が考慮されておらず,手作業で登録を行うので登録漏れがあった場合に単語の意味に関係なく処理を行うことが不可能となってしまう.
%重み付けは単語の出現頻度や文章の長さを使用して単語・文章に順位を付ける手法で必ずしも単語の登録が必要でないため多くの研究で使用されている.
%しかし,重み付けもまた単語を文字列のまま扱っており,意味までは考慮されていない.故に ,人間なら対応できる似た単語でも1文字違うだけで対処が困難となる.
議論支援に関する先行研究においてファシリテーターに対する支援を目的としたものは無く,殆どが議論の活性化や可視化を目的としている.
%=================================新手法
近年,自然言語処理の分野において分散表現が多くの研究で使われており.分散表現は文字列である単語を辞書データを使用して実数ベクトルへと変換する.辞書データにない単語には対応できないが,多様な処理を1つの辞書データで行うことができる.また,実数ベクトルの各数値が単語の意味を表現するものとなっており,数値を使用して処理を行うことができる.
分散表現を用いることで既存手法より人間の感覚に近しい処理を行うことができる.
%=================================
以上のような背景を踏まえて,分散表現を用いてファシリテーターの代わりに話題の変化を判定し,知らせることを目指す.
話題転換の検出は発言同士の近さ,すなわち発言に含まれる単語意味の近さと見ることができる.
分散表現ではベクトル同士の内積計算を行うことで単語同士の意味の近さを計算することができる.
また,分散表現を使用することで機械翻訳を始めとする複数の分野で精度の向上が確認されている.
\end{comment}
\section{研究の目的}
\label{intro:taget}
本論文では,分散表現を用いて議論中での発言に含まれる単語の関連度を計算し,話題の変化を観測する手法を提案する.

\section{本論文の構成}
本論文の構成を以下に示す.
\ref{relwork:chapter} 章では要約手法に関する研究と,分散表現に関する先行研究を紹介する.
次に,\ref{model:chapter}章では発言の要約手法の説明を行い,\ref{impl:chapter}章では分散表現を用いた単語集合間の関連度計算について説明する.
そして,\ref{exp:chapter}章では話題転換点の検出の評価実験について説明する.
最後に\ref{con:chapter}章で本論文のまとめと考察を示す.

 %-------------------------------------------------------------------------------
 \expandafter\ifx\csname MasterFile\endcsname\relax
	\def\BibFile{hoge}
	\expandafter\ifx\csname MasterFile\endcsname\relax
	\def\SubFile{hoge}
	\input{../thesis/thesis}
	\begin{document}
	\setcounter{chapter}{0}
	\fi
  %-------------------------------------------------------------------------------
\cleardoublepage
\chapter{序論}
\label{intro:chapter}
%本章では, 本研究を行なうに至った背景と目的について述べる.その後,本論文の構成について述べる.
\section{研究の背景}
\label{intro:background}
近年,Web上での大規模な議論活動が活発になっているが,現在一般的に使われている "2ちゃんねる" や "Twitter" といったシステムでは整理や収束を行うことが困難である.困難である原因として,議論の管理を行う者がいないことが挙げられる.
つまり,議論を整理・収束させるには議論のマネジメントを行う人物が必要である.
%
大規模意見集約システムCOLLAGREE\cite{collagreeTest}ではファシリテーターと呼ばれる人物が議論のマネジメントを行っている.
しかし,ファシリテーターは人間であり,長時間に渡って大人数での議論の動向をマネジメントし続けるのは困難である.
COLLAGREEで大規模な議論を収束させるためには,ファシリテーターが必要な時には画面を見るようにして,他の時は見なくても済むようにすることで画面に向き合う時間を減らす工夫があることが望ましい.ファシリテーターが画面を見るべきタイミングは議論の話題が変化したときである.以前の議論の内容から外れた発言がされた時,ファシリテーターが適切な発言をすることで,脱線や炎上を避けて議論を収束させることができる.
すなわち,ファシリテーターの代わりに自動的に議論中の話題の変化を観測することが求められている.
%
現在,COLLAGREE上で使用されている議論支援システムは「(1)投稿支援システム」と「(2)議論可視化システム」の2つに大別できる.
投稿支援システムはポイント機能やファシリテーションフレーズ簡易投稿機能のように,ユーザーが投稿をする際に何らかの補助やリアクションを行う.現行の機能では選択肢の提示に留まっており,作業量を減らすことには繋がりにくい.
一方,議論可視化システムは議論ツリーやキーワード抽出のように,ユーザーにスレッドとは異なる議論の見方を提供する.
\ref{Fig:argTree1}に議論ツリーの例を示す.
\begin{figure}[htbp]
 \begin{center}
  \includegraphics[width=\textwidth]{../images/2.Related_Work/argTree1.png}
  \caption{議論ツリー}
  \label{Fig:argTree1}
  \vspace{-10pt}
 \end{center}
\end{figure}
現行の機能では議論を見やすくすることに重点が置かれており,議論の把握の助けにはなるが画面に向き合う時間を減らすことにはなりにくい.むしろ,作業量を増やすことになり得ることもある.
従って,現行の支援機能ではファシリテーターの作業量の減少には繋がりにくい.
%
近年,自然言語処理の分野において分散表現が多くの研究で使われており,機械翻訳を始めとする単語の意味が重要となる分野で精度の向上が確認されている.分散表現を用いることで,人間に近い精度で話題の変化を観測することが可能となる.
%
以上のような背景を踏まえて,分散表現を用いて,話題の変化を観測し,話題の変化が確認された時にファシリテーターに伝えることが望ましい.
話題の変化の観測は,発言中に現れる単語の類似度の計算と見なすことができる.
分散表現を用いることで単語間の類似度を求めることができる,値が大きいほど単語がそれぞれ類似した実数ベクトルであることを表す.単語Aと単語Bの実数ベクトルが類似しているとは,単語Aと共に使われることの多い単語と単語Bと共に使われることの多い単語が多く共通していることを示す.故に,分散表現を使って単語の類似度を計算することができる.
%
発言文から単語を選ぶ際には自動要約を用いる.発言文から重要でない単語を取り除くことで関連度の計算の精度を高めることが可能となる.
要約の手法としてはokapi BM25 \cite{okapiBM25}とLexRankを組み合わせた抽出的要約手法を用いる.
\begin{comment}
%======================================= 社会的背景
2013年頃からWeb上での大規模な議論活動が活発になり,大規模な人数での議論が期待されている.
大規模な議論では意見を共有することは可能であるが,議論を整理させることや収束させることは難しい.以上から大規模意見集約システムCOLLAGREEが開発された.本システムではWeb上で適切に大規模な議論を行うことができるように議論をマネジメントするファシリテーターを導入した\cite{collagreeTest}.
過去の実験ではファシリテーターの存在が議論の集約に大きな役割を果たしていることが認識されており,大規模な議論のためにファシリテータは必要である.しかし,議論の規模に伴って議論時間が長くなる傾向があり,同時にファシリテーターは常に議論の動向を見続ける必要がある.故に,議論の規模が大きくなればなるほどファシリテーターは長時間かつ大規模な議論の動向の監視によって大きな負担がかかる.大規模な議論が増加する傾向を踏まえるとファシリテーターにかかる負担を軽減する支援が必要である.\\
以上の問題を解決するため,話題の変化を追い,重要な話題の転換点をファシリテーターの代わりに検出することが有用であると考える.必要な時にだけファシリテーターが画面を見れば良いようにすることでファシリテーターの負担軽減が期待できる.
%========================================= 現行手法問題点背景
%議論支援に関する先行研究において,既存の手法は全てが文字列を文字列のまま扱う手法である.
%既存手法は殆どがパターンマッチングと重み付けの2つに区分することができる.
%パターンマッチングでは事前に単語を登録して,単語がマッチした場合に処理を行うが,処理それぞれに対して単語を登録しなければならず手間が膨大になってしまう.また,単語の意味が考慮されておらず,手作業で登録を行うので登録漏れがあった場合に単語の意味に関係なく処理を行うことが不可能となってしまう.
%重み付けは単語の出現頻度や文章の長さを使用して単語・文章に順位を付ける手法で必ずしも単語の登録が必要でないため多くの研究で使用されている.
%しかし,重み付けもまた単語を文字列のまま扱っており,意味までは考慮されていない.故に ,人間なら対応できる似た単語でも1文字違うだけで対処が困難となる.
議論支援に関する先行研究においてファシリテーターに対する支援を目的としたものは無く,殆どが議論の活性化や可視化を目的としている.
%=================================新手法
近年,自然言語処理の分野において分散表現が多くの研究で使われており.分散表現は文字列である単語を辞書データを使用して実数ベクトルへと変換する.辞書データにない単語には対応できないが,多様な処理を1つの辞書データで行うことができる.また,実数ベクトルの各数値が単語の意味を表現するものとなっており,数値を使用して処理を行うことができる.
分散表現を用いることで既存手法より人間の感覚に近しい処理を行うことができる.
%=================================
以上のような背景を踏まえて,分散表現を用いてファシリテーターの代わりに話題の変化を判定し,知らせることを目指す.
話題転換の検出は発言同士の近さ,すなわち発言に含まれる単語意味の近さと見ることができる.
分散表現ではベクトル同士の内積計算を行うことで単語同士の意味の近さを計算することができる.
また,分散表現を使用することで機械翻訳を始めとする複数の分野で精度の向上が確認されている.
\end{comment}
\section{研究の目的}
\label{intro:taget}
本論文では,分散表現を用いて議論中での発言に含まれる単語の関連度を計算し,話題の変化を観測する手法を提案する.

\section{本論文の構成}
本論文の構成を以下に示す.
\ref{relwork:chapter} 章では要約手法に関する研究と,分散表現に関する先行研究を紹介する.
次に,\ref{model:chapter}章では発言の要約手法の説明を行い,\ref{impl:chapter}章では分散表現を用いた単語集合間の関連度計算について説明する.
そして,\ref{exp:chapter}章では話題転換点の検出の評価実験について説明する.
最後に\ref{con:chapter}章で本論文のまとめと考察を示す.

 %-------------------------------------------------------------------------------
 \expandafter\ifx\csname MasterFile\endcsname\relax
	\def\BibFile{hoge}
	\input{../Bibliography/chapter}
  \fi
  %-------------------------------------------------------------------------------
  \expandafter\ifx\csname MasterFile\endcsname\relax
  \end{document}
  \fi

  \fi
  %-------------------------------------------------------------------------------
  \expandafter\ifx\csname MasterFile\endcsname\relax
  \end{document}
  \fi
 % 投稿論文リスト
\expandafter\ifx\csname MasterFile\endcsname\relax
	\def\SubFile{hoge}
	\documentclass[a4j,12pt,twoside,openany]{jreport}
%\nofiles %tocファイルを更新させない
%\documentclass[12pt,a4j,twoside,openany]{jsbook}
\usepackage[dvipdfmx]{graphicx}
\usepackage{../dspc} % ベースラインスキップの指定
\usepackage{../slashbox} % 表に斜線を入れる
%\usepackage{../mediabb}
\usepackage{fancyvrb} % Verbatim環境
\usepackage{fancyhdr} % Headerの下線付き章見出し
\usepackage{here} % float[H]
\usepackage{multirow}
\usepackage{hhline} % 表の罫線の角を美しくする
\usepackage{amsmath} %コレがないとcasesが動かない
\usepackage{amsfonts} % 数学用フォント
\usepackage{bm} % 数式環境での bold
\usepackage{algorithm}
\usepackage{algorithmicx}
\usepackage[noend]{algpseudocode}
\usepackage[flushleft]{threeparttable} % 脚注付きテーブル
\usepackage{enumitem}
\usepackage{comment}
\usepackage{fancybox}
%\usepackage{csvsimple,booktabs,siunitx}
%\usepackage{filecontents}


\setlength{\evensidemargin}{5pt}
\setlength{\oddsidemargin}{40pt}
%\setlength{\headheight}{16.5pt}
%%\setlength{\headheight}{30pt}
\setcounter{secnumdepth}{3}
\setlist[description]{leftmargin=2\parindent,labelindent=\parindent}

\makeatletter
\def\@makechapterhead#1{%
	\vspace*{50\p@}%
	{
		\parindent \z@ \raggedright \normalfont
		\ifnum \c@secnumdepth >\m@ne
		% \if@mainmatter
			\huge\bfseries\@chapapp\thechapter\@chappos
			\par\nobreak
			\vskip 20\p@
		% \fi
		\fi
		\interlinepenalty\@M
		\Huge\bfseries #1\par\nobreak
		\vskip 40\p@
	}
}

%新しいコマンド定義
\newcounter{linenumber}
\newenvironment{listing}{%
  \begin{list}{%
    \small\arabic{linenumber}:}{%
      \usecounter{linenumber}%
      \setlength{\baselineskip}{18pt}%
      \setlength{\itemsep}{0pt}%
      \setlength{\parsep}{0pt}}}%
 {\end{list}}
\newcommand{\figcaption}[1]{\def\@captype{figure}\caption{#1}}
\newcommand{\tblcaption}[1]{\def\@captype{table}\caption{#1}}
\newcommand{\norm}[1]{\left\| #1 \right\|}
\newcommand{\cc}[1]{\multicolumn{1}{|c|}{#1}}
\newcommand{\circled}[1]{\raisebox{.5pt}{\textcircled{\raisebox{-.9pt} {#1}}}}
\newcommand{\specialcell}[2][c]{%
  \begin{tabular}[#1]{@{}c@{}}#2\end{tabular}}
\makeatother
%===============================================================================
\expandafter\ifx\csname SubFile\endcsname\relax
\begin{document}
\def\MasterFile{hoge}
%-------------------------------------------------------------------------------
%\maketitle
\thispagestyle{empty}
\input{../hyoushi/title}
%\addcontentsline{toc}{chapter}{表紙}
\thispagestyle{empty}
\mbox{}\newpage
%===============================================================================
%\frontmatter
%===============================================================================
%\mainmatter
%-------------------------------------------------------------------------------
\pagenumbering{arabic}
\cleardoublepage
\input{../0.Abstract/chapter}
%-------------------------------------------------------------------------------
\clearpage
\addcontentsline{toc}{chapter}{目次}
\tableofcontents

\clearpage
\addcontentsline{toc}{chapter}{図目次}
\listoffigures

\clearpage
\addcontentsline{toc}{chapter}{表目次}
\listoftables

%-------------------------------------------------------------------------------

%=====================
\pagestyle{fancy} % Headerをつける
\renewcommand{\sectionmark}[1]{\markright{\thesection\ \ \ #1}}
\renewcommand{\chaptermark}[1]{\markboth{#1}{}}
\lhead{}
\chead{}
\lfoot{}
\rfoot{}%-------------------------------------------------------------------------------
\input{../1.Introduction/chapter}
%-------------------------------------------------------------------------------
\input{../2.Related_Work/chapter}
%-------------------------------------------------------------------------------
\input{../3.The_Model/chapter}
%-------------------------------------------------------------------------------
\input{../4.Implementation/chapter}
%-------------------------------------------------------------------------------
\input{../5.Experiments/chapter}
%-------------------------------------------------------------------------------
\input{../6.Conclusion/chapter}

%===============================================================================
\pagestyle{plain}
%-------------------------------------------------------------------------------
\input{../7.Acknowledgement/chapter} %謝辞
%-------------------------------------------------------------------------------
\def\BibFile{../Bibliograhoy/database2}
\input{../Bibliography/chapter} %参考文献
% %===============================================================================
\appendix
\input{../A.Mypaper/chapter} % 投稿論文リスト
\input{../B.SIG-CCI2/chapter} %
\input{../C.IJCAI-16/chapter} %
%===============================================================================
\end{document}
\fi

	\begin{document}
	\setcounter{chapter}{0}
	\fi
  %-------------------------------------------------------------------------------
\cleardoublepage
\chapter{序論}
\label{intro:chapter}
%本章では, 本研究を行なうに至った背景と目的について述べる.その後,本論文の構成について述べる.
\section{研究の背景}
\label{intro:background}
近年,Web上での大規模な議論活動が活発になっているが,現在一般的に使われている "2ちゃんねる" や "Twitter" といったシステムでは整理や収束を行うことが困難である.困難である原因として,議論の管理を行う者がいないことが挙げられる.
つまり,議論を整理・収束させるには議論のマネジメントを行う人物が必要である.
%
大規模意見集約システムCOLLAGREE\cite{collagreeTest}ではファシリテーターと呼ばれる人物が議論のマネジメントを行っている.
しかし,ファシリテーターは人間であり,長時間に渡って大人数での議論の動向をマネジメントし続けるのは困難である.
COLLAGREEで大規模な議論を収束させるためには,ファシリテーターが必要な時には画面を見るようにして,他の時は見なくても済むようにすることで画面に向き合う時間を減らす工夫があることが望ましい.ファシリテーターが画面を見るべきタイミングは議論の話題が変化したときである.以前の議論の内容から外れた発言がされた時,ファシリテーターが適切な発言をすることで,脱線や炎上を避けて議論を収束させることができる.
すなわち,ファシリテーターの代わりに自動的に議論中の話題の変化を観測することが求められている.
%
現在,COLLAGREE上で使用されている議論支援システムは「(1)投稿支援システム」と「(2)議論可視化システム」の2つに大別できる.
投稿支援システムはポイント機能やファシリテーションフレーズ簡易投稿機能のように,ユーザーが投稿をする際に何らかの補助やリアクションを行う.現行の機能では選択肢の提示に留まっており,作業量を減らすことには繋がりにくい.
一方,議論可視化システムは議論ツリーやキーワード抽出のように,ユーザーにスレッドとは異なる議論の見方を提供する.
\ref{Fig:argTree1}に議論ツリーの例を示す.
\begin{figure}[htbp]
 \begin{center}
  \includegraphics[width=\textwidth]{../images/2.Related_Work/argTree1.png}
  \caption{議論ツリー}
  \label{Fig:argTree1}
  \vspace{-10pt}
 \end{center}
\end{figure}
現行の機能では議論を見やすくすることに重点が置かれており,議論の把握の助けにはなるが画面に向き合う時間を減らすことにはなりにくい.むしろ,作業量を増やすことになり得ることもある.
従って,現行の支援機能ではファシリテーターの作業量の減少には繋がりにくい.
%
近年,自然言語処理の分野において分散表現が多くの研究で使われており,機械翻訳を始めとする単語の意味が重要となる分野で精度の向上が確認されている.分散表現を用いることで,人間に近い精度で話題の変化を観測することが可能となる.
%
以上のような背景を踏まえて,分散表現を用いて,話題の変化を観測し,話題の変化が確認された時にファシリテーターに伝えることが望ましい.
話題の変化の観測は,発言中に現れる単語の類似度の計算と見なすことができる.
分散表現を用いることで単語間の類似度を求めることができる,値が大きいほど単語がそれぞれ類似した実数ベクトルであることを表す.単語Aと単語Bの実数ベクトルが類似しているとは,単語Aと共に使われることの多い単語と単語Bと共に使われることの多い単語が多く共通していることを示す.故に,分散表現を使って単語の類似度を計算することができる.
%
発言文から単語を選ぶ際には自動要約を用いる.発言文から重要でない単語を取り除くことで関連度の計算の精度を高めることが可能となる.
要約の手法としてはokapi BM25 \cite{okapiBM25}とLexRankを組み合わせた抽出的要約手法を用いる.
\begin{comment}
%======================================= 社会的背景
2013年頃からWeb上での大規模な議論活動が活発になり,大規模な人数での議論が期待されている.
大規模な議論では意見を共有することは可能であるが,議論を整理させることや収束させることは難しい.以上から大規模意見集約システムCOLLAGREEが開発された.本システムではWeb上で適切に大規模な議論を行うことができるように議論をマネジメントするファシリテーターを導入した\cite{collagreeTest}.
過去の実験ではファシリテーターの存在が議論の集約に大きな役割を果たしていることが認識されており,大規模な議論のためにファシリテータは必要である.しかし,議論の規模に伴って議論時間が長くなる傾向があり,同時にファシリテーターは常に議論の動向を見続ける必要がある.故に,議論の規模が大きくなればなるほどファシリテーターは長時間かつ大規模な議論の動向の監視によって大きな負担がかかる.大規模な議論が増加する傾向を踏まえるとファシリテーターにかかる負担を軽減する支援が必要である.\\
以上の問題を解決するため,話題の変化を追い,重要な話題の転換点をファシリテーターの代わりに検出することが有用であると考える.必要な時にだけファシリテーターが画面を見れば良いようにすることでファシリテーターの負担軽減が期待できる.
%========================================= 現行手法問題点背景
%議論支援に関する先行研究において,既存の手法は全てが文字列を文字列のまま扱う手法である.
%既存手法は殆どがパターンマッチングと重み付けの2つに区分することができる.
%パターンマッチングでは事前に単語を登録して,単語がマッチした場合に処理を行うが,処理それぞれに対して単語を登録しなければならず手間が膨大になってしまう.また,単語の意味が考慮されておらず,手作業で登録を行うので登録漏れがあった場合に単語の意味に関係なく処理を行うことが不可能となってしまう.
%重み付けは単語の出現頻度や文章の長さを使用して単語・文章に順位を付ける手法で必ずしも単語の登録が必要でないため多くの研究で使用されている.
%しかし,重み付けもまた単語を文字列のまま扱っており,意味までは考慮されていない.故に ,人間なら対応できる似た単語でも1文字違うだけで対処が困難となる.
議論支援に関する先行研究においてファシリテーターに対する支援を目的としたものは無く,殆どが議論の活性化や可視化を目的としている.
%=================================新手法
近年,自然言語処理の分野において分散表現が多くの研究で使われており.分散表現は文字列である単語を辞書データを使用して実数ベクトルへと変換する.辞書データにない単語には対応できないが,多様な処理を1つの辞書データで行うことができる.また,実数ベクトルの各数値が単語の意味を表現するものとなっており,数値を使用して処理を行うことができる.
分散表現を用いることで既存手法より人間の感覚に近しい処理を行うことができる.
%=================================
以上のような背景を踏まえて,分散表現を用いてファシリテーターの代わりに話題の変化を判定し,知らせることを目指す.
話題転換の検出は発言同士の近さ,すなわち発言に含まれる単語意味の近さと見ることができる.
分散表現ではベクトル同士の内積計算を行うことで単語同士の意味の近さを計算することができる.
また,分散表現を使用することで機械翻訳を始めとする複数の分野で精度の向上が確認されている.
\end{comment}
\section{研究の目的}
\label{intro:taget}
本論文では,分散表現を用いて議論中での発言に含まれる単語の関連度を計算し,話題の変化を観測する手法を提案する.

\section{本論文の構成}
本論文の構成を以下に示す.
\ref{relwork:chapter} 章では要約手法に関する研究と,分散表現に関する先行研究を紹介する.
次に,\ref{model:chapter}章では発言の要約手法の説明を行い,\ref{impl:chapter}章では分散表現を用いた単語集合間の関連度計算について説明する.
そして,\ref{exp:chapter}章では話題転換点の検出の評価実験について説明する.
最後に\ref{con:chapter}章で本論文のまとめと考察を示す.

 %-------------------------------------------------------------------------------
 \expandafter\ifx\csname MasterFile\endcsname\relax
	\def\BibFile{hoge}
	\expandafter\ifx\csname MasterFile\endcsname\relax
	\def\SubFile{hoge}
	\input{../thesis/thesis}
	\begin{document}
	\setcounter{chapter}{0}
	\fi
  %-------------------------------------------------------------------------------
\cleardoublepage
\chapter{序論}
\label{intro:chapter}
%本章では, 本研究を行なうに至った背景と目的について述べる.その後,本論文の構成について述べる.
\section{研究の背景}
\label{intro:background}
近年,Web上での大規模な議論活動が活発になっているが,現在一般的に使われている "2ちゃんねる" や "Twitter" といったシステムでは整理や収束を行うことが困難である.困難である原因として,議論の管理を行う者がいないことが挙げられる.
つまり,議論を整理・収束させるには議論のマネジメントを行う人物が必要である.
%
大規模意見集約システムCOLLAGREE\cite{collagreeTest}ではファシリテーターと呼ばれる人物が議論のマネジメントを行っている.
しかし,ファシリテーターは人間であり,長時間に渡って大人数での議論の動向をマネジメントし続けるのは困難である.
COLLAGREEで大規模な議論を収束させるためには,ファシリテーターが必要な時には画面を見るようにして,他の時は見なくても済むようにすることで画面に向き合う時間を減らす工夫があることが望ましい.ファシリテーターが画面を見るべきタイミングは議論の話題が変化したときである.以前の議論の内容から外れた発言がされた時,ファシリテーターが適切な発言をすることで,脱線や炎上を避けて議論を収束させることができる.
すなわち,ファシリテーターの代わりに自動的に議論中の話題の変化を観測することが求められている.
%
現在,COLLAGREE上で使用されている議論支援システムは「(1)投稿支援システム」と「(2)議論可視化システム」の2つに大別できる.
投稿支援システムはポイント機能やファシリテーションフレーズ簡易投稿機能のように,ユーザーが投稿をする際に何らかの補助やリアクションを行う.現行の機能では選択肢の提示に留まっており,作業量を減らすことには繋がりにくい.
一方,議論可視化システムは議論ツリーやキーワード抽出のように,ユーザーにスレッドとは異なる議論の見方を提供する.
\ref{Fig:argTree1}に議論ツリーの例を示す.
\begin{figure}[htbp]
 \begin{center}
  \includegraphics[width=\textwidth]{../images/2.Related_Work/argTree1.png}
  \caption{議論ツリー}
  \label{Fig:argTree1}
  \vspace{-10pt}
 \end{center}
\end{figure}
現行の機能では議論を見やすくすることに重点が置かれており,議論の把握の助けにはなるが画面に向き合う時間を減らすことにはなりにくい.むしろ,作業量を増やすことになり得ることもある.
従って,現行の支援機能ではファシリテーターの作業量の減少には繋がりにくい.
%
近年,自然言語処理の分野において分散表現が多くの研究で使われており,機械翻訳を始めとする単語の意味が重要となる分野で精度の向上が確認されている.分散表現を用いることで,人間に近い精度で話題の変化を観測することが可能となる.
%
以上のような背景を踏まえて,分散表現を用いて,話題の変化を観測し,話題の変化が確認された時にファシリテーターに伝えることが望ましい.
話題の変化の観測は,発言中に現れる単語の類似度の計算と見なすことができる.
分散表現を用いることで単語間の類似度を求めることができる,値が大きいほど単語がそれぞれ類似した実数ベクトルであることを表す.単語Aと単語Bの実数ベクトルが類似しているとは,単語Aと共に使われることの多い単語と単語Bと共に使われることの多い単語が多く共通していることを示す.故に,分散表現を使って単語の類似度を計算することができる.
%
発言文から単語を選ぶ際には自動要約を用いる.発言文から重要でない単語を取り除くことで関連度の計算の精度を高めることが可能となる.
要約の手法としてはokapi BM25 \cite{okapiBM25}とLexRankを組み合わせた抽出的要約手法を用いる.
\begin{comment}
%======================================= 社会的背景
2013年頃からWeb上での大規模な議論活動が活発になり,大規模な人数での議論が期待されている.
大規模な議論では意見を共有することは可能であるが,議論を整理させることや収束させることは難しい.以上から大規模意見集約システムCOLLAGREEが開発された.本システムではWeb上で適切に大規模な議論を行うことができるように議論をマネジメントするファシリテーターを導入した\cite{collagreeTest}.
過去の実験ではファシリテーターの存在が議論の集約に大きな役割を果たしていることが認識されており,大規模な議論のためにファシリテータは必要である.しかし,議論の規模に伴って議論時間が長くなる傾向があり,同時にファシリテーターは常に議論の動向を見続ける必要がある.故に,議論の規模が大きくなればなるほどファシリテーターは長時間かつ大規模な議論の動向の監視によって大きな負担がかかる.大規模な議論が増加する傾向を踏まえるとファシリテーターにかかる負担を軽減する支援が必要である.\\
以上の問題を解決するため,話題の変化を追い,重要な話題の転換点をファシリテーターの代わりに検出することが有用であると考える.必要な時にだけファシリテーターが画面を見れば良いようにすることでファシリテーターの負担軽減が期待できる.
%========================================= 現行手法問題点背景
%議論支援に関する先行研究において,既存の手法は全てが文字列を文字列のまま扱う手法である.
%既存手法は殆どがパターンマッチングと重み付けの2つに区分することができる.
%パターンマッチングでは事前に単語を登録して,単語がマッチした場合に処理を行うが,処理それぞれに対して単語を登録しなければならず手間が膨大になってしまう.また,単語の意味が考慮されておらず,手作業で登録を行うので登録漏れがあった場合に単語の意味に関係なく処理を行うことが不可能となってしまう.
%重み付けは単語の出現頻度や文章の長さを使用して単語・文章に順位を付ける手法で必ずしも単語の登録が必要でないため多くの研究で使用されている.
%しかし,重み付けもまた単語を文字列のまま扱っており,意味までは考慮されていない.故に ,人間なら対応できる似た単語でも1文字違うだけで対処が困難となる.
議論支援に関する先行研究においてファシリテーターに対する支援を目的としたものは無く,殆どが議論の活性化や可視化を目的としている.
%=================================新手法
近年,自然言語処理の分野において分散表現が多くの研究で使われており.分散表現は文字列である単語を辞書データを使用して実数ベクトルへと変換する.辞書データにない単語には対応できないが,多様な処理を1つの辞書データで行うことができる.また,実数ベクトルの各数値が単語の意味を表現するものとなっており,数値を使用して処理を行うことができる.
分散表現を用いることで既存手法より人間の感覚に近しい処理を行うことができる.
%=================================
以上のような背景を踏まえて,分散表現を用いてファシリテーターの代わりに話題の変化を判定し,知らせることを目指す.
話題転換の検出は発言同士の近さ,すなわち発言に含まれる単語意味の近さと見ることができる.
分散表現ではベクトル同士の内積計算を行うことで単語同士の意味の近さを計算することができる.
また,分散表現を使用することで機械翻訳を始めとする複数の分野で精度の向上が確認されている.
\end{comment}
\section{研究の目的}
\label{intro:taget}
本論文では,分散表現を用いて議論中での発言に含まれる単語の関連度を計算し,話題の変化を観測する手法を提案する.

\section{本論文の構成}
本論文の構成を以下に示す.
\ref{relwork:chapter} 章では要約手法に関する研究と,分散表現に関する先行研究を紹介する.
次に,\ref{model:chapter}章では発言の要約手法の説明を行い,\ref{impl:chapter}章では分散表現を用いた単語集合間の関連度計算について説明する.
そして,\ref{exp:chapter}章では話題転換点の検出の評価実験について説明する.
最後に\ref{con:chapter}章で本論文のまとめと考察を示す.

 %-------------------------------------------------------------------------------
 \expandafter\ifx\csname MasterFile\endcsname\relax
	\def\BibFile{hoge}
	\input{../Bibliography/chapter}
  \fi
  %-------------------------------------------------------------------------------
  \expandafter\ifx\csname MasterFile\endcsname\relax
  \end{document}
  \fi

  \fi
  %-------------------------------------------------------------------------------
  \expandafter\ifx\csname MasterFile\endcsname\relax
  \end{document}
  \fi
 %
\expandafter\ifx\csname MasterFile\endcsname\relax
	\def\SubFile{hoge}
	\documentclass[a4j,12pt,twoside,openany]{jreport}
%\nofiles %tocファイルを更新させない
%\documentclass[12pt,a4j,twoside,openany]{jsbook}
\usepackage[dvipdfmx]{graphicx}
\usepackage{../dspc} % ベースラインスキップの指定
\usepackage{../slashbox} % 表に斜線を入れる
%\usepackage{../mediabb}
\usepackage{fancyvrb} % Verbatim環境
\usepackage{fancyhdr} % Headerの下線付き章見出し
\usepackage{here} % float[H]
\usepackage{multirow}
\usepackage{hhline} % 表の罫線の角を美しくする
\usepackage{amsmath} %コレがないとcasesが動かない
\usepackage{amsfonts} % 数学用フォント
\usepackage{bm} % 数式環境での bold
\usepackage{algorithm}
\usepackage{algorithmicx}
\usepackage[noend]{algpseudocode}
\usepackage[flushleft]{threeparttable} % 脚注付きテーブル
\usepackage{enumitem}
\usepackage{comment}
\usepackage{fancybox}
%\usepackage{csvsimple,booktabs,siunitx}
%\usepackage{filecontents}


\setlength{\evensidemargin}{5pt}
\setlength{\oddsidemargin}{40pt}
%\setlength{\headheight}{16.5pt}
%%\setlength{\headheight}{30pt}
\setcounter{secnumdepth}{3}
\setlist[description]{leftmargin=2\parindent,labelindent=\parindent}

\makeatletter
\def\@makechapterhead#1{%
	\vspace*{50\p@}%
	{
		\parindent \z@ \raggedright \normalfont
		\ifnum \c@secnumdepth >\m@ne
		% \if@mainmatter
			\huge\bfseries\@chapapp\thechapter\@chappos
			\par\nobreak
			\vskip 20\p@
		% \fi
		\fi
		\interlinepenalty\@M
		\Huge\bfseries #1\par\nobreak
		\vskip 40\p@
	}
}

%新しいコマンド定義
\newcounter{linenumber}
\newenvironment{listing}{%
  \begin{list}{%
    \small\arabic{linenumber}:}{%
      \usecounter{linenumber}%
      \setlength{\baselineskip}{18pt}%
      \setlength{\itemsep}{0pt}%
      \setlength{\parsep}{0pt}}}%
 {\end{list}}
\newcommand{\figcaption}[1]{\def\@captype{figure}\caption{#1}}
\newcommand{\tblcaption}[1]{\def\@captype{table}\caption{#1}}
\newcommand{\norm}[1]{\left\| #1 \right\|}
\newcommand{\cc}[1]{\multicolumn{1}{|c|}{#1}}
\newcommand{\circled}[1]{\raisebox{.5pt}{\textcircled{\raisebox{-.9pt} {#1}}}}
\newcommand{\specialcell}[2][c]{%
  \begin{tabular}[#1]{@{}c@{}}#2\end{tabular}}
\makeatother
%===============================================================================
\expandafter\ifx\csname SubFile\endcsname\relax
\begin{document}
\def\MasterFile{hoge}
%-------------------------------------------------------------------------------
%\maketitle
\thispagestyle{empty}
\input{../hyoushi/title}
%\addcontentsline{toc}{chapter}{表紙}
\thispagestyle{empty}
\mbox{}\newpage
%===============================================================================
%\frontmatter
%===============================================================================
%\mainmatter
%-------------------------------------------------------------------------------
\pagenumbering{arabic}
\cleardoublepage
\input{../0.Abstract/chapter}
%-------------------------------------------------------------------------------
\clearpage
\addcontentsline{toc}{chapter}{目次}
\tableofcontents

\clearpage
\addcontentsline{toc}{chapter}{図目次}
\listoffigures

\clearpage
\addcontentsline{toc}{chapter}{表目次}
\listoftables

%-------------------------------------------------------------------------------

%=====================
\pagestyle{fancy} % Headerをつける
\renewcommand{\sectionmark}[1]{\markright{\thesection\ \ \ #1}}
\renewcommand{\chaptermark}[1]{\markboth{#1}{}}
\lhead{}
\chead{}
\lfoot{}
\rfoot{}%-------------------------------------------------------------------------------
\input{../1.Introduction/chapter}
%-------------------------------------------------------------------------------
\input{../2.Related_Work/chapter}
%-------------------------------------------------------------------------------
\input{../3.The_Model/chapter}
%-------------------------------------------------------------------------------
\input{../4.Implementation/chapter}
%-------------------------------------------------------------------------------
\input{../5.Experiments/chapter}
%-------------------------------------------------------------------------------
\input{../6.Conclusion/chapter}

%===============================================================================
\pagestyle{plain}
%-------------------------------------------------------------------------------
\input{../7.Acknowledgement/chapter} %謝辞
%-------------------------------------------------------------------------------
\def\BibFile{../Bibliograhoy/database2}
\input{../Bibliography/chapter} %参考文献
% %===============================================================================
\appendix
\input{../A.Mypaper/chapter} % 投稿論文リスト
\input{../B.SIG-CCI2/chapter} %
\input{../C.IJCAI-16/chapter} %
%===============================================================================
\end{document}
\fi

	\begin{document}
	\setcounter{chapter}{0}
	\fi
  %-------------------------------------------------------------------------------
\cleardoublepage
\chapter{序論}
\label{intro:chapter}
%本章では, 本研究を行なうに至った背景と目的について述べる.その後,本論文の構成について述べる.
\section{研究の背景}
\label{intro:background}
近年,Web上での大規模な議論活動が活発になっているが,現在一般的に使われている "2ちゃんねる" や "Twitter" といったシステムでは整理や収束を行うことが困難である.困難である原因として,議論の管理を行う者がいないことが挙げられる.
つまり,議論を整理・収束させるには議論のマネジメントを行う人物が必要である.
%
大規模意見集約システムCOLLAGREE\cite{collagreeTest}ではファシリテーターと呼ばれる人物が議論のマネジメントを行っている.
しかし,ファシリテーターは人間であり,長時間に渡って大人数での議論の動向をマネジメントし続けるのは困難である.
COLLAGREEで大規模な議論を収束させるためには,ファシリテーターが必要な時には画面を見るようにして,他の時は見なくても済むようにすることで画面に向き合う時間を減らす工夫があることが望ましい.ファシリテーターが画面を見るべきタイミングは議論の話題が変化したときである.以前の議論の内容から外れた発言がされた時,ファシリテーターが適切な発言をすることで,脱線や炎上を避けて議論を収束させることができる.
すなわち,ファシリテーターの代わりに自動的に議論中の話題の変化を観測することが求められている.
%
現在,COLLAGREE上で使用されている議論支援システムは「(1)投稿支援システム」と「(2)議論可視化システム」の2つに大別できる.
投稿支援システムはポイント機能やファシリテーションフレーズ簡易投稿機能のように,ユーザーが投稿をする際に何らかの補助やリアクションを行う.現行の機能では選択肢の提示に留まっており,作業量を減らすことには繋がりにくい.
一方,議論可視化システムは議論ツリーやキーワード抽出のように,ユーザーにスレッドとは異なる議論の見方を提供する.
\ref{Fig:argTree1}に議論ツリーの例を示す.
\begin{figure}[htbp]
 \begin{center}
  \includegraphics[width=\textwidth]{../images/2.Related_Work/argTree1.png}
  \caption{議論ツリー}
  \label{Fig:argTree1}
  \vspace{-10pt}
 \end{center}
\end{figure}
現行の機能では議論を見やすくすることに重点が置かれており,議論の把握の助けにはなるが画面に向き合う時間を減らすことにはなりにくい.むしろ,作業量を増やすことになり得ることもある.
従って,現行の支援機能ではファシリテーターの作業量の減少には繋がりにくい.
%
近年,自然言語処理の分野において分散表現が多くの研究で使われており,機械翻訳を始めとする単語の意味が重要となる分野で精度の向上が確認されている.分散表現を用いることで,人間に近い精度で話題の変化を観測することが可能となる.
%
以上のような背景を踏まえて,分散表現を用いて,話題の変化を観測し,話題の変化が確認された時にファシリテーターに伝えることが望ましい.
話題の変化の観測は,発言中に現れる単語の類似度の計算と見なすことができる.
分散表現を用いることで単語間の類似度を求めることができる,値が大きいほど単語がそれぞれ類似した実数ベクトルであることを表す.単語Aと単語Bの実数ベクトルが類似しているとは,単語Aと共に使われることの多い単語と単語Bと共に使われることの多い単語が多く共通していることを示す.故に,分散表現を使って単語の類似度を計算することができる.
%
発言文から単語を選ぶ際には自動要約を用いる.発言文から重要でない単語を取り除くことで関連度の計算の精度を高めることが可能となる.
要約の手法としてはokapi BM25 \cite{okapiBM25}とLexRankを組み合わせた抽出的要約手法を用いる.
\begin{comment}
%======================================= 社会的背景
2013年頃からWeb上での大規模な議論活動が活発になり,大規模な人数での議論が期待されている.
大規模な議論では意見を共有することは可能であるが,議論を整理させることや収束させることは難しい.以上から大規模意見集約システムCOLLAGREEが開発された.本システムではWeb上で適切に大規模な議論を行うことができるように議論をマネジメントするファシリテーターを導入した\cite{collagreeTest}.
過去の実験ではファシリテーターの存在が議論の集約に大きな役割を果たしていることが認識されており,大規模な議論のためにファシリテータは必要である.しかし,議論の規模に伴って議論時間が長くなる傾向があり,同時にファシリテーターは常に議論の動向を見続ける必要がある.故に,議論の規模が大きくなればなるほどファシリテーターは長時間かつ大規模な議論の動向の監視によって大きな負担がかかる.大規模な議論が増加する傾向を踏まえるとファシリテーターにかかる負担を軽減する支援が必要である.\\
以上の問題を解決するため,話題の変化を追い,重要な話題の転換点をファシリテーターの代わりに検出することが有用であると考える.必要な時にだけファシリテーターが画面を見れば良いようにすることでファシリテーターの負担軽減が期待できる.
%========================================= 現行手法問題点背景
%議論支援に関する先行研究において,既存の手法は全てが文字列を文字列のまま扱う手法である.
%既存手法は殆どがパターンマッチングと重み付けの2つに区分することができる.
%パターンマッチングでは事前に単語を登録して,単語がマッチした場合に処理を行うが,処理それぞれに対して単語を登録しなければならず手間が膨大になってしまう.また,単語の意味が考慮されておらず,手作業で登録を行うので登録漏れがあった場合に単語の意味に関係なく処理を行うことが不可能となってしまう.
%重み付けは単語の出現頻度や文章の長さを使用して単語・文章に順位を付ける手法で必ずしも単語の登録が必要でないため多くの研究で使用されている.
%しかし,重み付けもまた単語を文字列のまま扱っており,意味までは考慮されていない.故に ,人間なら対応できる似た単語でも1文字違うだけで対処が困難となる.
議論支援に関する先行研究においてファシリテーターに対する支援を目的としたものは無く,殆どが議論の活性化や可視化を目的としている.
%=================================新手法
近年,自然言語処理の分野において分散表現が多くの研究で使われており.分散表現は文字列である単語を辞書データを使用して実数ベクトルへと変換する.辞書データにない単語には対応できないが,多様な処理を1つの辞書データで行うことができる.また,実数ベクトルの各数値が単語の意味を表現するものとなっており,数値を使用して処理を行うことができる.
分散表現を用いることで既存手法より人間の感覚に近しい処理を行うことができる.
%=================================
以上のような背景を踏まえて,分散表現を用いてファシリテーターの代わりに話題の変化を判定し,知らせることを目指す.
話題転換の検出は発言同士の近さ,すなわち発言に含まれる単語意味の近さと見ることができる.
分散表現ではベクトル同士の内積計算を行うことで単語同士の意味の近さを計算することができる.
また,分散表現を使用することで機械翻訳を始めとする複数の分野で精度の向上が確認されている.
\end{comment}
\section{研究の目的}
\label{intro:taget}
本論文では,分散表現を用いて議論中での発言に含まれる単語の関連度を計算し,話題の変化を観測する手法を提案する.

\section{本論文の構成}
本論文の構成を以下に示す.
\ref{relwork:chapter} 章では要約手法に関する研究と,分散表現に関する先行研究を紹介する.
次に,\ref{model:chapter}章では発言の要約手法の説明を行い,\ref{impl:chapter}章では分散表現を用いた単語集合間の関連度計算について説明する.
そして,\ref{exp:chapter}章では話題転換点の検出の評価実験について説明する.
最後に\ref{con:chapter}章で本論文のまとめと考察を示す.

 %-------------------------------------------------------------------------------
 \expandafter\ifx\csname MasterFile\endcsname\relax
	\def\BibFile{hoge}
	\expandafter\ifx\csname MasterFile\endcsname\relax
	\def\SubFile{hoge}
	\input{../thesis/thesis}
	\begin{document}
	\setcounter{chapter}{0}
	\fi
  %-------------------------------------------------------------------------------
\cleardoublepage
\chapter{序論}
\label{intro:chapter}
%本章では, 本研究を行なうに至った背景と目的について述べる.その後,本論文の構成について述べる.
\section{研究の背景}
\label{intro:background}
近年,Web上での大規模な議論活動が活発になっているが,現在一般的に使われている "2ちゃんねる" や "Twitter" といったシステムでは整理や収束を行うことが困難である.困難である原因として,議論の管理を行う者がいないことが挙げられる.
つまり,議論を整理・収束させるには議論のマネジメントを行う人物が必要である.
%
大規模意見集約システムCOLLAGREE\cite{collagreeTest}ではファシリテーターと呼ばれる人物が議論のマネジメントを行っている.
しかし,ファシリテーターは人間であり,長時間に渡って大人数での議論の動向をマネジメントし続けるのは困難である.
COLLAGREEで大規模な議論を収束させるためには,ファシリテーターが必要な時には画面を見るようにして,他の時は見なくても済むようにすることで画面に向き合う時間を減らす工夫があることが望ましい.ファシリテーターが画面を見るべきタイミングは議論の話題が変化したときである.以前の議論の内容から外れた発言がされた時,ファシリテーターが適切な発言をすることで,脱線や炎上を避けて議論を収束させることができる.
すなわち,ファシリテーターの代わりに自動的に議論中の話題の変化を観測することが求められている.
%
現在,COLLAGREE上で使用されている議論支援システムは「(1)投稿支援システム」と「(2)議論可視化システム」の2つに大別できる.
投稿支援システムはポイント機能やファシリテーションフレーズ簡易投稿機能のように,ユーザーが投稿をする際に何らかの補助やリアクションを行う.現行の機能では選択肢の提示に留まっており,作業量を減らすことには繋がりにくい.
一方,議論可視化システムは議論ツリーやキーワード抽出のように,ユーザーにスレッドとは異なる議論の見方を提供する.
\ref{Fig:argTree1}に議論ツリーの例を示す.
\begin{figure}[htbp]
 \begin{center}
  \includegraphics[width=\textwidth]{../images/2.Related_Work/argTree1.png}
  \caption{議論ツリー}
  \label{Fig:argTree1}
  \vspace{-10pt}
 \end{center}
\end{figure}
現行の機能では議論を見やすくすることに重点が置かれており,議論の把握の助けにはなるが画面に向き合う時間を減らすことにはなりにくい.むしろ,作業量を増やすことになり得ることもある.
従って,現行の支援機能ではファシリテーターの作業量の減少には繋がりにくい.
%
近年,自然言語処理の分野において分散表現が多くの研究で使われており,機械翻訳を始めとする単語の意味が重要となる分野で精度の向上が確認されている.分散表現を用いることで,人間に近い精度で話題の変化を観測することが可能となる.
%
以上のような背景を踏まえて,分散表現を用いて,話題の変化を観測し,話題の変化が確認された時にファシリテーターに伝えることが望ましい.
話題の変化の観測は,発言中に現れる単語の類似度の計算と見なすことができる.
分散表現を用いることで単語間の類似度を求めることができる,値が大きいほど単語がそれぞれ類似した実数ベクトルであることを表す.単語Aと単語Bの実数ベクトルが類似しているとは,単語Aと共に使われることの多い単語と単語Bと共に使われることの多い単語が多く共通していることを示す.故に,分散表現を使って単語の類似度を計算することができる.
%
発言文から単語を選ぶ際には自動要約を用いる.発言文から重要でない単語を取り除くことで関連度の計算の精度を高めることが可能となる.
要約の手法としてはokapi BM25 \cite{okapiBM25}とLexRankを組み合わせた抽出的要約手法を用いる.
\begin{comment}
%======================================= 社会的背景
2013年頃からWeb上での大規模な議論活動が活発になり,大規模な人数での議論が期待されている.
大規模な議論では意見を共有することは可能であるが,議論を整理させることや収束させることは難しい.以上から大規模意見集約システムCOLLAGREEが開発された.本システムではWeb上で適切に大規模な議論を行うことができるように議論をマネジメントするファシリテーターを導入した\cite{collagreeTest}.
過去の実験ではファシリテーターの存在が議論の集約に大きな役割を果たしていることが認識されており,大規模な議論のためにファシリテータは必要である.しかし,議論の規模に伴って議論時間が長くなる傾向があり,同時にファシリテーターは常に議論の動向を見続ける必要がある.故に,議論の規模が大きくなればなるほどファシリテーターは長時間かつ大規模な議論の動向の監視によって大きな負担がかかる.大規模な議論が増加する傾向を踏まえるとファシリテーターにかかる負担を軽減する支援が必要である.\\
以上の問題を解決するため,話題の変化を追い,重要な話題の転換点をファシリテーターの代わりに検出することが有用であると考える.必要な時にだけファシリテーターが画面を見れば良いようにすることでファシリテーターの負担軽減が期待できる.
%========================================= 現行手法問題点背景
%議論支援に関する先行研究において,既存の手法は全てが文字列を文字列のまま扱う手法である.
%既存手法は殆どがパターンマッチングと重み付けの2つに区分することができる.
%パターンマッチングでは事前に単語を登録して,単語がマッチした場合に処理を行うが,処理それぞれに対して単語を登録しなければならず手間が膨大になってしまう.また,単語の意味が考慮されておらず,手作業で登録を行うので登録漏れがあった場合に単語の意味に関係なく処理を行うことが不可能となってしまう.
%重み付けは単語の出現頻度や文章の長さを使用して単語・文章に順位を付ける手法で必ずしも単語の登録が必要でないため多くの研究で使用されている.
%しかし,重み付けもまた単語を文字列のまま扱っており,意味までは考慮されていない.故に ,人間なら対応できる似た単語でも1文字違うだけで対処が困難となる.
議論支援に関する先行研究においてファシリテーターに対する支援を目的としたものは無く,殆どが議論の活性化や可視化を目的としている.
%=================================新手法
近年,自然言語処理の分野において分散表現が多くの研究で使われており.分散表現は文字列である単語を辞書データを使用して実数ベクトルへと変換する.辞書データにない単語には対応できないが,多様な処理を1つの辞書データで行うことができる.また,実数ベクトルの各数値が単語の意味を表現するものとなっており,数値を使用して処理を行うことができる.
分散表現を用いることで既存手法より人間の感覚に近しい処理を行うことができる.
%=================================
以上のような背景を踏まえて,分散表現を用いてファシリテーターの代わりに話題の変化を判定し,知らせることを目指す.
話題転換の検出は発言同士の近さ,すなわち発言に含まれる単語意味の近さと見ることができる.
分散表現ではベクトル同士の内積計算を行うことで単語同士の意味の近さを計算することができる.
また,分散表現を使用することで機械翻訳を始めとする複数の分野で精度の向上が確認されている.
\end{comment}
\section{研究の目的}
\label{intro:taget}
本論文では,分散表現を用いて議論中での発言に含まれる単語の関連度を計算し,話題の変化を観測する手法を提案する.

\section{本論文の構成}
本論文の構成を以下に示す.
\ref{relwork:chapter} 章では要約手法に関する研究と,分散表現に関する先行研究を紹介する.
次に,\ref{model:chapter}章では発言の要約手法の説明を行い,\ref{impl:chapter}章では分散表現を用いた単語集合間の関連度計算について説明する.
そして,\ref{exp:chapter}章では話題転換点の検出の評価実験について説明する.
最後に\ref{con:chapter}章で本論文のまとめと考察を示す.

 %-------------------------------------------------------------------------------
 \expandafter\ifx\csname MasterFile\endcsname\relax
	\def\BibFile{hoge}
	\input{../Bibliography/chapter}
  \fi
  %-------------------------------------------------------------------------------
  \expandafter\ifx\csname MasterFile\endcsname\relax
  \end{document}
  \fi

  \fi
  %-------------------------------------------------------------------------------
  \expandafter\ifx\csname MasterFile\endcsname\relax
  \end{document}
  \fi
 %
%===============================================================================
\end{document}
\fi

  \begin{document}
	\setcounter{chapter}{2}
  \fi
  %-------------------------------------------------------------------------------
  \cleardoublepage
\chapter{通知システム}
%
\label{model:chapter}

\section{序言}
\label{model:introduction}
本章では通知システムの概要について説明する. 以下に本章の構成を示す.まず\ref{model:wholeModel}節でシステム全体の動作の流れを示し, アルゴリズムについても説明を行う. 次に\ref{model:detail}節ではシステムの詳細を説明する.システムで扱う発言データの形式や発言間の類似度計算について述べる. 
最後に,\ref{model:conclusion} 節で本章のまとめを示す.

\section{システムの動作の流れ}
\label{model:wholeModel}
擬似コードを\textbf{Algorithm\ref{algo:wholeModel}}に示し,図示したものを図\ref{Fig:wholeModel}に示す
システムの動作の流れについて説明する.発言$R$が投稿された時,過去に投稿された発言と類似度の計算を行い類似度が閾値を超えていれば2つの発言が同じ話題であるとみなし,発言Rと同じ話題である発言の集合$SG$に登録する.作業を繰り返し全ての発言との計算が終了した後,$SG$が空集合である,すなわち発言Rと同じ話題である発言がない場合に話題を変化させる発言であると判定して通知を行う.
\begin{algorithm}
\caption{システムの流れ} \label{algo:wholeModel}
\begin{algorithmic}[1]
\State $Input:  発言R$
\State $Output: 通知判定Notify$
\State $PG = 過去の発言の集合;$%~~ Group ~ of ~ past ~ remark
\Procedure{topicChange}{$R$} 
	\State $SG = \{\};$%~~ Group ~ of ~ remark ~ whose ~ topic ~ is ~ same ~ with ~ R
	 \For{$Each ~ pastR \in PG$}\label{algo:wholeModel:for1-b}
	 	\State sim = similarity(R,pastR)
		\If{$sim > threshold$}
			\State SG.append(pastR)
		\EndIf
	 \EndFor\label{algo:wholeModel:for1-e}
	 \State Notify = False
	 \If{$SG == \{\}$}
		\State Notify = True
	\EndIf
	\State \textbf{return} Notify
\EndProcedure
\end{algorithmic}
\end{algorithm}

\begin{figure}[htbp]
 \begin{center}
  \includegraphics[width=\textwidth]{../images/3.The_Model/WholeModel.png}
  \caption{システムの流れ}
  \label{Fig:wholeModel}
  \vspace{-10pt}
 \end{center}
\end{figure}

\section{システム詳細}
\label{model:detail}
\subsection{発言データ}
本システムで扱う"発言データ"は単なる文字列ではなく,タイムスタンプ等の他のデータを持つリスト形式のデータである.表\ref{table:remarkData}にデータの一部を示す.
また,以下で本研究で使用する発言データの要素について説明する.
\subsubsection*{\circled{1} id}
発言データを識別するための番号で,全て整数値で表される.
\subsubsection*{\circled{2} title}
スレッドのタイトル名を表す文字列で,発言がスレッドの先頭でない限りはNULLとなる.
\subsubsection*{\circled{3} body}
発言の内容を表す文字列.
\subsubsection*{\circled{4} parent-id}
発言の返信先,すなわち親発言のidを表す番号で,親発言がある場合は親発言のidと同じ番号になり,ない場合はNULLとなる.
\subsubsection*{\circled{5} crreated-at}
発言が投稿された時間を示すタイムスタンプ.

\begin{table}[htbp]
  %\begin{tabular}{| c | p{2cm} | p{1cm} | p{2cm} | p{1cm} | p{2cm} | p{2cm} | c | c |} \hline
  \begin{tabular}{| c | p{4cm} | p{4cm} | c | c | c |} \hline
     %id & title & body & parent-id & user-id & facilitation & created-at & theme-id& has-reply \\ \hline \hline
     id & title & body & parent-id & user-id & $(以降は省略する)$ \\ \hline
     %(ID)&(スレッドタイトル)&(発言内容)&(返信先ID)&(発言者ID)&(ファシリテーター)&(投稿時間)&(議論ID)&(返信)\\ \hline
     18 & オンラインでの議論に関する実験 &  参加者の皆様が集まるまでお待ち下さい。 & NULL & 1 & $(以降は省略する)$ \\
     \hline
  \end{tabular}
  \caption{発言データ} \label{table:remarkData}
\end{table}

\subsection{発言間の類似度計算}
\label{model:simRemarkl}
発言間の類似度計算は次の2段階で行われる.
\subsubsection*{\circled{1} 文章間の類似度計算}
\label{model:simRemark:1}
発言データ中のtitleとbody,すなわち発言の内容の類似度の計算を行う.titleがNULLでない場合はtitleとbodyを改行コードで繋いで1つの文章とする.

文章間の類似度計算の手法については\ref{impl:chapter}章で詳しく述べる.
\subsubsection*{\circled{2} 総合類似度計算}
上記の\circled{1}で計算された発言の文章間の類似度に発言間の時間差と返信関係を組み合わせることで総合類似度を求める.
\paragraph{時間差評価値}\ \\
発言$new$と以前の発言$old$間の時間差を式\ref{eq:timeValue}に基いて正規化された評価値として求める.
\begin{equation}
\begin{aligned}
\label{eq:timeValue}
tValue & = 1 - \frac{ epoch(new.created) - epoch(old.created) }{ maxTime }
\end{aligned}
\end{equation}
ここで関数$epoch$,定数$maxTime$について説明する.$epoch$は与えられたタイムスタンプをエポック秒に変換する.$maxTime$は議論の制限時間を表す.%$x.created$は発言$x$が投稿された時間を表す.
時間差評価値は2発言間の時間差が小さいほど関連が強いとみなし,0から1に近づく.

\paragraph{返信距離}\ \\
発言$new$と以前の発言$old$間の返信距離をAlgorithm\ref{algo:replyDist}に基いて再帰的に求める.
\begin{algorithm}
\caption{返信距離} \label{algo:replyDist}
\begin{algorithmic}[1]
\State $Input:  発言new,発言old,返信距離dist$ \Comment{初期値$dist$=1}
\State $Output: 返信距離dist$
\State $PG = IDに対応づけられた過去の発言の集合;$%~~ Group ~ of ~ past ~ remark
\Procedure{replyDist}{$new,old,dist$} 
	\If{$new$.parent-id == NULL}\label{replyDist:if1-b}
		\State \textbf{return} 0\label{replyDist:if1-e}
	\ElsIf{$new$.parent-id==$old$.id}\label{replyDist:if2-b}
		\State \textbf{return} $dist$\label{replyDist:if2-e}
	 \Else\label{replyDist:if3-b}
	 	\State $parent$ = PG[$new$.parent-id]
		\State $dist+=1$
		\State \textbf{return} REPLYDIST($parent,old,dist$)\label{replyDist:if3-e}
	\EndIf
\EndProcedure
\end{algorithmic}
\end{algorithm}

\ref{replyDist:if2-b} $\sim$ \ref{replyDist:if2-e}行目 , \ref{replyDist:if3-b} $\sim$ \ref{replyDist:if3-e}行目で示すように発言のidが一致した場合は現在の返信距離を返し,一致しなかった場合は返信距離を1増やして$new$の親発言$parent$と$old$の返信関係を再帰呼び出しで求め,返り値を返す.
また,\ref{replyDist:if1-b} $\sim$ \ref{replyDist:if1-e} 行目で示すように2発言間が返信関係になかった場合は0を返す.

\paragraph{総合類似度}\ \\
総合類似度は前述の発言内容の類似度,時間差評価値,返信距離によって求められる.
返信距離が0である時,すなわち2発言が異なるスレッドに属している場合は類似度と時間差評価値から総合類似度を計算する.
類似度だけでなく時間差評価値を使用するのは,総合類似度だけで判断してしまうと議論の終盤になって発言数が多くなってきた時に新しく投稿された発言が多くの古い発言と類似していると判断されてしまうことがあり得るからである.議論は基本的に少し前の発言に関連して行われることが多いことから時間差評価値を使用して時間的に近しいものほど総合類似度が上昇するようにする.具体的には式\ref{eq:tSimilarity}のように計算される.
\begin{equation}
\begin{aligned}
\label{eq:tSimilarity}
tSim & = tValue*tWeight + sim*(1-tWeight)
\end{aligned}
\end{equation}
ここで変数$sim$,定数$tWeight$について説明する.$sim$は発言内容の類似度を表し,0から1の値を取る.$tWeight$は時間差評価値の総合類似度の計算における重要性を表し,0から1の値を取る.
また,返信距離が0でない,すなわち2発言が同じスレッドに属している場合は何らかの関連があると考えられることから,発言内容の類似度を総合類似度とする.

\section{結言}
\label{model:conclusion}
本章では通知システムの動作の流れやアルゴリズムについて説明し,扱うデータの形式や内容,及び発言内容の類似度を除く発言間の類似度計算の手法について説明した.また,時間的に近しい議論ほど類似度が上昇するように時間差評価値と発言内容の類似度の2つを用いて総合類似度計算することを示した.

 %-------------------------------------------------------------------------------
 \expandafter\ifx\csname MasterFile\endcsname\relax
	\def\BibFile{hoge}
	\expandafter\ifx\csname MasterFile\endcsname\relax
	\def\SubFile{hoge}
	\documentclass[a4j,12pt,twoside,openany]{jreport}
%\nofiles %tocファイルを更新させない
%\documentclass[12pt,a4j,twoside,openany]{jsbook}
\usepackage[dvipdfmx]{graphicx}
\usepackage{../dspc} % ベースラインスキップの指定
\usepackage{../slashbox} % 表に斜線を入れる
%\usepackage{../mediabb}
\usepackage{fancyvrb} % Verbatim環境
\usepackage{fancyhdr} % Headerの下線付き章見出し
\usepackage{here} % float[H]
\usepackage{multirow}
\usepackage{hhline} % 表の罫線の角を美しくする
\usepackage{amsmath} %コレがないとcasesが動かない
\usepackage{amsfonts} % 数学用フォント
\usepackage{bm} % 数式環境での bold
\usepackage{algorithm}
\usepackage{algorithmicx}
\usepackage[noend]{algpseudocode}
\usepackage[flushleft]{threeparttable} % 脚注付きテーブル
\usepackage{enumitem}
\usepackage{comment}
\usepackage{fancybox}
%\usepackage{csvsimple,booktabs,siunitx}
%\usepackage{filecontents}


\setlength{\evensidemargin}{5pt}
\setlength{\oddsidemargin}{40pt}
%\setlength{\headheight}{16.5pt}
%%\setlength{\headheight}{30pt}
\setcounter{secnumdepth}{3}
\setlist[description]{leftmargin=2\parindent,labelindent=\parindent}

\makeatletter
\def\@makechapterhead#1{%
	\vspace*{50\p@}%
	{
		\parindent \z@ \raggedright \normalfont
		\ifnum \c@secnumdepth >\m@ne
		% \if@mainmatter
			\huge\bfseries\@chapapp\thechapter\@chappos
			\par\nobreak
			\vskip 20\p@
		% \fi
		\fi
		\interlinepenalty\@M
		\Huge\bfseries #1\par\nobreak
		\vskip 40\p@
	}
}

%新しいコマンド定義
\newcounter{linenumber}
\newenvironment{listing}{%
  \begin{list}{%
    \small\arabic{linenumber}:}{%
      \usecounter{linenumber}%
      \setlength{\baselineskip}{18pt}%
      \setlength{\itemsep}{0pt}%
      \setlength{\parsep}{0pt}}}%
 {\end{list}}
\newcommand{\figcaption}[1]{\def\@captype{figure}\caption{#1}}
\newcommand{\tblcaption}[1]{\def\@captype{table}\caption{#1}}
\newcommand{\norm}[1]{\left\| #1 \right\|}
\newcommand{\cc}[1]{\multicolumn{1}{|c|}{#1}}
\newcommand{\circled}[1]{\raisebox{.5pt}{\textcircled{\raisebox{-.9pt} {#1}}}}
\newcommand{\specialcell}[2][c]{%
  \begin{tabular}[#1]{@{}c@{}}#2\end{tabular}}
\makeatother
%===============================================================================
\expandafter\ifx\csname SubFile\endcsname\relax
\begin{document}
\def\MasterFile{hoge}
%-------------------------------------------------------------------------------
%\maketitle
\thispagestyle{empty}
\begin{titlepage}

% 題名
\def\title{分散表現を用いた\\話題変化判定}
% 補助題名
\def\subtitle{卒業論文}
% 著者
\def\author{芳野 魁}
% 入学年度(平成)
\def\year{29}
% 学籍番号
\def\number{26115162}
% 指導教官
\def\kyoukan{伊藤 孝行}
% 指導教官役職
\def\kyoukanrank{教授}
% 提出日
\def\teisyutubi{平成29年12月28日}

\pagestyle{empty}

\begin{center}

\vspace*{20mm}
{\Large\mc 平成29年度 \hspace{7mm} 卒 業 論 文}
\vspace{15mm}

%\setlength{\unitlength}{1mm}
\begin{picture}(100,60)
  \put(0,0){\makebox(100,60){\huge\bf\shortstack{\title}}}
\end{picture}
\\
%\begin{picture}(100,5)
%  \put(0,0){\makebox(100,5){\Large\bf\shortstack{\subtitle}}}
%\end{picture}
\end{center}
\vspace{10mm}
\begin{flushright}
\begin{tabular}{ll}
{\large 提出日} & {\large {\teisyutubi}} \\
{\large 所属}  & {\large 名古屋工業大学 情報工学科} \\
{\large 指導教員} & {\large {\kyoukan} {\kyoukanrank}} \\
 & \\
{\large 入学年度} & {\large 平成26年度入学}\\
{\large 学籍番号} &{\large {\number}} \\
 & \\
%{\large 氏名} & {\huge {\author}}
{\large 氏名} & {\huge\mc {\author}}
\end{tabular}
\end{flushright}

\end{titlepage}

%\addcontentsline{toc}{chapter}{表紙}
\thispagestyle{empty}
\mbox{}\newpage
%===============================================================================
%\frontmatter
%===============================================================================
%\mainmatter
%-------------------------------------------------------------------------------
\pagenumbering{arabic}
\cleardoublepage
\expandafter\ifx\csname MasterFile\endcsname\relax
	\def\SubFile{hoge}
	\input{../thesis/thesis}
	\begin{document}
	\setcounter{chapter}{0}
	\fi
  %-------------------------------------------------------------------------------
\cleardoublepage
\chapter{序論}
\label{intro:chapter}
%本章では, 本研究を行なうに至った背景と目的について述べる.その後,本論文の構成について述べる.
\section{研究の背景}
\label{intro:background}
近年,Web上での大規模な議論活動が活発になっているが,現在一般的に使われている "2ちゃんねる" や "Twitter" といったシステムでは整理や収束を行うことが困難である.困難である原因として,議論の管理を行う者がいないことが挙げられる.
つまり,議論を整理・収束させるには議論のマネジメントを行う人物が必要である.
%
大規模意見集約システムCOLLAGREE\cite{collagreeTest}ではファシリテーターと呼ばれる人物が議論のマネジメントを行っている.
しかし,ファシリテーターは人間であり,長時間に渡って大人数での議論の動向をマネジメントし続けるのは困難である.
COLLAGREEで大規模な議論を収束させるためには,ファシリテーターが必要な時には画面を見るようにして,他の時は見なくても済むようにすることで画面に向き合う時間を減らす工夫があることが望ましい.ファシリテーターが画面を見るべきタイミングは議論の話題が変化したときである.以前の議論の内容から外れた発言がされた時,ファシリテーターが適切な発言をすることで,脱線や炎上を避けて議論を収束させることができる.
すなわち,ファシリテーターの代わりに自動的に議論中の話題の変化を観測することが求められている.
%
現在,COLLAGREE上で使用されている議論支援システムは「(1)投稿支援システム」と「(2)議論可視化システム」の2つに大別できる.
投稿支援システムはポイント機能やファシリテーションフレーズ簡易投稿機能のように,ユーザーが投稿をする際に何らかの補助やリアクションを行う.現行の機能では選択肢の提示に留まっており,作業量を減らすことには繋がりにくい.
一方,議論可視化システムは議論ツリーやキーワード抽出のように,ユーザーにスレッドとは異なる議論の見方を提供する.
\ref{Fig:argTree1}に議論ツリーの例を示す.
\begin{figure}[htbp]
 \begin{center}
  \includegraphics[width=\textwidth]{../images/2.Related_Work/argTree1.png}
  \caption{議論ツリー}
  \label{Fig:argTree1}
  \vspace{-10pt}
 \end{center}
\end{figure}
現行の機能では議論を見やすくすることに重点が置かれており,議論の把握の助けにはなるが画面に向き合う時間を減らすことにはなりにくい.むしろ,作業量を増やすことになり得ることもある.
従って,現行の支援機能ではファシリテーターの作業量の減少には繋がりにくい.
%
近年,自然言語処理の分野において分散表現が多くの研究で使われており,機械翻訳を始めとする単語の意味が重要となる分野で精度の向上が確認されている.分散表現を用いることで,人間に近い精度で話題の変化を観測することが可能となる.
%
以上のような背景を踏まえて,分散表現を用いて,話題の変化を観測し,話題の変化が確認された時にファシリテーターに伝えることが望ましい.
話題の変化の観測は,発言中に現れる単語の類似度の計算と見なすことができる.
分散表現を用いることで単語間の類似度を求めることができる,値が大きいほど単語がそれぞれ類似した実数ベクトルであることを表す.単語Aと単語Bの実数ベクトルが類似しているとは,単語Aと共に使われることの多い単語と単語Bと共に使われることの多い単語が多く共通していることを示す.故に,分散表現を使って単語の類似度を計算することができる.
%
発言文から単語を選ぶ際には自動要約を用いる.発言文から重要でない単語を取り除くことで関連度の計算の精度を高めることが可能となる.
要約の手法としてはokapi BM25 \cite{okapiBM25}とLexRankを組み合わせた抽出的要約手法を用いる.
\begin{comment}
%======================================= 社会的背景
2013年頃からWeb上での大規模な議論活動が活発になり,大規模な人数での議論が期待されている.
大規模な議論では意見を共有することは可能であるが,議論を整理させることや収束させることは難しい.以上から大規模意見集約システムCOLLAGREEが開発された.本システムではWeb上で適切に大規模な議論を行うことができるように議論をマネジメントするファシリテーターを導入した\cite{collagreeTest}.
過去の実験ではファシリテーターの存在が議論の集約に大きな役割を果たしていることが認識されており,大規模な議論のためにファシリテータは必要である.しかし,議論の規模に伴って議論時間が長くなる傾向があり,同時にファシリテーターは常に議論の動向を見続ける必要がある.故に,議論の規模が大きくなればなるほどファシリテーターは長時間かつ大規模な議論の動向の監視によって大きな負担がかかる.大規模な議論が増加する傾向を踏まえるとファシリテーターにかかる負担を軽減する支援が必要である.\\
以上の問題を解決するため,話題の変化を追い,重要な話題の転換点をファシリテーターの代わりに検出することが有用であると考える.必要な時にだけファシリテーターが画面を見れば良いようにすることでファシリテーターの負担軽減が期待できる.
%========================================= 現行手法問題点背景
%議論支援に関する先行研究において,既存の手法は全てが文字列を文字列のまま扱う手法である.
%既存手法は殆どがパターンマッチングと重み付けの2つに区分することができる.
%パターンマッチングでは事前に単語を登録して,単語がマッチした場合に処理を行うが,処理それぞれに対して単語を登録しなければならず手間が膨大になってしまう.また,単語の意味が考慮されておらず,手作業で登録を行うので登録漏れがあった場合に単語の意味に関係なく処理を行うことが不可能となってしまう.
%重み付けは単語の出現頻度や文章の長さを使用して単語・文章に順位を付ける手法で必ずしも単語の登録が必要でないため多くの研究で使用されている.
%しかし,重み付けもまた単語を文字列のまま扱っており,意味までは考慮されていない.故に ,人間なら対応できる似た単語でも1文字違うだけで対処が困難となる.
議論支援に関する先行研究においてファシリテーターに対する支援を目的としたものは無く,殆どが議論の活性化や可視化を目的としている.
%=================================新手法
近年,自然言語処理の分野において分散表現が多くの研究で使われており.分散表現は文字列である単語を辞書データを使用して実数ベクトルへと変換する.辞書データにない単語には対応できないが,多様な処理を1つの辞書データで行うことができる.また,実数ベクトルの各数値が単語の意味を表現するものとなっており,数値を使用して処理を行うことができる.
分散表現を用いることで既存手法より人間の感覚に近しい処理を行うことができる.
%=================================
以上のような背景を踏まえて,分散表現を用いてファシリテーターの代わりに話題の変化を判定し,知らせることを目指す.
話題転換の検出は発言同士の近さ,すなわち発言に含まれる単語意味の近さと見ることができる.
分散表現ではベクトル同士の内積計算を行うことで単語同士の意味の近さを計算することができる.
また,分散表現を使用することで機械翻訳を始めとする複数の分野で精度の向上が確認されている.
\end{comment}
\section{研究の目的}
\label{intro:taget}
本論文では,分散表現を用いて議論中での発言に含まれる単語の関連度を計算し,話題の変化を観測する手法を提案する.

\section{本論文の構成}
本論文の構成を以下に示す.
\ref{relwork:chapter} 章では要約手法に関する研究と,分散表現に関する先行研究を紹介する.
次に,\ref{model:chapter}章では発言の要約手法の説明を行い,\ref{impl:chapter}章では分散表現を用いた単語集合間の関連度計算について説明する.
そして,\ref{exp:chapter}章では話題転換点の検出の評価実験について説明する.
最後に\ref{con:chapter}章で本論文のまとめと考察を示す.

 %-------------------------------------------------------------------------------
 \expandafter\ifx\csname MasterFile\endcsname\relax
	\def\BibFile{hoge}
	\input{../Bibliography/chapter}
  \fi
  %-------------------------------------------------------------------------------
  \expandafter\ifx\csname MasterFile\endcsname\relax
  \end{document}
  \fi

%-------------------------------------------------------------------------------
\clearpage
\addcontentsline{toc}{chapter}{目次}
\tableofcontents

\clearpage
\addcontentsline{toc}{chapter}{図目次}
\listoffigures

\clearpage
\addcontentsline{toc}{chapter}{表目次}
\listoftables

%-------------------------------------------------------------------------------

%=====================
\pagestyle{fancy} % Headerをつける
\renewcommand{\sectionmark}[1]{\markright{\thesection\ \ \ #1}}
\renewcommand{\chaptermark}[1]{\markboth{#1}{}}
\lhead{}
\chead{}
\lfoot{}
\rfoot{}%-------------------------------------------------------------------------------
\expandafter\ifx\csname MasterFile\endcsname\relax
	\def\SubFile{hoge}
	\input{../thesis/thesis}
	\begin{document}
	\setcounter{chapter}{0}
	\fi
  %-------------------------------------------------------------------------------
\cleardoublepage
\chapter{序論}
\label{intro:chapter}
%本章では, 本研究を行なうに至った背景と目的について述べる.その後,本論文の構成について述べる.
\section{研究の背景}
\label{intro:background}
近年,Web上での大規模な議論活動が活発になっているが,現在一般的に使われている "2ちゃんねる" や "Twitter" といったシステムでは整理や収束を行うことが困難である.困難である原因として,議論の管理を行う者がいないことが挙げられる.
つまり,議論を整理・収束させるには議論のマネジメントを行う人物が必要である.
%
大規模意見集約システムCOLLAGREE\cite{collagreeTest}ではファシリテーターと呼ばれる人物が議論のマネジメントを行っている.
しかし,ファシリテーターは人間であり,長時間に渡って大人数での議論の動向をマネジメントし続けるのは困難である.
COLLAGREEで大規模な議論を収束させるためには,ファシリテーターが必要な時には画面を見るようにして,他の時は見なくても済むようにすることで画面に向き合う時間を減らす工夫があることが望ましい.ファシリテーターが画面を見るべきタイミングは議論の話題が変化したときである.以前の議論の内容から外れた発言がされた時,ファシリテーターが適切な発言をすることで,脱線や炎上を避けて議論を収束させることができる.
すなわち,ファシリテーターの代わりに自動的に議論中の話題の変化を観測することが求められている.
%
現在,COLLAGREE上で使用されている議論支援システムは「(1)投稿支援システム」と「(2)議論可視化システム」の2つに大別できる.
投稿支援システムはポイント機能やファシリテーションフレーズ簡易投稿機能のように,ユーザーが投稿をする際に何らかの補助やリアクションを行う.現行の機能では選択肢の提示に留まっており,作業量を減らすことには繋がりにくい.
一方,議論可視化システムは議論ツリーやキーワード抽出のように,ユーザーにスレッドとは異なる議論の見方を提供する.
\ref{Fig:argTree1}に議論ツリーの例を示す.
\begin{figure}[htbp]
 \begin{center}
  \includegraphics[width=\textwidth]{../images/2.Related_Work/argTree1.png}
  \caption{議論ツリー}
  \label{Fig:argTree1}
  \vspace{-10pt}
 \end{center}
\end{figure}
現行の機能では議論を見やすくすることに重点が置かれており,議論の把握の助けにはなるが画面に向き合う時間を減らすことにはなりにくい.むしろ,作業量を増やすことになり得ることもある.
従って,現行の支援機能ではファシリテーターの作業量の減少には繋がりにくい.
%
近年,自然言語処理の分野において分散表現が多くの研究で使われており,機械翻訳を始めとする単語の意味が重要となる分野で精度の向上が確認されている.分散表現を用いることで,人間に近い精度で話題の変化を観測することが可能となる.
%
以上のような背景を踏まえて,分散表現を用いて,話題の変化を観測し,話題の変化が確認された時にファシリテーターに伝えることが望ましい.
話題の変化の観測は,発言中に現れる単語の類似度の計算と見なすことができる.
分散表現を用いることで単語間の類似度を求めることができる,値が大きいほど単語がそれぞれ類似した実数ベクトルであることを表す.単語Aと単語Bの実数ベクトルが類似しているとは,単語Aと共に使われることの多い単語と単語Bと共に使われることの多い単語が多く共通していることを示す.故に,分散表現を使って単語の類似度を計算することができる.
%
発言文から単語を選ぶ際には自動要約を用いる.発言文から重要でない単語を取り除くことで関連度の計算の精度を高めることが可能となる.
要約の手法としてはokapi BM25 \cite{okapiBM25}とLexRankを組み合わせた抽出的要約手法を用いる.
\begin{comment}
%======================================= 社会的背景
2013年頃からWeb上での大規模な議論活動が活発になり,大規模な人数での議論が期待されている.
大規模な議論では意見を共有することは可能であるが,議論を整理させることや収束させることは難しい.以上から大規模意見集約システムCOLLAGREEが開発された.本システムではWeb上で適切に大規模な議論を行うことができるように議論をマネジメントするファシリテーターを導入した\cite{collagreeTest}.
過去の実験ではファシリテーターの存在が議論の集約に大きな役割を果たしていることが認識されており,大規模な議論のためにファシリテータは必要である.しかし,議論の規模に伴って議論時間が長くなる傾向があり,同時にファシリテーターは常に議論の動向を見続ける必要がある.故に,議論の規模が大きくなればなるほどファシリテーターは長時間かつ大規模な議論の動向の監視によって大きな負担がかかる.大規模な議論が増加する傾向を踏まえるとファシリテーターにかかる負担を軽減する支援が必要である.\\
以上の問題を解決するため,話題の変化を追い,重要な話題の転換点をファシリテーターの代わりに検出することが有用であると考える.必要な時にだけファシリテーターが画面を見れば良いようにすることでファシリテーターの負担軽減が期待できる.
%========================================= 現行手法問題点背景
%議論支援に関する先行研究において,既存の手法は全てが文字列を文字列のまま扱う手法である.
%既存手法は殆どがパターンマッチングと重み付けの2つに区分することができる.
%パターンマッチングでは事前に単語を登録して,単語がマッチした場合に処理を行うが,処理それぞれに対して単語を登録しなければならず手間が膨大になってしまう.また,単語の意味が考慮されておらず,手作業で登録を行うので登録漏れがあった場合に単語の意味に関係なく処理を行うことが不可能となってしまう.
%重み付けは単語の出現頻度や文章の長さを使用して単語・文章に順位を付ける手法で必ずしも単語の登録が必要でないため多くの研究で使用されている.
%しかし,重み付けもまた単語を文字列のまま扱っており,意味までは考慮されていない.故に ,人間なら対応できる似た単語でも1文字違うだけで対処が困難となる.
議論支援に関する先行研究においてファシリテーターに対する支援を目的としたものは無く,殆どが議論の活性化や可視化を目的としている.
%=================================新手法
近年,自然言語処理の分野において分散表現が多くの研究で使われており.分散表現は文字列である単語を辞書データを使用して実数ベクトルへと変換する.辞書データにない単語には対応できないが,多様な処理を1つの辞書データで行うことができる.また,実数ベクトルの各数値が単語の意味を表現するものとなっており,数値を使用して処理を行うことができる.
分散表現を用いることで既存手法より人間の感覚に近しい処理を行うことができる.
%=================================
以上のような背景を踏まえて,分散表現を用いてファシリテーターの代わりに話題の変化を判定し,知らせることを目指す.
話題転換の検出は発言同士の近さ,すなわち発言に含まれる単語意味の近さと見ることができる.
分散表現ではベクトル同士の内積計算を行うことで単語同士の意味の近さを計算することができる.
また,分散表現を使用することで機械翻訳を始めとする複数の分野で精度の向上が確認されている.
\end{comment}
\section{研究の目的}
\label{intro:taget}
本論文では,分散表現を用いて議論中での発言に含まれる単語の関連度を計算し,話題の変化を観測する手法を提案する.

\section{本論文の構成}
本論文の構成を以下に示す.
\ref{relwork:chapter} 章では要約手法に関する研究と,分散表現に関する先行研究を紹介する.
次に,\ref{model:chapter}章では発言の要約手法の説明を行い,\ref{impl:chapter}章では分散表現を用いた単語集合間の関連度計算について説明する.
そして,\ref{exp:chapter}章では話題転換点の検出の評価実験について説明する.
最後に\ref{con:chapter}章で本論文のまとめと考察を示す.

 %-------------------------------------------------------------------------------
 \expandafter\ifx\csname MasterFile\endcsname\relax
	\def\BibFile{hoge}
	\input{../Bibliography/chapter}
  \fi
  %-------------------------------------------------------------------------------
  \expandafter\ifx\csname MasterFile\endcsname\relax
  \end{document}
  \fi

%-------------------------------------------------------------------------------
\expandafter\ifx\csname MasterFile\endcsname\relax
	\def\SubFile{hoge}
	\input{../thesis/thesis}
	\begin{document}
	\setcounter{chapter}{0}
	\fi
  %-------------------------------------------------------------------------------
\cleardoublepage
\chapter{序論}
\label{intro:chapter}
%本章では, 本研究を行なうに至った背景と目的について述べる.その後,本論文の構成について述べる.
\section{研究の背景}
\label{intro:background}
近年,Web上での大規模な議論活動が活発になっているが,現在一般的に使われている "2ちゃんねる" や "Twitter" といったシステムでは整理や収束を行うことが困難である.困難である原因として,議論の管理を行う者がいないことが挙げられる.
つまり,議論を整理・収束させるには議論のマネジメントを行う人物が必要である.
%
大規模意見集約システムCOLLAGREE\cite{collagreeTest}ではファシリテーターと呼ばれる人物が議論のマネジメントを行っている.
しかし,ファシリテーターは人間であり,長時間に渡って大人数での議論の動向をマネジメントし続けるのは困難である.
COLLAGREEで大規模な議論を収束させるためには,ファシリテーターが必要な時には画面を見るようにして,他の時は見なくても済むようにすることで画面に向き合う時間を減らす工夫があることが望ましい.ファシリテーターが画面を見るべきタイミングは議論の話題が変化したときである.以前の議論の内容から外れた発言がされた時,ファシリテーターが適切な発言をすることで,脱線や炎上を避けて議論を収束させることができる.
すなわち,ファシリテーターの代わりに自動的に議論中の話題の変化を観測することが求められている.
%
現在,COLLAGREE上で使用されている議論支援システムは「(1)投稿支援システム」と「(2)議論可視化システム」の2つに大別できる.
投稿支援システムはポイント機能やファシリテーションフレーズ簡易投稿機能のように,ユーザーが投稿をする際に何らかの補助やリアクションを行う.現行の機能では選択肢の提示に留まっており,作業量を減らすことには繋がりにくい.
一方,議論可視化システムは議論ツリーやキーワード抽出のように,ユーザーにスレッドとは異なる議論の見方を提供する.
\ref{Fig:argTree1}に議論ツリーの例を示す.
\begin{figure}[htbp]
 \begin{center}
  \includegraphics[width=\textwidth]{../images/2.Related_Work/argTree1.png}
  \caption{議論ツリー}
  \label{Fig:argTree1}
  \vspace{-10pt}
 \end{center}
\end{figure}
現行の機能では議論を見やすくすることに重点が置かれており,議論の把握の助けにはなるが画面に向き合う時間を減らすことにはなりにくい.むしろ,作業量を増やすことになり得ることもある.
従って,現行の支援機能ではファシリテーターの作業量の減少には繋がりにくい.
%
近年,自然言語処理の分野において分散表現が多くの研究で使われており,機械翻訳を始めとする単語の意味が重要となる分野で精度の向上が確認されている.分散表現を用いることで,人間に近い精度で話題の変化を観測することが可能となる.
%
以上のような背景を踏まえて,分散表現を用いて,話題の変化を観測し,話題の変化が確認された時にファシリテーターに伝えることが望ましい.
話題の変化の観測は,発言中に現れる単語の類似度の計算と見なすことができる.
分散表現を用いることで単語間の類似度を求めることができる,値が大きいほど単語がそれぞれ類似した実数ベクトルであることを表す.単語Aと単語Bの実数ベクトルが類似しているとは,単語Aと共に使われることの多い単語と単語Bと共に使われることの多い単語が多く共通していることを示す.故に,分散表現を使って単語の類似度を計算することができる.
%
発言文から単語を選ぶ際には自動要約を用いる.発言文から重要でない単語を取り除くことで関連度の計算の精度を高めることが可能となる.
要約の手法としてはokapi BM25 \cite{okapiBM25}とLexRankを組み合わせた抽出的要約手法を用いる.
\begin{comment}
%======================================= 社会的背景
2013年頃からWeb上での大規模な議論活動が活発になり,大規模な人数での議論が期待されている.
大規模な議論では意見を共有することは可能であるが,議論を整理させることや収束させることは難しい.以上から大規模意見集約システムCOLLAGREEが開発された.本システムではWeb上で適切に大規模な議論を行うことができるように議論をマネジメントするファシリテーターを導入した\cite{collagreeTest}.
過去の実験ではファシリテーターの存在が議論の集約に大きな役割を果たしていることが認識されており,大規模な議論のためにファシリテータは必要である.しかし,議論の規模に伴って議論時間が長くなる傾向があり,同時にファシリテーターは常に議論の動向を見続ける必要がある.故に,議論の規模が大きくなればなるほどファシリテーターは長時間かつ大規模な議論の動向の監視によって大きな負担がかかる.大規模な議論が増加する傾向を踏まえるとファシリテーターにかかる負担を軽減する支援が必要である.\\
以上の問題を解決するため,話題の変化を追い,重要な話題の転換点をファシリテーターの代わりに検出することが有用であると考える.必要な時にだけファシリテーターが画面を見れば良いようにすることでファシリテーターの負担軽減が期待できる.
%========================================= 現行手法問題点背景
%議論支援に関する先行研究において,既存の手法は全てが文字列を文字列のまま扱う手法である.
%既存手法は殆どがパターンマッチングと重み付けの2つに区分することができる.
%パターンマッチングでは事前に単語を登録して,単語がマッチした場合に処理を行うが,処理それぞれに対して単語を登録しなければならず手間が膨大になってしまう.また,単語の意味が考慮されておらず,手作業で登録を行うので登録漏れがあった場合に単語の意味に関係なく処理を行うことが不可能となってしまう.
%重み付けは単語の出現頻度や文章の長さを使用して単語・文章に順位を付ける手法で必ずしも単語の登録が必要でないため多くの研究で使用されている.
%しかし,重み付けもまた単語を文字列のまま扱っており,意味までは考慮されていない.故に ,人間なら対応できる似た単語でも1文字違うだけで対処が困難となる.
議論支援に関する先行研究においてファシリテーターに対する支援を目的としたものは無く,殆どが議論の活性化や可視化を目的としている.
%=================================新手法
近年,自然言語処理の分野において分散表現が多くの研究で使われており.分散表現は文字列である単語を辞書データを使用して実数ベクトルへと変換する.辞書データにない単語には対応できないが,多様な処理を1つの辞書データで行うことができる.また,実数ベクトルの各数値が単語の意味を表現するものとなっており,数値を使用して処理を行うことができる.
分散表現を用いることで既存手法より人間の感覚に近しい処理を行うことができる.
%=================================
以上のような背景を踏まえて,分散表現を用いてファシリテーターの代わりに話題の変化を判定し,知らせることを目指す.
話題転換の検出は発言同士の近さ,すなわち発言に含まれる単語意味の近さと見ることができる.
分散表現ではベクトル同士の内積計算を行うことで単語同士の意味の近さを計算することができる.
また,分散表現を使用することで機械翻訳を始めとする複数の分野で精度の向上が確認されている.
\end{comment}
\section{研究の目的}
\label{intro:taget}
本論文では,分散表現を用いて議論中での発言に含まれる単語の関連度を計算し,話題の変化を観測する手法を提案する.

\section{本論文の構成}
本論文の構成を以下に示す.
\ref{relwork:chapter} 章では要約手法に関する研究と,分散表現に関する先行研究を紹介する.
次に,\ref{model:chapter}章では発言の要約手法の説明を行い,\ref{impl:chapter}章では分散表現を用いた単語集合間の関連度計算について説明する.
そして,\ref{exp:chapter}章では話題転換点の検出の評価実験について説明する.
最後に\ref{con:chapter}章で本論文のまとめと考察を示す.

 %-------------------------------------------------------------------------------
 \expandafter\ifx\csname MasterFile\endcsname\relax
	\def\BibFile{hoge}
	\input{../Bibliography/chapter}
  \fi
  %-------------------------------------------------------------------------------
  \expandafter\ifx\csname MasterFile\endcsname\relax
  \end{document}
  \fi

%-------------------------------------------------------------------------------
\expandafter\ifx\csname MasterFile\endcsname\relax
	\def\SubFile{hoge}
	\input{../thesis/thesis}
	\begin{document}
	\setcounter{chapter}{0}
	\fi
  %-------------------------------------------------------------------------------
\cleardoublepage
\chapter{序論}
\label{intro:chapter}
%本章では, 本研究を行なうに至った背景と目的について述べる.その後,本論文の構成について述べる.
\section{研究の背景}
\label{intro:background}
近年,Web上での大規模な議論活動が活発になっているが,現在一般的に使われている "2ちゃんねる" や "Twitter" といったシステムでは整理や収束を行うことが困難である.困難である原因として,議論の管理を行う者がいないことが挙げられる.
つまり,議論を整理・収束させるには議論のマネジメントを行う人物が必要である.
%
大規模意見集約システムCOLLAGREE\cite{collagreeTest}ではファシリテーターと呼ばれる人物が議論のマネジメントを行っている.
しかし,ファシリテーターは人間であり,長時間に渡って大人数での議論の動向をマネジメントし続けるのは困難である.
COLLAGREEで大規模な議論を収束させるためには,ファシリテーターが必要な時には画面を見るようにして,他の時は見なくても済むようにすることで画面に向き合う時間を減らす工夫があることが望ましい.ファシリテーターが画面を見るべきタイミングは議論の話題が変化したときである.以前の議論の内容から外れた発言がされた時,ファシリテーターが適切な発言をすることで,脱線や炎上を避けて議論を収束させることができる.
すなわち,ファシリテーターの代わりに自動的に議論中の話題の変化を観測することが求められている.
%
現在,COLLAGREE上で使用されている議論支援システムは「(1)投稿支援システム」と「(2)議論可視化システム」の2つに大別できる.
投稿支援システムはポイント機能やファシリテーションフレーズ簡易投稿機能のように,ユーザーが投稿をする際に何らかの補助やリアクションを行う.現行の機能では選択肢の提示に留まっており,作業量を減らすことには繋がりにくい.
一方,議論可視化システムは議論ツリーやキーワード抽出のように,ユーザーにスレッドとは異なる議論の見方を提供する.
\ref{Fig:argTree1}に議論ツリーの例を示す.
\begin{figure}[htbp]
 \begin{center}
  \includegraphics[width=\textwidth]{../images/2.Related_Work/argTree1.png}
  \caption{議論ツリー}
  \label{Fig:argTree1}
  \vspace{-10pt}
 \end{center}
\end{figure}
現行の機能では議論を見やすくすることに重点が置かれており,議論の把握の助けにはなるが画面に向き合う時間を減らすことにはなりにくい.むしろ,作業量を増やすことになり得ることもある.
従って,現行の支援機能ではファシリテーターの作業量の減少には繋がりにくい.
%
近年,自然言語処理の分野において分散表現が多くの研究で使われており,機械翻訳を始めとする単語の意味が重要となる分野で精度の向上が確認されている.分散表現を用いることで,人間に近い精度で話題の変化を観測することが可能となる.
%
以上のような背景を踏まえて,分散表現を用いて,話題の変化を観測し,話題の変化が確認された時にファシリテーターに伝えることが望ましい.
話題の変化の観測は,発言中に現れる単語の類似度の計算と見なすことができる.
分散表現を用いることで単語間の類似度を求めることができる,値が大きいほど単語がそれぞれ類似した実数ベクトルであることを表す.単語Aと単語Bの実数ベクトルが類似しているとは,単語Aと共に使われることの多い単語と単語Bと共に使われることの多い単語が多く共通していることを示す.故に,分散表現を使って単語の類似度を計算することができる.
%
発言文から単語を選ぶ際には自動要約を用いる.発言文から重要でない単語を取り除くことで関連度の計算の精度を高めることが可能となる.
要約の手法としてはokapi BM25 \cite{okapiBM25}とLexRankを組み合わせた抽出的要約手法を用いる.
\begin{comment}
%======================================= 社会的背景
2013年頃からWeb上での大規模な議論活動が活発になり,大規模な人数での議論が期待されている.
大規模な議論では意見を共有することは可能であるが,議論を整理させることや収束させることは難しい.以上から大規模意見集約システムCOLLAGREEが開発された.本システムではWeb上で適切に大規模な議論を行うことができるように議論をマネジメントするファシリテーターを導入した\cite{collagreeTest}.
過去の実験ではファシリテーターの存在が議論の集約に大きな役割を果たしていることが認識されており,大規模な議論のためにファシリテータは必要である.しかし,議論の規模に伴って議論時間が長くなる傾向があり,同時にファシリテーターは常に議論の動向を見続ける必要がある.故に,議論の規模が大きくなればなるほどファシリテーターは長時間かつ大規模な議論の動向の監視によって大きな負担がかかる.大規模な議論が増加する傾向を踏まえるとファシリテーターにかかる負担を軽減する支援が必要である.\\
以上の問題を解決するため,話題の変化を追い,重要な話題の転換点をファシリテーターの代わりに検出することが有用であると考える.必要な時にだけファシリテーターが画面を見れば良いようにすることでファシリテーターの負担軽減が期待できる.
%========================================= 現行手法問題点背景
%議論支援に関する先行研究において,既存の手法は全てが文字列を文字列のまま扱う手法である.
%既存手法は殆どがパターンマッチングと重み付けの2つに区分することができる.
%パターンマッチングでは事前に単語を登録して,単語がマッチした場合に処理を行うが,処理それぞれに対して単語を登録しなければならず手間が膨大になってしまう.また,単語の意味が考慮されておらず,手作業で登録を行うので登録漏れがあった場合に単語の意味に関係なく処理を行うことが不可能となってしまう.
%重み付けは単語の出現頻度や文章の長さを使用して単語・文章に順位を付ける手法で必ずしも単語の登録が必要でないため多くの研究で使用されている.
%しかし,重み付けもまた単語を文字列のまま扱っており,意味までは考慮されていない.故に ,人間なら対応できる似た単語でも1文字違うだけで対処が困難となる.
議論支援に関する先行研究においてファシリテーターに対する支援を目的としたものは無く,殆どが議論の活性化や可視化を目的としている.
%=================================新手法
近年,自然言語処理の分野において分散表現が多くの研究で使われており.分散表現は文字列である単語を辞書データを使用して実数ベクトルへと変換する.辞書データにない単語には対応できないが,多様な処理を1つの辞書データで行うことができる.また,実数ベクトルの各数値が単語の意味を表現するものとなっており,数値を使用して処理を行うことができる.
分散表現を用いることで既存手法より人間の感覚に近しい処理を行うことができる.
%=================================
以上のような背景を踏まえて,分散表現を用いてファシリテーターの代わりに話題の変化を判定し,知らせることを目指す.
話題転換の検出は発言同士の近さ,すなわち発言に含まれる単語意味の近さと見ることができる.
分散表現ではベクトル同士の内積計算を行うことで単語同士の意味の近さを計算することができる.
また,分散表現を使用することで機械翻訳を始めとする複数の分野で精度の向上が確認されている.
\end{comment}
\section{研究の目的}
\label{intro:taget}
本論文では,分散表現を用いて議論中での発言に含まれる単語の関連度を計算し,話題の変化を観測する手法を提案する.

\section{本論文の構成}
本論文の構成を以下に示す.
\ref{relwork:chapter} 章では要約手法に関する研究と,分散表現に関する先行研究を紹介する.
次に,\ref{model:chapter}章では発言の要約手法の説明を行い,\ref{impl:chapter}章では分散表現を用いた単語集合間の関連度計算について説明する.
そして,\ref{exp:chapter}章では話題転換点の検出の評価実験について説明する.
最後に\ref{con:chapter}章で本論文のまとめと考察を示す.

 %-------------------------------------------------------------------------------
 \expandafter\ifx\csname MasterFile\endcsname\relax
	\def\BibFile{hoge}
	\input{../Bibliography/chapter}
  \fi
  %-------------------------------------------------------------------------------
  \expandafter\ifx\csname MasterFile\endcsname\relax
  \end{document}
  \fi

%-------------------------------------------------------------------------------
\expandafter\ifx\csname MasterFile\endcsname\relax
	\def\SubFile{hoge}
	\input{../thesis/thesis}
	\begin{document}
	\setcounter{chapter}{0}
	\fi
  %-------------------------------------------------------------------------------
\cleardoublepage
\chapter{序論}
\label{intro:chapter}
%本章では, 本研究を行なうに至った背景と目的について述べる.その後,本論文の構成について述べる.
\section{研究の背景}
\label{intro:background}
近年,Web上での大規模な議論活動が活発になっているが,現在一般的に使われている "2ちゃんねる" や "Twitter" といったシステムでは整理や収束を行うことが困難である.困難である原因として,議論の管理を行う者がいないことが挙げられる.
つまり,議論を整理・収束させるには議論のマネジメントを行う人物が必要である.
%
大規模意見集約システムCOLLAGREE\cite{collagreeTest}ではファシリテーターと呼ばれる人物が議論のマネジメントを行っている.
しかし,ファシリテーターは人間であり,長時間に渡って大人数での議論の動向をマネジメントし続けるのは困難である.
COLLAGREEで大規模な議論を収束させるためには,ファシリテーターが必要な時には画面を見るようにして,他の時は見なくても済むようにすることで画面に向き合う時間を減らす工夫があることが望ましい.ファシリテーターが画面を見るべきタイミングは議論の話題が変化したときである.以前の議論の内容から外れた発言がされた時,ファシリテーターが適切な発言をすることで,脱線や炎上を避けて議論を収束させることができる.
すなわち,ファシリテーターの代わりに自動的に議論中の話題の変化を観測することが求められている.
%
現在,COLLAGREE上で使用されている議論支援システムは「(1)投稿支援システム」と「(2)議論可視化システム」の2つに大別できる.
投稿支援システムはポイント機能やファシリテーションフレーズ簡易投稿機能のように,ユーザーが投稿をする際に何らかの補助やリアクションを行う.現行の機能では選択肢の提示に留まっており,作業量を減らすことには繋がりにくい.
一方,議論可視化システムは議論ツリーやキーワード抽出のように,ユーザーにスレッドとは異なる議論の見方を提供する.
\ref{Fig:argTree1}に議論ツリーの例を示す.
\begin{figure}[htbp]
 \begin{center}
  \includegraphics[width=\textwidth]{../images/2.Related_Work/argTree1.png}
  \caption{議論ツリー}
  \label{Fig:argTree1}
  \vspace{-10pt}
 \end{center}
\end{figure}
現行の機能では議論を見やすくすることに重点が置かれており,議論の把握の助けにはなるが画面に向き合う時間を減らすことにはなりにくい.むしろ,作業量を増やすことになり得ることもある.
従って,現行の支援機能ではファシリテーターの作業量の減少には繋がりにくい.
%
近年,自然言語処理の分野において分散表現が多くの研究で使われており,機械翻訳を始めとする単語の意味が重要となる分野で精度の向上が確認されている.分散表現を用いることで,人間に近い精度で話題の変化を観測することが可能となる.
%
以上のような背景を踏まえて,分散表現を用いて,話題の変化を観測し,話題の変化が確認された時にファシリテーターに伝えることが望ましい.
話題の変化の観測は,発言中に現れる単語の類似度の計算と見なすことができる.
分散表現を用いることで単語間の類似度を求めることができる,値が大きいほど単語がそれぞれ類似した実数ベクトルであることを表す.単語Aと単語Bの実数ベクトルが類似しているとは,単語Aと共に使われることの多い単語と単語Bと共に使われることの多い単語が多く共通していることを示す.故に,分散表現を使って単語の類似度を計算することができる.
%
発言文から単語を選ぶ際には自動要約を用いる.発言文から重要でない単語を取り除くことで関連度の計算の精度を高めることが可能となる.
要約の手法としてはokapi BM25 \cite{okapiBM25}とLexRankを組み合わせた抽出的要約手法を用いる.
\begin{comment}
%======================================= 社会的背景
2013年頃からWeb上での大規模な議論活動が活発になり,大規模な人数での議論が期待されている.
大規模な議論では意見を共有することは可能であるが,議論を整理させることや収束させることは難しい.以上から大規模意見集約システムCOLLAGREEが開発された.本システムではWeb上で適切に大規模な議論を行うことができるように議論をマネジメントするファシリテーターを導入した\cite{collagreeTest}.
過去の実験ではファシリテーターの存在が議論の集約に大きな役割を果たしていることが認識されており,大規模な議論のためにファシリテータは必要である.しかし,議論の規模に伴って議論時間が長くなる傾向があり,同時にファシリテーターは常に議論の動向を見続ける必要がある.故に,議論の規模が大きくなればなるほどファシリテーターは長時間かつ大規模な議論の動向の監視によって大きな負担がかかる.大規模な議論が増加する傾向を踏まえるとファシリテーターにかかる負担を軽減する支援が必要である.\\
以上の問題を解決するため,話題の変化を追い,重要な話題の転換点をファシリテーターの代わりに検出することが有用であると考える.必要な時にだけファシリテーターが画面を見れば良いようにすることでファシリテーターの負担軽減が期待できる.
%========================================= 現行手法問題点背景
%議論支援に関する先行研究において,既存の手法は全てが文字列を文字列のまま扱う手法である.
%既存手法は殆どがパターンマッチングと重み付けの2つに区分することができる.
%パターンマッチングでは事前に単語を登録して,単語がマッチした場合に処理を行うが,処理それぞれに対して単語を登録しなければならず手間が膨大になってしまう.また,単語の意味が考慮されておらず,手作業で登録を行うので登録漏れがあった場合に単語の意味に関係なく処理を行うことが不可能となってしまう.
%重み付けは単語の出現頻度や文章の長さを使用して単語・文章に順位を付ける手法で必ずしも単語の登録が必要でないため多くの研究で使用されている.
%しかし,重み付けもまた単語を文字列のまま扱っており,意味までは考慮されていない.故に ,人間なら対応できる似た単語でも1文字違うだけで対処が困難となる.
議論支援に関する先行研究においてファシリテーターに対する支援を目的としたものは無く,殆どが議論の活性化や可視化を目的としている.
%=================================新手法
近年,自然言語処理の分野において分散表現が多くの研究で使われており.分散表現は文字列である単語を辞書データを使用して実数ベクトルへと変換する.辞書データにない単語には対応できないが,多様な処理を1つの辞書データで行うことができる.また,実数ベクトルの各数値が単語の意味を表現するものとなっており,数値を使用して処理を行うことができる.
分散表現を用いることで既存手法より人間の感覚に近しい処理を行うことができる.
%=================================
以上のような背景を踏まえて,分散表現を用いてファシリテーターの代わりに話題の変化を判定し,知らせることを目指す.
話題転換の検出は発言同士の近さ,すなわち発言に含まれる単語意味の近さと見ることができる.
分散表現ではベクトル同士の内積計算を行うことで単語同士の意味の近さを計算することができる.
また,分散表現を使用することで機械翻訳を始めとする複数の分野で精度の向上が確認されている.
\end{comment}
\section{研究の目的}
\label{intro:taget}
本論文では,分散表現を用いて議論中での発言に含まれる単語の関連度を計算し,話題の変化を観測する手法を提案する.

\section{本論文の構成}
本論文の構成を以下に示す.
\ref{relwork:chapter} 章では要約手法に関する研究と,分散表現に関する先行研究を紹介する.
次に,\ref{model:chapter}章では発言の要約手法の説明を行い,\ref{impl:chapter}章では分散表現を用いた単語集合間の関連度計算について説明する.
そして,\ref{exp:chapter}章では話題転換点の検出の評価実験について説明する.
最後に\ref{con:chapter}章で本論文のまとめと考察を示す.

 %-------------------------------------------------------------------------------
 \expandafter\ifx\csname MasterFile\endcsname\relax
	\def\BibFile{hoge}
	\input{../Bibliography/chapter}
  \fi
  %-------------------------------------------------------------------------------
  \expandafter\ifx\csname MasterFile\endcsname\relax
  \end{document}
  \fi

%-------------------------------------------------------------------------------
\expandafter\ifx\csname MasterFile\endcsname\relax
	\def\SubFile{hoge}
	\input{../thesis/thesis}
	\begin{document}
	\setcounter{chapter}{0}
	\fi
  %-------------------------------------------------------------------------------
\cleardoublepage
\chapter{序論}
\label{intro:chapter}
%本章では, 本研究を行なうに至った背景と目的について述べる.その後,本論文の構成について述べる.
\section{研究の背景}
\label{intro:background}
近年,Web上での大規模な議論活動が活発になっているが,現在一般的に使われている "2ちゃんねる" や "Twitter" といったシステムでは整理や収束を行うことが困難である.困難である原因として,議論の管理を行う者がいないことが挙げられる.
つまり,議論を整理・収束させるには議論のマネジメントを行う人物が必要である.
%
大規模意見集約システムCOLLAGREE\cite{collagreeTest}ではファシリテーターと呼ばれる人物が議論のマネジメントを行っている.
しかし,ファシリテーターは人間であり,長時間に渡って大人数での議論の動向をマネジメントし続けるのは困難である.
COLLAGREEで大規模な議論を収束させるためには,ファシリテーターが必要な時には画面を見るようにして,他の時は見なくても済むようにすることで画面に向き合う時間を減らす工夫があることが望ましい.ファシリテーターが画面を見るべきタイミングは議論の話題が変化したときである.以前の議論の内容から外れた発言がされた時,ファシリテーターが適切な発言をすることで,脱線や炎上を避けて議論を収束させることができる.
すなわち,ファシリテーターの代わりに自動的に議論中の話題の変化を観測することが求められている.
%
現在,COLLAGREE上で使用されている議論支援システムは「(1)投稿支援システム」と「(2)議論可視化システム」の2つに大別できる.
投稿支援システムはポイント機能やファシリテーションフレーズ簡易投稿機能のように,ユーザーが投稿をする際に何らかの補助やリアクションを行う.現行の機能では選択肢の提示に留まっており,作業量を減らすことには繋がりにくい.
一方,議論可視化システムは議論ツリーやキーワード抽出のように,ユーザーにスレッドとは異なる議論の見方を提供する.
\ref{Fig:argTree1}に議論ツリーの例を示す.
\begin{figure}[htbp]
 \begin{center}
  \includegraphics[width=\textwidth]{../images/2.Related_Work/argTree1.png}
  \caption{議論ツリー}
  \label{Fig:argTree1}
  \vspace{-10pt}
 \end{center}
\end{figure}
現行の機能では議論を見やすくすることに重点が置かれており,議論の把握の助けにはなるが画面に向き合う時間を減らすことにはなりにくい.むしろ,作業量を増やすことになり得ることもある.
従って,現行の支援機能ではファシリテーターの作業量の減少には繋がりにくい.
%
近年,自然言語処理の分野において分散表現が多くの研究で使われており,機械翻訳を始めとする単語の意味が重要となる分野で精度の向上が確認されている.分散表現を用いることで,人間に近い精度で話題の変化を観測することが可能となる.
%
以上のような背景を踏まえて,分散表現を用いて,話題の変化を観測し,話題の変化が確認された時にファシリテーターに伝えることが望ましい.
話題の変化の観測は,発言中に現れる単語の類似度の計算と見なすことができる.
分散表現を用いることで単語間の類似度を求めることができる,値が大きいほど単語がそれぞれ類似した実数ベクトルであることを表す.単語Aと単語Bの実数ベクトルが類似しているとは,単語Aと共に使われることの多い単語と単語Bと共に使われることの多い単語が多く共通していることを示す.故に,分散表現を使って単語の類似度を計算することができる.
%
発言文から単語を選ぶ際には自動要約を用いる.発言文から重要でない単語を取り除くことで関連度の計算の精度を高めることが可能となる.
要約の手法としてはokapi BM25 \cite{okapiBM25}とLexRankを組み合わせた抽出的要約手法を用いる.
\begin{comment}
%======================================= 社会的背景
2013年頃からWeb上での大規模な議論活動が活発になり,大規模な人数での議論が期待されている.
大規模な議論では意見を共有することは可能であるが,議論を整理させることや収束させることは難しい.以上から大規模意見集約システムCOLLAGREEが開発された.本システムではWeb上で適切に大規模な議論を行うことができるように議論をマネジメントするファシリテーターを導入した\cite{collagreeTest}.
過去の実験ではファシリテーターの存在が議論の集約に大きな役割を果たしていることが認識されており,大規模な議論のためにファシリテータは必要である.しかし,議論の規模に伴って議論時間が長くなる傾向があり,同時にファシリテーターは常に議論の動向を見続ける必要がある.故に,議論の規模が大きくなればなるほどファシリテーターは長時間かつ大規模な議論の動向の監視によって大きな負担がかかる.大規模な議論が増加する傾向を踏まえるとファシリテーターにかかる負担を軽減する支援が必要である.\\
以上の問題を解決するため,話題の変化を追い,重要な話題の転換点をファシリテーターの代わりに検出することが有用であると考える.必要な時にだけファシリテーターが画面を見れば良いようにすることでファシリテーターの負担軽減が期待できる.
%========================================= 現行手法問題点背景
%議論支援に関する先行研究において,既存の手法は全てが文字列を文字列のまま扱う手法である.
%既存手法は殆どがパターンマッチングと重み付けの2つに区分することができる.
%パターンマッチングでは事前に単語を登録して,単語がマッチした場合に処理を行うが,処理それぞれに対して単語を登録しなければならず手間が膨大になってしまう.また,単語の意味が考慮されておらず,手作業で登録を行うので登録漏れがあった場合に単語の意味に関係なく処理を行うことが不可能となってしまう.
%重み付けは単語の出現頻度や文章の長さを使用して単語・文章に順位を付ける手法で必ずしも単語の登録が必要でないため多くの研究で使用されている.
%しかし,重み付けもまた単語を文字列のまま扱っており,意味までは考慮されていない.故に ,人間なら対応できる似た単語でも1文字違うだけで対処が困難となる.
議論支援に関する先行研究においてファシリテーターに対する支援を目的としたものは無く,殆どが議論の活性化や可視化を目的としている.
%=================================新手法
近年,自然言語処理の分野において分散表現が多くの研究で使われており.分散表現は文字列である単語を辞書データを使用して実数ベクトルへと変換する.辞書データにない単語には対応できないが,多様な処理を1つの辞書データで行うことができる.また,実数ベクトルの各数値が単語の意味を表現するものとなっており,数値を使用して処理を行うことができる.
分散表現を用いることで既存手法より人間の感覚に近しい処理を行うことができる.
%=================================
以上のような背景を踏まえて,分散表現を用いてファシリテーターの代わりに話題の変化を判定し,知らせることを目指す.
話題転換の検出は発言同士の近さ,すなわち発言に含まれる単語意味の近さと見ることができる.
分散表現ではベクトル同士の内積計算を行うことで単語同士の意味の近さを計算することができる.
また,分散表現を使用することで機械翻訳を始めとする複数の分野で精度の向上が確認されている.
\end{comment}
\section{研究の目的}
\label{intro:taget}
本論文では,分散表現を用いて議論中での発言に含まれる単語の関連度を計算し,話題の変化を観測する手法を提案する.

\section{本論文の構成}
本論文の構成を以下に示す.
\ref{relwork:chapter} 章では要約手法に関する研究と,分散表現に関する先行研究を紹介する.
次に,\ref{model:chapter}章では発言の要約手法の説明を行い,\ref{impl:chapter}章では分散表現を用いた単語集合間の関連度計算について説明する.
そして,\ref{exp:chapter}章では話題転換点の検出の評価実験について説明する.
最後に\ref{con:chapter}章で本論文のまとめと考察を示す.

 %-------------------------------------------------------------------------------
 \expandafter\ifx\csname MasterFile\endcsname\relax
	\def\BibFile{hoge}
	\input{../Bibliography/chapter}
  \fi
  %-------------------------------------------------------------------------------
  \expandafter\ifx\csname MasterFile\endcsname\relax
  \end{document}
  \fi

%-------------------------------------------------------------------------------
\expandafter\ifx\csname MasterFile\endcsname\relax
	\def\SubFile{hoge}
	\input{../thesis/thesis}
	\begin{document}
	\setcounter{chapter}{0}
	\fi
  %-------------------------------------------------------------------------------
\cleardoublepage
\chapter{序論}
\label{intro:chapter}
%本章では, 本研究を行なうに至った背景と目的について述べる.その後,本論文の構成について述べる.
\section{研究の背景}
\label{intro:background}
近年,Web上での大規模な議論活動が活発になっているが,現在一般的に使われている "2ちゃんねる" や "Twitter" といったシステムでは整理や収束を行うことが困難である.困難である原因として,議論の管理を行う者がいないことが挙げられる.
つまり,議論を整理・収束させるには議論のマネジメントを行う人物が必要である.
%
大規模意見集約システムCOLLAGREE\cite{collagreeTest}ではファシリテーターと呼ばれる人物が議論のマネジメントを行っている.
しかし,ファシリテーターは人間であり,長時間に渡って大人数での議論の動向をマネジメントし続けるのは困難である.
COLLAGREEで大規模な議論を収束させるためには,ファシリテーターが必要な時には画面を見るようにして,他の時は見なくても済むようにすることで画面に向き合う時間を減らす工夫があることが望ましい.ファシリテーターが画面を見るべきタイミングは議論の話題が変化したときである.以前の議論の内容から外れた発言がされた時,ファシリテーターが適切な発言をすることで,脱線や炎上を避けて議論を収束させることができる.
すなわち,ファシリテーターの代わりに自動的に議論中の話題の変化を観測することが求められている.
%
現在,COLLAGREE上で使用されている議論支援システムは「(1)投稿支援システム」と「(2)議論可視化システム」の2つに大別できる.
投稿支援システムはポイント機能やファシリテーションフレーズ簡易投稿機能のように,ユーザーが投稿をする際に何らかの補助やリアクションを行う.現行の機能では選択肢の提示に留まっており,作業量を減らすことには繋がりにくい.
一方,議論可視化システムは議論ツリーやキーワード抽出のように,ユーザーにスレッドとは異なる議論の見方を提供する.
\ref{Fig:argTree1}に議論ツリーの例を示す.
\begin{figure}[htbp]
 \begin{center}
  \includegraphics[width=\textwidth]{../images/2.Related_Work/argTree1.png}
  \caption{議論ツリー}
  \label{Fig:argTree1}
  \vspace{-10pt}
 \end{center}
\end{figure}
現行の機能では議論を見やすくすることに重点が置かれており,議論の把握の助けにはなるが画面に向き合う時間を減らすことにはなりにくい.むしろ,作業量を増やすことになり得ることもある.
従って,現行の支援機能ではファシリテーターの作業量の減少には繋がりにくい.
%
近年,自然言語処理の分野において分散表現が多くの研究で使われており,機械翻訳を始めとする単語の意味が重要となる分野で精度の向上が確認されている.分散表現を用いることで,人間に近い精度で話題の変化を観測することが可能となる.
%
以上のような背景を踏まえて,分散表現を用いて,話題の変化を観測し,話題の変化が確認された時にファシリテーターに伝えることが望ましい.
話題の変化の観測は,発言中に現れる単語の類似度の計算と見なすことができる.
分散表現を用いることで単語間の類似度を求めることができる,値が大きいほど単語がそれぞれ類似した実数ベクトルであることを表す.単語Aと単語Bの実数ベクトルが類似しているとは,単語Aと共に使われることの多い単語と単語Bと共に使われることの多い単語が多く共通していることを示す.故に,分散表現を使って単語の類似度を計算することができる.
%
発言文から単語を選ぶ際には自動要約を用いる.発言文から重要でない単語を取り除くことで関連度の計算の精度を高めることが可能となる.
要約の手法としてはokapi BM25 \cite{okapiBM25}とLexRankを組み合わせた抽出的要約手法を用いる.
\begin{comment}
%======================================= 社会的背景
2013年頃からWeb上での大規模な議論活動が活発になり,大規模な人数での議論が期待されている.
大規模な議論では意見を共有することは可能であるが,議論を整理させることや収束させることは難しい.以上から大規模意見集約システムCOLLAGREEが開発された.本システムではWeb上で適切に大規模な議論を行うことができるように議論をマネジメントするファシリテーターを導入した\cite{collagreeTest}.
過去の実験ではファシリテーターの存在が議論の集約に大きな役割を果たしていることが認識されており,大規模な議論のためにファシリテータは必要である.しかし,議論の規模に伴って議論時間が長くなる傾向があり,同時にファシリテーターは常に議論の動向を見続ける必要がある.故に,議論の規模が大きくなればなるほどファシリテーターは長時間かつ大規模な議論の動向の監視によって大きな負担がかかる.大規模な議論が増加する傾向を踏まえるとファシリテーターにかかる負担を軽減する支援が必要である.\\
以上の問題を解決するため,話題の変化を追い,重要な話題の転換点をファシリテーターの代わりに検出することが有用であると考える.必要な時にだけファシリテーターが画面を見れば良いようにすることでファシリテーターの負担軽減が期待できる.
%========================================= 現行手法問題点背景
%議論支援に関する先行研究において,既存の手法は全てが文字列を文字列のまま扱う手法である.
%既存手法は殆どがパターンマッチングと重み付けの2つに区分することができる.
%パターンマッチングでは事前に単語を登録して,単語がマッチした場合に処理を行うが,処理それぞれに対して単語を登録しなければならず手間が膨大になってしまう.また,単語の意味が考慮されておらず,手作業で登録を行うので登録漏れがあった場合に単語の意味に関係なく処理を行うことが不可能となってしまう.
%重み付けは単語の出現頻度や文章の長さを使用して単語・文章に順位を付ける手法で必ずしも単語の登録が必要でないため多くの研究で使用されている.
%しかし,重み付けもまた単語を文字列のまま扱っており,意味までは考慮されていない.故に ,人間なら対応できる似た単語でも1文字違うだけで対処が困難となる.
議論支援に関する先行研究においてファシリテーターに対する支援を目的としたものは無く,殆どが議論の活性化や可視化を目的としている.
%=================================新手法
近年,自然言語処理の分野において分散表現が多くの研究で使われており.分散表現は文字列である単語を辞書データを使用して実数ベクトルへと変換する.辞書データにない単語には対応できないが,多様な処理を1つの辞書データで行うことができる.また,実数ベクトルの各数値が単語の意味を表現するものとなっており,数値を使用して処理を行うことができる.
分散表現を用いることで既存手法より人間の感覚に近しい処理を行うことができる.
%=================================
以上のような背景を踏まえて,分散表現を用いてファシリテーターの代わりに話題の変化を判定し,知らせることを目指す.
話題転換の検出は発言同士の近さ,すなわち発言に含まれる単語意味の近さと見ることができる.
分散表現ではベクトル同士の内積計算を行うことで単語同士の意味の近さを計算することができる.
また,分散表現を使用することで機械翻訳を始めとする複数の分野で精度の向上が確認されている.
\end{comment}
\section{研究の目的}
\label{intro:taget}
本論文では,分散表現を用いて議論中での発言に含まれる単語の関連度を計算し,話題の変化を観測する手法を提案する.

\section{本論文の構成}
本論文の構成を以下に示す.
\ref{relwork:chapter} 章では要約手法に関する研究と,分散表現に関する先行研究を紹介する.
次に,\ref{model:chapter}章では発言の要約手法の説明を行い,\ref{impl:chapter}章では分散表現を用いた単語集合間の関連度計算について説明する.
そして,\ref{exp:chapter}章では話題転換点の検出の評価実験について説明する.
最後に\ref{con:chapter}章で本論文のまとめと考察を示す.

 %-------------------------------------------------------------------------------
 \expandafter\ifx\csname MasterFile\endcsname\relax
	\def\BibFile{hoge}
	\input{../Bibliography/chapter}
  \fi
  %-------------------------------------------------------------------------------
  \expandafter\ifx\csname MasterFile\endcsname\relax
  \end{document}
  \fi


%===============================================================================
\pagestyle{plain}
%-------------------------------------------------------------------------------
\expandafter\ifx\csname MasterFile\endcsname\relax
	\def\SubFile{hoge}
	\input{../thesis/thesis}
	\begin{document}
	\setcounter{chapter}{0}
	\fi
  %-------------------------------------------------------------------------------
\cleardoublepage
\chapter{序論}
\label{intro:chapter}
%本章では, 本研究を行なうに至った背景と目的について述べる.その後,本論文の構成について述べる.
\section{研究の背景}
\label{intro:background}
近年,Web上での大規模な議論活動が活発になっているが,現在一般的に使われている "2ちゃんねる" や "Twitter" といったシステムでは整理や収束を行うことが困難である.困難である原因として,議論の管理を行う者がいないことが挙げられる.
つまり,議論を整理・収束させるには議論のマネジメントを行う人物が必要である.
%
大規模意見集約システムCOLLAGREE\cite{collagreeTest}ではファシリテーターと呼ばれる人物が議論のマネジメントを行っている.
しかし,ファシリテーターは人間であり,長時間に渡って大人数での議論の動向をマネジメントし続けるのは困難である.
COLLAGREEで大規模な議論を収束させるためには,ファシリテーターが必要な時には画面を見るようにして,他の時は見なくても済むようにすることで画面に向き合う時間を減らす工夫があることが望ましい.ファシリテーターが画面を見るべきタイミングは議論の話題が変化したときである.以前の議論の内容から外れた発言がされた時,ファシリテーターが適切な発言をすることで,脱線や炎上を避けて議論を収束させることができる.
すなわち,ファシリテーターの代わりに自動的に議論中の話題の変化を観測することが求められている.
%
現在,COLLAGREE上で使用されている議論支援システムは「(1)投稿支援システム」と「(2)議論可視化システム」の2つに大別できる.
投稿支援システムはポイント機能やファシリテーションフレーズ簡易投稿機能のように,ユーザーが投稿をする際に何らかの補助やリアクションを行う.現行の機能では選択肢の提示に留まっており,作業量を減らすことには繋がりにくい.
一方,議論可視化システムは議論ツリーやキーワード抽出のように,ユーザーにスレッドとは異なる議論の見方を提供する.
\ref{Fig:argTree1}に議論ツリーの例を示す.
\begin{figure}[htbp]
 \begin{center}
  \includegraphics[width=\textwidth]{../images/2.Related_Work/argTree1.png}
  \caption{議論ツリー}
  \label{Fig:argTree1}
  \vspace{-10pt}
 \end{center}
\end{figure}
現行の機能では議論を見やすくすることに重点が置かれており,議論の把握の助けにはなるが画面に向き合う時間を減らすことにはなりにくい.むしろ,作業量を増やすことになり得ることもある.
従って,現行の支援機能ではファシリテーターの作業量の減少には繋がりにくい.
%
近年,自然言語処理の分野において分散表現が多くの研究で使われており,機械翻訳を始めとする単語の意味が重要となる分野で精度の向上が確認されている.分散表現を用いることで,人間に近い精度で話題の変化を観測することが可能となる.
%
以上のような背景を踏まえて,分散表現を用いて,話題の変化を観測し,話題の変化が確認された時にファシリテーターに伝えることが望ましい.
話題の変化の観測は,発言中に現れる単語の類似度の計算と見なすことができる.
分散表現を用いることで単語間の類似度を求めることができる,値が大きいほど単語がそれぞれ類似した実数ベクトルであることを表す.単語Aと単語Bの実数ベクトルが類似しているとは,単語Aと共に使われることの多い単語と単語Bと共に使われることの多い単語が多く共通していることを示す.故に,分散表現を使って単語の類似度を計算することができる.
%
発言文から単語を選ぶ際には自動要約を用いる.発言文から重要でない単語を取り除くことで関連度の計算の精度を高めることが可能となる.
要約の手法としてはokapi BM25 \cite{okapiBM25}とLexRankを組み合わせた抽出的要約手法を用いる.
\begin{comment}
%======================================= 社会的背景
2013年頃からWeb上での大規模な議論活動が活発になり,大規模な人数での議論が期待されている.
大規模な議論では意見を共有することは可能であるが,議論を整理させることや収束させることは難しい.以上から大規模意見集約システムCOLLAGREEが開発された.本システムではWeb上で適切に大規模な議論を行うことができるように議論をマネジメントするファシリテーターを導入した\cite{collagreeTest}.
過去の実験ではファシリテーターの存在が議論の集約に大きな役割を果たしていることが認識されており,大規模な議論のためにファシリテータは必要である.しかし,議論の規模に伴って議論時間が長くなる傾向があり,同時にファシリテーターは常に議論の動向を見続ける必要がある.故に,議論の規模が大きくなればなるほどファシリテーターは長時間かつ大規模な議論の動向の監視によって大きな負担がかかる.大規模な議論が増加する傾向を踏まえるとファシリテーターにかかる負担を軽減する支援が必要である.\\
以上の問題を解決するため,話題の変化を追い,重要な話題の転換点をファシリテーターの代わりに検出することが有用であると考える.必要な時にだけファシリテーターが画面を見れば良いようにすることでファシリテーターの負担軽減が期待できる.
%========================================= 現行手法問題点背景
%議論支援に関する先行研究において,既存の手法は全てが文字列を文字列のまま扱う手法である.
%既存手法は殆どがパターンマッチングと重み付けの2つに区分することができる.
%パターンマッチングでは事前に単語を登録して,単語がマッチした場合に処理を行うが,処理それぞれに対して単語を登録しなければならず手間が膨大になってしまう.また,単語の意味が考慮されておらず,手作業で登録を行うので登録漏れがあった場合に単語の意味に関係なく処理を行うことが不可能となってしまう.
%重み付けは単語の出現頻度や文章の長さを使用して単語・文章に順位を付ける手法で必ずしも単語の登録が必要でないため多くの研究で使用されている.
%しかし,重み付けもまた単語を文字列のまま扱っており,意味までは考慮されていない.故に ,人間なら対応できる似た単語でも1文字違うだけで対処が困難となる.
議論支援に関する先行研究においてファシリテーターに対する支援を目的としたものは無く,殆どが議論の活性化や可視化を目的としている.
%=================================新手法
近年,自然言語処理の分野において分散表現が多くの研究で使われており.分散表現は文字列である単語を辞書データを使用して実数ベクトルへと変換する.辞書データにない単語には対応できないが,多様な処理を1つの辞書データで行うことができる.また,実数ベクトルの各数値が単語の意味を表現するものとなっており,数値を使用して処理を行うことができる.
分散表現を用いることで既存手法より人間の感覚に近しい処理を行うことができる.
%=================================
以上のような背景を踏まえて,分散表現を用いてファシリテーターの代わりに話題の変化を判定し,知らせることを目指す.
話題転換の検出は発言同士の近さ,すなわち発言に含まれる単語意味の近さと見ることができる.
分散表現ではベクトル同士の内積計算を行うことで単語同士の意味の近さを計算することができる.
また,分散表現を使用することで機械翻訳を始めとする複数の分野で精度の向上が確認されている.
\end{comment}
\section{研究の目的}
\label{intro:taget}
本論文では,分散表現を用いて議論中での発言に含まれる単語の関連度を計算し,話題の変化を観測する手法を提案する.

\section{本論文の構成}
本論文の構成を以下に示す.
\ref{relwork:chapter} 章では要約手法に関する研究と,分散表現に関する先行研究を紹介する.
次に,\ref{model:chapter}章では発言の要約手法の説明を行い,\ref{impl:chapter}章では分散表現を用いた単語集合間の関連度計算について説明する.
そして,\ref{exp:chapter}章では話題転換点の検出の評価実験について説明する.
最後に\ref{con:chapter}章で本論文のまとめと考察を示す.

 %-------------------------------------------------------------------------------
 \expandafter\ifx\csname MasterFile\endcsname\relax
	\def\BibFile{hoge}
	\input{../Bibliography/chapter}
  \fi
  %-------------------------------------------------------------------------------
  \expandafter\ifx\csname MasterFile\endcsname\relax
  \end{document}
  \fi
 %謝辞
%-------------------------------------------------------------------------------
\def\BibFile{../Bibliograhoy/database2}
\expandafter\ifx\csname MasterFile\endcsname\relax
	\def\SubFile{hoge}
	\input{../thesis/thesis}
	\begin{document}
	\setcounter{chapter}{0}
	\fi
  %-------------------------------------------------------------------------------
\cleardoublepage
\chapter{序論}
\label{intro:chapter}
%本章では, 本研究を行なうに至った背景と目的について述べる.その後,本論文の構成について述べる.
\section{研究の背景}
\label{intro:background}
近年,Web上での大規模な議論活動が活発になっているが,現在一般的に使われている "2ちゃんねる" や "Twitter" といったシステムでは整理や収束を行うことが困難である.困難である原因として,議論の管理を行う者がいないことが挙げられる.
つまり,議論を整理・収束させるには議論のマネジメントを行う人物が必要である.
%
大規模意見集約システムCOLLAGREE\cite{collagreeTest}ではファシリテーターと呼ばれる人物が議論のマネジメントを行っている.
しかし,ファシリテーターは人間であり,長時間に渡って大人数での議論の動向をマネジメントし続けるのは困難である.
COLLAGREEで大規模な議論を収束させるためには,ファシリテーターが必要な時には画面を見るようにして,他の時は見なくても済むようにすることで画面に向き合う時間を減らす工夫があることが望ましい.ファシリテーターが画面を見るべきタイミングは議論の話題が変化したときである.以前の議論の内容から外れた発言がされた時,ファシリテーターが適切な発言をすることで,脱線や炎上を避けて議論を収束させることができる.
すなわち,ファシリテーターの代わりに自動的に議論中の話題の変化を観測することが求められている.
%
現在,COLLAGREE上で使用されている議論支援システムは「(1)投稿支援システム」と「(2)議論可視化システム」の2つに大別できる.
投稿支援システムはポイント機能やファシリテーションフレーズ簡易投稿機能のように,ユーザーが投稿をする際に何らかの補助やリアクションを行う.現行の機能では選択肢の提示に留まっており,作業量を減らすことには繋がりにくい.
一方,議論可視化システムは議論ツリーやキーワード抽出のように,ユーザーにスレッドとは異なる議論の見方を提供する.
\ref{Fig:argTree1}に議論ツリーの例を示す.
\begin{figure}[htbp]
 \begin{center}
  \includegraphics[width=\textwidth]{../images/2.Related_Work/argTree1.png}
  \caption{議論ツリー}
  \label{Fig:argTree1}
  \vspace{-10pt}
 \end{center}
\end{figure}
現行の機能では議論を見やすくすることに重点が置かれており,議論の把握の助けにはなるが画面に向き合う時間を減らすことにはなりにくい.むしろ,作業量を増やすことになり得ることもある.
従って,現行の支援機能ではファシリテーターの作業量の減少には繋がりにくい.
%
近年,自然言語処理の分野において分散表現が多くの研究で使われており,機械翻訳を始めとする単語の意味が重要となる分野で精度の向上が確認されている.分散表現を用いることで,人間に近い精度で話題の変化を観測することが可能となる.
%
以上のような背景を踏まえて,分散表現を用いて,話題の変化を観測し,話題の変化が確認された時にファシリテーターに伝えることが望ましい.
話題の変化の観測は,発言中に現れる単語の類似度の計算と見なすことができる.
分散表現を用いることで単語間の類似度を求めることができる,値が大きいほど単語がそれぞれ類似した実数ベクトルであることを表す.単語Aと単語Bの実数ベクトルが類似しているとは,単語Aと共に使われることの多い単語と単語Bと共に使われることの多い単語が多く共通していることを示す.故に,分散表現を使って単語の類似度を計算することができる.
%
発言文から単語を選ぶ際には自動要約を用いる.発言文から重要でない単語を取り除くことで関連度の計算の精度を高めることが可能となる.
要約の手法としてはokapi BM25 \cite{okapiBM25}とLexRankを組み合わせた抽出的要約手法を用いる.
\begin{comment}
%======================================= 社会的背景
2013年頃からWeb上での大規模な議論活動が活発になり,大規模な人数での議論が期待されている.
大規模な議論では意見を共有することは可能であるが,議論を整理させることや収束させることは難しい.以上から大規模意見集約システムCOLLAGREEが開発された.本システムではWeb上で適切に大規模な議論を行うことができるように議論をマネジメントするファシリテーターを導入した\cite{collagreeTest}.
過去の実験ではファシリテーターの存在が議論の集約に大きな役割を果たしていることが認識されており,大規模な議論のためにファシリテータは必要である.しかし,議論の規模に伴って議論時間が長くなる傾向があり,同時にファシリテーターは常に議論の動向を見続ける必要がある.故に,議論の規模が大きくなればなるほどファシリテーターは長時間かつ大規模な議論の動向の監視によって大きな負担がかかる.大規模な議論が増加する傾向を踏まえるとファシリテーターにかかる負担を軽減する支援が必要である.\\
以上の問題を解決するため,話題の変化を追い,重要な話題の転換点をファシリテーターの代わりに検出することが有用であると考える.必要な時にだけファシリテーターが画面を見れば良いようにすることでファシリテーターの負担軽減が期待できる.
%========================================= 現行手法問題点背景
%議論支援に関する先行研究において,既存の手法は全てが文字列を文字列のまま扱う手法である.
%既存手法は殆どがパターンマッチングと重み付けの2つに区分することができる.
%パターンマッチングでは事前に単語を登録して,単語がマッチした場合に処理を行うが,処理それぞれに対して単語を登録しなければならず手間が膨大になってしまう.また,単語の意味が考慮されておらず,手作業で登録を行うので登録漏れがあった場合に単語の意味に関係なく処理を行うことが不可能となってしまう.
%重み付けは単語の出現頻度や文章の長さを使用して単語・文章に順位を付ける手法で必ずしも単語の登録が必要でないため多くの研究で使用されている.
%しかし,重み付けもまた単語を文字列のまま扱っており,意味までは考慮されていない.故に ,人間なら対応できる似た単語でも1文字違うだけで対処が困難となる.
議論支援に関する先行研究においてファシリテーターに対する支援を目的としたものは無く,殆どが議論の活性化や可視化を目的としている.
%=================================新手法
近年,自然言語処理の分野において分散表現が多くの研究で使われており.分散表現は文字列である単語を辞書データを使用して実数ベクトルへと変換する.辞書データにない単語には対応できないが,多様な処理を1つの辞書データで行うことができる.また,実数ベクトルの各数値が単語の意味を表現するものとなっており,数値を使用して処理を行うことができる.
分散表現を用いることで既存手法より人間の感覚に近しい処理を行うことができる.
%=================================
以上のような背景を踏まえて,分散表現を用いてファシリテーターの代わりに話題の変化を判定し,知らせることを目指す.
話題転換の検出は発言同士の近さ,すなわち発言に含まれる単語意味の近さと見ることができる.
分散表現ではベクトル同士の内積計算を行うことで単語同士の意味の近さを計算することができる.
また,分散表現を使用することで機械翻訳を始めとする複数の分野で精度の向上が確認されている.
\end{comment}
\section{研究の目的}
\label{intro:taget}
本論文では,分散表現を用いて議論中での発言に含まれる単語の関連度を計算し,話題の変化を観測する手法を提案する.

\section{本論文の構成}
本論文の構成を以下に示す.
\ref{relwork:chapter} 章では要約手法に関する研究と,分散表現に関する先行研究を紹介する.
次に,\ref{model:chapter}章では発言の要約手法の説明を行い,\ref{impl:chapter}章では分散表現を用いた単語集合間の関連度計算について説明する.
そして,\ref{exp:chapter}章では話題転換点の検出の評価実験について説明する.
最後に\ref{con:chapter}章で本論文のまとめと考察を示す.

 %-------------------------------------------------------------------------------
 \expandafter\ifx\csname MasterFile\endcsname\relax
	\def\BibFile{hoge}
	\input{../Bibliography/chapter}
  \fi
  %-------------------------------------------------------------------------------
  \expandafter\ifx\csname MasterFile\endcsname\relax
  \end{document}
  \fi
 %参考文献
% %===============================================================================
\appendix
\expandafter\ifx\csname MasterFile\endcsname\relax
	\def\SubFile{hoge}
	\input{../thesis/thesis}
	\begin{document}
	\setcounter{chapter}{0}
	\fi
  %-------------------------------------------------------------------------------
\cleardoublepage
\chapter{序論}
\label{intro:chapter}
%本章では, 本研究を行なうに至った背景と目的について述べる.その後,本論文の構成について述べる.
\section{研究の背景}
\label{intro:background}
近年,Web上での大規模な議論活動が活発になっているが,現在一般的に使われている "2ちゃんねる" や "Twitter" といったシステムでは整理や収束を行うことが困難である.困難である原因として,議論の管理を行う者がいないことが挙げられる.
つまり,議論を整理・収束させるには議論のマネジメントを行う人物が必要である.
%
大規模意見集約システムCOLLAGREE\cite{collagreeTest}ではファシリテーターと呼ばれる人物が議論のマネジメントを行っている.
しかし,ファシリテーターは人間であり,長時間に渡って大人数での議論の動向をマネジメントし続けるのは困難である.
COLLAGREEで大規模な議論を収束させるためには,ファシリテーターが必要な時には画面を見るようにして,他の時は見なくても済むようにすることで画面に向き合う時間を減らす工夫があることが望ましい.ファシリテーターが画面を見るべきタイミングは議論の話題が変化したときである.以前の議論の内容から外れた発言がされた時,ファシリテーターが適切な発言をすることで,脱線や炎上を避けて議論を収束させることができる.
すなわち,ファシリテーターの代わりに自動的に議論中の話題の変化を観測することが求められている.
%
現在,COLLAGREE上で使用されている議論支援システムは「(1)投稿支援システム」と「(2)議論可視化システム」の2つに大別できる.
投稿支援システムはポイント機能やファシリテーションフレーズ簡易投稿機能のように,ユーザーが投稿をする際に何らかの補助やリアクションを行う.現行の機能では選択肢の提示に留まっており,作業量を減らすことには繋がりにくい.
一方,議論可視化システムは議論ツリーやキーワード抽出のように,ユーザーにスレッドとは異なる議論の見方を提供する.
\ref{Fig:argTree1}に議論ツリーの例を示す.
\begin{figure}[htbp]
 \begin{center}
  \includegraphics[width=\textwidth]{../images/2.Related_Work/argTree1.png}
  \caption{議論ツリー}
  \label{Fig:argTree1}
  \vspace{-10pt}
 \end{center}
\end{figure}
現行の機能では議論を見やすくすることに重点が置かれており,議論の把握の助けにはなるが画面に向き合う時間を減らすことにはなりにくい.むしろ,作業量を増やすことになり得ることもある.
従って,現行の支援機能ではファシリテーターの作業量の減少には繋がりにくい.
%
近年,自然言語処理の分野において分散表現が多くの研究で使われており,機械翻訳を始めとする単語の意味が重要となる分野で精度の向上が確認されている.分散表現を用いることで,人間に近い精度で話題の変化を観測することが可能となる.
%
以上のような背景を踏まえて,分散表現を用いて,話題の変化を観測し,話題の変化が確認された時にファシリテーターに伝えることが望ましい.
話題の変化の観測は,発言中に現れる単語の類似度の計算と見なすことができる.
分散表現を用いることで単語間の類似度を求めることができる,値が大きいほど単語がそれぞれ類似した実数ベクトルであることを表す.単語Aと単語Bの実数ベクトルが類似しているとは,単語Aと共に使われることの多い単語と単語Bと共に使われることの多い単語が多く共通していることを示す.故に,分散表現を使って単語の類似度を計算することができる.
%
発言文から単語を選ぶ際には自動要約を用いる.発言文から重要でない単語を取り除くことで関連度の計算の精度を高めることが可能となる.
要約の手法としてはokapi BM25 \cite{okapiBM25}とLexRankを組み合わせた抽出的要約手法を用いる.
\begin{comment}
%======================================= 社会的背景
2013年頃からWeb上での大規模な議論活動が活発になり,大規模な人数での議論が期待されている.
大規模な議論では意見を共有することは可能であるが,議論を整理させることや収束させることは難しい.以上から大規模意見集約システムCOLLAGREEが開発された.本システムではWeb上で適切に大規模な議論を行うことができるように議論をマネジメントするファシリテーターを導入した\cite{collagreeTest}.
過去の実験ではファシリテーターの存在が議論の集約に大きな役割を果たしていることが認識されており,大規模な議論のためにファシリテータは必要である.しかし,議論の規模に伴って議論時間が長くなる傾向があり,同時にファシリテーターは常に議論の動向を見続ける必要がある.故に,議論の規模が大きくなればなるほどファシリテーターは長時間かつ大規模な議論の動向の監視によって大きな負担がかかる.大規模な議論が増加する傾向を踏まえるとファシリテーターにかかる負担を軽減する支援が必要である.\\
以上の問題を解決するため,話題の変化を追い,重要な話題の転換点をファシリテーターの代わりに検出することが有用であると考える.必要な時にだけファシリテーターが画面を見れば良いようにすることでファシリテーターの負担軽減が期待できる.
%========================================= 現行手法問題点背景
%議論支援に関する先行研究において,既存の手法は全てが文字列を文字列のまま扱う手法である.
%既存手法は殆どがパターンマッチングと重み付けの2つに区分することができる.
%パターンマッチングでは事前に単語を登録して,単語がマッチした場合に処理を行うが,処理それぞれに対して単語を登録しなければならず手間が膨大になってしまう.また,単語の意味が考慮されておらず,手作業で登録を行うので登録漏れがあった場合に単語の意味に関係なく処理を行うことが不可能となってしまう.
%重み付けは単語の出現頻度や文章の長さを使用して単語・文章に順位を付ける手法で必ずしも単語の登録が必要でないため多くの研究で使用されている.
%しかし,重み付けもまた単語を文字列のまま扱っており,意味までは考慮されていない.故に ,人間なら対応できる似た単語でも1文字違うだけで対処が困難となる.
議論支援に関する先行研究においてファシリテーターに対する支援を目的としたものは無く,殆どが議論の活性化や可視化を目的としている.
%=================================新手法
近年,自然言語処理の分野において分散表現が多くの研究で使われており.分散表現は文字列である単語を辞書データを使用して実数ベクトルへと変換する.辞書データにない単語には対応できないが,多様な処理を1つの辞書データで行うことができる.また,実数ベクトルの各数値が単語の意味を表現するものとなっており,数値を使用して処理を行うことができる.
分散表現を用いることで既存手法より人間の感覚に近しい処理を行うことができる.
%=================================
以上のような背景を踏まえて,分散表現を用いてファシリテーターの代わりに話題の変化を判定し,知らせることを目指す.
話題転換の検出は発言同士の近さ,すなわち発言に含まれる単語意味の近さと見ることができる.
分散表現ではベクトル同士の内積計算を行うことで単語同士の意味の近さを計算することができる.
また,分散表現を使用することで機械翻訳を始めとする複数の分野で精度の向上が確認されている.
\end{comment}
\section{研究の目的}
\label{intro:taget}
本論文では,分散表現を用いて議論中での発言に含まれる単語の関連度を計算し,話題の変化を観測する手法を提案する.

\section{本論文の構成}
本論文の構成を以下に示す.
\ref{relwork:chapter} 章では要約手法に関する研究と,分散表現に関する先行研究を紹介する.
次に,\ref{model:chapter}章では発言の要約手法の説明を行い,\ref{impl:chapter}章では分散表現を用いた単語集合間の関連度計算について説明する.
そして,\ref{exp:chapter}章では話題転換点の検出の評価実験について説明する.
最後に\ref{con:chapter}章で本論文のまとめと考察を示す.

 %-------------------------------------------------------------------------------
 \expandafter\ifx\csname MasterFile\endcsname\relax
	\def\BibFile{hoge}
	\input{../Bibliography/chapter}
  \fi
  %-------------------------------------------------------------------------------
  \expandafter\ifx\csname MasterFile\endcsname\relax
  \end{document}
  \fi
 % 投稿論文リスト
\expandafter\ifx\csname MasterFile\endcsname\relax
	\def\SubFile{hoge}
	\input{../thesis/thesis}
	\begin{document}
	\setcounter{chapter}{0}
	\fi
  %-------------------------------------------------------------------------------
\cleardoublepage
\chapter{序論}
\label{intro:chapter}
%本章では, 本研究を行なうに至った背景と目的について述べる.その後,本論文の構成について述べる.
\section{研究の背景}
\label{intro:background}
近年,Web上での大規模な議論活動が活発になっているが,現在一般的に使われている "2ちゃんねる" や "Twitter" といったシステムでは整理や収束を行うことが困難である.困難である原因として,議論の管理を行う者がいないことが挙げられる.
つまり,議論を整理・収束させるには議論のマネジメントを行う人物が必要である.
%
大規模意見集約システムCOLLAGREE\cite{collagreeTest}ではファシリテーターと呼ばれる人物が議論のマネジメントを行っている.
しかし,ファシリテーターは人間であり,長時間に渡って大人数での議論の動向をマネジメントし続けるのは困難である.
COLLAGREEで大規模な議論を収束させるためには,ファシリテーターが必要な時には画面を見るようにして,他の時は見なくても済むようにすることで画面に向き合う時間を減らす工夫があることが望ましい.ファシリテーターが画面を見るべきタイミングは議論の話題が変化したときである.以前の議論の内容から外れた発言がされた時,ファシリテーターが適切な発言をすることで,脱線や炎上を避けて議論を収束させることができる.
すなわち,ファシリテーターの代わりに自動的に議論中の話題の変化を観測することが求められている.
%
現在,COLLAGREE上で使用されている議論支援システムは「(1)投稿支援システム」と「(2)議論可視化システム」の2つに大別できる.
投稿支援システムはポイント機能やファシリテーションフレーズ簡易投稿機能のように,ユーザーが投稿をする際に何らかの補助やリアクションを行う.現行の機能では選択肢の提示に留まっており,作業量を減らすことには繋がりにくい.
一方,議論可視化システムは議論ツリーやキーワード抽出のように,ユーザーにスレッドとは異なる議論の見方を提供する.
\ref{Fig:argTree1}に議論ツリーの例を示す.
\begin{figure}[htbp]
 \begin{center}
  \includegraphics[width=\textwidth]{../images/2.Related_Work/argTree1.png}
  \caption{議論ツリー}
  \label{Fig:argTree1}
  \vspace{-10pt}
 \end{center}
\end{figure}
現行の機能では議論を見やすくすることに重点が置かれており,議論の把握の助けにはなるが画面に向き合う時間を減らすことにはなりにくい.むしろ,作業量を増やすことになり得ることもある.
従って,現行の支援機能ではファシリテーターの作業量の減少には繋がりにくい.
%
近年,自然言語処理の分野において分散表現が多くの研究で使われており,機械翻訳を始めとする単語の意味が重要となる分野で精度の向上が確認されている.分散表現を用いることで,人間に近い精度で話題の変化を観測することが可能となる.
%
以上のような背景を踏まえて,分散表現を用いて,話題の変化を観測し,話題の変化が確認された時にファシリテーターに伝えることが望ましい.
話題の変化の観測は,発言中に現れる単語の類似度の計算と見なすことができる.
分散表現を用いることで単語間の類似度を求めることができる,値が大きいほど単語がそれぞれ類似した実数ベクトルであることを表す.単語Aと単語Bの実数ベクトルが類似しているとは,単語Aと共に使われることの多い単語と単語Bと共に使われることの多い単語が多く共通していることを示す.故に,分散表現を使って単語の類似度を計算することができる.
%
発言文から単語を選ぶ際には自動要約を用いる.発言文から重要でない単語を取り除くことで関連度の計算の精度を高めることが可能となる.
要約の手法としてはokapi BM25 \cite{okapiBM25}とLexRankを組み合わせた抽出的要約手法を用いる.
\begin{comment}
%======================================= 社会的背景
2013年頃からWeb上での大規模な議論活動が活発になり,大規模な人数での議論が期待されている.
大規模な議論では意見を共有することは可能であるが,議論を整理させることや収束させることは難しい.以上から大規模意見集約システムCOLLAGREEが開発された.本システムではWeb上で適切に大規模な議論を行うことができるように議論をマネジメントするファシリテーターを導入した\cite{collagreeTest}.
過去の実験ではファシリテーターの存在が議論の集約に大きな役割を果たしていることが認識されており,大規模な議論のためにファシリテータは必要である.しかし,議論の規模に伴って議論時間が長くなる傾向があり,同時にファシリテーターは常に議論の動向を見続ける必要がある.故に,議論の規模が大きくなればなるほどファシリテーターは長時間かつ大規模な議論の動向の監視によって大きな負担がかかる.大規模な議論が増加する傾向を踏まえるとファシリテーターにかかる負担を軽減する支援が必要である.\\
以上の問題を解決するため,話題の変化を追い,重要な話題の転換点をファシリテーターの代わりに検出することが有用であると考える.必要な時にだけファシリテーターが画面を見れば良いようにすることでファシリテーターの負担軽減が期待できる.
%========================================= 現行手法問題点背景
%議論支援に関する先行研究において,既存の手法は全てが文字列を文字列のまま扱う手法である.
%既存手法は殆どがパターンマッチングと重み付けの2つに区分することができる.
%パターンマッチングでは事前に単語を登録して,単語がマッチした場合に処理を行うが,処理それぞれに対して単語を登録しなければならず手間が膨大になってしまう.また,単語の意味が考慮されておらず,手作業で登録を行うので登録漏れがあった場合に単語の意味に関係なく処理を行うことが不可能となってしまう.
%重み付けは単語の出現頻度や文章の長さを使用して単語・文章に順位を付ける手法で必ずしも単語の登録が必要でないため多くの研究で使用されている.
%しかし,重み付けもまた単語を文字列のまま扱っており,意味までは考慮されていない.故に ,人間なら対応できる似た単語でも1文字違うだけで対処が困難となる.
議論支援に関する先行研究においてファシリテーターに対する支援を目的としたものは無く,殆どが議論の活性化や可視化を目的としている.
%=================================新手法
近年,自然言語処理の分野において分散表現が多くの研究で使われており.分散表現は文字列である単語を辞書データを使用して実数ベクトルへと変換する.辞書データにない単語には対応できないが,多様な処理を1つの辞書データで行うことができる.また,実数ベクトルの各数値が単語の意味を表現するものとなっており,数値を使用して処理を行うことができる.
分散表現を用いることで既存手法より人間の感覚に近しい処理を行うことができる.
%=================================
以上のような背景を踏まえて,分散表現を用いてファシリテーターの代わりに話題の変化を判定し,知らせることを目指す.
話題転換の検出は発言同士の近さ,すなわち発言に含まれる単語意味の近さと見ることができる.
分散表現ではベクトル同士の内積計算を行うことで単語同士の意味の近さを計算することができる.
また,分散表現を使用することで機械翻訳を始めとする複数の分野で精度の向上が確認されている.
\end{comment}
\section{研究の目的}
\label{intro:taget}
本論文では,分散表現を用いて議論中での発言に含まれる単語の関連度を計算し,話題の変化を観測する手法を提案する.

\section{本論文の構成}
本論文の構成を以下に示す.
\ref{relwork:chapter} 章では要約手法に関する研究と,分散表現に関する先行研究を紹介する.
次に,\ref{model:chapter}章では発言の要約手法の説明を行い,\ref{impl:chapter}章では分散表現を用いた単語集合間の関連度計算について説明する.
そして,\ref{exp:chapter}章では話題転換点の検出の評価実験について説明する.
最後に\ref{con:chapter}章で本論文のまとめと考察を示す.

 %-------------------------------------------------------------------------------
 \expandafter\ifx\csname MasterFile\endcsname\relax
	\def\BibFile{hoge}
	\input{../Bibliography/chapter}
  \fi
  %-------------------------------------------------------------------------------
  \expandafter\ifx\csname MasterFile\endcsname\relax
  \end{document}
  \fi
 %
\expandafter\ifx\csname MasterFile\endcsname\relax
	\def\SubFile{hoge}
	\input{../thesis/thesis}
	\begin{document}
	\setcounter{chapter}{0}
	\fi
  %-------------------------------------------------------------------------------
\cleardoublepage
\chapter{序論}
\label{intro:chapter}
%本章では, 本研究を行なうに至った背景と目的について述べる.その後,本論文の構成について述べる.
\section{研究の背景}
\label{intro:background}
近年,Web上での大規模な議論活動が活発になっているが,現在一般的に使われている "2ちゃんねる" や "Twitter" といったシステムでは整理や収束を行うことが困難である.困難である原因として,議論の管理を行う者がいないことが挙げられる.
つまり,議論を整理・収束させるには議論のマネジメントを行う人物が必要である.
%
大規模意見集約システムCOLLAGREE\cite{collagreeTest}ではファシリテーターと呼ばれる人物が議論のマネジメントを行っている.
しかし,ファシリテーターは人間であり,長時間に渡って大人数での議論の動向をマネジメントし続けるのは困難である.
COLLAGREEで大規模な議論を収束させるためには,ファシリテーターが必要な時には画面を見るようにして,他の時は見なくても済むようにすることで画面に向き合う時間を減らす工夫があることが望ましい.ファシリテーターが画面を見るべきタイミングは議論の話題が変化したときである.以前の議論の内容から外れた発言がされた時,ファシリテーターが適切な発言をすることで,脱線や炎上を避けて議論を収束させることができる.
すなわち,ファシリテーターの代わりに自動的に議論中の話題の変化を観測することが求められている.
%
現在,COLLAGREE上で使用されている議論支援システムは「(1)投稿支援システム」と「(2)議論可視化システム」の2つに大別できる.
投稿支援システムはポイント機能やファシリテーションフレーズ簡易投稿機能のように,ユーザーが投稿をする際に何らかの補助やリアクションを行う.現行の機能では選択肢の提示に留まっており,作業量を減らすことには繋がりにくい.
一方,議論可視化システムは議論ツリーやキーワード抽出のように,ユーザーにスレッドとは異なる議論の見方を提供する.
\ref{Fig:argTree1}に議論ツリーの例を示す.
\begin{figure}[htbp]
 \begin{center}
  \includegraphics[width=\textwidth]{../images/2.Related_Work/argTree1.png}
  \caption{議論ツリー}
  \label{Fig:argTree1}
  \vspace{-10pt}
 \end{center}
\end{figure}
現行の機能では議論を見やすくすることに重点が置かれており,議論の把握の助けにはなるが画面に向き合う時間を減らすことにはなりにくい.むしろ,作業量を増やすことになり得ることもある.
従って,現行の支援機能ではファシリテーターの作業量の減少には繋がりにくい.
%
近年,自然言語処理の分野において分散表現が多くの研究で使われており,機械翻訳を始めとする単語の意味が重要となる分野で精度の向上が確認されている.分散表現を用いることで,人間に近い精度で話題の変化を観測することが可能となる.
%
以上のような背景を踏まえて,分散表現を用いて,話題の変化を観測し,話題の変化が確認された時にファシリテーターに伝えることが望ましい.
話題の変化の観測は,発言中に現れる単語の類似度の計算と見なすことができる.
分散表現を用いることで単語間の類似度を求めることができる,値が大きいほど単語がそれぞれ類似した実数ベクトルであることを表す.単語Aと単語Bの実数ベクトルが類似しているとは,単語Aと共に使われることの多い単語と単語Bと共に使われることの多い単語が多く共通していることを示す.故に,分散表現を使って単語の類似度を計算することができる.
%
発言文から単語を選ぶ際には自動要約を用いる.発言文から重要でない単語を取り除くことで関連度の計算の精度を高めることが可能となる.
要約の手法としてはokapi BM25 \cite{okapiBM25}とLexRankを組み合わせた抽出的要約手法を用いる.
\begin{comment}
%======================================= 社会的背景
2013年頃からWeb上での大規模な議論活動が活発になり,大規模な人数での議論が期待されている.
大規模な議論では意見を共有することは可能であるが,議論を整理させることや収束させることは難しい.以上から大規模意見集約システムCOLLAGREEが開発された.本システムではWeb上で適切に大規模な議論を行うことができるように議論をマネジメントするファシリテーターを導入した\cite{collagreeTest}.
過去の実験ではファシリテーターの存在が議論の集約に大きな役割を果たしていることが認識されており,大規模な議論のためにファシリテータは必要である.しかし,議論の規模に伴って議論時間が長くなる傾向があり,同時にファシリテーターは常に議論の動向を見続ける必要がある.故に,議論の規模が大きくなればなるほどファシリテーターは長時間かつ大規模な議論の動向の監視によって大きな負担がかかる.大規模な議論が増加する傾向を踏まえるとファシリテーターにかかる負担を軽減する支援が必要である.\\
以上の問題を解決するため,話題の変化を追い,重要な話題の転換点をファシリテーターの代わりに検出することが有用であると考える.必要な時にだけファシリテーターが画面を見れば良いようにすることでファシリテーターの負担軽減が期待できる.
%========================================= 現行手法問題点背景
%議論支援に関する先行研究において,既存の手法は全てが文字列を文字列のまま扱う手法である.
%既存手法は殆どがパターンマッチングと重み付けの2つに区分することができる.
%パターンマッチングでは事前に単語を登録して,単語がマッチした場合に処理を行うが,処理それぞれに対して単語を登録しなければならず手間が膨大になってしまう.また,単語の意味が考慮されておらず,手作業で登録を行うので登録漏れがあった場合に単語の意味に関係なく処理を行うことが不可能となってしまう.
%重み付けは単語の出現頻度や文章の長さを使用して単語・文章に順位を付ける手法で必ずしも単語の登録が必要でないため多くの研究で使用されている.
%しかし,重み付けもまた単語を文字列のまま扱っており,意味までは考慮されていない.故に ,人間なら対応できる似た単語でも1文字違うだけで対処が困難となる.
議論支援に関する先行研究においてファシリテーターに対する支援を目的としたものは無く,殆どが議論の活性化や可視化を目的としている.
%=================================新手法
近年,自然言語処理の分野において分散表現が多くの研究で使われており.分散表現は文字列である単語を辞書データを使用して実数ベクトルへと変換する.辞書データにない単語には対応できないが,多様な処理を1つの辞書データで行うことができる.また,実数ベクトルの各数値が単語の意味を表現するものとなっており,数値を使用して処理を行うことができる.
分散表現を用いることで既存手法より人間の感覚に近しい処理を行うことができる.
%=================================
以上のような背景を踏まえて,分散表現を用いてファシリテーターの代わりに話題の変化を判定し,知らせることを目指す.
話題転換の検出は発言同士の近さ,すなわち発言に含まれる単語意味の近さと見ることができる.
分散表現ではベクトル同士の内積計算を行うことで単語同士の意味の近さを計算することができる.
また,分散表現を使用することで機械翻訳を始めとする複数の分野で精度の向上が確認されている.
\end{comment}
\section{研究の目的}
\label{intro:taget}
本論文では,分散表現を用いて議論中での発言に含まれる単語の関連度を計算し,話題の変化を観測する手法を提案する.

\section{本論文の構成}
本論文の構成を以下に示す.
\ref{relwork:chapter} 章では要約手法に関する研究と,分散表現に関する先行研究を紹介する.
次に,\ref{model:chapter}章では発言の要約手法の説明を行い,\ref{impl:chapter}章では分散表現を用いた単語集合間の関連度計算について説明する.
そして,\ref{exp:chapter}章では話題転換点の検出の評価実験について説明する.
最後に\ref{con:chapter}章で本論文のまとめと考察を示す.

 %-------------------------------------------------------------------------------
 \expandafter\ifx\csname MasterFile\endcsname\relax
	\def\BibFile{hoge}
	\input{../Bibliography/chapter}
  \fi
  %-------------------------------------------------------------------------------
  \expandafter\ifx\csname MasterFile\endcsname\relax
  \end{document}
  \fi
 %
%===============================================================================
\end{document}
\fi

	\begin{document}
	\setcounter{chapter}{0}
	\fi
  %-------------------------------------------------------------------------------
\cleardoublepage
\chapter{序論}
\label{intro:chapter}
%本章では, 本研究を行なうに至った背景と目的について述べる.その後,本論文の構成について述べる.
\section{研究の背景}
\label{intro:background}
近年,Web上での大規模な議論活動が活発になっているが,現在一般的に使われている "2ちゃんねる" や "Twitter" といったシステムでは整理や収束を行うことが困難である.困難である原因として,議論の管理を行う者がいないことが挙げられる.
つまり,議論を整理・収束させるには議論のマネジメントを行う人物が必要である.
%
大規模意見集約システムCOLLAGREE\cite{collagreeTest}ではファシリテーターと呼ばれる人物が議論のマネジメントを行っている.
しかし,ファシリテーターは人間であり,長時間に渡って大人数での議論の動向をマネジメントし続けるのは困難である.
COLLAGREEで大規模な議論を収束させるためには,ファシリテーターが必要な時には画面を見るようにして,他の時は見なくても済むようにすることで画面に向き合う時間を減らす工夫があることが望ましい.ファシリテーターが画面を見るべきタイミングは議論の話題が変化したときである.以前の議論の内容から外れた発言がされた時,ファシリテーターが適切な発言をすることで,脱線や炎上を避けて議論を収束させることができる.
すなわち,ファシリテーターの代わりに自動的に議論中の話題の変化を観測することが求められている.
%
現在,COLLAGREE上で使用されている議論支援システムは「(1)投稿支援システム」と「(2)議論可視化システム」の2つに大別できる.
投稿支援システムはポイント機能やファシリテーションフレーズ簡易投稿機能のように,ユーザーが投稿をする際に何らかの補助やリアクションを行う.現行の機能では選択肢の提示に留まっており,作業量を減らすことには繋がりにくい.
一方,議論可視化システムは議論ツリーやキーワード抽出のように,ユーザーにスレッドとは異なる議論の見方を提供する.
\ref{Fig:argTree1}に議論ツリーの例を示す.
\begin{figure}[htbp]
 \begin{center}
  \includegraphics[width=\textwidth]{../images/2.Related_Work/argTree1.png}
  \caption{議論ツリー}
  \label{Fig:argTree1}
  \vspace{-10pt}
 \end{center}
\end{figure}
現行の機能では議論を見やすくすることに重点が置かれており,議論の把握の助けにはなるが画面に向き合う時間を減らすことにはなりにくい.むしろ,作業量を増やすことになり得ることもある.
従って,現行の支援機能ではファシリテーターの作業量の減少には繋がりにくい.
%
近年,自然言語処理の分野において分散表現が多くの研究で使われており,機械翻訳を始めとする単語の意味が重要となる分野で精度の向上が確認されている.分散表現を用いることで,人間に近い精度で話題の変化を観測することが可能となる.
%
以上のような背景を踏まえて,分散表現を用いて,話題の変化を観測し,話題の変化が確認された時にファシリテーターに伝えることが望ましい.
話題の変化の観測は,発言中に現れる単語の類似度の計算と見なすことができる.
分散表現を用いることで単語間の類似度を求めることができる,値が大きいほど単語がそれぞれ類似した実数ベクトルであることを表す.単語Aと単語Bの実数ベクトルが類似しているとは,単語Aと共に使われることの多い単語と単語Bと共に使われることの多い単語が多く共通していることを示す.故に,分散表現を使って単語の類似度を計算することができる.
%
発言文から単語を選ぶ際には自動要約を用いる.発言文から重要でない単語を取り除くことで関連度の計算の精度を高めることが可能となる.
要約の手法としてはokapi BM25 \cite{okapiBM25}とLexRankを組み合わせた抽出的要約手法を用いる.
\begin{comment}
%======================================= 社会的背景
2013年頃からWeb上での大規模な議論活動が活発になり,大規模な人数での議論が期待されている.
大規模な議論では意見を共有することは可能であるが,議論を整理させることや収束させることは難しい.以上から大規模意見集約システムCOLLAGREEが開発された.本システムではWeb上で適切に大規模な議論を行うことができるように議論をマネジメントするファシリテーターを導入した\cite{collagreeTest}.
過去の実験ではファシリテーターの存在が議論の集約に大きな役割を果たしていることが認識されており,大規模な議論のためにファシリテータは必要である.しかし,議論の規模に伴って議論時間が長くなる傾向があり,同時にファシリテーターは常に議論の動向を見続ける必要がある.故に,議論の規模が大きくなればなるほどファシリテーターは長時間かつ大規模な議論の動向の監視によって大きな負担がかかる.大規模な議論が増加する傾向を踏まえるとファシリテーターにかかる負担を軽減する支援が必要である.\\
以上の問題を解決するため,話題の変化を追い,重要な話題の転換点をファシリテーターの代わりに検出することが有用であると考える.必要な時にだけファシリテーターが画面を見れば良いようにすることでファシリテーターの負担軽減が期待できる.
%========================================= 現行手法問題点背景
%議論支援に関する先行研究において,既存の手法は全てが文字列を文字列のまま扱う手法である.
%既存手法は殆どがパターンマッチングと重み付けの2つに区分することができる.
%パターンマッチングでは事前に単語を登録して,単語がマッチした場合に処理を行うが,処理それぞれに対して単語を登録しなければならず手間が膨大になってしまう.また,単語の意味が考慮されておらず,手作業で登録を行うので登録漏れがあった場合に単語の意味に関係なく処理を行うことが不可能となってしまう.
%重み付けは単語の出現頻度や文章の長さを使用して単語・文章に順位を付ける手法で必ずしも単語の登録が必要でないため多くの研究で使用されている.
%しかし,重み付けもまた単語を文字列のまま扱っており,意味までは考慮されていない.故に ,人間なら対応できる似た単語でも1文字違うだけで対処が困難となる.
議論支援に関する先行研究においてファシリテーターに対する支援を目的としたものは無く,殆どが議論の活性化や可視化を目的としている.
%=================================新手法
近年,自然言語処理の分野において分散表現が多くの研究で使われており.分散表現は文字列である単語を辞書データを使用して実数ベクトルへと変換する.辞書データにない単語には対応できないが,多様な処理を1つの辞書データで行うことができる.また,実数ベクトルの各数値が単語の意味を表現するものとなっており,数値を使用して処理を行うことができる.
分散表現を用いることで既存手法より人間の感覚に近しい処理を行うことができる.
%=================================
以上のような背景を踏まえて,分散表現を用いてファシリテーターの代わりに話題の変化を判定し,知らせることを目指す.
話題転換の検出は発言同士の近さ,すなわち発言に含まれる単語意味の近さと見ることができる.
分散表現ではベクトル同士の内積計算を行うことで単語同士の意味の近さを計算することができる.
また,分散表現を使用することで機械翻訳を始めとする複数の分野で精度の向上が確認されている.
\end{comment}
\section{研究の目的}
\label{intro:taget}
本論文では,分散表現を用いて議論中での発言に含まれる単語の関連度を計算し,話題の変化を観測する手法を提案する.

\section{本論文の構成}
本論文の構成を以下に示す.
\ref{relwork:chapter} 章では要約手法に関する研究と,分散表現に関する先行研究を紹介する.
次に,\ref{model:chapter}章では発言の要約手法の説明を行い,\ref{impl:chapter}章では分散表現を用いた単語集合間の関連度計算について説明する.
そして,\ref{exp:chapter}章では話題転換点の検出の評価実験について説明する.
最後に\ref{con:chapter}章で本論文のまとめと考察を示す.

 %-------------------------------------------------------------------------------
 \expandafter\ifx\csname MasterFile\endcsname\relax
	\def\BibFile{hoge}
	\expandafter\ifx\csname MasterFile\endcsname\relax
	\def\SubFile{hoge}
	\documentclass[a4j,12pt,twoside,openany]{jreport}
%\nofiles %tocファイルを更新させない
%\documentclass[12pt,a4j,twoside,openany]{jsbook}
\usepackage[dvipdfmx]{graphicx}
\usepackage{../dspc} % ベースラインスキップの指定
\usepackage{../slashbox} % 表に斜線を入れる
%\usepackage{../mediabb}
\usepackage{fancyvrb} % Verbatim環境
\usepackage{fancyhdr} % Headerの下線付き章見出し
\usepackage{here} % float[H]
\usepackage{multirow}
\usepackage{hhline} % 表の罫線の角を美しくする
\usepackage{amsmath} %コレがないとcasesが動かない
\usepackage{amsfonts} % 数学用フォント
\usepackage{bm} % 数式環境での bold
\usepackage{algorithm}
\usepackage{algorithmicx}
\usepackage[noend]{algpseudocode}
\usepackage[flushleft]{threeparttable} % 脚注付きテーブル
\usepackage{enumitem}
\usepackage{comment}
\usepackage{fancybox}
%\usepackage{csvsimple,booktabs,siunitx}
%\usepackage{filecontents}


\setlength{\evensidemargin}{5pt}
\setlength{\oddsidemargin}{40pt}
%\setlength{\headheight}{16.5pt}
%%\setlength{\headheight}{30pt}
\setcounter{secnumdepth}{3}
\setlist[description]{leftmargin=2\parindent,labelindent=\parindent}

\makeatletter
\def\@makechapterhead#1{%
	\vspace*{50\p@}%
	{
		\parindent \z@ \raggedright \normalfont
		\ifnum \c@secnumdepth >\m@ne
		% \if@mainmatter
			\huge\bfseries\@chapapp\thechapter\@chappos
			\par\nobreak
			\vskip 20\p@
		% \fi
		\fi
		\interlinepenalty\@M
		\Huge\bfseries #1\par\nobreak
		\vskip 40\p@
	}
}

%新しいコマンド定義
\newcounter{linenumber}
\newenvironment{listing}{%
  \begin{list}{%
    \small\arabic{linenumber}:}{%
      \usecounter{linenumber}%
      \setlength{\baselineskip}{18pt}%
      \setlength{\itemsep}{0pt}%
      \setlength{\parsep}{0pt}}}%
 {\end{list}}
\newcommand{\figcaption}[1]{\def\@captype{figure}\caption{#1}}
\newcommand{\tblcaption}[1]{\def\@captype{table}\caption{#1}}
\newcommand{\norm}[1]{\left\| #1 \right\|}
\newcommand{\cc}[1]{\multicolumn{1}{|c|}{#1}}
\newcommand{\circled}[1]{\raisebox{.5pt}{\textcircled{\raisebox{-.9pt} {#1}}}}
\newcommand{\specialcell}[2][c]{%
  \begin{tabular}[#1]{@{}c@{}}#2\end{tabular}}
\makeatother
%===============================================================================
\expandafter\ifx\csname SubFile\endcsname\relax
\begin{document}
\def\MasterFile{hoge}
%-------------------------------------------------------------------------------
%\maketitle
\thispagestyle{empty}
\input{../hyoushi/title}
%\addcontentsline{toc}{chapter}{表紙}
\thispagestyle{empty}
\mbox{}\newpage
%===============================================================================
%\frontmatter
%===============================================================================
%\mainmatter
%-------------------------------------------------------------------------------
\pagenumbering{arabic}
\cleardoublepage
\input{../0.Abstract/chapter}
%-------------------------------------------------------------------------------
\clearpage
\addcontentsline{toc}{chapter}{目次}
\tableofcontents

\clearpage
\addcontentsline{toc}{chapter}{図目次}
\listoffigures

\clearpage
\addcontentsline{toc}{chapter}{表目次}
\listoftables

%-------------------------------------------------------------------------------

%=====================
\pagestyle{fancy} % Headerをつける
\renewcommand{\sectionmark}[1]{\markright{\thesection\ \ \ #1}}
\renewcommand{\chaptermark}[1]{\markboth{#1}{}}
\lhead{}
\chead{}
\lfoot{}
\rfoot{}%-------------------------------------------------------------------------------
\input{../1.Introduction/chapter}
%-------------------------------------------------------------------------------
\input{../2.Related_Work/chapter}
%-------------------------------------------------------------------------------
\input{../3.The_Model/chapter}
%-------------------------------------------------------------------------------
\input{../4.Implementation/chapter}
%-------------------------------------------------------------------------------
\input{../5.Experiments/chapter}
%-------------------------------------------------------------------------------
\input{../6.Conclusion/chapter}

%===============================================================================
\pagestyle{plain}
%-------------------------------------------------------------------------------
\input{../7.Acknowledgement/chapter} %謝辞
%-------------------------------------------------------------------------------
\def\BibFile{../Bibliograhoy/database2}
\input{../Bibliography/chapter} %参考文献
% %===============================================================================
\appendix
\input{../A.Mypaper/chapter} % 投稿論文リスト
\input{../B.SIG-CCI2/chapter} %
\input{../C.IJCAI-16/chapter} %
%===============================================================================
\end{document}
\fi

	\begin{document}
	\setcounter{chapter}{0}
	\fi
  %-------------------------------------------------------------------------------
\cleardoublepage
\chapter{序論}
\label{intro:chapter}
%本章では, 本研究を行なうに至った背景と目的について述べる.その後,本論文の構成について述べる.
\section{研究の背景}
\label{intro:background}
近年,Web上での大規模な議論活動が活発になっているが,現在一般的に使われている "2ちゃんねる" や "Twitter" といったシステムでは整理や収束を行うことが困難である.困難である原因として,議論の管理を行う者がいないことが挙げられる.
つまり,議論を整理・収束させるには議論のマネジメントを行う人物が必要である.
%
大規模意見集約システムCOLLAGREE\cite{collagreeTest}ではファシリテーターと呼ばれる人物が議論のマネジメントを行っている.
しかし,ファシリテーターは人間であり,長時間に渡って大人数での議論の動向をマネジメントし続けるのは困難である.
COLLAGREEで大規模な議論を収束させるためには,ファシリテーターが必要な時には画面を見るようにして,他の時は見なくても済むようにすることで画面に向き合う時間を減らす工夫があることが望ましい.ファシリテーターが画面を見るべきタイミングは議論の話題が変化したときである.以前の議論の内容から外れた発言がされた時,ファシリテーターが適切な発言をすることで,脱線や炎上を避けて議論を収束させることができる.
すなわち,ファシリテーターの代わりに自動的に議論中の話題の変化を観測することが求められている.
%
現在,COLLAGREE上で使用されている議論支援システムは「(1)投稿支援システム」と「(2)議論可視化システム」の2つに大別できる.
投稿支援システムはポイント機能やファシリテーションフレーズ簡易投稿機能のように,ユーザーが投稿をする際に何らかの補助やリアクションを行う.現行の機能では選択肢の提示に留まっており,作業量を減らすことには繋がりにくい.
一方,議論可視化システムは議論ツリーやキーワード抽出のように,ユーザーにスレッドとは異なる議論の見方を提供する.
\ref{Fig:argTree1}に議論ツリーの例を示す.
\begin{figure}[htbp]
 \begin{center}
  \includegraphics[width=\textwidth]{../images/2.Related_Work/argTree1.png}
  \caption{議論ツリー}
  \label{Fig:argTree1}
  \vspace{-10pt}
 \end{center}
\end{figure}
現行の機能では議論を見やすくすることに重点が置かれており,議論の把握の助けにはなるが画面に向き合う時間を減らすことにはなりにくい.むしろ,作業量を増やすことになり得ることもある.
従って,現行の支援機能ではファシリテーターの作業量の減少には繋がりにくい.
%
近年,自然言語処理の分野において分散表現が多くの研究で使われており,機械翻訳を始めとする単語の意味が重要となる分野で精度の向上が確認されている.分散表現を用いることで,人間に近い精度で話題の変化を観測することが可能となる.
%
以上のような背景を踏まえて,分散表現を用いて,話題の変化を観測し,話題の変化が確認された時にファシリテーターに伝えることが望ましい.
話題の変化の観測は,発言中に現れる単語の類似度の計算と見なすことができる.
分散表現を用いることで単語間の類似度を求めることができる,値が大きいほど単語がそれぞれ類似した実数ベクトルであることを表す.単語Aと単語Bの実数ベクトルが類似しているとは,単語Aと共に使われることの多い単語と単語Bと共に使われることの多い単語が多く共通していることを示す.故に,分散表現を使って単語の類似度を計算することができる.
%
発言文から単語を選ぶ際には自動要約を用いる.発言文から重要でない単語を取り除くことで関連度の計算の精度を高めることが可能となる.
要約の手法としてはokapi BM25 \cite{okapiBM25}とLexRankを組み合わせた抽出的要約手法を用いる.
\begin{comment}
%======================================= 社会的背景
2013年頃からWeb上での大規模な議論活動が活発になり,大規模な人数での議論が期待されている.
大規模な議論では意見を共有することは可能であるが,議論を整理させることや収束させることは難しい.以上から大規模意見集約システムCOLLAGREEが開発された.本システムではWeb上で適切に大規模な議論を行うことができるように議論をマネジメントするファシリテーターを導入した\cite{collagreeTest}.
過去の実験ではファシリテーターの存在が議論の集約に大きな役割を果たしていることが認識されており,大規模な議論のためにファシリテータは必要である.しかし,議論の規模に伴って議論時間が長くなる傾向があり,同時にファシリテーターは常に議論の動向を見続ける必要がある.故に,議論の規模が大きくなればなるほどファシリテーターは長時間かつ大規模な議論の動向の監視によって大きな負担がかかる.大規模な議論が増加する傾向を踏まえるとファシリテーターにかかる負担を軽減する支援が必要である.\\
以上の問題を解決するため,話題の変化を追い,重要な話題の転換点をファシリテーターの代わりに検出することが有用であると考える.必要な時にだけファシリテーターが画面を見れば良いようにすることでファシリテーターの負担軽減が期待できる.
%========================================= 現行手法問題点背景
%議論支援に関する先行研究において,既存の手法は全てが文字列を文字列のまま扱う手法である.
%既存手法は殆どがパターンマッチングと重み付けの2つに区分することができる.
%パターンマッチングでは事前に単語を登録して,単語がマッチした場合に処理を行うが,処理それぞれに対して単語を登録しなければならず手間が膨大になってしまう.また,単語の意味が考慮されておらず,手作業で登録を行うので登録漏れがあった場合に単語の意味に関係なく処理を行うことが不可能となってしまう.
%重み付けは単語の出現頻度や文章の長さを使用して単語・文章に順位を付ける手法で必ずしも単語の登録が必要でないため多くの研究で使用されている.
%しかし,重み付けもまた単語を文字列のまま扱っており,意味までは考慮されていない.故に ,人間なら対応できる似た単語でも1文字違うだけで対処が困難となる.
議論支援に関する先行研究においてファシリテーターに対する支援を目的としたものは無く,殆どが議論の活性化や可視化を目的としている.
%=================================新手法
近年,自然言語処理の分野において分散表現が多くの研究で使われており.分散表現は文字列である単語を辞書データを使用して実数ベクトルへと変換する.辞書データにない単語には対応できないが,多様な処理を1つの辞書データで行うことができる.また,実数ベクトルの各数値が単語の意味を表現するものとなっており,数値を使用して処理を行うことができる.
分散表現を用いることで既存手法より人間の感覚に近しい処理を行うことができる.
%=================================
以上のような背景を踏まえて,分散表現を用いてファシリテーターの代わりに話題の変化を判定し,知らせることを目指す.
話題転換の検出は発言同士の近さ,すなわち発言に含まれる単語意味の近さと見ることができる.
分散表現ではベクトル同士の内積計算を行うことで単語同士の意味の近さを計算することができる.
また,分散表現を使用することで機械翻訳を始めとする複数の分野で精度の向上が確認されている.
\end{comment}
\section{研究の目的}
\label{intro:taget}
本論文では,分散表現を用いて議論中での発言に含まれる単語の関連度を計算し,話題の変化を観測する手法を提案する.

\section{本論文の構成}
本論文の構成を以下に示す.
\ref{relwork:chapter} 章では要約手法に関する研究と,分散表現に関する先行研究を紹介する.
次に,\ref{model:chapter}章では発言の要約手法の説明を行い,\ref{impl:chapter}章では分散表現を用いた単語集合間の関連度計算について説明する.
そして,\ref{exp:chapter}章では話題転換点の検出の評価実験について説明する.
最後に\ref{con:chapter}章で本論文のまとめと考察を示す.

 %-------------------------------------------------------------------------------
 \expandafter\ifx\csname MasterFile\endcsname\relax
	\def\BibFile{hoge}
	\expandafter\ifx\csname MasterFile\endcsname\relax
	\def\SubFile{hoge}
	\input{../thesis/thesis}
	\begin{document}
	\setcounter{chapter}{0}
	\fi
  %-------------------------------------------------------------------------------
\cleardoublepage
\chapter{序論}
\label{intro:chapter}
%本章では, 本研究を行なうに至った背景と目的について述べる.その後,本論文の構成について述べる.
\section{研究の背景}
\label{intro:background}
近年,Web上での大規模な議論活動が活発になっているが,現在一般的に使われている "2ちゃんねる" や "Twitter" といったシステムでは整理や収束を行うことが困難である.困難である原因として,議論の管理を行う者がいないことが挙げられる.
つまり,議論を整理・収束させるには議論のマネジメントを行う人物が必要である.
%
大規模意見集約システムCOLLAGREE\cite{collagreeTest}ではファシリテーターと呼ばれる人物が議論のマネジメントを行っている.
しかし,ファシリテーターは人間であり,長時間に渡って大人数での議論の動向をマネジメントし続けるのは困難である.
COLLAGREEで大規模な議論を収束させるためには,ファシリテーターが必要な時には画面を見るようにして,他の時は見なくても済むようにすることで画面に向き合う時間を減らす工夫があることが望ましい.ファシリテーターが画面を見るべきタイミングは議論の話題が変化したときである.以前の議論の内容から外れた発言がされた時,ファシリテーターが適切な発言をすることで,脱線や炎上を避けて議論を収束させることができる.
すなわち,ファシリテーターの代わりに自動的に議論中の話題の変化を観測することが求められている.
%
現在,COLLAGREE上で使用されている議論支援システムは「(1)投稿支援システム」と「(2)議論可視化システム」の2つに大別できる.
投稿支援システムはポイント機能やファシリテーションフレーズ簡易投稿機能のように,ユーザーが投稿をする際に何らかの補助やリアクションを行う.現行の機能では選択肢の提示に留まっており,作業量を減らすことには繋がりにくい.
一方,議論可視化システムは議論ツリーやキーワード抽出のように,ユーザーにスレッドとは異なる議論の見方を提供する.
\ref{Fig:argTree1}に議論ツリーの例を示す.
\begin{figure}[htbp]
 \begin{center}
  \includegraphics[width=\textwidth]{../images/2.Related_Work/argTree1.png}
  \caption{議論ツリー}
  \label{Fig:argTree1}
  \vspace{-10pt}
 \end{center}
\end{figure}
現行の機能では議論を見やすくすることに重点が置かれており,議論の把握の助けにはなるが画面に向き合う時間を減らすことにはなりにくい.むしろ,作業量を増やすことになり得ることもある.
従って,現行の支援機能ではファシリテーターの作業量の減少には繋がりにくい.
%
近年,自然言語処理の分野において分散表現が多くの研究で使われており,機械翻訳を始めとする単語の意味が重要となる分野で精度の向上が確認されている.分散表現を用いることで,人間に近い精度で話題の変化を観測することが可能となる.
%
以上のような背景を踏まえて,分散表現を用いて,話題の変化を観測し,話題の変化が確認された時にファシリテーターに伝えることが望ましい.
話題の変化の観測は,発言中に現れる単語の類似度の計算と見なすことができる.
分散表現を用いることで単語間の類似度を求めることができる,値が大きいほど単語がそれぞれ類似した実数ベクトルであることを表す.単語Aと単語Bの実数ベクトルが類似しているとは,単語Aと共に使われることの多い単語と単語Bと共に使われることの多い単語が多く共通していることを示す.故に,分散表現を使って単語の類似度を計算することができる.
%
発言文から単語を選ぶ際には自動要約を用いる.発言文から重要でない単語を取り除くことで関連度の計算の精度を高めることが可能となる.
要約の手法としてはokapi BM25 \cite{okapiBM25}とLexRankを組み合わせた抽出的要約手法を用いる.
\begin{comment}
%======================================= 社会的背景
2013年頃からWeb上での大規模な議論活動が活発になり,大規模な人数での議論が期待されている.
大規模な議論では意見を共有することは可能であるが,議論を整理させることや収束させることは難しい.以上から大規模意見集約システムCOLLAGREEが開発された.本システムではWeb上で適切に大規模な議論を行うことができるように議論をマネジメントするファシリテーターを導入した\cite{collagreeTest}.
過去の実験ではファシリテーターの存在が議論の集約に大きな役割を果たしていることが認識されており,大規模な議論のためにファシリテータは必要である.しかし,議論の規模に伴って議論時間が長くなる傾向があり,同時にファシリテーターは常に議論の動向を見続ける必要がある.故に,議論の規模が大きくなればなるほどファシリテーターは長時間かつ大規模な議論の動向の監視によって大きな負担がかかる.大規模な議論が増加する傾向を踏まえるとファシリテーターにかかる負担を軽減する支援が必要である.\\
以上の問題を解決するため,話題の変化を追い,重要な話題の転換点をファシリテーターの代わりに検出することが有用であると考える.必要な時にだけファシリテーターが画面を見れば良いようにすることでファシリテーターの負担軽減が期待できる.
%========================================= 現行手法問題点背景
%議論支援に関する先行研究において,既存の手法は全てが文字列を文字列のまま扱う手法である.
%既存手法は殆どがパターンマッチングと重み付けの2つに区分することができる.
%パターンマッチングでは事前に単語を登録して,単語がマッチした場合に処理を行うが,処理それぞれに対して単語を登録しなければならず手間が膨大になってしまう.また,単語の意味が考慮されておらず,手作業で登録を行うので登録漏れがあった場合に単語の意味に関係なく処理を行うことが不可能となってしまう.
%重み付けは単語の出現頻度や文章の長さを使用して単語・文章に順位を付ける手法で必ずしも単語の登録が必要でないため多くの研究で使用されている.
%しかし,重み付けもまた単語を文字列のまま扱っており,意味までは考慮されていない.故に ,人間なら対応できる似た単語でも1文字違うだけで対処が困難となる.
議論支援に関する先行研究においてファシリテーターに対する支援を目的としたものは無く,殆どが議論の活性化や可視化を目的としている.
%=================================新手法
近年,自然言語処理の分野において分散表現が多くの研究で使われており.分散表現は文字列である単語を辞書データを使用して実数ベクトルへと変換する.辞書データにない単語には対応できないが,多様な処理を1つの辞書データで行うことができる.また,実数ベクトルの各数値が単語の意味を表現するものとなっており,数値を使用して処理を行うことができる.
分散表現を用いることで既存手法より人間の感覚に近しい処理を行うことができる.
%=================================
以上のような背景を踏まえて,分散表現を用いてファシリテーターの代わりに話題の変化を判定し,知らせることを目指す.
話題転換の検出は発言同士の近さ,すなわち発言に含まれる単語意味の近さと見ることができる.
分散表現ではベクトル同士の内積計算を行うことで単語同士の意味の近さを計算することができる.
また,分散表現を使用することで機械翻訳を始めとする複数の分野で精度の向上が確認されている.
\end{comment}
\section{研究の目的}
\label{intro:taget}
本論文では,分散表現を用いて議論中での発言に含まれる単語の関連度を計算し,話題の変化を観測する手法を提案する.

\section{本論文の構成}
本論文の構成を以下に示す.
\ref{relwork:chapter} 章では要約手法に関する研究と,分散表現に関する先行研究を紹介する.
次に,\ref{model:chapter}章では発言の要約手法の説明を行い,\ref{impl:chapter}章では分散表現を用いた単語集合間の関連度計算について説明する.
そして,\ref{exp:chapter}章では話題転換点の検出の評価実験について説明する.
最後に\ref{con:chapter}章で本論文のまとめと考察を示す.

 %-------------------------------------------------------------------------------
 \expandafter\ifx\csname MasterFile\endcsname\relax
	\def\BibFile{hoge}
	\input{../Bibliography/chapter}
  \fi
  %-------------------------------------------------------------------------------
  \expandafter\ifx\csname MasterFile\endcsname\relax
  \end{document}
  \fi

  \fi
  %-------------------------------------------------------------------------------
  \expandafter\ifx\csname MasterFile\endcsname\relax
  \end{document}
  \fi

  \fi
  %-------------------------------------------------------------------------------
  \expandafter\ifx\csname MasterFile\endcsname\relax
  \end{document}
  \fi

  \fi
  %-------------------------------------------------------------------------------
  \expandafter\ifx\csname MasterFile\endcsname\relax
  \end{document}
  \fi
